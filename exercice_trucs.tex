\documentclass[11pt,french]{article}
  \usepackage[utf8]{inputenc}
  \usepackage[T1]{fontenc}
  \usepackage{babel}
  \usepackage[autolanguage]{numprint}
  \usepackage{fontawesome}      % icônes de la commande \doc
  \usepackage{metalogo}         % logo \XeLaTeX
  \usepackage{hyperref}         % hyperliens; chargé en dernier!

  \newcommand{\class}[1]{\textsf{#1}}
  \newcommand{\fichier}[1]{\texttt{#1}}

  \newcommand{\link}[2]{\href{#1}{#2~\raisebox{-0.2ex}{\faExternalLink}}}
  \newcommand{\doc}[3][documentation]{\link{#3}{#1}%
    \marginpar[\hfill\faBookmark~\fichier{#2}]{\faBookmark~\fichier{#2}}}

  \title{\class{ulthese}: la classe pour les thèses et mémoires de
    l'Université Laval}
  \author{Faculté des études supérieures et postdoctorales\thanks{%
      Cette classe et sa documentation ont été rédigées par Vincent
      Goulet~(Faculté des sciences et de génie) avec la collaboration de
      Koassi D'Almeida~(Faculté des études supérieures et postdoctorales) et
      Pierre Lasou~(Bibliothèque).}}

\begin{document}

\maketitle

\tableofcontents

\section{Introduction}

La classe \class{ulthese} permet de composer avec {\LaTeX} ou
{\XeLaTeX} des thèses et mémoires immédiatement conformes aux règles
générales de présentation matérielle de la Faculté des études
supérieures et postdoctorales (FESP) de l'Université Laval. Ces
règles définissent principalement la présentation des pages de titre
des thèses et mémoires ainsi que la disposition du texte sur la
page. La classe en elle-même est donc relativement simple.

La classe \class{ulthese} est basée sur la classe \class{memoir},
une extension de la classe standard \class{book} facilitant à
plusieurs égards la préparation de documents d'allure
professionnelle dans {\LaTeX}. La classe \class{memoir} est très
configurable et incorpore d'office plus de 30 des paquetages
(\emph{packages}) les plus populaires\footnote{%
Consulter la section~18.24 de la documentation de \class{memoir}
pour la liste ou encore le fichier journal (\emph{log}) de la
compilation d'un document utilisant la classe \class{ulthese}.}. %
L'intégralité des fonctionnalités de \class{memoir} se retrouve donc
disponible dans \class{ulthese}.

La classe \class{memoir} fait partie des distributions
{\LaTeX} modernes; elle devrait donc être installée et disponible
sur votre système. La classe est livrée avec une %
\doc{memman.pdf}{http://texdoc.net/pkg/memoir}\footnote{%
Première occurence d'une convention de ce document quand il s'agit
de documentation d'un paquetage: un hyperlien mène vers une version
en ligne dans le site \href{http://texdoc.net}{TeXdoc Online} et on
trouve dans la marge le nom du fichier correspondant sur un système
doté d'une installation de {\TeX}~Live.} %
exhaustive: le manuel d'instructions fait près de 600~pages! Il peut
être utile de s'y référer de temps à autre pour réaliser une mise en
page particulière.

\section{Installation}

La classe est livrée avec la distribution {\TeX}~Live. Si vous
utilisez cette distribution et qu'elle est à jour, vous devriez
pouvoir utiliser \class{ulthese} sans autre intervention. Dans ce
cas, passez directement à la section suivante.

Le reste de cette section explique comment installer la classe si
elle n'est pas disponible sur votre système ou si la version n'est
pas à jour.

La classe est distribuée sous forme d'une archive
\fichier{ulthese.zip} via le réseau de sites \emph{Comprehensive R
  Archive Network} (CTAN):
\begin{quote}
  \url{http://www.ctan.org/pkg/ulthese}
\end{quote}

L'installation de la classe consiste à créer le fichier
\fichier{ulthese.cls} et plusieurs gabarits \fichier{.tex} à partir du
code source documenté se trouvant dans le fichier
\fichier{ulthese.dtx}. Il est recommandé de simplement créer ces
fichiers dans le dossier de travail de la thèse ou du mémoire.

Pour procéder à l'installation, décompresser l'archive
\fichier{ulthese.zip} dans le dossier de travail, puis compiler avec
{\LaTeX} le fichier \fichier{ulthese.ins} en exécutant
\begin{verbatim}
latex ulthese.ins
\end{verbatim}
depuis une invite de commande. Si l'on est peu familier avec l'invite
de commande, on peut aussi procéder comme avec tout document {\LaTeX},
soit ouvrir le fichier \fichier{ulthese.ins} dans son éditeur de texte
favori et lancer depuis celui-ci la compilation avec {\LaTeX},
pdf{\LaTeX}, {\XeLaTeX} ou un autre moteur {\TeX}.

\end{document}
