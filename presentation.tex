\chapter{Présentation de {\TeX} et {\LaTeX}}
\label{chap:presentation}


{\LaTeX} est certainement

{\TeX}, {\LaTeX}, {\XeLaTeX}, {\LuaLaTeX}

Un peu de phonétique avant d'aller plus loin. La dernière lettre de
tous les acronymes ci-dessus étant non pas un X, mais plutôt la lettre
grecque khi majuscule (visuellement identique), la terminaison se
prononce «tek». Voilà qui devrait permettre de distinguer
immédiatement le système de mise en page du matériau élastique.

de {\TeX}, {\LaTeX} et consorts

\section{Ce que c'est}
\label{sec:presentation:c-est}

À la base de {\LaTeX} et de ses dérivés, il y a toujours le système
{\TeX} développé par Donald Knuth à partir de la fin des années 1970
alors qu'il travaillait à la rédaction de son {\oe}uvre phare
\emph{The Art of Computer Programming}. Comme il n'était pas satisfait
de la qualité typographique des systèmes de mise en page alors
disponibles, il a tout naturellement décidé d'en créer un à la hauteur
de ses exigences!

\begin{itemize}
\item {\TeX} est un système de mise en page (\emph{typesetting}) ou de
  préparation de documents.
\item {\LaTeX} est un ensemble de macro-commandes développé par Leslie
  Lamport en 1983 pour faciliter l'utilisation de {\TeX}. Le terme
  {\LaTeX} en est venu, chez les utilisateurs, à nommer l'ensemble du
  système.
\item {\TeX} et {\LaTeX} sont des langages de balisage (\emph{Markup
    Languages}) qui indiquent la mise en forme du texte à l'intérieur
  de celui-ci.
\item {\TeX} met l'accent sur la production de documents de grande
  qualité à la typographie soignée, surtout pour les mathématiques.
\end{itemize}

\begin{exemple}
  Les traitements de texte sont d'abord et avant tout conçus pour
  reproduire à l'identique ce que l'utilisateur produit à l'écran
  (d'où l'appellation \emph{What You See is What You Get}, WYSIWYG).
  Les systèmes de mise en page, eux, visent plutôt à produire une mise
  en page et une typographie de grande qualité pour un texte donné.

  Voici deux exemples de typographie soignée. D'abord, l'utilisation
  de \emph{ligatures} (jonctions) entre certaines lettres. À gauche,
  ce que produisent les traitements de texte, qui ne voient qu'une
  série de lettres individuelles. À droite, le résultat avec {\LaTeX},
  qui peut analyser le texte et identifier les ligatures.
  \begin{demo}
    \centering
    \begin{minipage}{0.3\linewidth}
      \rmfamily f\/f \quad f\/i \quad f\/l \quad f\/f\/i \quad
      f\/f\/l
    \end{minipage}
    \qquad
    \begin{minipage}{0.3\linewidth}
      \rmfamily ff \quad fi \quad fl \quad ffi \quad ffl
    \end{minipage}
  \end{demo}

  Ensuite, l'espacement entre les lettres. {\LaTeX} l'ajustera selon
  le contexte. Comparer la disposition pour du texte normal, à gauche,
  à celle pour des mathématiques, à droite.
  \begin{demo}
    \centering
    \begin{minipage}{0.3\linewidth}
      \rmfamily xy \quad \emph{xy}
    \end{minipage}
    \qquad
    \begin{minipage}{0.3\linewidth}
      $xy$
    \end{minipage}
  \end{demo}
  \qed
\end{exemple}


\section{Ce que ce n'est pas}
\label{sec:presentation:n-est-pas}

\begin{itemize}
\item Un traitement de texte --- {\LaTeX} impose un mode de travail
  qui permet de séparer \emph{structure} et \emph{apparence} du texte.
\item WYSIWYG --- un système de mise en page est davantage qualité de
  \emph{What You See is What You Mean}.
\item Incompatible --- le code source {\LaTeX} peut être lu et le
  document reproduit à l'identique sur tous les types de systèmes
  informatiques.
\item Instable --- le moteur {\TeX} est considéré exempt de bogues.
\item Imprévisible --- {\LaTeX} fait uniquement et exactement ce qu'on
  lui demande, sans prétendre pouvoir deviner ce que nous voulons
  faire ou, pire, le savoir mieux que nous.
\end{itemize}


\section{Processus de création d'un document}

Le processus de création d'un document {\LaTeX} compte trois phases:
la rédaction, la compilation (ou composition par le système) et la
visualisation du résultat. On peut le représenter schématiquement
ainsi:
\begin{center}
  \begin{minipage}[t]{0.25\linewidth}
    \centering
    {\Huge\faFileTextO} \\ \medskip
    rédaction du texte et balisage avec un \emph{éditeur de texte}
  \end{minipage}
  \quad{\Large\faArrowRight}\quad
  \begin{minipage}[t]{0.25\linewidth}
    \centering
    {\Huge\faCogs} \\ \medskip
    compilation avec un \emph{moteur} {\TeX} depuis la ligne de commande
  \end{minipage}
  \quad{\Large\faArrowRight}\quad
  \begin{minipage}[t]{0.25\linewidth}
    \centering
    {\Huge\faFilePdfO} \\ \medskip
    visualisation avec une visionneuse externe (Aperçu,
    SumatraPDF, etc.)
  \end{minipage}
\end{center}

Les logiciels de rédaction intégrés facilitent grandement les deux
premières étapes --- certains intègrent même une visionneuse PDF pour
englober le processus au complet. Il existe plusieurs de ces
logiciels. Mentionnons, par exemple:
\begin{itemize}
\item \link{http://www.xm1math.net/texmaker/index_fr.html}{Texmaker}
  (multiplateforme);
\item \link{http://www.tug.org/texworks/}{TeXworks} (multiplateforme);
\item \link{http://www.texshop.org/}{TeXShop} (OS~X seulement);
\item \link{http://www.winedt.com}{WinEdt} (Windows seulement);
\item \link{http://kile.sourceforge.net}{Kile} (Linux);
\item à peu près tous les bons éditeurs de texte pour programmeur.
\end{itemize}

\section{Quelques choses simples à réaliser avec {\LaTeX} }

Quiconque a travaillé un tant soit peu les logiciels de traitement de
texte reconnaîtra ci-dessous des éléments de mise en page qui ne sont
pas toujours faciles à réaliser. C'est tout le contraire avec
{\LaTeX}: quand ce n'est pas le comportement par défaut, il suffit en
général d'insérer une commande dans le code source pour obtenir le
résultat souhaité.

\begin{itemize}
\item Page titre standard avec le titre du document, le nom de
  l'auteur et la date (une commande).
\item Table des matières (une commande).
\item Numérotation des pages (automatique).
\item Disposition sur la page des figures et tableaux, numérotation et
  renvois.
\item Numérotation des équations mathématiques et renvois (automatique).
\item Citations et composition de la bibliographie.
\item Coupure de mots (automatique).
\item Document recto-verso avec marges distinctes pour le recto et le
  verso (une commande).
\end{itemize}


\section{Outils de production}

Un petit mot d'abord sur les arcanes de {\TeX}. Dans ce monde, il
existe des \emph{moteurs} et des \emph{formats}. Un moteur {\TeX} est
un programme informatique qui transforme du code source en
représentation d'un document sur une page. Un format est simplement un
ensemble de macro commandes comprises par un moteur et qui est chargé
par défaut lorsque le moteur est invoqué avec un nom de commande
spécifique.

Le \autoref{tab:presentation:moteurs} dresse la liste des divers
\emph{moteurs} {\TeX} et formats (ensembles de macro commandes)
couramment utilisés aujourd'hui.

\begin{table}
  \centering
  \begin{tabular}{llc}
    \toprule
    Moteur & Format & Fichier de sortie \\
    \midrule
    \code{tex} & \emph{plain} \TeX & DVI \\
    \code{tex} (\code{latex}) & \LaTeX & DVI \\
    \code{pdftex} (\code{pdflatex}) & pdf\LaTeX & PDF \\
    \code{xetex} (\code{xelatex}) & \XeLaTeX & PDF \\
    \code{luatex} (\code{lualatex}) & \LuaLaTeX & PDF \\
    \bottomrule
  \end{tabular}
  \caption{Moteurs et formats les plus courants}
  \label{tab:presentation:moteurs}
\end{table}

\begin{itemize}
\item Les formats les plus usuels sont pdf{\LaTeX} et {\XeLaTeX}. La
  classe \class{ulthese} est compatible avec les deux.
\item Le moteur \code{pdftex} est le moteur par défaut des
  distributions {\LaTeX} modernes. Comme son nom l'indique, ce moteur
  produit directement un fichier de sortie en format PDF. C'est la
  principale différence par rapport au moteur \code{tex}.
\item Le moteur \code{xetex} peut utiliser directement les polices de
  caractères du système d'exploitation. On traite de l'utilisation des
  polices de caractères plus en détail à la \autoref{trucs:polices}.
\item Le moteur \code{luatex} et le format {\LuaLaTeX} offrent les mêmes
  avantages que \code{xetex} et {\XeLaTeX} en plus d'intégrer les
  fonctionnalités du langage de programmation
  \link{http://www.lua.org}{Lua}.
\item Le format de fichier de sortie DVI, qui est antérieur aux
  formats PostScript et PDF, permet de décrire la disposition d'un
  document exactement telle qu'elle devrait apparaître à l'écran ou à
  l'impression. C'est un format aujourd'hui plus ou moins tombé en
  désuétude depuis la standardisation autour du format PDF.
\end{itemize}

Le système {\LaTeX} est formé d'un grand nombre de composantes réunies
sont forme d'une \emph{distribution}. La plus populaire distribution
aujourd'hui est %
\link{https://www.tug.org/texlive}{{\TeX}~Live}. %
Elle est administrée par le {\TeX} Users Group. C'est la distribution que
recommandent la Bibliothèque et la Faculté des études supérieures de
l'Université Laval.

\begin{itemize}
\item Sous OS~X, on installera plutôt la distribution %
  \link{https://www.tug.org/mactex/}{Mac{\TeX}}, %
  qui est étroitement dérivée de {\TeX}~Live.
\item Sous Windows, la distribution %
  \link{http://www.miktex.org}{MiK\TeX} %
  demeure aussi très populaire.
\item La philosophie de {\TeX}~Live: tout installer. Cette façon de
  faire est aujourd'hui réalisable puisque l'espace disque est
  disponible à profusion dans les ordinateurs. C'est également la plus
  simple puisque à peu près tout ce que vous êtes susceptible
  d'utiliser dans un système {\TeX} est déjà installé.
\end{itemize}


\begin{information}
  Quelques faits amusants au sujet de {\TeX}.
  \begin{itemize}
  \item {\TeX} est aujourd'hui considéré essentiellement exempt de
    bogue.
  \item Donald Knuth vous offre une récompense (symbolique) si vous en
    trouvez un!
  \item Le numéro de version de {\TeX} converge vers $\pi$:
\begin{lstlisting}
$ tex --version
TeX `\textbf{3.14159265}' (TeX Live 2015)
kpathsea version 6.2.1
Copyright 2015 D.E. Knuth.
[...]
\end{lstlisting}
  \end{itemize}
  Pour en savoir plus:
  \begin{itemize}
  \item \link{https://www.tug.org/whatis.html}{Histoire de \TeX} sur le
    site du {\TeX} Users Group (en anglais);
  \item {\TeX} sur Wikipedia
    (\link{http://fr.wikipedia.org/wiki/TeX}{version française};
    \link{http://en.wikipedia.org/wiki/TeX}{version anglaise}, plus
    complète).
  \end{itemize}
\end{information}


%%%
%%% Exercices
%%%

\section{Exercices}
\label{sec:presentation:exercices}

\begin{exercice}[nosol]
  Démarrer le logiciel intégré de rédaction de votre choix (Texmaker
  et TeXShop constituent des bonnes options pour débuter), puis ouvrir
  et compiler le fichier \fichier{exercice\_minimal.tex}.
\end{exercice}

\begin{exercice}[nosol]
  Question de voir ce que {\LaTeX} peut faire, compiler le document
  élaboré \fichier{exercice\_demo.tex} de la manière suivante:
  \begin{enumerate}[i)]
  \item une fois avec {\LaTeX};
  \item une fois avec {\BibTeX};
  \item deux à trois autres fois avec {\LaTeX}.
  \end{enumerate}
\end{exercice}


%%% Local Variables:
%%% mode: latex
%%% TeX-engine: xetex
%%% TeX-master: "formation-latex-ul"
%%% coding: utf-8
%%% End:
