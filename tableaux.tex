\chapter{Tableaux et figures}
\label{chap:tableaux}

Les tableaux et graphiques ne sont pas les éléments de texte les plus
simples et rapides à créer avec {\LaTeX}. Les traitements de texte
brillent, ici, avec leurs interfaces graphiques permettant de composer
un tableau ou un graphique simple pièce par pièce avec la souris.

En revanche, pour ce type de contenu comme pour tout autre, {\LaTeX}
fait exactement ce qu'on lui demande. À ce chapitre, les traitements
de texte ne brillent plus! Si vous avez déjà eu de la difficulté à
contrôler les bordures d'un tableau, la hauteur des lignes ou la
largeur des colonnes, vous comprenez ce à quoi nous faisons référence.

Avant de discuter de la création ou de l'insertion de tableaux,
graphiques et images, il convient ...


\section{Éléments flottants}
\label{sec:tableaux:floats}

Les tableaux et figures sont des éléments de contenu qui occupent
souvent beaucoup d'espace vertical dans la page. S'il ne reste plus
assez de place pour afficher un tel élément sur une page, {\TeX} devra
le déplacer au début de page suivante et cela risque de produire une
page inesthétique car insuffisamment remplie\footnote{%
  \emph{Underful \verb=\vbox=} dans le jargon de {\TeX}.}. %

Pour éviter cela, il serait souhaitable que {\TeX} puisse insérer un
tableau à l'endroit indiqué dans le code source s'il y a suffisamment
d'espace sur la page pour l'accueillir


De plus, tout tableau et toute figure dans un document soigné devrait
comporter un titre/légende/description ainsi qu'un numéro afin de
pouvoir y faire référence dans le texte («comme l'illustre la
figure~3\dots»).

fonctionnalité très élaborée de LaTeX


éléments flottants
conseils de booktabs
tableaux
figures (insertion de graphiques (Sweave), mention de picture)

LATEX offre à ses valeureux utilisateurs la possibilité d'utiliser des environnements flottants. Ces environnements ont la particularité de rendre « flottants » leur contenu. C'est-à-dire que LATEX choisit à partir d'un algorithme qui tient compte d'un certain nombre de paramètres, la position de l'environnement dans le document.

%%% Local Variables:
%%% mode: latex
%%% TeX-engine: xetex
%%% TeX-master: "formation_latex-partie_2"
%%% coding: utf-8
%%% End:
