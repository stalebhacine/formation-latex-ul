\chapter{Tableaux et figures}
\label{chap:tableaux}

Les tableaux et graphiques ne sont pas les éléments de texte les plus
simples et rapides à créer avec {\LaTeX}. Les traitements de texte
brillent, ici, avec leurs interfaces graphiques permettant de composer
un tableau ou un graphique simple pièce par pièce avec la souris.

En revanche, pour ce type de contenu comme pour tout autre, {\LaTeX}
fait exactement ce qu'on lui demande. À ce chapitre, les traitements
de texte ne brillent plus! Si vous avez déjà eu de la difficulté à
contrôler les bordures d'un tableau, la hauteur des lignes ou la
largeur des colonnes, vous comprenez ce à quoi nous faisons référence.

Avant de discuter de la création ou de l'insertion de tableaux,
graphiques et images, il convient ...


\section{De la conception de beaux tableaux}
\label{sec:tableaux:booktabs}

On utilise les tableaux pour disposer de l'information sous forme de
grille. Ainsi, le premier réflexe pour les mettre en forme consiste-t-il
souvent à mettre en évidence cette grille par le biais de
filets\footnote{%
  Communément appelés «lignes» dans le langage courant ou «bordures»
  dans les logiciels de traitement de texte.} %
horizontaux et verticaux.

C'est une mauvaise idée, une pratique à éviter. Vraiment!

Comparer les deux tableaux ci-dessous. Le premier est mis en forme
selon une approche classique supportée depuis toujours par {\LaTeX}:
filets doubles en entête et en pied de tableau, filets simples entre
chaque ligne et entre les colonnes.

\begin{center}
  \begin{tabular}{|>{$}c<{$}|>{$}r<{$}|>{$}r<{$}|>{$}r<{$}|>{$}c<{$}|>{$}c<{$}|}
    \hline\hline
    i &
    \multicolumn{1}{c|}{$v$} &
    \multicolumn{1}{c|}{$b_i$} &
    \lfloor v/b_i \rfloor & v \bmod b_i & x_i \\
    \hline
    0 & \nombre{91492} &  60 & \nombre{1524} & 52 & 52 \\
    1 &  \nombre{1524} &  60 &           25  & 24 & 24 \\
    2 &            25  &  24 &            1  &  1 &  1 \\
    3 &             1  & 365 &            0  &  1 &  1 \\
    \hline\hline
  \end{tabular}
\end{center}

Le second tableau tire profit des fonctionnalités du paquetage
\pkg{booktabs} et des recommandations de son auteur: les filets
horizontaux sont d'épaisseur différente selon qu'ils sont situés dans
l'entête et dans le pied du tableau ou entre les lignes, l'espace
autour des filets horizontaux est plus grand et, surtout, il n'y a pas
de filets verticaux.

\begin{center}
  \begin{tabular}{>{$}c<{$}>{$}r<{$}>{$}r<{$}>{$}r<{$}>{$}c<{$}>{$}c<{$}}
    \toprule
    i &
    \multicolumn{1}{c}{$v$} &
    \multicolumn{1}{c}{$b_i$} &
    \lfloor v/b_i \rfloor & v \bmod b_i & x_i \\
    \midrule
    0 & \nombre{91492} &  60 & \nombre{1524} & 52 & 52 \\
    1 &  \nombre{1524} &  60 &           25  & 24 & 24 \\
    2 &            25  &  24 &            1  &  1 &  1 \\
    3 &             1  & 365 &            0  &  1 &  1 \\
    \bottomrule
  \end{tabular}
\end{center}

La seconde version n'est-elle pas la plus aérée et la plus facile à
consulter? Vous constatez que, contrairement à ce que l'on pourrait
penser, les filets verticaux ne sont pas du tout requis pour bien
délimiter les colonnes?

Tel que mentionné ci-dessus, le paquetage \pkg{booktabs} ajoute des
fonctionnalités à {\LaTeX} pour améliorer la qualité typographique des
tableaux. Dans la documentation du paquetage, son auteur énonce
quelques règles à suivre pour la mise en forme des tableaux:
\begin{enumerate}
\item ne \emph{jamais} utiliser de filets verticaux. Si l'information du côté
  gauche du tableau semble si différente de celle du côté droit
  qu'un filet apparait vertical nécessaire, scinder simplement
  l'information dans deux tableaux;
\item ne jamais utiliser de filets doubles;
\item placer les unités dans le titre de la colonne plutôt qu'après
  chaque valeur dans le corps du tableau;
\item toujours inscrire un chiffre du côté gauche du séparateur
  décimal: $0,1$ et non $,1$ (pratique plus répandue en anglais où le
  séparateur décimal est le point);
\item ne pas utiliser de symbole pour représenter une valeur répétée
  (par exemple: $''$ ou ---). Laisser un blanc ou répéter la valeur
  s'il subsiste une ambiguïté.
\end{enumerate}

Nous recommandons évidemment de suivre ces règles et c'est pourquoi la
présente documentation ainsi que les fichiers d'exemples font usage
des commandes de \pkg{booktabs}.

Les fonctionnalités de \pkg{booktabs} sont intégrées à la classe
\class{memoir} et par conséquent à \class{ulthese}. Il n'est donc pas
nécessaire de charger le paquetage avec ces deux classes.



\section{Tableaux}
\label{sec:tableaux:tableaux}

Peu importe l'outil utilisé, la création d'un tableau requiert de
préciser à l'ordinateur le nombre de colonnes que contiendra le
tableau, l'entête du tableau si nécessaire et le contenu des
différentes cellules. Cette dernière étape nécessite à son tour une
convention pour pour identifier les passages à la colonne suivante
ainsi que le passage à la ligne suivante.

On crée des tableaux dans {\LaTeX} principalement avec les
environnements \texttt{tabular}, \texttt{tabular*} et \texttt{tabularx}
(ce dernier fourni par le paquetage \pkg{tabularx} ou par la classe
\class{memoir}). La syntaxe de ces environnements est la suivante:
\begin{lstlisting}
\begin{tabular}[`\textit{pos}']{`\textit{format}'}           `\textit{lignes}' \end{tabular}
\begin{tabular*}{`\textit{largeur}'}[`\textit{pos}']{`\textit{format}'} `\textit{lignes}' \end{tabular*}
\begin{tabularx}{`\textit{largeur}'}[`\texttt{pos}']{`\texttt{format}'} `\textit{lignes}' \end{tabularx}
\end{lstlisting}
La signification des arguments est la suivante. Nous ne traitons ici
que les options plus souvent employées. Pour une liste plus
exhaustive, consulter la documentation de la classe \class{memoir}
(chapitre 11) ou \cite{wikibooks:latex} (section
\href{http://en.wikibooks.org/wiki/LaTeX/Tables}{Tables}).

\begin{list}{}{%
    \setlength{\labelsep}{1.5ex}
    \settowidth{\labelwidth}{\textit{largeur}}
    \setlength{\leftmargin}{\labelwidth}
    \addtolength{\leftmargin}{\labelsep}
    \setlength{\parsep}{0.5ex plus0.2ex minus0.2ex}
    \setlength{\itemsep}{0.3ex}
    \renewcommand{\makelabel}[1]{\textit{#1}\hfill}}
%
\item[largeur] Largeur hors tout d'un tableau avec les
  environnements \texttt{tabular*} et \texttt{tabularx}. Autrement,
  avec l'environnement \texttt{tabular}, la largeur d'un tableau est
  déterminée automatiquement pour contenir tout le contenu du tableau,
  quitte à dépasser dans la marge de droite.

  La largeur du tableau est généralement exprimée en fraction de la
  largeur du bloc de texte. Celle-ci est accessible avec la commande
  \verb=\textwidth=. Par exemple, les déclarations suivantes
  définissent respectivement des tableaux occupant toute la largeur
  d'une page et 80~\% de la largeur de la page:
\begin{lstlisting}
\begin{tabular*}{\textwidth}[`\textit{pos}']{`\textit{format}'}
\end{lstlisting}
\begin{lstlisting}
\begin{tabularx}{0.8\textwidth}[`\textit{pos}']{`\textit{format}'}
\end{lstlisting}
  L'environnement \code{tabular*} joue sur l'espace entre les colonnes
  pour parvenir à la largeur prescrite, alors que \code{tabularx} joue
  sur la largeur des colonnes (voir ci-dessous).
%
\item[pos] [Argument optionnel à peu près jamais utilisé.] Alignement
  vertical du tableau par rapport à la ligne de base externe. Les
  valeurs possibles sont \code{t} (le haut du tableau est aligné avec
  la ligne de base) et \code{b} (le bas du tableau est aligné avec la
  ligne de base). Par défaut, le tableau est centré par rapport à la
  ligne de base externe.
%
\item[format] Le format des colonnes et, par le fait même, le
  nombre de colonnes puisque l'argument doit compter un symbole pour
  chaque colonne du tableau. Les principaux symboles de mise en forme
  des colonnes sont:
  \begin{description}
  \item[\normalfont\code{l}] contenu de la colonne aligné à gauche;
  \item[\normalfont\code{r}] contenu de la colonne aligné à droite;
  \item[\normalfont\code{c}] contenu de la colonne centré;
  \item[\normalfont\code{p\{}\textit{lgr}\code{\}}] contenu de la colonne traité comme un
    paragraphe de texte de largeur \textit{lgr};
  \item[\normalfont\code{X}] [environnement \code{tabularx} seulement]
    colonne dont la largeur peut être ajustée pour obtenir un tableau
    de la largeur prescrite; identique à \code{p} par ailleurs.
  \end{description}
  Les symboles \verb=|= et \verb=||= dans \textit{format} servent à
  insérer des filets verticaux simples et doubles entre les colonnes,
  mais nous avons vu à la \autoref{sec:tableaux:booktabs} que c'est
  une pratique à proscrire.
%
\item[lignes]
\end{list}


\section{Éléments flottants}
\label{sec:tableaux:floats}

Les tableaux et figures sont des éléments de contenu qui occupent
souvent beaucoup d'espace vertical dans la page. S'il ne reste plus
assez de place pour afficher un tel élément sur une page, {\TeX} devra
le déplacer au début de page suivante et cela risque de produire une
page inesthétique car insuffisamment remplie\footnote{%
  \emph{Underful \textbackslash\texttt{vbox}} dans le jargon de {\TeX}.}. %

Pour éviter cela, il serait souhaitable que {\TeX} puisse insérer un
tableau à l'endroit indiqué dans le code source s'il y a suffisamment
d'espace sur la page pour l'accueillir


De plus, tout tableau et toute figure dans un document soigné devrait
comporter un titre/légende/description ainsi qu'un numéro afin de
pouvoir y faire référence dans le texte («comme l'illustre la
figure~3\dots»).

fonctionnalité très élaborée de LaTeX


éléments flottants
conseils de booktabs
tableaux
figures (insertion de graphiques (Sweave), mention de picture)

LATEX offre à ses valeureux utilisateurs la possibilité d'utiliser des environnements flottants. Ces environnements ont la particularité de rendre « flottants » leur contenu. C'est-à-dire que LATEX choisit à partir d'un algorithme qui tient compte d'un certain nombre de paramètres, la position de l'environnement dans le document.

%%% Local Variables:
%%% mode: latex
%%% TeX-engine: xetex
%%% TeX-master: "formation_latex-partie_2"
%%% coding: utf-8
%%% End:
