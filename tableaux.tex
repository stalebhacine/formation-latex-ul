\chapter{Boîtes, tableaux et figures}
\label{chap:tableaux}

Les tableaux et graphiques ne sont pas les éléments de texte les plus
simples et rapides à créer avec {\LaTeX}. Les traitements de texte
brillent, ici, avec leurs interfaces graphiques permettant de composer
un tableau ou un graphique simple pièce par pièce avec la souris.

En revanche, pour ce type de contenu comme pour tout autre, {\LaTeX}
fait exactement ce qu'on lui demande. À ce chapitre, les traitements
de texte ne brillent plus! Si vous avez déjà eu de la difficulté à
contrôler les bordures d'un tableau, la hauteur des lignes ou la
largeur des colonnes, vous comprenez ce à quoi nous faisons référence.

Avant de discuter de la création ou de l'insertion de tableaux,
graphiques et images, il convient ...


\section{Boîtes}
\label{sec:tableaux:boites}

Il arrive que l'on doive traiter de manière spéciale une aire
rectangulaire de texte; pour l'encadrer, la mettre en surbrillance ou
la mettre en exergue, par exemple.

Avec les traitements de texte, on aura souvent recours aux tableaux à
de telles fins. {\LaTeX} offre ici une solution plus générale. Les
tableaux servent spécifiquement à disposer de l'information sous forme
de grille. Autrement, on utilise des «boîtes» (\emph{boxes} en
anglais) pour la disposition et la mise en forme de contenu quelconque
se présentant sous forme rectangulaire.

Il n'est pas inutile de savoir que {\TeX} ne manipule que cela, des
boîtes. Pour {\TeX}, chaque caractère, chaque lettre n'est qu'un
rectangle d'une certaine largeur qui s'élève au-dessus de la ligne de
base (les lignes d'une feuille lignée) et qui, parfois, se prolonge
sous la ligne de base (pensons aux lettres \emph{p}, \emph{y} ou
\emph{Q}). Les commandes ci-dessous permettent simplement de créer
d'autres boîtes dont le contrôle des dimensions et du contenu est
laissé à l'usager. Une fois créée, une boîte ne peut être scindée en
parties.

\subsection{Boîtes horizontales}

Le plus simple concept de boîte dans {\LaTeX} est celui de boîte
«horizontale», c'est-à-dire dont le contenu est disposé latéralement
de gauche à droite\footnote{%
  D'où l'appellation \emph{LR (left-right) box} en anglais.}. %
Le contenu est normalement du texte, mais conceptuellement ce pourrait
être n'importe quoi, y compris d'autres boîtes.

Les commandes de base pour créer des boîtes horizontales:
\begin{lstlisting}
\mbox`\marg{texte}'
\fbox`\marg{texte}'
\end{lstlisting}
Elles produisent une boîte de la largeur précise de \meta{texte}. Avec
la commande \cmd{\fbox}, le texte est au surplus \fbox{encadré}. En
usage courant, la commande \cmd{!mbox} sert principalement à deux
choses:
\begin{enumerate}
\item réunir en un bloc du texte que l'on ne veut pas voir scindé
  entre les lignes ou entre les pages;
\item \label{item:tableaux:mbox} avec \verb=\mbox{}=, laisser croire à
  {\TeX} que du contenu apparait à un endroit sans toutefois qu'il
  n'occupe aucun espace.
  %% ça prendrait un exemple de cela dans les exercices!
\end{enumerate}

Il existe également des versions plus générales des commandes
\cmd{\mbox} et \cmd{\fbox}. Il s'agit de:
\begin{lstlisting}
\makebox`\oarg{largeur}\oarg{pos}\marg{texte}'
\framebox`\oarg{largeur}\oarg{pos}\marg{texte}'
\end{lstlisting}
Les arguments optionnels \meta{largeur} et \meta{pos} déterminent
respectivement la largeur de la boîte et la position du texte
dans la boîte. Les valeurs possibles de \meta{pos} sont: \code{l} pour
du texte aligné à gauche, \code{r} pour du texte aligné à droite et
\code{c} (la valeur par défaut) pour du texte centré. Ainsi, la commande
\begin{lstlisting}
\framebox[3.5cm][l]{aligné à gauche}
\end{lstlisting}
produit \framebox[3.5cm][l]{aligné à gauche}, alors que
\begin{lstlisting}
\makebox[3.5cm]{centré}
\end{lstlisting}
produit \makebox[3.5cm]{centré}.

\subsection{Boîtes verticales}

Les boîtes verticales se distinguent des boîtes horizontales par le
fait qu'elles peuvent contenir plusieurs lignes de contenu empilées
les unes au-dessus des autres. Lorsque le contenu en question est du
texte, on obtient des paragraphes\footnote{%
  D'où l'appellation, cette fois, de \emph{paragraph boxes} en anglais
  ou \emph{parboxes} dans le jargon {\LaTeX}.}. %

La commande de base pour créer une boîte verticale est
\begin{lstlisting}
\parbox`\oarg{pos}\marg{largeur}\marg{texte}'
\end{lstlisting}
Ici, l'argument optionnel \meta{pos} permet d'ajuster l'alignement
vertical de la boîte avec la ligne de base: \code{b} ou \code{t} selon
que l'on souhaite aligner, respectivement, le bas ou le haut de la
boîte avec la ligne de base. Par défaut, la boîte est centrée avec la
ligne de base. Cet argument n'a aucun effet si la boîte est le seul
élément de contenu du paragraphe.

On remarquera que l'argument \meta{largeur} est ici obligatoire.
Autrement dit, on doit nécessairement définir la largeur des boîtes
verticales, un peu comme il faut bien définir la largeur de la page
pour le texte normal.

Les boîtes créées avec \cmd{\parbox} ne peuvent contenir de structures
«complexes» comme des listes ou des tableaux. Parce que plus général,
l'outil véritablement utile de création de boîtes verticales est
l'environnement \Ie{minipage}. Cet environnement peut contenir à peu
n'importe quel type de contenu. Comme son nom l'indique, c'est ni plus
ni moins qu'une page miniature à l'intérieur de la page standard.

La syntaxe de l'environnement \Ie{minipage} est la suivante:
\begin{lstlisting}
\begin{minipage}`\oarg{pos}\marg{largeur}'
  `\meta{texte}'
\end{minipage}
\end{lstlisting}

L'environnement \Ie{minipage} est ouvent utilisé pour disposer des
éléments de contenu de manière spécifique sur la page, notamment des
tableaux ou des figures côte-à-côte ou en grille.
%%% Exemple plus loin! %%%

Voici un exemple d'utilisation des boîtes verticales. Le code suivant

\begin{lstlisting}
\begin{minipage}[b]{0.3\textwidth}
  La ligne inférieure de cette \emph{minipage} est alignée avec
\end{minipage}
\hfill
\parbox{0.3\textwidth}{la centre de cette boîte verticale, qui
  est à son tour alignée avec}
\hfill
\begin{minipage}[t]{0.3\textwidth}
  la ligne supérieure de cette \emph{minipage}.
\end{minipage}
\end{lstlisting}
produit \\
\medskip

\noindent%
\begin{minipage}{\textwidth}
  \begin{minipage}[b]{0.3\textwidth}
    La ligne inférieure de cette \emph{minipage} est alignée avec
  \end{minipage}
  \hfill
  \parbox{0.3\textwidth}{la centre de cette boîte verticale, qui est à
    son tour alignée avec}
  \hfill
  \begin{minipage}[t]{0.3\textwidth}
    la ligne supérieure de cette \emph{minipage}.
  \end{minipage}
\end{minipage}
\medskip

La commande \cmd{\hfill} utilisée entre les boîtes ci-dessus indique à
{\LaTeX} d'insérer de l'espace blanc entre les éléments de contenu de
manière à remplir entièrement la ligne de texte. C'est une commande
très utile pour disposer automatiquement des éléments à intervalles
égaux sur la largeur du bloc de texte \hfill comme ceci
\begin{lstlisting}
[...] de texte  \hfill comme ceci
\end{lstlisting}
ou encore \hfill comme \hfill ceci.
\begin{lstlisting}
ou encore \hfill comme \hfill ceci.
\end{lstlisting}

%%% Exercices: faire le contenu de 4.7.4 de Kopka et Daly (disposition
%%% verticale des boîtes)

\subsection{Boîtes de réglure}

Réglure est un terme d'imprimerie

\begin{lstlisting}
\rule`\oarg{delta}\marg{largeur}\marg{hauteur}'
\end{lstlisting}


\section{De la conception de beaux tableaux}
\label{sec:tableaux:booktabs}

On utilise les tableaux pour disposer de l'information sous forme de
grille. Ainsi, le premier réflexe pour les mettre en forme
consiste-t-il souvent à mettre en évidence cette grille par le biais
de filets\footnote{%
  Communément appelés «lignes» dans le langage courant ou «bordures»
  dans les logiciels de traitement de texte. Dans la documentation en
  anglais, on parle de \emph{rules}.} %
horizontaux et verticaux.

C'est une mauvaise idée, une pratique à éviter. Vraiment!

Comparer les deux tableaux ci-dessous. Le premier est mis en forme
selon une approche classique supportée depuis toujours par {\LaTeX}:
filets doubles en entête et en pied de tableau, filets simples entre
chaque ligne et entre les colonnes.

\begin{center}
  \begin{tabular}{|>{$}c<{$}|>{$}r<{$}|>{$}r<{$}|>{$}r<{$}|>{$}c<{$}|>{$}c<{$}|}
    \hline\hline
    i &
    \multicolumn{1}{c|}{$v$} &
    \multicolumn{1}{c|}{$b_i$} &
    \lfloor v/b_i \rfloor & v \bmod b_i & x_i \\
    \hline
    0 & \nombre{91492} &  60 & \nombre{1524} & 52 & 52 \\
    1 &  \nombre{1524} &  60 &           25  & 24 & 24 \\
    2 &            25  &  24 &            1  &  1 &  1 \\
    3 &             1  & 365 &            0  &  1 &  1 \\
    \hline\hline
  \end{tabular}
\end{center}

Le second tableau tire profit des fonctionnalités du paquetage
\pkg{booktabs} et des recommandations de son auteur: les filets
horizontaux sont d'épaisseur différente selon qu'ils sont situés dans
l'entête et dans le pied du tableau ou entre les lignes, l'espace
autour des filets horizontaux est plus grand et, surtout, il n'y a pas
de filets verticaux.

\begin{center}
  \begin{tabular}{>{$}c<{$}>{$}r<{$}>{$}r<{$}>{$}r<{$}>{$}c<{$}>{$}c<{$}}
    \toprule
    i &
    \multicolumn{1}{c}{$v$} &
    \multicolumn{1}{c}{$b_i$} &
    \lfloor v/b_i \rfloor & v \bmod b_i & x_i \\
    \midrule
    0 & \nombre{91492} &  60 & \nombre{1524} & 52 & 52 \\
    1 &  \nombre{1524} &  60 &           25  & 24 & 24 \\
    2 &            25  &  24 &            1  &  1 &  1 \\
    3 &             1  & 365 &            0  &  1 &  1 \\
    \bottomrule
  \end{tabular}
\end{center}

La seconde version n'est-elle pas la plus aérée et la plus facile à
consulter? Vous constatez que, contrairement à ce que l'on pourrait
penser, les filets verticaux ne sont pas du tout requis pour bien
délimiter les colonnes?

Tel que mentionné ci-dessus, le paquetage \pkg{booktabs} ajoute des
fonctionnalités à {\LaTeX} pour améliorer la qualité typographique des
tableaux. Dans la documentation du paquetage, son auteur énonce
quelques règles à suivre pour la mise en forme des tableaux:
\begin{enumerate}
\item ne \emph{jamais} utiliser de filets verticaux. Si l'information du côté
  gauche du tableau semble si différente de celle du côté droit
  qu'un filet apparait vertical nécessaire, scinder simplement
  l'information dans deux tableaux;
\item ne jamais utiliser de filets doubles;
\item placer les unités dans le titre de la colonne plutôt qu'après
  chaque valeur dans le corps du tableau;
\item toujours inscrire un chiffre du côté gauche du séparateur
  décimal: $0,1$ et non $,1$ (pratique plus répandue en anglais, où le
  séparateur décimal est le point);
\item ne pas utiliser un symbole pour représenter une valeur
  répétée (comme $''$ ou ---). Laisser un blanc ou répéter la
  valeur s'il subsiste une ambiguïté.
\end{enumerate}

Nous recommandons évidemment de suivre ces règles et c'est pourquoi la
présente documentation ainsi que les fichiers d'exemples font usage
des commandes de \pkg{booktabs}.

Les fonctionnalités de \pkg{booktabs} sont intégrées à la classe
\class{memoir} et par conséquent à \class{ulthese}. Il n'est donc pas
nécessaire de charger le paquetage avec ces deux classes.



\section{Tableaux}
\label{sec:tableaux:tableaux}

Peu importe l'outil utilisé, la création d'un tableau requiert de
préciser à l'ordinateur le nombre de colonnes que contiendra le
tableau, l'entête du tableau si nécessaire et le contenu des
différentes cellules. Cette dernière étape nécessite à son tour une
convention pour pour identifier les passages à la colonne suivante
ainsi que le passage à la ligne suivante.

On crée des tableaux dans {\LaTeX} principalement avec les
environnements \Ie{tabular}, \Ie{tabular*} et \Ie{tabularx} (ce
dernier fourni par le paquetage \pkg{tabularx} ou par la classe
\class{memoir}). La syntaxe de ces environnements est:
\begin{lstlisting}
\begin{tabular}`\marg{format}'           `\textit{lignes}' \end{tabular}
\begin{tabular*}`\marg{largeur}\marg{format}' `\textit{lignes}' \end{tabular*}
\begin{tabularx}`\marg{largeur}\marg{format}' `\textit{lignes}' \end{tabularx}
\end{lstlisting}
La signification des arguments\footnote{%
  Nous avons omis un argument optionnel à peu près jamais utilisé
  servant à spécifier l'alignement vertical du tableau par rapport à
  la ligne de base externe.} %
est la suivante. Nous ne traitons ici que les options les plus souvent
employées. Pour une liste plus exhaustive, consulter la documentation
de la classe \class{memoir} (chapitre 11) ou \cite{wikilivres:latex}
(section \href{http://fr.wikibooks.org/wiki/LaTeX/Tableaux}{Tableaux}).

\begin{list}{}{%
    \setlength{\labelsep}{1.5ex}
    \settowidth{\labelwidth}{\meta{largeur}}
    \setlength{\leftmargin}{\labelwidth}
    \addtolength{\leftmargin}{\labelsep}
    \setlength{\parsep}{0.5ex plus0.2ex minus0.2ex}
    \setlength{\itemsep}{0.3ex}
    \renewcommand{\makelabel}[1]{\meta{#1}\hfill}}
%
\item[largeur] Largeur hors tout d'un tableau avec les
  environnements \Ie{tabular*} et \Ie{tabularx}. Autrement,
  avec l'environnement \Ie{tabular}, la largeur d'un tableau est
  déterminée automatiquement pour contenir tout le contenu du tableau,
  quitte à dépasser dans la marge de droite.

  La largeur du tableau est généralement exprimée en fraction de la
  largeur du bloc de texte. Celle-ci est accessible avec la commande
  \verb=\textwidth=. Par exemple, les déclarations suivantes
  définissent respectivement des tableaux occupant toute la largeur
  d'une page et 80~\% de la largeur de la page:
\begin{lstlisting}
\begin{tabular*}{\textwidth}`\marg{format}'
\end{lstlisting}
\begin{lstlisting}
\begin{tabularx}{0.8\textwidth}`\marg{format}'
\end{lstlisting}
  L'environnement \Ie{tabular*} joue sur l'espace entre les colonnes
  pour parvenir à la largeur prescrite, alors que \Ie{tabularx} joue
  sur la largeur des colonnes (voir ci-dessous).
  %
\item[format] Le format des colonnes et, par le fait même, le nombre
  de colonnes puisque l'argument doit compter un symbole pour chaque
  colonne du tableau. Les principaux symboles de mise en forme des
  colonnes sont:
  \begin{description}
  \item[\normalfont\code{l}] contenu de la colonne aligné à gauche;
  \item[\normalfont\code{r}] contenu de la colonne aligné à droite;
  \item[\normalfont\code{c}] contenu de la colonne centré;
  \item[\normalfont\code{p\{}\textit{lgr}\code{\}}] contenu de la colonne traité comme un
    paragraphe de texte de largeur \textit{lgr};
  \item[\normalfont\code{X}] [environnement \Ie{tabularx} seulement]
    colonne dont la largeur peut être ajustée pour obtenir un tableau
    de la largeur prescrite; identique à \code{p} par ailleurs.
  \end{description}
  Par exemple, pour définir un tableau à trois colonnes dont le
  contenu de la première est aligné à gauche, celui de la seconde à
  droite et celui de la troisième en texte libre dans une cellule de
  5~cm de largeur, on utiliserait:
\begin{lstlisting}
\begin{tabular}{lrp{5cm}}
\end{lstlisting}
  Avec la déclaration suivante, la largeur de la troisième colonne
  sera automatiquement adaptée pour que le tableau occupe toute la
  largeur de la page:
\begin{lstlisting}
\begin{tabularx}{\textwidth}{lrX}
\end{lstlisting}

  Les symboles \verb=|= et \verb=||= dans \textit{format} servent à
  insérer des filets verticaux simples et doubles entre les colonnes,
  mais nous avons vu à la \autoref{sec:tableaux:booktabs} que c'est
  une pratique à proscrire.
  %
\item[lignes] Le contenu des cellules du tableau. Les entrées des
  cellules sont séparées par le symbole \verb=&= et les lignes par
  \verb=\\=. Une cellule peut être vide.

  Les lignes de contenu peuvent également contenir certaines commandes
  spéciales pour contrôler la mise en forme. La commande ci-dessous
  permet de fusionner des cellules:
  \begin{description}
  \item[\normalfont\cmd{\multicolumn\marg{num}\marg{fmt}\marg{texte}}]
    fusionne les \meta{num} cellules suivantes en une seule de format
    \meta{fmt} et contenant le texte \meta{texte}. \par%
    Cette commande ne peut apparaître qu'au début d'une ligne ou après
    un symbole de changement de colonne \verb=&=. \par%
    La commande est souvent utilisée avec $\text{\meta{num}} = 1$ pour
    changer le format d'une cellule, par exemple pour centrer le
    titre d'une colonne autrement alignée à gauche ou à droite.
  \end{description}

  Les commandes suivantes\footnote{%
    Ce sont les commandes de \pkg{booktabs} et \class{memoir}
auxquelles nous faisions référence à la \autoref{sec:tableaux:booktabs}.} %
  servent à insérer des filets horizontaux dans un tableau:
  \begin{description}
  \item[\normalfont\cmd{\toprule}] insère un filet horizontal épais
    suivi d'un espace vertical au début d'un tableau;
  \item[\normalfont\cmd{\midrule}] insère un filet horizontal mince
    précédé et suivi d'un espace vertical entre deux lignes;
  \item[\normalfont\cmd{\cmidrule\marg{n-m}}] insère un filet
    horizontal comme \cmd{\midrule}, mais seulement de la gauche de la
    colonne \meta{n} à la droite de la colonne \meta{m};
  \item[\normalfont\cmd{\bottomrule}] insère un filet horizontal épais
    précédé d'un espace vertical à la fin d'un tableau.
  \end{description}
  Ces commandes doivent toutes obligatoirement être précédées d'une
  fin de ligne \verb=\\=, sauf évidemment \cmd{\toprule}.

  Un exemple simple de lignes de contenu serait:
\begin{lstlisting}
\toprule
Produit & Quantité & Prix unitaire (\$) & Prix (\$) \\
\midrule
Vis à bois    & 2 & 9,90 & 19,80 \\
Clous vrillés & 5 & 4,35 & 21,75 \\
\midrule
TOTAL         & 7 &      & 41,55 \\
\bottomrule
\end{lstlisting}
\end{list}

\begin{exemple}
  On reprend le contenu ci-dessus pour en faire un tableau d'une
  largeur ajustée automatiquement au contenu. La première colonne est
  alignée à gauche et toutes les autres à droite.
\begin{lstlisting}
\begin{tabular}{lrrr}
  \toprule
  Produit & Quantité & Prix unitaire (\$) & Prix (\$) \\
  \midrule
  Vis à bois    & 2 & 9,90 & 19,80 \\
  Clous vrillés & 5 & 4,35 & 21,75 \\
  \midrule
  TOTAL         & 7 &      & 41,55 \\
  \bottomrule
\end{tabular}
\end{lstlisting}
  \begin{center}
    \begin{tabular}{lrrr}
      \toprule
      Produit & Quantité & Prix unitaire (\$) & Prix (\$) \\
      \midrule
      Vis à bois    & 2 & 9,90 & 19,80 \\
      Clous vrillés & 5 & 4,35 & 21,75 \\
      \midrule
      TOTAL         & 7 &      & 41,55 \\
      \bottomrule
    \end{tabular}
  \end{center}
  Avec quelques modifications, le tableau occupe maintenant 80~\% de
  la largeur de la page, la largeur de la première colonne étant
  ajustée pour combler l'espace nécessaire. De plus, on modifie
  l'entête de la première colonne avec la commande \cmd{\multicolumn}
  afin de centrer le titre.
\begin{lstlisting}
\begin{tabularx}{0.8\textwidth}{Xrrr}
  \toprule
  \multicolumn{1}{c}{Produit} &
    Quantité & Prix unitaire (\$) & Prix (\$) \\
  \midrule
  Vis à bois    & 2 & 9,90 & 19,80 \\
  Clous vrillés & 5 & 4,35 & 21,75 \\
  \midrule
  TOTAL         & 7 &      & 41,55 \\
  \bottomrule
\end{tabularx}
\end{lstlisting}
  \begin{center}
    \begin{tabularx}{0.8\textwidth}{Xrrr}
      \toprule
      \multicolumn{1}{c}{Produit} &
        Quantité & Prix unitaire (\$) & Prix (\$) \\
      \midrule
      Vis à bois    & 2 & 9,90 & 19,80 \\
      Clous vrillés & 5 & 4,35 & 21,75 \\
      \midrule
      TOTAL         & 7 &      & 41,55 \\
      \bottomrule
    \end{tabularx}
  \end{center}
\end{exemple}


\section{Éléments flottants}
\label{sec:tableaux:floats}

Les tableaux et figures sont des éléments de contenu qui occupent
souvent beaucoup d'espace vertical dans la page. S'il ne reste plus
assez de place pour afficher un tel élément sur une page, {\TeX} devra
le déplacer au début de page suivante et cela risque de produire une
page inesthétique car insuffisamment remplie\footnote{%
  \emph{Underful \cs{vbox}} dans le jargon de {\TeX}.}. %

Pour éviter cela, il serait souhaitable que {\TeX} puisse insérer un
tableau à l'endroit indiqué dans le code source s'il y a suffisamment
d'espace sur la page pour l'accueillir


De plus, tout tableau et toute figure dans un document soigné devrait
comporter un titre/légende/description ainsi qu'un numéro afin de
pouvoir y faire référence dans le texte («comme l'illustre la
figure~3\dots»).

fonctionnalité très élaborée de LaTeX


éléments flottants
conseils de booktabs
tableaux
figures (insertion de graphiques (Sweave), mention de picture)

LATEX offre à ses valeureux utilisateurs la possibilité d'utiliser des environnements flottants. Ces environnements ont la particularité de rendre « flottants » leur contenu. C'est-à-dire que LATEX choisit à partir d'un algorithme qui tient compte d'un certain nombre de paramètres, la position de l'environnement dans le document.

%%% Local Variables:
%%% mode: latex
%%% TeX-engine: xetex
%%% TeX-master: "formation_latex-partie_2"
%%% coding: utf-8
%%% End:
