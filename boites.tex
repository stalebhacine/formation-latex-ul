%%% Copyright (C) 2018 Vincent Goulet
%%%
%%% Ce fichier fait partie du projet
%%% «Rédaction avec LaTeX»
%%% http://github.com/vigou3/formation-latex-ul
%%%
%%% Cette création est mise à disposition selon le contrat
%%% Attribution-Partage dans les mêmes conditions 4.0
%%% International de Creative Commons.
%%% http://creativecommons.org/licenses/by-sa/4.0/

\chapter{Boites}
\label{chap:boites}

Il arrive que l'on doive traiter de manière spéciale une aire
rectangulaire de texte; pour l'encadrer, la mettre en surbrillance ou
la mettre en exergue, par exemple.

Avec les traitements de texte, on aura souvent recours aux tableaux à
de telles fins. Or, les tableaux devraient être réservés à la
disposition d'information sous forme de lignes et de colonnes. Pour
disposer et mettre en forme tout autre type contenu se présentant sous
forme rectangulaire, {\LaTeX} offre la solution plus générale des
«boites».

Il existe trois sortes de boites en {\LaTeX}: les boites horizontales,
dont le contenu est disposé exclusivement côte à côte; les boites
verticales, qui peuvent contenir plusieurs lignes de contenu; les
boites de réglure pour former des lignes pleines de largeur et de
hauteur quelconques.

Il n'est pas inutile de savoir, au passage, que {\TeX} ne manipule que
cela, des boites. Pour {\TeX}, chaque caractère, chaque lettre n'est
qu'un rectangle d'une certaine largeur qui s'élève au-dessus de la
ligne de base (les lignes d'une feuille lignée) et qui, parfois, se
prolonge sous la ligne de base (pensons aux lettres \emph{p}, \emph{y}
ou \emph{Q}). La \autoref{fig:boites:glyphs} illustre cela.

\begin{figure}[t]
  \begin{minipage}{0.45\linewidth}
    \centering\huge
    fantastique
  \end{minipage}
  \hfill
  \begin{minipage}{0.45\linewidth}
    %% creéation de boites autour des caractères
    %% https://tex.stackexchange.com/questions/57812/bounding-box-for-each-letter
    \fboxrule=0.1pt
    \fboxsep=-\fboxrule
    \makeatletter
    \def\SOUL@soeverytoken{%
      \fbox{\color{white}\the\SOUL@token}}
    \makeatother
    \centering
    \so{\huge fantastique}
  \end{minipage}
  \caption{À gauche, ce que nous voyons dans le document fini: un
    alignement de lettres. À droite, ce que {\TeX} manipule: une série
    de boites.}
  \label{fig:boites:glyphs}
\end{figure}

Les commandes et environnements présentés ci-dessous permettent
simplement de créer d'autres boites dont le contrôle des dimensions et
du contenu est laissé à l'usager.

Une fois créée, une boite ne peut être scindée en parties, notamment
entre les lignes ou entre les pages.


\section{Boites horizontales}
\label{sec:boites:lrbox}

Le concept de boite le plus simple dans {\LaTeX} est celui de boite
«horizontale», c'est-à-dire dont le contenu est disposé latéralement
de gauche à droite\footnote{%
  D'où l'appellation \emph{LR (left-right) box} en anglais.}. %
Le contenu est normalement du texte, mais conceptuellement ce pourrait
être n'importe quoi, y compris d'autres boites.

Les commandes de base pour créer des boites horizontales sont:
\begin{lstlisting}
\mbox`\marg{texte}'
\fbox`\marg{texte}'
\end{lstlisting}
Elles produisent une boite de la largeur précise de \meta{texte}. Avec
la commande \cmd{\fbox}, le texte est au surplus \fbox{encadré}.

Il existe également des versions plus générales des commandes
\cmd{\mbox} et \cmd{\fbox}:
\begin{lstlisting}
\makebox`\oarg{largeur}\oarg{pos}\marg{texte}'
\framebox`\oarg{largeur}\oarg{pos}\marg{texte}'
\end{lstlisting}
Les arguments optionnels \meta{largeur} et \meta{pos} déterminent
respectivement la largeur de la boite et la position du texte
dans la boite. Les valeurs possibles de \meta{pos} sont: \code{l} pour
du texte aligné à gauche, \code{r} pour du texte aligné à droite et
\code{c} (la valeur par défaut) pour du texte centré. Ainsi, la commande
\begin{lstlisting}
\framebox[3.5cm][l]{aligné à gauche}
\end{lstlisting}
produit \framebox[3.5cm][l]{aligné à gauche}, alors que
\begin{lstlisting}
\makebox[3.5cm]{centré}
\end{lstlisting}
produit \makebox[3.5cm]{centré}.

\tipbox{En usage courant, la commande \cmd{\mbox} sert principalement
  à deux choses:
  \begin{enumerate}
  \item réunir en un bloc du texte que l'on ne veut pas voir scindé
    entre les lignes ou entre les pages;
  \item \label{item:boites:mbox} créer une boite vide avec
    \cs{mbox\{\}} afin de laisser croire à {\TeX} que du contenu
    apparait à un endroit, sans toutefois qu'il n'occupe aucun espace.
  \end{enumerate}
  La seconde utilisation fait l'objet de
  l'\autoref{ex:boites:alignement-v}.}

Il est parfois nécessaire d'ajuster le positionnement vertical
d'éléments de contenu, notamment pour les symboles ou les images. La
commande
\begin{lstlisting}
\raisebox`\marg{déplacement}\marg{texte}'
\end{lstlisting}
produit une boite horizontale dont le contenu \meta{texte} est
surélevé de la longueur \meta{déplacement} par rapport à la ligne de
base. Si \meta{déplacement} est négatif, la boite est positionnée sous
la ligne de base.
\begin{demo}
  \begin{texample}[0.55\textwidth]
\begin{lstlisting}
Texte \raisebox{1ex}{au-dessus}
de la ligne de base.
\end{lstlisting}
    \producing
    Texte \raisebox{1ex}{au-dessus}
    de la ligne de base.
  \end{texample}

  \begin{texample}[0.55\textwidth]
\begin{lstlisting}
Texte \raisebox{-1ex}{au-dessous}
de la ligne de base.
\end{lstlisting}
    \producing
    Texte \raisebox{-1ex}{au-dessous}
    de la ligne de base.
  \end{texample}
\end{demo}

Attention, toutefois, de ne pas utiliser \cmd{\raisebox} pour placer
du texte en exposant ou en indice. Selon la nature du texte, employer
plutôt les commandes \cmd{\textsuperscript} et \cmd{\textsubscript},
les commandes de la famille \cmd{\ieme} de \pkg{babel} (section~1.1 de
la %
\doc{frenchb}{http://texdoc.net/pkg/babel-french/}) ou, pour des
symboles mathématiques, les commandes d'exposant et d'indice
spécifiques au mode mathématique (\autoref{sec:math:bases:exposants}).

\section{Boites verticales}
\label{sec:boites:parbox}

Les boites verticales se distinguent des boites horizontales par le
fait qu'elles peuvent contenir plusieurs lignes de contenu empilées
les unes au-dessus des autres. Lorsque le contenu en question est du
texte, on obtient des paragraphes\footnote{%
  D'où l'appellation de \emph{paragraph boxes} en anglais
  ou \emph{parboxes} dans le jargon {\LaTeX}.}. %

La commande de base pour créer une boite verticale est:
\begin{lstlisting}
\parbox`\oarg{pos}\marg{largeur}\marg{texte}'
\end{lstlisting}
Ici, l'argument optionnel \meta{pos} permet d'ajuster l'alignement
vertical de la boite avec la ligne de base: \code{b} ou \code{t} selon
que l'on souhaite aligner, respectivement, le bas ou le haut de la
boite avec la ligne de base. Par défaut, la boite est centrée avec la
ligne de base. Cet argument n'a aucun effet si la boite est le seul
élément de contenu du paragraphe.

On remarquera que l'argument \meta{largeur} est ici obligatoire.
Autrement dit, on doit nécessairement définir la largeur des boites
verticales, un peu comme il faut bien définir la largeur de la page
pour le texte normal (la classe se charge de ce détail).

Les boites créées avec \cmd{\parbox} ne peuvent contenir de structures
«complexes» comme des listes ou des tableaux. Parce que plus général,
l'outil véritablement utile pour la création de boites verticales est
l'environnement \Pe{minipage}. Cet environnement peut contenir à peu
n'importe quel type de contenu. Comme son nom l'indique, c'est ni plus
ni moins qu'une page miniature à l'intérieur de la page standard.

La syntaxe de l'environnement \Ie{minipage} est la suivante:
\begin{lstlisting}
\begin{minipage}`\oarg{pos}\marg{largeur}'
  `\meta{texte}'
\end{minipage}
\end{lstlisting}
La signification des arguments \meta{largeur} et \meta{pos} est la
même que pour la commande \cmd{parbox}.

L'environnement \Pe{minipage} est fréquemment utilisé pour disposer
des éléments de contenu de manière spécifique sur la page, notamment
des tableaux ou des figures côte à côte ou en grille (voir
l'\autoref{exemple:tableaux:grille} à la
\autopageref{exemple:tableaux:grille}).

\begin{exemple}
  L'agencement de boites ci-dessous est produit avec le code qui suit
  immédiatement.  \\[0.5\baselineskip]
  \begin{minipage}{\textwidth}
    \makebox[0pt][l]{\color{lightgray}\rule{\textwidth}{0.7pt}}\relax
    \fbox{\begin{minipage}[b]{0.3\textwidth} La ligne inférieure de
        cette \emph{minipage} est alignée avec
      \end{minipage}} \hfill \fbox{\parbox{0.3\textwidth}{le centre de
        cette boite verticale, qui est à son tour alignée avec}}
    \hfill \fbox{\begin{minipage}[t]{0.3\textwidth} la ligne
        supérieure de cette \emph{minipage}. Le filet horizontal grisé
        représente la ligne de base du paragraphe contenant les trois
        boites.
      \end{minipage}}
  \end{minipage}
\begin{lstlisting}
\begin{minipage}[b]{0.3\textwidth}
  La ligne inférieure de cette \emph{minipage} [...]
\end{minipage}
\hfill
\parbox{0.3\textwidth}{le centre de cette boite [...] }
\hfill
\begin{minipage}[t]{0.3\textwidth}
  la ligne supérieure de cette \emph{minipage}. [...]
\end{minipage}
\end{lstlisting}
  \qed
\end{exemple}

La commande \cmd{\hfill} utilisée entre les boites dans l'exemple
ci-dessus indique à {\LaTeX} d'insérer de l'espace entre les éléments
de contenu de manière à remplir entièrement la ligne de texte. C'est
une commande très utile pour disposer automatiquement des éléments à
intervalles égaux sur la largeur du bloc de texte. Ainsi,
\begin{lstlisting}
\framebox[\linewidth]{gauche \hfill droite}
\end{lstlisting}
produit \\[0.5\baselineskip]
\framebox[\linewidth]{gauche \hfill droite} \\[0.5\baselineskip]
alors que
\begin{lstlisting}
\framebox[\linewidth]{gauche \hfill centre \hfill droite.}
\end{lstlisting}
produit \\[0.5\baselineskip]
\framebox[\linewidth]{gauche \hfill centre \hfill droite.}



\section{Boites de réglure}
\label{sec:boites:rulebox}

En imprimerie, une réglure est une ligne droite continue ou
pointillée. Une ligne n'étant jamais rien d'autre qu'un rectangle
plein, si mince fut-il, la réglure est le troisième type de
boite\footnote{%
  \emph{Rule box}, en anglais} %
dans {\LaTeX}.

La commande
\begin{lstlisting}
\rule`\oarg{déplacement}\marg{largeur}\marg{hauteur}'
\end{lstlisting}
crée une réglure de dimensions \meta{largeur} $\times$ \meta{hauteur}.
Par défaut, la réglure s'appuie sur la ligne de base. Le résultat de
\begin{lstlisting}
\rule{2cm}{6pt}
\end{lstlisting}
est donc une ligne pleine de $2$~cm de long et de $6$~points d'épais:
\rule{2cm}{6pt}.

L'argument optionnel \meta{déplacement} permet de déplacer
verticalement la réglure au-dessus ou au-dessous de la ligne de base
selon que la longueur \meta{déplacement} est positive ou négative. Avec les deux
commandes
\begin{lstlisting}
\rule[3pt]{2cm}{6pt}
\rule[-3pt]{2cm}{6pt}
\end{lstlisting}
on crée respectivement les réglures \rule[3pt]{2cm}{6pt} et
\rule[-3pt]{2cm}{6pt}.

Un usage intéressant de la réglure consiste à faire croire à {\TeX}
qu'une ligne est plus haute qu'il n'y parait en insérant dans celle-ci
une réglure de largeur nulle. Par exemple, la distance entre
\rule[-12pt]{0mm}{30pt}\relax la présente ligne et les autres du paragraphe est
plus grande que la normale parce que nous y avons inséré une réglure
invisible avec
\begin{lstlisting}
\rule[-12pt]{0mm}{30pt}
\end{lstlisting}
Ce truc est particulièrement utile pour augmenter la hauteur des
lignes dans un tableau; voir la \autoref{sec:tableaux:tableaux}.



%%%
%%% Exercices
%%%

\section{Exercices}
\label{sec:boites:exercices}

\Opensolutionfile{solutions}[solutions-boites]

\begin{Filesave}{solutions}
\section*{Chapitre \ref*{chap:boites}}
\addcontentsline{toc}{section}{Chapitre \protect\ref*{chap:boites}}

\end{Filesave}

\noindent%
Utiliser comme canevas le fichier \fichier{exercice\_gabarit.tex} pour
tous les exercices ci-dessous.

\begin{exercice}
  Une fois qu'une boite est définie, {\TeX} n'y voit qu'une unité de
  contenu avec ses dimensions propres. Il est donc possible de définir
  une boite à l'intérieur d'une autre, et ce, peu importe le type de
  boite.

  Avec ceci en tête, définir la boite suivante:
  \begin{center}
    \fbox{\fbox{%
        \parbox{10cm}{Ce bloc de texte est une boite verticale de
          10~cm de large, doublement encadrée et centrée sur la
          ligne.}}}
  \end{center}

  \begin{sol}
    Une première boite verticale de 10~cm de large contient le texte:
\begin{lstlisting}
\parbox{10cm}{Ce bloc [...] la ligne.}
\end{lstlisting}
    Cette boite peut être placée dans une boite horizontale encadrée
    avec \cmd{\fbox}. Celle-ci peut à son tour être placée dans une autre
    boite horizontale encadrée, de manière à obtenir un cadrage
    double. Pour centrer le tout sur la ligne, on a recours à
    l'environnement \Ie{center}:
\begin{lstlisting}
\begin{center}
  \fbox{\fbox{\parbox{10cm}{Ce bloc [...] la ligne.}}}
\end{center}
\end{lstlisting}
  \end{sol}
\end{exercice}

\begin{exercice}
  \label{ex:boites:alignement-v}
  Réaliser l'agencement de boites verticales suivant:
  \begin{center}
    \begin{minipage}{0.8\linewidth}
      \makebox[0pt][l]{\color{lightgray}\rule{\linewidth}{0.7pt}}\relax
      \hfill
        \begin{minipage}[b]{0.95\linewidth}
          \small
          \parbox[t]{0.45\linewidth}{Deux boites verticales de
            hauteurs différentes placées côte à côte}
          \hfill
          \parbox[t]{0.45\linewidth}{alignées sur
            leurs premières lignes et le bas de la boite
            la plus haute alignée sur la ligne de base (représentée
            ici par le filet horizontal grisé).} \\
          \mbox{}
        \end{minipage}
      \hfill
    \end{minipage}
  \end{center}

  La solution intuitive serait la suivante:
\begin{lstlisting}
\begin{minipage}[b]{...}
  \parbox[t]{...}{...} \hfill \parbox[t]{...}{...}
\end{minipage}
\end{lstlisting}
  Cependant, cette solution produit le résultat suivant (les boites
  sont rendues visibles par des cadres):
  \begin{center}
    \begin{minipage}{0.8\linewidth}
      \makebox[0pt][l]{\color{lightgray}\rule{\linewidth}{0.7pt}}\relax
      \hfill
      \fbox{%
        \begin{minipage}[b]{0.95\linewidth}
          \small
          \fbox{%
            \parbox[t]{0.45\linewidth}{Les deux boites sont
              correctement alignées l'une par rapport à l'autre}} \hfill
          \fbox{\parbox[t]{0.45\linewidth}{mais l'alignement avec la
              ligne de base est incorrect.}}
        \end{minipage}}
      \hfill
    \end{minipage}
  \end{center}
  La raison: pour {\TeX}, la \Ie{minipage} externe ne contient que
  deux «caractères» sur une seule ligne de «texte». La \Pe{minipage}
  est donc correctement alignée sur sa ligne du bas, mais celle-ci se
  trouve aussi être la ligne du haut.

  Pour parvenir au résultat escompté, utiliser la commande \cmd{\mbox}
  pour créer une seconde ligne (vide) dans la \Pe{minipage} externe.
  \begin{sol}
    L'idée consiste à créer une seconde ligne dans la \Pe{minipage}
    externe sans que celle-ci n'occupe aucun espace. Pour ce faire, on
    insère du contenu vide avec \cs{mbox\{\}}, tel qu'expliqué à la
    \autopageref{item:boites:mbox}. Le code
\begin{lstlisting}
\begin{minipage}[b]{...}
  \parbox[t]{...}{...} \hfill \parbox[t]{...}{...} \\
  \mbox{}
\end{minipage}
\end{lstlisting}
    produit donc le résultat voulu:
    \begin{center}
      \begin{minipage}{0.8\linewidth}
        \makebox[0pt][l]{\color{lightgray}\rule{\linewidth}{0.7pt}}\relax
        \hfill
        \fbox{%
          \begin{minipage}[b]{0.95\linewidth}
            \small
            \fbox{%
              \parbox[t]{0.45\linewidth}{Les boites sont rendues visibles
                par des cadres}} \hfill
            \fbox{\parbox[t]{0.45\linewidth}{et le filet horizontal grisé
                représente la ligne de base du paragraphe courant.}} \\
            \fbox{\mbox{}}
          \end{minipage}}
        \hfill
      \end{minipage}
  \end{center}
  (Sans le cadre, la boite de la seconde ligne n'occupe aucun espace.)
  \end{sol}
\end{exercice}

\begin{exercice}
  Réaliser l'agencement de boites verticales ci-dessous. (La taille de
  la police est \cs{footnotesize}.)
  \begin{center}
    \begin{minipage}{120mm}
      \footnotesize
      \begin{minipage}[b]{80mm}
        \parbox[t]{30mm}{La première ligne de cette \emph{parbox} de
          $30$~mm de large est alignée avec celle de la boite
          voisine.}
        \hfill
        \parbox[t]{45mm}{Cette \emph{parbox} de $45$~mm de large est
          positionnée de telle sorte que sa première ligne soit
          alignée avec le haut de la boite à gauche et la dernière
          avec le bas de la boite à droite. La solution intuitive
          consistant à placer côte à côte trois boites avec des
          arguments de positionnement \code{t}, \code{t} et \code{b}
          ne fonctionne pas.} \\
        \mbox{}
      \end{minipage}
      \hfill
      \parbox[b]{35mm}{Pour parvenir à cette disposition, il faut
        avoir recours à des lignes invisibles comme dans l'exercice
        précédent.}
    \end{minipage}
  \end{center}
  La troisième boite fait $35$~mm de large et l'espace entre les
  boites, $5$~mm.
  \begin{sol}
    La solution la plus simple consiste à réunir les deux premières
    boites dans une \Pe{minipage} dans laquelle les deux boites seront
    alignées tel que désiré, puis à aligner la \Pe{minipage} avec la
    troisième boite. Cependant, il faut insérer une seconde ligne
    invisible dans la \Pe{minipage} afin de pouvoir l'aligner par le
    bas avec la boite de droite:
\begin{lstlisting}
\begin{minipage}[b]{80mm}
  \parbox[t]{30mm}{...} \hfill \parbox[t]{45mm}{...} \\
  \mbox{}
\end{minipage}
\hfill
\parbox[b]{35mm}{...}
\end{lstlisting}
  \end{sol}
\end{exercice}

\Closesolutionfile{solutions}

%%% Local Variables:
%%% mode: latex
%%% TeX-master: "formation-latex-ul"
%%% TeX-engine: xetex
%%% coding: utf-8
%%% End:
