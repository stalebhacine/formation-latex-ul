\chapter{Boîtes}
\label{chap:boites}

Il arrive que l'on doive traiter de manière spéciale une aire
rectangulaire de texte; pour l'encadrer, la mettre en surbrillance ou
la mettre en exergue, par exemple.

Avec les traitements de texte, on aura souvent recours aux tableaux à
de telles fins. Or, les tableaux devraient être réservés pour disposer
de l'information sous forme de lignes et de colonnes. (Un tableau
d'une seule cellule n'a donc pas vraiment de sens d'un point de vue
sémantique ou typographique.) Pour disposer et mettre en forme du tout
autre type contenu se présentant sous forme rectangulaire, {\LaTeX}
offre la solution plus générale des «boîtes».

Il existe trois sortes de boîtes en {\LaTeX}: les boîtes horizontales
dont le contenu est disposé exclusivement côte à côte; les boîtes
verticales qui peuvent contenir plusieurs lignes de contenu; les
boîtes de réglure pour former des lignes pleines de largeur et de
hauteur quelconques.

Il n'est pas inutile de savoir, au passage, que {\TeX} ne manipule que
cela, des boîtes. Pour {\TeX}, chaque caractère, chaque lettre n'est
qu'un rectangle d'une certaine largeur qui s'élève au-dessus de la
ligne de base (les lignes d'une feuille lignée) et qui, parfois, se
prolonge sous la ligne de base (pensons aux lettres \emph{p}, \emph{y}
ou \emph{Q}). Les commandes et environnements présentés ci-dessous
permettent simplement de créer d'autres boîtes dont le contrôle des
dimensions et du contenu est laissé à l'usager.

Une fois créée, une boîte ne peut être scindée en parties, notamment
entre les lignes ou entre les pages.



\section{Boîtes horizontales}
\label{sec:boites:lrbox}

Le plus simple concept de boîte dans {\LaTeX} est celui de boîte
«horizontale», c'est-à-dire dont le contenu est disposé latéralement
de gauche à droite\footnote{%
  D'où l'appellation \emph{LR (left-right) box} en anglais.}. %
Le contenu est normalement du texte, mais conceptuellement ce pourrait
être n'importe quoi, y compris d'autres boîtes.

Les commandes de base pour créer des boîtes horizontales sont:
\begin{lstlisting}
\mbox`\marg{texte}'
\fbox`\marg{texte}'
\end{lstlisting}
Elles produisent une boîte de la largeur précise de \meta{texte}. Avec
la commande \cmd{\fbox}, le texte est au surplus \fbox{encadré}. En
usage courant, la commande \cmd{\mbox} sert principalement à deux
choses:
\begin{enumerate}
\item réunir en un bloc du texte que l'on ne veut pas voir scindé
  entre les lignes ou entre les pages;
\item \label{item:tableaux:mbox} créer une boîte vide avec
  \cs{mbox\{\}} afin de laisser croire à {\TeX} que du contenu
  apparaît à un endroit, sans toutefois qu'il n'occupe aucun espace.
  %% ça prendrait un exemple de cela dans les exercices!
\end{enumerate}

Il existe également des versions plus générales des commandes
\cmd{\mbox} et \cmd{\fbox}:
\begin{lstlisting}
\makebox`\oarg{largeur}\oarg{pos}\marg{texte}'
\framebox`\oarg{largeur}\oarg{pos}\marg{texte}'
\end{lstlisting}
Les arguments optionnels \meta{largeur} et \meta{pos} déterminent
respectivement la largeur de la boîte et la position du texte
dans la boîte. Les valeurs possibles de \meta{pos} sont: \code{l} pour
du texte aligné à gauche, \code{r} pour du texte aligné à droite et
\code{c} (la valeur par défaut) pour du texte centré. Ainsi, la commande
\begin{lstlisting}
\framebox[3.5cm][l]{aligné à gauche}
\end{lstlisting}
produit \framebox[3.5cm][l]{aligné à gauche}, alors que
\begin{lstlisting}
\makebox[3.5cm]{centré}
\end{lstlisting}
produit \makebox[3.5cm]{centré}.

Il arrive que l'on doive ajuster légèrement le positionnement vertical
d'éléments de contenu, notamment pour les images. La commande
\begin{lstlisting}
\raisebox`\marg{déplacement}\marg{texte}'
\end{lstlisting}
produit une boîte horizontale avec le contenu \meta{texte} surélevé de
\meta{déplacement} par rapport à la ligne de base. Si
\meta{déplacement} est négatif, la boîte est positionnée sous la ligne
de base. Ainsi
\begin{lstlisting}
\raisebox{1ex}{au-dessus} de la ligne de base
\end{lstlisting}
produit  \raisebox{1ex}{au-dessus} de la ligne de base et
\begin{lstlisting}
\raisebox{-1ex}{au-dessous} de la ligne de base
\end{lstlisting}
produit \raisebox{-1ex}{au-dessous} de la ligne de base.

Attention, toutefois, de ne pas utiliser \cmd{\raisebox} pour placer
du texte en exposant ou en indice. Employer plutôt les commandes
\cmd{\textsuperscript} ou \cmd{\textsubscript}, le mode mathématique
(\autoref{chap:math}) ou les commandes de la famille \cmd{\ieme} de
\pkg{babel} (section~1.1 de la documentation%
\doc{frenchb}{http://texdoc.net/pkg/babel-french/}).

\section{Boîtes verticales}
\label{sec:boites:parbox}

Les boîtes verticales se distinguent des boîtes horizontales par le
fait qu'elles peuvent contenir plusieurs lignes de contenu empilées
les unes au-dessus des autres. Lorsque le contenu en question est du
texte, on obtient des paragraphes\footnote{%
  D'où l'appellation, cette fois, de \emph{paragraph boxes} en anglais
  ou \emph{parboxes} dans le jargon {\LaTeX}.}. %

La commande de base pour créer une boîte verticale est:
\begin{lstlisting}
\parbox`\oarg{pos}\marg{largeur}\marg{texte}'
\end{lstlisting}
Ici, l'argument optionnel \meta{pos} permet d'ajuster l'alignement
vertical de la boîte avec la ligne de base: \code{b} ou \code{t} selon
que l'on souhaite aligner, respectivement, le bas ou le haut de la
boîte avec la ligne de base. Par défaut, la boîte est centrée avec la
ligne de base. Cet argument n'a aucun effet si la boîte est le seul
élément de contenu du paragraphe.

On remarquera que l'argument \meta{largeur} est ici obligatoire.
Autrement dit, on doit nécessairement définir la largeur des boîtes
verticales, un peu comme il faut bien définir la largeur de la page
pour le texte normal.

Les boîtes créées avec \cmd{\parbox} ne peuvent contenir de structures
«complexes» comme des listes ou des tableaux. Parce que plus général,
l'outil véritablement utile pour la création de boîtes verticales est
l'environnement \Ie{minipage}. Cet environnement peut contenir à peu
n'importe quel type de contenu. Comme son nom l'indique, c'est ni plus
ni moins qu'une page miniature à l'intérieur de la page standard.

La syntaxe de l'environnement \Ie{minipage} est la suivante:
\begin{lstlisting}
\begin{minipage}`\oarg{pos}\marg{largeur}'
  `\meta{texte}'
\end{minipage}
\end{lstlisting}
La signification des arguments \meta{largeur} et \meta{pos} est la
même que pour la commande \cmd{parbox}.

L'environnement \Ie{minipage} est souvent utilisé pour disposer des
éléments de contenu de manière spécifique sur la page, notamment des
tableaux ou des figures côte à côte ou en grille (voir
l'\autoref{exemple:tableaux:grille} à la
\autopageref{exemple:tableaux:grille}).

\begin{exemple}
  Le code ci-dessous
\begin{lstlisting}
\begin{minipage}[b]{0.3\textwidth}
  La ligne inférieure de cette \emph{minipage} [...]
\end{minipage}
\hfill
\parbox{0.3\textwidth}{le centre de cette boîte [...] }
\hfill
\begin{minipage}[t]{0.3\textwidth}
  la ligne supérieure de cette \emph{minipage}. [...]
\end{minipage}
\end{lstlisting}
  produit: \\[0.5\baselineskip]
  \begin{minipage}{\textwidth}
    \makebox[0pt][l]{\color{lightgray}\rule{\textwidth}{0.7pt}}\relax
    \fbox{\begin{minipage}[b]{0.3\textwidth} La ligne inférieure de
        cette \emph{minipage} est alignée avec
      \end{minipage}} \hfill \fbox{\parbox{0.3\textwidth}{le centre de
        cette boîte verticale, qui est à son tour alignée avec}}
    \hfill \fbox{\begin{minipage}[t]{0.3\textwidth} la ligne
        supérieure de cette \emph{minipage}. Le filet horizontal grisé
        représente la ligne de base du paragraphe contenant les trois
        boîtes.
      \end{minipage}}
  \end{minipage}
  \qed
\end{exemple}

La commande \cmd{\hfill} utilisée entre les boîtes dans l'exemple
indique à {\LaTeX} d'insérer de l'espace blanc entre les éléments de
contenu de manière à remplir entièrement la ligne de texte. C'est une
commande très utile pour disposer automatiquement des éléments à
intervalles égaux sur la largeur du bloc de texte. Ainsi,
\begin{lstlisting}
\framebox[\textwidth]{gauche  \hfill droite}
\end{lstlisting}
produit \\[0.5\baselineskip]
\framebox[\textwidth]{gauche \hfill droite} \\[0.5\baselineskip]
alors que
\begin{lstlisting}
\framebox[\textwidth]{gauche \hfill centre \hfill droite.}
\end{lstlisting}
produit \\[0.5\baselineskip]
\framebox[\textwidth]{gauche \hfill centre \hfill droite.}



%%% Exercices: faire le contenu de 4.7.4 de Kopka et Daly (disposition
%%% verticale des boîtes)

\section{Boîtes de réglure}
\label{sec:boites:rulebox}

En imprimerie, une réglure est une ligne droite continue ou
pointillée. Une ligne n'étant jamais rien d'autre qu'un rectangle
plein, si mince fut-il, la réglure est le troisième type de
boîte\footnote{%
  \emph{Rule box}, en anglais} %
dans {\LaTeX}.

La commande
\begin{lstlisting}
\rule`\oarg{déplacement}\marg{largeur}\marg{hauteur}'
\end{lstlisting}
crée une réglure de dimensions \meta{largeur} $\times$ \meta{hauteur}.
Par défaut, la réglure s'appuie sur la ligne de base. Le résultat de
\begin{lstlisting}
\rule{2cm}{6pt}
\end{lstlisting}
est donc une ligne pleine de $2$~cm de long et de $6$~points d'épais:
\rule{2cm}{6pt}.

L'argument optionnel \meta{déplacement} permet de déplacer
verticalement la réglure au-dessus ou au-dessous de la ligne de base
selon que \meta{déplacement} est positif ou négatif. Avec les deux
commandes
\begin{lstlisting}
\rule[3pt]{2cm}{6pt}
\rule[-3pt]{2cm}{6pt}
\end{lstlisting}
on crée respectivement les réglures \rule[3pt]{2cm}{6pt} et
\rule[-3pt]{2cm}{6pt}.

Un usage intéressant de la réglure consiste à faire croire à {\TeX}
qu'une ligne est plus haute qu'il n'y parait en insérant dans celle-ci
une réglure de largeur nulle. Par exemple, la distance entre
\rule[-12pt]{0mm}{30pt}\relax la présente ligne et les autres du paragraphe est
plus grande que la normale parce que nous y avons inséré une réglure
invisible avec
\begin{lstlisting}
\rule[-12pt]{0mm}{30pt}
\end{lstlisting}
Ce truc est particulièrement utile pour augmenter la hauteur des
lignes dans un tableau; voir la \autoref{sec:tableaux:tableaux}.


%%% Local Variables:
%%% mode: latex
%%% TeX-engine: xetex
%%% TeX-master: "formation_latex-partie_2"
%%% coding: utf-8
%%% End:
