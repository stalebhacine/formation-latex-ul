\documentclass[letterpaper,11pt,x11names,english,french]{memoir}
  \usepackage{natbib,url,bibentry}
  \usepackage{babel}
  \usepackage[autolanguage]{numprint}
  \usepackage{amsmath,amsthm}
  \usepackage[shortlabels]{enumitem}
  \usepackage{graphicx}
  \usepackage{framed}                  % environnement titled-frame
  \usepackage{manfnt}                  % \mantriangleright (puce)
  \usepackage{dirtree}                 % arbre pour exercice sur \include
  \usepackage{metalogo}                % \XeLaTeX logo
  \usepackage{fontawesome}             % plusieurs icônes
  \usepackage{answers}                 % exercices et solutions
  \usepackage{listings}                % code informatique
  \usepackage{pdfpages}                % couvertures
  \usepackage{eso-pic}                 % disposition d'images


  %%% ===================
  %%%  Style du document
  %%% ===================

  %% Polices de caractères
  \usepackage{fontspec}
  \usepackage[bold-style=ISO]{unicode-math}
  \defaultfontfeatures{Scale=0.92}
  \setmainfont[Ligatures=TeX,Numbers=OldStyle]{Lucida Bright OT}
  \setmathfont{Lucida Bright Math OT}
  \setmonofont{Lucida Grande Mono DK}
  \setsansfont[Scale=1.0,Numbers=OldStyle]{Myriad Pro}
  \newfontfamily\fullcaps[Letters=Uppercase,Numbers=Uppercase]{Myriad Pro}
  \usepackage[babel=true]{microtype}
  \usepackage{icomma}

  %% Polices additionnelles pour le chapitre trucs et astuces
  \newfontfamily\CM{cmunrm.otf}                       % Computer Modern
  \newfontfamily\Times{texgyretermes-regular.otf}     % Times
  \newfontfamily\Palatino{texgyrepagella-regular.otf} % Palatino

  %% Couleurs
  \usepackage{xcolor}
  \definecolor{comments}{rgb}{0.7,0,0}    % rouge foncé
  \definecolor{link}{rgb}{0,0.4,0.6}      % ~RoyalBlue de dvips
  \definecolor{url}{rgb}{0.6,0,0}         % rouge-brun
  \definecolor{citation}{rgb}{0,0.5,0}    % vert foncé
  \definecolor{ULlinkcolor}{rgb}{0,0,0.3} % de ulthese.cls

  %% Hyperliens
  \usepackage{hyperref}
  \hypersetup{colorlinks, linktocpage,
    urlcolor=url, linkcolor=link, citecolor=citation,
    bookmarksopen, bookmarksnumbered, bookmarksdepth=subsubsection,
    pdftitle={Rédaction de thèses et mémoires avec LaTeX - Partie II},
    pdfauthor={Université Laval}}
  \setlength{\XeTeXLinkMargin}{1pt}

  %% Étiquettes de \autoref (redéfinitions compatibles avec babel).
  %% Attention! Les % à la fin des lignes sont importants sinon des
  %% blancs apparaissent dès que la commande \selectlanguage est
  %% utilisée, notamment dans la bibliographie.
  \addto\extrasfrench{%
    \def\subsectionautorefname{section}%
    \def\figureautorefname{figure}%
    \def\tableautorefname{tableau}%
    \def\exempleautorefname{exemple}%
    \def\exerciceautorefname{exercice}%
  }

  %% Table des matières (inspirée de classicthesis.sty)
  \renewcommand{\cftchapterleader}{\hspace{1.5em}}
  \renewcommand{\cftchapterafterpnum}{\cftparfillskip}
  \renewcommand{\cftsectionleader}{\hspace{1.5em}}
  \renewcommand{\cftsectionafterpnum}{\cftparfillskip}

  %% Titres des chapitres
  \chapterstyle{hangnum}
  \renewcommand{\chaptitlefont}{\normalfont\Huge\sffamily\bfseries\raggedright}

  %% Marges, entêtes et pieds de page
  \setlength{\marginparsep}{7mm}
  \setlength{\marginparwidth}{20mm}
  \setlength{\headwidth}{\textwidth}
  \addtolength{\headwidth}{\marginparsep}
  \addtolength{\headwidth}{\marginparwidth}
  \addtolength{\marginparwidth}{15mm} % plus d'espace pour titres de documentation

  %% Titres des sections et sous-sections
  \setsecheadstyle{\normalfont\Large\sffamily\bfseries\raggedright}
  \setsubsecheadstyle{\normalfont\large\sffamily\bfseries\raggedright}
  \maxsecnumdepth{subsection}
  \setsecnumdepth{subsection}

  %% Listes. Paramétrage avec enumitem.
  \setlist[enumerate]{leftmargin=*,align=left}
  \setlist[enumerate,2]{label=\alph*),labelsep=*,leftmargin=1.5em}
  \setlist[enumerate,3]{label=\roman*),labelsep=*,leftmargin=1.5em,align=right}
  \setlist[itemize]{leftmargin=*,align=left}

  %% Paramétrage de babel
  \frenchbsetup{CompactItemize=false,%
    ThinSpaceInFrenchNumbers=true,
    ItemLabeli=\mantriangleright,
    ItemLabelii=\textendash,
    og=«, fg=»}
  \def\frenchfigurename{{\scshape Fig.}}
  \def\frenchtablename{{\scshape Tab.}}

  %% Sections de code source
  \lstloadlanguages{[LaTeX]TeX}
  \lstset{language=[LaTeX]TeX,
    basicstyle=\ttfamily\NoAutoSpacing,
    keywordstyle=\mdseries,
    commentstyle=\color{comments}\slshape,
    emphstyle=\bfseries,
    escapeinside=`',
    extendedchars=true,
    showstringspaces=false,
    backgroundcolor=\color{LightYellow1},
    frame=lr, rulecolor=\color{LightYellow1},
    xleftmargin=3.4pt, xrightmargin=3.4pt}

  %%% =========================
  %%%  Nouveaux environnements
  %%% =========================

  %% Exemples
  \theoremstyle{definition}
  \newtheorem{exemple}{Exemple}[chapter]

  %% Exercices et réponses
  \Newassociation{sol}{solution}{solutions}
  \newcounter{exercice}[chapter]
  \renewcommand{\theexercice}{\thechapter.\arabic{exercice}}
  \newenvironment{exercice}[1][]{%
    \begin{list}{}{%
        \refstepcounter{exercice}
        \ifthenelse{\equal{#1}{nosol}}{%
          \renewcommand{\makelabel}{\bfseries\theexercice}}{%
          \hypertarget{ex:\theexercice}{}
          \Writetofile{solutions}{\protect\hypertarget{sol:\theexercice}{}}
          \renewcommand{\makelabel}{%
            \bfseries\protect\hyperlink{sol:\theexercice}{\theexercice}}}
        \settowidth{\labelwidth}{\bfseries\theexercice}
        \setlength{\leftmargin}{\labelwidth}
        \addtolength{\leftmargin}{\labelsep}
        \setlist[enumerate,1]{label=\alph*),labelsep=*,leftmargin=1.5em}
        \setlist[enumerate,2]{label=\roman*),labelsep=0.5em,align=right}}
      \item}
    {\end{list}}
  \renewenvironment{solution}[1]{%
    \begin{list}{}{%
        \renewcommand{\makelabel}{%
          \bfseries\protect\hyperlink{ex:#1}{#1}}
        \settowidth{\labelwidth}{\bfseries #1}
        \setlength{\leftmargin}{\labelwidth}
        \addtolength{\leftmargin}{\labelsep}
        \setlist[enumerate,1]{label=\alph*),labelsep=*,leftmargin=1.5em}
        \setlist[enumerate,2]{label=\roman*),labelsep=0.5em,align=right}}
      \item}
    {\end{list}}

  %% Démo de code LaTeX. Le code de 'texample' et 'eqxample' est
  %% repris de amsldoc.tex avec des petits ajustements.
  \newenvironment{demo}{%
    \begin{trivlist}\item}{%
    \end{trivlist}}
  \newenvironment{texample}[1][0.5\linewidth]{%
    \noindent\begin{minipage}{#1}%
      \def\producing{\end{minipage}\hfill\begin{minipage}{\dimexpr0.97\linewidth-#1}%
        \hbox\bgroup\kern-.2pt%
        \vbox\bgroup\parindent0pt\relax
        % The 3pt is to cancel the -\lineskip from \displ@y
        \abovedisplayskip3pt \abovedisplayshortskip\abovedisplayskip
        \belowdisplayskip0pt \belowdisplayshortskip\belowdisplayskip
        \noindent}
    }{%
      \par
      % Ensure that a lonely \[\] structure doesn't take up width less than
      % \hsize.
      \hrule height0pt width\hsize
      \egroup\kern-.2pt\egroup
    \end{minipage}%
    \par
  }
  \newenvironment{eqxample}{%
    \noindent\begin{minipage}{.5\columnwidth}%
      \def\producing{\end{minipage}\hfill\begin{minipage}{.45\columnwidth}%
        \hbox\bgroup\kern-.2pt\vrule width.2pt%
        \vbox\bgroup\parindent0pt\relax
        % The 3pt is to cancel the -\lineskip from \displ@y
        \abovedisplayskip3pt \abovedisplayshortskip\abovedisplayskip
        \belowdisplayskip0pt \belowdisplayshortskip\belowdisplayskip
        \noindent}
    }{%
      \par
      % Ensure that a lonely \[\] structure doesn't take up width less than
      % \hsize.
      \hrule height0pt width\hsize
      \egroup\vrule width.2pt\kern-.2pt\egroup
    \end{minipage}%
    \par
  }

  %% Un exemple du chapitre Trucs et astuces nécessite des
  %% environnements 'lstlisting' imbriqués, ce que ne digère pas
  %% LaTeX. La ruse consiste à définir un environnement équivalent qui
  %% porte simplement un autre nom.
  \lstnewenvironment{vglisting}{\lstset{deletetexcs={int,include}}}{}

  %% Exemples de notices bibliographiques
  \newenvironment{bibexample}[1][\linewidth]{%
    \begin{minipage}[t]{#1}%
      \begin{trivlist}}
      {\end{trivlist}\end{minipage}}

  %% Encadré générique pour les remarques importantes et autres
  %% comportant une icône sur la gauche. Argument: symbole à
  %% placer sur la gauche (obligatoire).
  \newenvironment{infobox}[1]{%
    \setlength{\FrameRule}{1pt}
    \begin{table}[h]%
      \begin{framed}%
        \noindent
        \begin{minipage}{0.1\linewidth}
          \raisebox{-1.5em}[0em][0em]{\HUGE#1}
        \end{minipage}
        \begin{minipage}[t]{0.88\linewidth}}%
        {\end{minipage}\end{framed}\end{table}}

  %% Remarques importantes
  \newenvironment{important}{%
    \begin{infobox}{\faExclamationCircle}}%
    {\end{infobox}}

  %% Informations
  \newenvironment{information}{%
    \begin{infobox}{\faInfoCircle}}%
    {\end{infobox}}

  %% Encadré avec titre (basé sur 'titled-frame' de framed pour les
  %% conseils du TeXpert. Cet environnement est laissé flottant.
  \newenvironment{conseil}{%
    \colorlet{TFFrameColor}{black}%
    \colorlet{TFTitleColor}{white}%
    \begin{table}%
      \begin{titled-frame}{\sffamily Conseil du {\TeX}pert}%
        \noindent
        \begin{minipage}{0.1\linewidth}
          \raisebox{-1.5em}[0em][0em]{\HUGE\faThumbsOUp}
        \end{minipage}
        \begin{minipage}[t]{0.88\linewidth}}%
        {\end{minipage}\end{titled-frame}\end{table}}

  %%% =======
  %%%  Index
  %%% =======
  \renewcommand{\preindexhook}{%
    Cet index contient des références aux commandes et environnements
    {\LaTeX}, ainsi qu'aux noms de paquetages et de classes.%
    \vskip\onelineskip%
    Le premier numéro indique habituellement, mais pas toujours,
    la page où un concept est introduit, défini ou expliqué.%
    \vskip\onelineskip}
  \lstset{language=[AlLaTeX]TeX,
    morekeywords={align,align*,aligned,bmatrix,cases,equation*,%
      figure,gather,lstlisting,multline,quote,split,%
      table,tabular,tabularx},
    deletekeywords={document},   % répéter dans deletetexcs
    moretexcs={toprule,midrule,bottomrule,%
      includegraphics,reflectbox,resizebox,rotatebox,scalebox,%
      includepdf,frenchfigurename,frenchtablename,%
      newsubfloat,subcaption,%
      bm,dfrac,tfrac,iint,text,mathcal,mathbb,eqref,symbf,%
      citet,citep,citeauthor,citeyear,%
      setmainfont,setsansfont,setmonofont,setmathfont,%
      color,definecolor,colorlet,hypersetup},
    deletetexcs={document,documentclass,usepackage,begin,end,LaTeX,TeX,%
      normalfont,bfseries,textbf,itshape,scshape,sffamily,ttfamily,texttt,%
      emph,small,Huge,raggedright,%
      hfill,def,a,b,c,d,em,i,j,l,r,t},
    index=[1][keywords],        % environnements
    indexstyle=[1]\ixenv,
    index=[2][texcs],           % commandes
    indexstyle=[2]\ixcmd}
  \newcommand{\ixenv}[1]{\index{#1 env@\Pe{#1} (environnement)}%
    \index{environnement!#1@\Pe{#1}}}
  \newcommand{\ixcmd}[1]{\index{#1@\string\cs{#1}}}
  \makeindex

  %%% =====================
  %%%  Nouvelles commandes
  %%% =====================

  %% Noms de fonctions, code, environnement, etc.
  \newcommand{\code}[1]{\texttt{#1}}
  \newcommand{\fichier}[1]{\texttt{#1}}
  \newcommand{\class}[1]{\textsf{#1}\index{#1 class@\textsf{#1} (classe)}%
    \index{classe!#1}}
  \newcommand{\pkg}[1]{\textbf{#1}\index{#1 pkg@\textbf{#1} (paquetage)}%
    \index{paquetage!#1}}
  \newcommand{\Pe}[1]{\code{#1}}         % tiré de la doc de memoir
  \newcommand{\Ie}[1]{\Pe{#1}\ixenv{#1}} % idem
  \newcommand{\mat}[1]{\symbf{#1}}       % en mode mathématique

  %% Modification de commandes tirées de memoir.cls servant à afficher
  %% des noms de commandes.
  %% - \cmdprint est modifiée pour que le nom de la commande ne soit
  %%   pas en italique;
  %% - \cmd est modifiée pour utiliser @ comme séparateur dans \index
  %%   et pour utiliser \cs plutôt que \cmdprint pour afficher le nom de
  %%   la commande (afin d'obtenir le même format d'entrée d'index
  %%   qu'avec \ixcmd ci-dessus).
  \renewcommand{\cmdprint}[1]{\textup{\texttt{\string#1}}}
  \makeatletter
  \renewcommand{\cmd}[1]{\cmdprint{#1}%
    \index{\expandafter\@gobble\string#1@\string\cs{\expandafter\@gobble\string#1}}}
  \makeatother

  %% Indications de capsule vidéo dans la marge
  \newcommand{\capsule}[2]{\href{#1}{#2}\marginpar{%
      \href{#1}{\raisebox{-0.5em}[0em][0em]{\HUGE\faYoutubePlay}}}}

  %% Hyperlien avec symbole de lien externe juste après
  \newcommand{\link}[2]{\href{#1}{#2~\raisebox{-0.2ex}{\faExternalLink}}}

  %% Lien vers documentation dans la marge
  %% usage: \doc[documentation]{nom_fichier}{url}
  \newcommand{\doc}[3][documentation]{\link{#3}{#1}%
    \ifthenelse{\equal{#2}{}}{}{\marginpar%
      [\hfill\faBook~\fichier{#2}]%
      {\faBook~\fichier{#2}}}}

  %% Suppression de l'hyperlien
  \newcommand{\nolink}[1]{\begin{NoHyper}#1\end{NoHyper}}

  %% «Bouton» de la page de copyright
  \newcommand{\browsebutton}{%
    \setlength{\fboxrule}{1pt}%
    \framebox[40mm][l]{%
      \rule[-5pt]{0mm}{16pt}%
      \makebox[7mm]{\raisebox{-3pt}{\LARGE\faExternalLink}}\;%
      {\sffamily Accéder au dépôt}}}

  %% «Bouton» pour accéder à CTAN
  \newcommand{\ctanbutton}{%
    \setlength{\fboxrule}{1pt}%
    \framebox[62mm][l]{%
      \rule[-5pt]{0mm}{16pt}%
      \makebox[7mm]{\raisebox{-3pt}{\LARGE\faExternalLink}}\;%
      {\sffamily Accéder aux fichiers dans CTAN}}}

  %% Pour le tableau des commandes d'espacement en mode mathématique.
  %% Pris de la doc de amsmath.
  \newcommand{\lspx}{\mathord{\dashv\mkern-3mu}}
  \newcommand{\rspx}{\mathord{\mkern-2mu\vdash}}
  \newcommand{\spx}[1]{$\lspx #1\rspx$}

  %% Logo BIBTeX; tiré de http://bit.ly/1RQqUnG
  \newcommand{\BibTeX}{\rmfamily B\kern-.05em{\scshape i\kern-.025em b}\kern-.08em \TeX}

  %% Chapitre sur les bibliographies: des références bibliographiques
  %% sont insérées dans le texte avec \bibentry. Certaines commandes
  %% de francaisbst.tex sont alors utilisées, mais non encore
  %% définies. Répétées ici. De plus, il faut définir ici la commande
  %% \enquote plutôt que dans francais.bst. C'est pourquoi il y a une
  %% version modifiée de ce fichier dans ces sources.
  %% Voir http://bit.ly/1MORZmp
  \global\def\bbland{et}
  \global\def\bbledn{\'ed.}
  \global\def\bblfourtho{4{\ieme}}
  \global\def\bblth{{\ieme}}
  \global\def\bblvol{vol.}
  \def\bblno{\no{}}
  \def\bblpp{p.}
  \newcommand{\enquote}[1]{\guillemotleft#1\guillemotright}

  %%% =======
  %%%  Varia
  %%% =======

  %% Style de la bibliographie
  \bibliographystyle{francais}

  %%% ==================================================
  %%%  Page titre et autres informations de publication
  %%% ==================================================
  \title{\protect\raggedright%
    \bfseries\fontsize{32}{32}\selectfont Rédaction de thèses \\
                                          et de mémoires avec {\LaTeX} \\[10mm]
    \bfseries\fontsize{24}{24}\selectfont {\fullcaps 2} \textbar\ Concepts avancés}
  \author{\protect\raggedright%
    \bfseries\fontsize{16}{20}\selectfont Vincent Goulet \\
    \mdseries\fontsize{14}{18}\selectfont Professeur titulaire \textbar\
                                          École d'actuariat}
  \date{%
    \mdseries\fontsize{14}{18}\selectfont Édition \fullcaps\year}
  \renewcommand{\year}{2016}
  \newcommand{\ISBN}{978-2-9811416-5-1}
  \newcommand{\ctanurl}{https://ctan.org/pkg/formation-latex-ul/}

%  \includeonly{frontispice}

\begin{document}

\frontmatter

\pagestyle{empty}

%% Page couverture avant.
\includepdf[pages=1]{couvertures-partie_2}
\cleardoublepage

%% Page frontispice
\AddToShipoutPictureBG*{%
  \AtTextLowerLeft{\hspace*{40mm}\includegraphics[height=100mm]{ctanlion}}}
\begin{adjustwidth*}{-14.2mm}{0mm}
  \sffamily
  \raggedright
  \vspace*{-15mm}
  \thetitle \\
  \vspace*{50mm}
  \theauthor \\
  \vspace*{\fill}
  \thedate
\end{adjustwidth*}

%%% Local Variables:
%%% mode: latex
%%% TeX-master: "formation-latex-ul"
%%% End:

\clearpage

%% Page de copyright
\include{licence-partie_2}
\clearpage

%% Corps du document
\pagestyle{companion}

\chapter{Introduction}
\label{chap:introduction}

Ce document constitue la seconde partie d'une formation sur la
rédaction de thèses et de mémoires avec {\LaTeX} développée pour la
Bibliothèque de l'Université Laval. La première partie de
la formation se déroulant en classe, la documentation qui
l'accompagne consiste en une série de diapositives
\citep{UL:latex:1}.

Nous reprenons ici la formation une fois présentés les concepts de
base de {\LaTeX} pour un nouvel utilisateur: processus d'édition,
compilation, visualisation; séparation du contenu et de l'apparence du
texte; mise en forme du texte; séparation du document en parties;
rudiments du mode mathématique. Avec cette seconde partie, une
personne devrait être en mesure de composer des documents relativement
complexes comportant des tableaux, des figures, des équations
mathématiques élaborées, une bibliographie, etc.

Le présent ouvrage n'a aucune prétention d'exhaustivité. La
consultation de documentation additionnelle peut s'avérer nécessaire
pour réaliser des mises en page plus élaborées. À cet égard, nous
recommandons chaudement le livre de \citet{Kopka:latex:4e} --- il a
servi d'inspiration pour ce document à maints endroits. La très
complète documentation (plus de 600 pages!) de la classe
\class{memoir} \citep{memoir}, sur laquelle se base la classe
\class{ulthese} pour les thèses et mémoires de l'Université Laval,
constitue une autre référence de choix. Nous recommandons également:
\begin{itemize}
\item \link{http://fr.wikibooks.org/wiki/LaTeX}{\emph{LaTeX} dans
    Wikilivre} pour de la documentation en ligne, en français et
  libre;
\item le très actif forum de discussion
  \link{http://tex.stackexchange.com}{{\TeX}--{\LaTeX} Stack Exchange}
  (avant de penser y poser une question, vérifier que la réponse ne se trouve
  pas déjà dans le forum\dots\ ce qui risque fort d'être le cas);
\item la très complète
  \link{http://www.tex.ac.uk/cgi-bin/texfaq2html}{%
    \emph{foire aux questions}} (en anglais) du groupe des
  utilisateurs de {\LaTeX} du Royaume-Uni.
\end{itemize}

%%% Local Variables:
%%% mode: latex
%%% TeX-engine: xetex
%%% TeX-master: "formation_latex_UL-partie_2"
%%% encoding: utf-8
%%% End:

\chapter{Mode d'emploi}
\label{chap:modedemploi}

\section*{Hyperliens vers la documentation}

À plusieurs endroits dans le document nous renvoyons le lecteur vers
la documentation d'un paquetage ou d'une classe, par exemple vers la %
\doc{ulthese}{http://texdoc.net/pkg/ulthese/} %
de la classe \class{ulthese}. Le format du renvoi est toujours tel
qu'illustré ici: un hyperlien mène vers la version en ligne de la
documentation dans le site %
\link{http://texdoc.net}{TeXdoc Online}; on trouve dans la marge le
nom du fichier correspondant (sans l'extension \code{.pdf}) sur un
système doté de {\TeX}~Live.

Sur la plupart des systèmes, il est possible de consulter hors ligne
le fichier de documentation \meta{fichier}\code{.pdf} en entrant à une
invite de commande
\begin{quote}
\begin{lstlisting}[backgroundcolor=\color{white}]
texdoc `\meta{fichier}'
\end{lstlisting}
\end{quote}

Plusieurs logiciels intégrés de rédaction offrent une inferface pour
accéder à cette documentation. Quelques exemples.
\begin{itemize}
\item TeXShop: menu \code{Aide|Afficher l'aide pour le
    package} (\optkey\,\cmdkey\, I);
\item Texmaker: menu \code{Aide|TeXDoc [selection]};
\item GNU Emacs: commande \code{TeX-doc} (\code{C-c ?}).
\end{itemize}
Consulter l'aide de votre éditeur pour savoir s'il offre une
interface à \code{texdoc}.

% \section{Capsules d'aide additionnelle}
%
% Un symbole de lecture vidéo dans la marge indique qu'une capsule vidéo
% est disponible dans la %
% \capsule{http://www.youtube.com/user/ULFormationLaTeX}{chaîne
%   YouTube} %
% de la formation sur le sujet en hyperlien.

\section*{Fichiers d'accompagnement}
\enlargethispage{5mm}

Ce document devrait être accompagné des fichiers nécessaires pour
compléter certains exercices figurant à la fin des chapitres, ainsi
que d'un fichier \fichier{exercice\_gabarit.tex} pouvant servir de
gabarit pour composer les solutions des autres exercices.

Vous pouvez
récupérer ces fichiers dans le site \emph{Comprehensive TeX Archive Network}
(CTAN).
\begin{center}
  \href{\ctanurl}{\ctanbutton}
\end{center}


%%% Local Variables:
%%% mode: latex
%%% TeX-engine: xetex
%%% TeX-master: "formation_latex_UL-partie_2"
%%% encoding: utf-8
%%% End:


\cleartorecto
\tableofcontents*

\mainmatter

\chapter{Document contenu dans plusieurs fichiers}
\label{chap:include}

Un document {\LaTeX} comporte toujours un préambule suivi du corps du
texte. Lorsque ceux-ci sont relativement courts (peu de commandes
spéciales et moins d'une vingtaine de pages de texte), il demeure
assez simple et convivial d'en faire l'édition dans un seul fichier à
l'aide de son éditeur de texte favori.

Cependant, si le préambule devient long et complexe ou, surtout,
lorsque l'ampleur du document augmente jusqu'à compter un grand nombre
de pages sur plusieurs chapitres, il convient de répartir les divers
éléments du document dans des fichiers séparés.

La segmentation en plusieurs fichiers rend l'édition du texte plus
simple et plus efficace. De plus, durant la phase de rédaction, elle
peut significativement accélérer, la compilation des documents très
longs ou comptant plusieurs images.


\section{Insertion du contenu d'un autre fichier}
\label{sec:include:input}

La commande \cmd{\input} permet d'insérer le contenu d'un autre
fichier dans un document {\LaTeX}. La syntaxe de la commande est
\begin{lstlisting}
\input`\marg{fichier}'
\end{lstlisting}
où le nom du fichier à insérer est \meta{fichier}\code{.tex}. On
laisse donc tomber l'extension \code{.tex}, qui est implicite. Le
contenu du fichier est inséré tel quel dans le document, comme s'il
avait été tapé dans le fichier qui contient l'appel à \cmd{\input}.

Le procédé est surtout utile pour sauvegarder séparément des bouts de
code qui pourraient nuire à l'édition du texte (figures, longs
tableaux) ou qui sont communs entre plusieurs documents (licence
d'utilisation, auteur et affiliation).

La commande peut aussi être utilisée dans le préambule pour charger
une partie ou l'ensemble de celui-ci. Cela permet de composer un même
préambule pour plusieurs documents. Il suffit alors de faire
d'éventuelles modifications à un seul endroit pour les voir prendre
effet dans tous les documents.


\section{Insertion de parties d'un document}
\label{sec:include:include}

Extrait de la %
\doc{ulthese}{http://texdoc.net/pkg/ulthese/} %
de la classe \class{ulthese}:
\begin{quote}
  «Il est recommandé de segmenter tout document d'une certaine ampleur
  dans des fichiers \verb=.tex= distincts pour chaque partie ---
  habituellement un fichier par chapitre. Le document complet est
  composé à l'aide d'un fichier maître qui contient le préambule
  {\LaTeX} et un ensemble de commandes \verb=\include= pour réunir les
  parties dans un tout.»
\end{quote}

Comme \cmd{\input}, la commande \cmd{\include} insère le contenu
d'un autre fichier dans un document {\LaTeX}. Son effet est cependant
différent et c'est son utilisation qui permet d'accélérer la
compilation d'un long document.

L'insertion d'un fichier avec \cmd{\include} débute toujours une
nouvelle page. On utilisera donc \cmd{\include} principalement pour
insérer des chapitres entiers plutôt que seulement des portions de
texte. De plus, un fichier inséré avec \cmd{\include} peut contenir
des appels à \cmd{\input}, mais pas à \cmd{\include}.

La syntaxe de la commande \cmd{\include} est
\begin{lstlisting}
\include`\marg{fichier}'
\end{lstlisting}
où le nom du fichier à insérer est \meta{fichier}\code{.tex}. Ici
aussi on laisse tomber l'extension \code{.tex} qui est implicite.

 La structure type d'un fichier maître est la suivante:
\begin{lstlisting}
\documentclass{ulthese}
  [...]

\begin{document}

\frontmatter

\chapter{Introduction}
\label{chap:introduction}

Ce document constitue la seconde partie d'une formation sur la
rédaction de thèses et de mémoires avec {\LaTeX} développée pour la
Bibliothèque de l'Université Laval. La première partie de
la formation se déroulant en classe, la documentation qui
l'accompagne consiste en une série de diapositives
\citep{UL:latex:1}.

Nous reprenons ici la formation une fois présentés les concepts de
base de {\LaTeX} pour un nouvel utilisateur: processus d'édition,
compilation, visualisation; séparation du contenu et de l'apparence du
texte; mise en forme du texte; séparation du document en parties;
rudiments du mode mathématique. Avec cette seconde partie, une
personne devrait être en mesure de composer des documents relativement
complexes comportant des tableaux, des figures, des équations
mathématiques élaborées, une bibliographie, etc.

Le présent ouvrage n'a aucune prétention d'exhaustivité. La
consultation de documentation additionnelle peut s'avérer nécessaire
pour réaliser des mises en page plus élaborées. À cet égard, nous
recommandons chaudement le livre de \citet{Kopka:latex:4e} --- il a
servi d'inspiration pour ce document à maints endroits. La très
complète documentation (plus de 600 pages!) de la classe
\class{memoir} \citep{memoir}, sur laquelle se base la classe
\class{ulthese} pour les thèses et mémoires de l'Université Laval,
constitue une autre référence de choix. Nous recommandons également:
\begin{itemize}
\item \link{http://fr.wikibooks.org/wiki/LaTeX}{\emph{LaTeX} dans
    Wikilivre} pour de la documentation en ligne, en français et
  libre;
\item le très actif forum de discussion
  \link{http://tex.stackexchange.com}{{\TeX}--{\LaTeX} Stack Exchange}
  (avant de penser y poser une question, vérifier que la réponse ne se trouve
  pas déjà dans le forum\dots\ ce qui risque fort d'être le cas);
\item la très complète
  \link{http://www.tex.ac.uk/cgi-bin/texfaq2html}{%
    \emph{foire aux questions}} (en anglais) du groupe des
  utilisateurs de {\LaTeX} du Royaume-Uni.
\end{itemize}

%%% Local Variables:
%%% mode: latex
%%% TeX-engine: xetex
%%% TeX-master: "formation_latex_UL-partie_2"
%%% encoding: utf-8
%%% End:

\tableofcontents*

\mainmatter
\include{historique}            % premier chapitre
\include{modele}                % deuxième chapitre
[...]

\end{document}
\end{lstlisting}

\begin{conseil}
  Utiliser des noms des fichiers qui permettent de facilement
  identifier leur contenu. Par exemple, un nom comme
  \fichier{rappels.tex} est plus parlant et résiste mieux aux
  changements à l'ordre des chapitres que \fichier{chapitre1.tex}.
\end{conseil}

Le principal avantage de \cmd{\include} par rapport à \cmd{\input}
réside dans le fait que {\LaTeX} peut préserver entre les compilations
les informations telles que les numéros de pages, de sections ou
d'équations, ainsi que les références. Cela permet, par exemple, de
compiler le texte d'un seul chapitre --- plutôt que le document entier
--- et néanmoins obtenir une image représentative du chapitre.
Procéder ainsi accélère significativement la compilation des documents
longs ou complexes.

La commande \cmd{\includeonly}, que l'on utilise exclusivement dans le
préambule, sert à spécifier le ou les fichiers à compiler tout en
préservant la numérotation et les références. Sa syntaxe est
\begin{lstlisting}
\includeonly`\marg{liste\_fichiers}'
\end{lstlisting}
où \meta{liste\_fichiers} contient les noms des fichiers à
inclure dans la compilation, séparés par des virgules et sans
l'extension \code{.tex}.

Lors de l'utilisation de la commande \cmd{\includeonly}, toute la
numérotation dans les fichiers \meta{liste\_fichiers} suivra celle
établie lors de la compilation précédente. Si l'édition des fichiers
de \meta{liste\_fichiers} cause des changements dans la numérotation
et les références dans les autres parties du document, une nouvelle
compilation de l'ensemble ou d'une partie de celui-ci s'avérera
nécessaire.

\begin{exemple}
  Un document est composé en plusieurs parties avec les commandes
  suivantes:
\begin{lstlisting}
\include{historique}            % chapitre 1
\include{rappels}               % chapitre 2
\include{modele}                % chapitre 3
\end{lstlisting}
  Les chapitres débutent respectivement aux pages~1, 23 et 41.
  \begin{itemize}
  \item Si on ajoute au préambule du document la commande
\begin{lstlisting}
\includeonly{rappels}
\end{lstlisting}
    le numéro du chapitre sera toujours 2 et le folio de
    la première page sera toujours 23, même si les 22 pages
    précédentes ne se trouvent pas dans le document.
  \item Si l'on modifie le fichier \fichier{rappels.tex} de telle
    sorte que le chapitre se termine maintenant à la page 46, il
    faudra recompiler le document avec au moins les fichiers
    \fichier{rappels.tex} et \code{modele.tex} pour que les pages du
    chapitre~3 soient renumérotées à partir de 47.
  \end{itemize}
  \qed
\end{exemple}

L'\autoref{ex:include} illustre mieux le cycle typique
d'utilisation des commandes \cmd{\include} et \cmd{\includeonly}.



%%%
%%% Exercices
%%%

\section{Exercices}
\label{sec:include:exercices}

\begin{exercice}[nosol]
  \label{ex:include}
  Cet exercice fait appel au fichier maître
  \fichier{exercice\_include.tex} et à plusieurs fichiers auxiliaires.
  Schématiquement, le document est composé ainsi:

  \medskip
  \begin{minipage}{\linewidth}
    \dirtree{%
      .1 exercice\_include.tex.
      .2 {\textbackslash}input pagetitre.tex.
      .2 {\textbackslash}include presentation.tex.
      .3 {\textbackslash}includegraphics console-screenshot.pdf.
      .2 {\textbackslash}include emacs.tex.
    }
  \end{minipage}
  \medskip

  La commande \cmd{\includegraphics} permet d'insérer une image dans
  un document {\LaTeX}. Elle provient du paquetage \pkg{graphicx}.

  \begin{enumerate}
  \item Étudier le code source du fichier maître
    \fichier{exercice\_include.tex}, puis le compiler deux à trois
    fois jusqu'à ce que toutes les références internes soient à jour.
    Il est normal à ce stade que la figure~1 du document soit vide.
  \item Ajouter dans le préambule du fichier maître la commande
\begin{lstlisting}
\includeonly{emacs}
\end{lstlisting}
    puis compiler le document.

    Observer que, malgré l'absence du chapitre~1, la numérotation et
    les références demeurent à jour, notamment la table des matières.
  \item Remplacer la commande ajoutée en b) dans le préambule du
    fichier maître par la commande
\begin{lstlisting}
\includeonly{presentation}
\end{lstlisting}
    Vers la fin du fichier \fichier{presentation.tex}, activer la
    commande
\begin{lstlisting}
\includegraphics[width=\textwidth]{console-screenshot}
\end{lstlisting}
    en supprimant le symbole \% au début de la ligne. Compiler de
    nouveau le document deux fois.

    Les modifications ont eu pour effet d'ajouter une page au
    chapitre~1. Observer que selon la table des matières, le
    chapitre~2 débute toujours à la page~3 alors que celle-ci est
    maintenant occupée par la figure~1.
  \item Afin de corriger la table des matières, désactiver dans le
    préambule du fichier maître la commande \cmd{\includeonly}, puis
    compiler de nouveau le document quelques fois.
  \end{enumerate}
\end{exercice}

\begin{exercice}[nosol]
  Déplacer dans un fichier \fichier{preambule.tex} toutes les lignes
  du préambule du fichier \fichier{exercice\_include.tex} utilisé à
  l'exercice précédent, à l'exception de celles relatives à la page
  titre (titre, auteur, date). Insérer le préambule dans
  \fichier{exercice\_include.tex} avec la commande \cmd{\input}.
\end{exercice}

%%% Local Variables:
%%% mode: latex
%%% TeX-engine: xetex
%%% TeX-master: "formation_latex-partie_2"
%%% coding: utf-8
%%% End:

\chapter{Boîtes}
\label{chap:boites}

Il arrive que l'on doive traiter de manière spéciale une aire
rectangulaire de texte; pour l'encadrer, la mettre en surbrillance ou
la mettre en exergue, par exemple.

Avec les traitements de texte, on aura souvent recours aux tableaux à
de telles fins. Or, les tableaux devraient être réservés pour disposer
de l'information sous forme de lignes et de colonnes. (Un tableau
d'une seule cellule n'a donc pas vraiment de sens d'un point de vue
sémantique ou typographique.) Pour disposer et mettre en forme du tout
autre type contenu se présentant sous forme rectangulaire, {\LaTeX}
offre la solution plus générale des «boîtes».

Il existe trois sortes de boîtes en {\LaTeX}: les boîtes horizontales
dont le contenu est disposé exclusivement côte à côte; les boîtes
verticales qui peuvent contenir plusieurs lignes de contenu; les
boîtes de réglure pour former des lignes pleines de largeur et de
hauteur quelconques.

Il n'est pas inutile de savoir, au passage, que {\TeX} ne manipule que
cela, des boîtes. Pour {\TeX}, chaque caractère, chaque lettre n'est
qu'un rectangle d'une certaine largeur qui s'élève au-dessus de la
ligne de base (les lignes d'une feuille lignée) et qui, parfois, se
prolonge sous la ligne de base (pensons aux lettres \emph{p}, \emph{y}
ou \emph{Q}). Les commandes et environnements présentés ci-dessous
permettent simplement de créer d'autres boîtes dont le contrôle des
dimensions et du contenu est laissé à l'usager.

Une fois créée, une boîte ne peut être scindée en parties, notamment
entre les lignes ou entre les pages.



\section{Boîtes horizontales}
\label{sec:boites:lrbox}

Le plus simple concept de boîte dans {\LaTeX} est celui de boîte
«horizontale», c'est-à-dire dont le contenu est disposé latéralement
de gauche à droite\footnote{%
  D'où l'appellation \emph{LR (left-right) box} en anglais.}. %
Le contenu est normalement du texte, mais conceptuellement ce pourrait
être n'importe quoi, y compris d'autres boîtes.

Les commandes de base pour créer des boîtes horizontales sont:
\begin{lstlisting}
\mbox`\marg{texte}'
\fbox`\marg{texte}'
\end{lstlisting}
Elles produisent une boîte de la largeur précise de \meta{texte}. Avec
la commande \cmd{\fbox}, le texte est au surplus \fbox{encadré}. En
usage courant, la commande \cmd{\mbox} sert principalement à deux
choses:
\begin{enumerate}
\item réunir en un bloc du texte que l'on ne veut pas voir scindé
  entre les lignes ou entre les pages;
\item \label{item:tableaux:mbox} créer une boîte vide avec
  \cs{mbox\{\}} afin de laisser croire à {\TeX} que du contenu
  apparaît à un endroit, sans toutefois qu'il n'occupe aucun espace.
  %% ça prendrait un exemple de cela dans les exercices!
\end{enumerate}

Il existe également des versions plus générales des commandes
\cmd{\mbox} et \cmd{\fbox}:
\begin{lstlisting}
\makebox`\oarg{largeur}\oarg{pos}\marg{texte}'
\framebox`\oarg{largeur}\oarg{pos}\marg{texte}'
\end{lstlisting}
Les arguments optionnels \meta{largeur} et \meta{pos} déterminent
respectivement la largeur de la boîte et la position du texte
dans la boîte. Les valeurs possibles de \meta{pos} sont: \code{l} pour
du texte aligné à gauche, \code{r} pour du texte aligné à droite et
\code{c} (la valeur par défaut) pour du texte centré. Ainsi, la commande
\begin{lstlisting}
\framebox[3.5cm][l]{aligné à gauche}
\end{lstlisting}
produit \framebox[3.5cm][l]{aligné à gauche}, alors que
\begin{lstlisting}
\makebox[3.5cm]{centré}
\end{lstlisting}
produit \makebox[3.5cm]{centré}.



\section{Boîtes verticales}
\label{sec:boites:parbox}

Les boîtes verticales se distinguent des boîtes horizontales par le
fait qu'elles peuvent contenir plusieurs lignes de contenu empilées
les unes au-dessus des autres. Lorsque le contenu en question est du
texte, on obtient des paragraphes\footnote{%
  D'où l'appellation, cette fois, de \emph{paragraph boxes} en anglais
  ou \emph{parboxes} dans le jargon {\LaTeX}.}. %

La commande de base pour créer une boîte verticale est:
\begin{lstlisting}
\parbox`\oarg{pos}\marg{largeur}\marg{texte}'
\end{lstlisting}
Ici, l'argument optionnel \meta{pos} permet d'ajuster l'alignement
vertical de la boîte avec la ligne de base: \code{b} ou \code{t} selon
que l'on souhaite aligner, respectivement, le bas ou le haut de la
boîte avec la ligne de base. Par défaut, la boîte est centrée avec la
ligne de base. Cet argument n'a aucun effet si la boîte est le seul
élément de contenu du paragraphe.

On remarquera que l'argument \meta{largeur} est ici obligatoire.
Autrement dit, on doit nécessairement définir la largeur des boîtes
verticales, un peu comme il faut bien définir la largeur de la page
pour le texte normal.

Les boîtes créées avec \cmd{\parbox} ne peuvent contenir de structures
«complexes» comme des listes ou des tableaux. Parce que plus général,
l'outil véritablement utile pour la création de boîtes verticales est
l'environnement \Ie{minipage}. Cet environnement peut contenir à peu
n'importe quel type de contenu. Comme son nom l'indique, c'est ni plus
ni moins qu'une page miniature à l'intérieur de la page standard.

La syntaxe de l'environnement \Ie{minipage} est la suivante:
\begin{lstlisting}
\begin{minipage}`\oarg{pos}\marg{largeur}'
  `\meta{texte}'
\end{minipage}
\end{lstlisting}
La signification des arguments \meta{largeur} et \meta{pos} est la
même que pour la commande \cmd{parbox}.

L'environnement \Ie{minipage} est souvent utilisé pour disposer des
éléments de contenu de manière spécifique sur la page, notamment des
tableaux ou des figures côte à côte ou en grille (voir
l'\autoref{exemple:tableaux:grille} à la
\autopageref{exemple:tableaux:grille}).

\begin{exemple}
  Le code ci-dessous
\begin{lstlisting}
\begin{minipage}[b]{0.3\textwidth}
  La ligne inférieure de cette \emph{minipage} [...]
\end{minipage}
\hfill
\parbox{0.3\textwidth}{le centre de cette boîte [...] }
\hfill
\begin{minipage}[t]{0.3\textwidth}
  la ligne supérieure de cette \emph{minipage}. [...]
\end{minipage}
\end{lstlisting}
  produit: \\[0.5\baselineskip]
  \begin{minipage}{\textwidth}
    \makebox[0pt][l]{\color{lightgray}\rule{\textwidth}{0.7pt}}\relax
    \fbox{\begin{minipage}[b]{0.3\textwidth} La ligne inférieure de
        cette \emph{minipage} est alignée avec
      \end{minipage}} \hfill \fbox{\parbox{0.3\textwidth}{le centre de
        cette boîte verticale, qui est à son tour alignée avec}}
    \hfill \fbox{\begin{minipage}[t]{0.3\textwidth} la ligne
        supérieure de cette \emph{minipage}. Le filet horizontal grisé
        représente la ligne de base du paragraphe contenant les trois
        boîtes.
      \end{minipage}}
  \end{minipage}
  \qed
\end{exemple}

La commande \cmd{\hfill} utilisée entre les boîtes dans l'exemple
indique à {\LaTeX} d'insérer de l'espace blanc entre les éléments de
contenu de manière à remplir entièrement la ligne de texte. C'est une
commande très utile pour disposer automatiquement des éléments à
intervalles égaux sur la largeur du bloc de texte. Ainsi,
\begin{lstlisting}
\framebox[\textwidth]{gauche  \hfill droite}
\end{lstlisting}
produit \\[0.5\baselineskip]
\framebox[\textwidth]{gauche \hfill droite} \\[0.5\baselineskip]
alors que
\begin{lstlisting}
\framebox[\textwidth]{gauche \hfill centre \hfill droite.}
\end{lstlisting}
produit \\[0.5\baselineskip]
\framebox[\textwidth]{gauche \hfill centre \hfill droite.}



%%% Exercices: faire le contenu de 4.7.4 de Kopka et Daly (disposition
%%% verticale des boîtes)

\section{Boîtes de réglure}
\label{sec:boites:rulebox}

En imprimerie, une réglure est une ligne droite continue ou
pointillée. Une ligne n'étant jamais rien d'autre qu'un rectangle
plein, si mince fut-il, la réglure est le troisième type de
boîte\footnote{%
  \emph{Rule box}, en anglais} %
dans {\LaTeX}.

La commande
\begin{lstlisting}
\rule`\oarg{déplacement}\marg{largeur}\marg{hauteur}'
\end{lstlisting}
crée une réglure de dimensions \meta{largeur} $\times$ \meta{hauteur}.
Par défaut, la réglure s'appuie sur la ligne de base. Le résultat de
\begin{lstlisting}
\rule{2cm}{6pt}
\end{lstlisting}
est donc une ligne pleine de $2$~cm de long et de $6$~points d'épais:
\rule{2cm}{6pt}.

L'argument optionnel \meta{déplacement} permet de déplacer
verticalement la réglure au-dessus ou au-dessous de la ligne de base
selon que \meta{déplacement} est positif ou négatif. Avec les deux
commandes
\begin{lstlisting}
\rule[3pt]{2cm}{6pt}
\rule[-3pt]{2cm}{6pt}
\end{lstlisting}
on crée respectivement les réglures \rule[3pt]{2cm}{6pt} et
\rule[-3pt]{2cm}{6pt}.

Un usage intéressant de la réglure consiste à faire croire à {\TeX}
qu'une ligne est plus haute qu'il n'y parait en insérant dans celle-ci
une réglure de largeur nulle. Par exemple, la distance entre
\rule[-12pt]{0mm}{30pt}\relax la présente ligne et les autres du paragraphe est
plus grande que la normale parce que nous y avons inséré une réglure
invisible avec
\begin{lstlisting}
\rule[-12pt]{0mm}{30pt}
\end{lstlisting}
Ce truc est particulièrement utile pour augmenter la hauteur des
lignes dans un tableau; voir la \autoref{sec:tableaux:tableaux}.


%%% Local Variables:
%%% mode: latex
%%% TeX-engine: xetex
%%% TeX-master: "formation_latex-partie_2"
%%% coding: utf-8
%%% End:

\include{tableaux+figures}
\chapter{Mathématiques}
\label{chap:math}


S'il est un domaine où {\LaTeX} brille particulièrement, c'est bien
dans la préparation et la présentation d'équations mathématiques ---
des plus simples aux plus complexes. Après tout, l'amélioration de la
qualité typographique des équations mathématiques dans son ouvrage
phare \emph{The Art of Computer Programming} figurait parmi les
objectifs premiers de Knuth lorsqu'il a développé {\TeX}.

La première partie de cette formation aborde le sujet des
mathématiques, mais en s'en tenant qu'aux principes de base
\citep[section~7]{UL:latex:1}. [...]

Trop de symboles, consulter \citet{comprehensive} %
\doc[???]{comprehensive}{http://texdoc.net/pkg/comprehensive}


\section{Rappel des principes de base}
\label{sec:math:rappel}

La mise en forme d'équations mathématiques requiert d'indiquer à
l'ordinateur, dans un langage spécial, le contenu des dites équations
et la position des symboles: en exposant, en indice, sous forme de
fraction, etc. L'ordinateur peut ensuite assembler le tout à partir de
règles typographiques portant, par exemple, sur la représentation des
variables et des constantes, l'espacement entre les symboles ou la
disposition des équations selon qu'elles apparaissent au fil du texte
ou hors d'un paragraphe.

On indique à {\LaTeX} que l'on change de «langage», par l'utilisation
d'un mode mathématique. Il y a deux grandes manière d'activer le mode
mathématique:
\begin{enumerate}
\item en insérant le code entre les symboles \verb=$ $= pour générer
  une équation «en ligne», ou au fil du texte;
  \begin{trivlist}
  \item
    \begin{texinput}{0.48\linewidth}
\begin{lstlisting}
on sait que $(a + b)^2 = a^2
+ b^2$, d'où on obtient...
\end{lstlisting}
    \end{texinput}
    \hfill
    \begin{texoutput}[c]{0.48\linewidth}
      on sait que $(a + b)^2 = a^2 + b^2$, d'où on obtient...
    \end{texoutput}
  \end{trivlist}
\item en utilisant un environnement servant à créer une équation hors
  paragraphe;
\begin{trivlist}
  \item
    \begin{texinput}{0.48\linewidth}
\begin{lstlisting}
on sait que
\begin{equation*}
  (a + b)^2 = a^2 + b^2,
\end{equation*}
d'où on obtient...
\end{lstlisting}
    \end{texinput}
    \hfill
    \begin{texoutput}[c]{0.48\linewidth}
      on sait que
      \begin{equation*}
        (a + b)^2 = a^2 + b^2,
      \end{equation*}
      d'où on obtient...
    \end{texoutput}
  \end{trivlist}
\end{enumerate}

En mode mathématique, les chiffres sont automatiquement considérés
comme des constantes, les lettres comme des variables et une suite de
lettres comme un produit de variables (nous verrons plus loin comment
représenter des fonctions mathématiques comme $\sin$, $\log$ ou
$\lim$). Ceci a trois conséquences principales:
\begin{enumerate}
\item conformément aux conventions typographiques, les chiffres sont
  représentés en caractère \textrm{romain} et les variables en
  \emph{italique};
  \begin{trivlist}
  \item
    \begin{texinput}{0.48\linewidth}
\begin{lstlisting}
123xyz
\end{lstlisting}
    \end{texinput}
    \hfill
    \begin{texoutput}{0.48\linewidth}
      $123xyz$
    \end{texoutput}
  \end{trivlist}
\item l'espace entre les constantes, les variables et les opérateurs
  mathématiques est géré automatiquement;
  \begin{trivlist}
  \item
    \begin{texinput}{0.48\linewidth}
\begin{lstlisting}
z = 2 x + 3 x y
\end{lstlisting}
    \end{texinput}
    \hfill
    \begin{texoutput}{0.48\linewidth}
        $z = 2 x + 3 x y$
    \end{texoutput}
  \end{trivlist}
\item les espaces dans le code source n'ont aucun impact sur la
  disposition d'une équation.
  \begin{trivlist}
  \item
    \begin{texinput}{0.48\linewidth}
\begin{lstlisting}
z=2x + 3xy
\end{lstlisting}
    \end{texinput}
    \hfill
    \begin{texoutput}{0.48\linewidth}
      $z=2x + 3xy$
    \end{texoutput}
  \end{trivlist}
\end{enumerate}


% Pensez simplement
% à ce que pourrait être en mots la représentation de l'équation
% \begin{equation*}
%   \int_0^\infty \int_0^1 x_1^{\alpha + 1} x_2^{\beta + 1}\, dx_1 dx_2.
% \end{equation*}
% Cela ressemblerait sans doute à ceci (avec entre parenthèses les mots
% habituellement omis et en gras les symboles qu'il faut pouvoir décrire
% autrement que part un caractère disponible sur le clavier):
% \begin{quote}
%   \textbf{intégrale} de (indice) zéro à (exposant) l'\textbf{infini} \\
%   \textbf{intégrale} de (indice) zéro à (exposant) à un \\
%   $x$ (indice) un, puissance \textbf{alpha} plus un \\
%   $x$ (indice) deux, puissance \textbf{beta} plus un \\
%   $d$ $x$ (indice) un \\
%   $d$ $x$ (indice) deux.
% \end{quote}

Quant au langage retenu par {\LaTeX} pour décrire les équations
mathématiques, il est très similaire à celui que l'on utiliserait pour
le faire à voix haute; nous y reviendrons. Il faut simplement avoir
recours à des commandes pour identifier les symboles mathématiques que
l'on ne retrouve pas sur un clavier usuel, comme les lettres grecques,
les opérateurs d'inégalité ou les symboles de sommes et d'intégrales.


\section{Un paquetage incontournable}
\label{sec:math:amsmath}

Le paquetage \pkg{amsmath} produit par la prestigieuse \emph{American
  Mathematical Society} fournit diverses extensions à {\LaTeX} pour
faciliter encore davantage la saisie d'équations mathématiques
complexes et en améliorer le rendu. L'utilisation de ce paquetage doit
être considérée incontournable pour tout document contenant plus que
quelques équations très simples.

Au chapitre des améliorations fournies par \pkg{amsmath}, notons
particulièrement:
\begin{itemize}
\item plusieurs environnements pour les équations hors paragraphe, en
  particulier pour les équations multi-lignes;
\item une meilleure gestion de l'espacement autour des signes
  d'égalité dans les équations multi-lignes;
\item une commande pour faciliter l'entrée de texte à l'intérieur du
  mode mathématique;
\item un environnement pour la saisie des matrices et des coefficients
  binomiaux;
\item des commandes pour les intégrales multiples;
\item la possibilité de définir de nouveaux opérateurs mathématiques.
\end{itemize}
Nous décrivons certaines de ces fonctionnalités dans la suite, mais
l'utilisateur le moindrement avancé devrait impérativement consulter
la %
\doc[documentation complète]{amsldoc}{http://texdoc.net/pkg/amsmath}
du paquetage.


\section{Principaux éléments du mode mathématique}
\label{sec:math:bases}

\begin{equation*}
  \frac{\Gamma(\alpha)}{\lambda^\alpha} =
  \sum_{j = 0}^\infty \int_j^{j + 1} x^{\alpha - 1} e^{-\lambda x}\,
  dx,
  \quad
  \alpha > 0 \text{ et } \lambda > 0.
\end{equation*}

\subsection{Exposants et indices}
\label{sec:math:bases:exposants}

{\LaTeX} permet de créer facilement et avec la bonne taille de
symboles n'importe quelle combinaison d'exposants et d'indices.

\begin{itemize}
\item On place un caractère en exposant d'un autre avec la commande
  \verb=^= et en indice avec la commande \verb=_=. Les indices et
  exposants se combinent naturellement.
  \begin{trivlist}
  \item
    \begin{texinput}{0.2\linewidth}
\begin{lstlisting}
x^2
\end{lstlisting}
    \end{texinput}
    \quad
    \begin{texoutput}{0.1\linewidth}
      $x^2$
    \end{texoutput}
    \hfill
    \begin{texinput}{0.2\linewidth}
\begin{lstlisting}
a_n
\end{lstlisting}
    \end{texinput}
    \quad
    \begin{texoutput}{0.1\linewidth}
      $a_n$
    \end{texoutput}
    \hfill
    \begin{texinput}{0.2\linewidth}
\begin{lstlisting}
x_i^k
\end{lstlisting}
    \end{texinput}
    \quad
    \begin{texoutput}{0.1\linewidth}
      $x_i^k$
    \end{texoutput}
  \end{trivlist}
  (L'ordre de saisie n'a pas d'importance; le troisième exemple
  donnerait le même résultat avec \verb=x^k_i=.)
\item Si l'exposant ou l'indice compte plus d'un caractère, il faut
  regrouper le tout entre accolades \verb={ }=.
  \begin{trivlist}
  \item
    \begin{texinput}{0.2\linewidth}
\begin{lstlisting}
x^{2k + 1}
\end{lstlisting}
    \end{texinput}
    \quad
    \begin{texoutput}{0.1\linewidth}
      $x^{2k + 1}$
    \end{texoutput}
    \hfill
    \begin{texinput}{0.2\linewidth}
\begin{lstlisting}
x_{i,j}
\end{lstlisting}
    \end{texinput}
    \quad
    \begin{texoutput}{0.1\linewidth}
      $x_{i,j}$
    \end{texoutput}
    \hfill
    \begin{texinput}{0.2\linewidth}
\begin{lstlisting}
x_{ij}^{2n}
\end{lstlisting}
    \end{texinput}
    \quad
    \begin{texoutput}{0.1\linewidth}
      $x_{ij}^{2n}$
    \end{texoutput}
  \end{trivlist}
\item Toutes les combinaisons d'exposants et d'indices sont possibles,
  y compris les puissances de puissances ou les indices d'indices.
  \begin{trivlist}
  \item
    \begin{texinput}{0.2\linewidth}
\begin{lstlisting}
e^{-x^2}
\end{lstlisting}
    \end{texinput}
    \quad
    \begin{texoutput}{0.1\linewidth}
      $e^{-x^2}$
    \end{texoutput}
    \hfill
    \begin{texinput}{0.4\linewidth}
\begin{lstlisting}
A_{i_s, k^n}^{y_i}
\end{lstlisting}
    \end{texinput}
    \quad
    \begin{texoutput}{0.2\linewidth}
      $A_{i_s,k^n}^{y_i}$
    \end{texoutput}
    \quad \mbox{}     % pour alignement avec bloc d'exemples ci-dessus
  \end{trivlist}
\end{itemize}

\begin{important}
  Les commandes \verb=^= et \verb=_= sont permises dans le mode
  mathématique seulement. En fait, si {\TeX} rencontre l'une de ces
  commandes en mode texte, il tentera automatiquement de passer au
  mode mathématique après avoir émis l'avertissement
\begin{verbatim}
! Missing $ inserted.
\end{verbatim}
  Il est assez rare que le résultat soit celui souhaité.
\end{important}

\subsection{Fractions}
\label{sec:math:bases:fractions}

Il y a plusieurs façons de représenter une fraction selon qu'elle se
trouve au fil du texte, dans une équation hors paragraphe ou à
l'intérieur d'une autre fraction.
\begin{itemize}
\item Pour les fractions au fil du texte, il vaut souvent mieux
  utiliser simplement la barre oblique \verb=/= pour séparer le
  numérateur du dénominateur, quitte à utiliser des parenthèses.
  Ainsi, on utilise \verb=$(n + 1)/2$= pour obtenir $(n + 1)/2$.
\item De manière plus générale, la commande
\begin{lstlisting}
\frac`\marg{numérateur}\marg{dénominateur}'
\end{lstlisting}
  dispose \meta{numérateur} au-dessus de \meta{dénominateur}, séparé
  par une ligne horizontale. La taille des caractères s'ajuste
  automatiquement selon que la fraction se trouve au fil
  du texte ou dans une équation hors paragraphe, ainsi que selon la
  position de la fraction dans l'équation.
  \begin{trivlist}
  \item
    \begin{texinput}{0.48\linewidth}
\begin{lstlisting}
% taille au fil du texte
On a $z_1 = \frac{x}{y}$ et
$z_2 = xy$.
\end{lstlisting}
    \end{texinput}
    \hfill
    \begin{texoutput}[c]{0.48\linewidth}
      On a $z_1 = \frac{x}{y}$ et $z_2 = xy$.
    \end{texoutput}
  \item
    \begin{texinput}{0.48\linewidth}
\begin{lstlisting}
% taille hors paragraphe
On a
\begin{equation*}
  z_1 = \frac{x}{y}
\end{equation*}
et $z_2 = xy$.
\end{lstlisting}
    \end{texinput}
    \hfill
    \begin{texoutput}[c]{0.48\linewidth}
      On a
      \begin{equation*}
        z_1 = \frac{x}{y}
      \end{equation*}
      et $z_2 = xy$.
    \end{texoutput}
  \item
    \begin{texinput}{0.48\linewidth}
\begin{lstlisting}
% deux tailles combinées
Soit
\begin{equation*}
  z = \frac{\frac{x}{2}
    + 1}{y}.
\end{equation*}
\end{lstlisting}
    \end{texinput}
    \hfill
    \begin{texoutput}[c]{0.48\linewidth}
      Soit
      \begin{equation*}
        z = \frac{\frac{x}{2} + 1}{y}.
      \end{equation*}
    \end{texoutput}
  \end{trivlist}
\item Les commandes
\begin{lstlisting}
\dfrac`\marg{numérateur}\marg{dénominateur}'
\tfrac`\marg{numérateur}\marg{dénominateur}'
\end{lstlisting}
  de \pkg{amsmath} permettent de forcer une fraction à adopter la
  taille d'une fraction hors paragraphe (\emph{displayed}) dans le cas
  de \cmd{\dfrac} ou de celle d'une fraction au fil du texte
  (\emph{text}) dans le cas de \cmd{\tfrac}.
  %% quelque chose à cet effet dans exercices ou dans présentation de 'cases'
\item Il est parfois visuellement plus intéressant, surtout au fil du
  texte, d'écrire une fraction comme $1/x$ sous le forme $x^{-1}$.
\end{itemize}

\subsection{Racines}
\label{sec:math:bases:racines}

La commande
\begin{lstlisting}
\sqrt`\oarg{n}\marg{radicande}'
\end{lstlisting}
construit un symbole de radical autour de \meta{radicande}, par défaut
la racine carrée. Si l'argument optionnel \meta{n} est spécifié, c'est
plutôt un symbole de racine d'ordre $n$ qui est tracé. La longueur et
la hauteur du radical s'adapte toujours à celles du radicande.
\begin{trivlist}
\item
  \begin{texinput}{0.2\linewidth}
\begin{lstlisting}
\sqrt{2}
\end{lstlisting}
  \end{texinput}
  \quad
  \begin{texoutput}{0.1\linewidth}
    $\sqrt{2}$
  \end{texoutput}
  \hfill
  \begin{texinput}{0.2\linewidth}
\begin{lstlisting}
\sqrt{625}
\end{lstlisting}
  \end{texinput}
  \quad
  \begin{texoutput}{0.1\linewidth}
    $\sqrt{625}$
  \end{texoutput}
  \hfill
  \begin{texinput}{0.2\linewidth}
\begin{lstlisting}
\sqrt[3]{8}
\end{lstlisting}
  \end{texinput}
  \quad
  \begin{texoutput}{0.1\linewidth}
    $\sqrt[3]{8}$
  \end{texoutput}
\item
  \begin{texinput}{0.48\linewidth}
\begin{lstlisting}
\sqrt[n]{x + y + z}
\end{lstlisting}
  \end{texinput}
  \hfill
  \begin{texoutput}{0.48\linewidth}
    $\sqrt[n]{x + y + z}$
  \end{texoutput}
\item
  \begin{texinput}{0.48\linewidth}
\begin{lstlisting}
\sqrt{\frac{x + y}{x^2 - y^2}}
\end{lstlisting}
  \end{texinput}
  \hfill
  \begin{texoutput}[c]{0.48\linewidth}
    $\displaystyle \sqrt{\frac{x + y}{x^2 - y^2}}$
  \end{texoutput}
\end{trivlist}

\subsection{Sommes et intégrales}
\label{sec:math:bases:sommes-et-integrales}

Les sommes et intégrales requièrent un symbole spécial ainsi que des
limites inférieures et supérieures, le cas échéant.
\begin{itemize}
\item Les commandes \cmd{\sum} et \cmd{\int} servent respectivement à tracer les
  symboles de somme $\sum$ et d'intégrale $\int$.
\item On entre les éventuelles limites inférieures et supérieures
  comme des indices et des exposants.
  \begin{trivlist}
  \item
    \begin{texinput}{0.48\linewidth}
\begin{lstlisting}
\sum_{i = 0}^n x_i
\end{lstlisting}
    \end{texinput}
    \hfill
    \begin{texoutput}{0.48\linewidth}
      $\displaystyle \sum_{i = 0}^n x_i$
    \end{texoutput}
  \item
    \begin{texinput}{0.48\linewidth}
\begin{lstlisting}
\int_0^{10} f(x)\, dx
\end{lstlisting}
    \end{texinput}
    \hfill
    \begin{texoutput}{0.48\linewidth}
      $\displaystyle \int_0^{10} f(x)\, dx$
    \end{texoutput}
  \end{trivlist}
\item La taille des symboles et la position des limites s'ajustent
  automatiquement selon le contexte. Au fil du texte, la somme et
  l'intégrale ci-dessus apparaîtraient comme $\sum_{i = 0}^n x_i$ et
  $\int_0^{10} f(x)\, dx$.
\item Dans une intégrale il est recommandé de séparer l'intégrande de
  l'opérateur de différentiation $dx$ par une espace fine. C'est ce à
  quoi sert la commande \cmd{\,} ci-dessus.
\end{itemize}


\subsection{Texte et espaces}
\label{sec:math:bases:texte}



symboles

suites d'équations
  left ... right
  numérotation

vecteurs et matrices

%%%
%%% Exercices
%%%

\section{Exercices}
\label{sec:math:exercices}

\Opensolutionfile{solutions}[solutions-mathematiques]

\begin{Filesave}{solutions}
\section*{Chapitre \ref*{chap:math}}
\addcontentsline{toc}{section}{Chapitre \protect\ref*{chap:math}}

\end{Filesave}



\Closesolutionfile{solutions}

%%% Local Variables:
%%% mode: latex
%%% TeX-engine: xetex
%%% TeX-master: "formation_latex-partie_2"
%%% coding: utf-8
%%% End:

\chapter{Bibliographie et citations}
\label{chap:bibliographie}

\nobibliography*

La production de la bibliographie d'un ouvrage d'une certaine ampleur
--- qu'il s'agisse d'un article scientifique, d'un mémoire, d'une
thèse --- est une tâche d'une grande importance qui peut rapidement devenir
laborieuse\dots\ lorsqu'elle n'est pas réalisée avec les outils appropriés.

L'ordinateur est bien meilleur qu'un humain pour accomplir certaines
opérations propres à la production d'une bibliographie. Un auteur ne
devrait se préoccuper que de colliger les informations
bibliographiques, puis de sélectionner les ouvrages à citer. La
machine peut ensuite se charger:
\begin{itemize}
\item d'inclure dans la bibliographie tous les ouvrages cités dans le
  document et seulement ceux-ci;
\item de trier les entrées de la bibliographie;
\item de composer les entrées de manière uniforme;
\item de recommencer ces opérations autant de fois que nécessaire pour
  un même document ou pour chaque nouveau document.
\end{itemize}
Avec en main une base de données bibliographique, la création de la
bibliographie devient une tâche triviale qui ne prend guère plus que
les quelques secondes de compilation nécessaires pour la composer.


\section{Quel système utiliser?}
\label{sec:bibliographie:systeme}

La gestion des citations et la composition d'une bibliogrpahie sont
des tâches hautement spécialisées. Comme la plupart des traitements de
texte, {\LaTeX} les confie donc à des outils externes.

%% hack ici pour afficher le logo BibTeX en police sans serif qui n'a
%% pas de version petites capitales tout en utilisant la commande
%% usuelle dans la table des matières
\subsection[{\BibTeX} et \pkg{natbib}]{%
  {B\kern-.025em{\small I}\kern-.025em{\small  B}\kern-.08em\TeX} %
  et \pkg{natbib}}
\label{sec:bibliographie:systeme:bibtex}

Avec plus de 25 années d'utilisation, {\BibTeX} \citep{bibtex} est le
système standard de traitement des bibliographies dans {\LaTeX}. Il
est stable et prévisible --- ce que d'aucuns considéreraient des
bogues passent pour des caractéristiques --- et, surtout, il existe un
vaste catalogue de références bibliographiques en format {\BibTeX}.
C'est généralement le seul format accepté par les revues
scientifiques. Non qu'il s'agisse d'un argument massue, mais même
Wikipedia, dans les rubriques «Citer cette page», offre les citations
en format {\BibTeX}.

{\BibTeX} est principalement un système de tri d'entrées
bibliographiques et d'interface avec la base de données. Contrôler
l'apparence des citations et de la bibliographie requiert un
\emph{style}. Les styles standards sont \code{plain}, \code{unsrt},
\code{alpha} et \code{abbrv}; nous y reviendrons. \textbf{*******}

Fonctionnant de pair --- et exclusivement --- avec {\BibTeX},
\pkg{natbib} \citep{natbib} est un paquetage qui fournit des styles et
des commandes pour composer des bibliographies dans le format
auteur-année\footnote{%
  Comme on peut le voir dans cette phrase, c'est le style utilisé dans
  le présent document.} %
fréquemment utilisé dans les sciences naturelles et sociales. Il est
également compatible avec les styles de citation standards mentionnés
ci-dessus.

Parce qu'il est flexible et qu'il rend facile de produire des
extensions compatibles, \pkg{natbib} est en quelque sorte devenu un
standard \emph{de facto} pour la composition des bibliogrpahies.
D'ailleurs, la classe \class{ulthese} pour les thèses et mémoires de
l'Université Laval charge par défaut le paquetage.

Il existe plusieurs autres paquetages pour rencontrer des exigences
particulières avec {\BibTeX}: bibliographies multiples, bibliographies
par chapitre, etc. \citet{Mori:bibliographies:2009} en offre un bon
survol. Consulter aussi la section \emph{Bibliographies and citations}
de la formidable %
\doc[\emph{UK List of {\TeX} Frequently Asked
  Questions}]{letterfaq}{http://www.tex.ac.uk/}.


\subsection{Biber et biblatex}
\label{sec:bibliographie:systeme:biblatex}

Au moment d'écrire ces lignes, un nouveau système de traitement des
bibliographies dans {\LaTeX} est en émergence. Il est formé du moteur
de traitement Biber \citep{biber} et du paquetage \pkg{biblatex}
\citep{biblatex}. Ensemble, ils visent tout à la fois à remplacer
l'infrastructure bâtie autour de {\BibTeX} et à proposer des
fonctionnalités additionnelles. Citons le support natif des caractères
UTF-8 et de nombreux modes de citation, dont le mode
auteur-titre populaire en sciences humaines.

Le duo Biber-\pkg{biblatex} bénéficie d'un développement récent en
phase avec les technologies et les préoccupations actuelles. Certains
enjoignent aux débutants de sauter dans ce train. Difficile,
cependant, de dire si ce nouveau système saura s'établir comme nouveau
standard, surtout compte tenu de la masse de matériel disponible pour
{\BibTeX}.

Pour de l'information additionnelle, consulter
\doc[cette entrée]{}{http://tex.stackexchange.com/questions/25701/bibtex-vs-biber-and-biblatex-vs-natbib} %
du site \url{tex.stackexchange.com} qui fournit un excellent sommaire des
mérites et des inconvénients respectifs des deux systèmes de
traitement de bibliographie.

En l'absence d'un concensus clair, nous avons choisi de traiter dans
ce chapitre à la fois du système le plus répandu et de celui avec
lequel nous sommes le plus familier, soit la combinaison {\BibTeX} et
\pkg{natbib}.

\subsection{EndNote}
\label{sec:bibliographie:systeme:endnote}

EndNote est un logiciel commercial de gestion bibliographique très
répandu dans certaines disciplines scientifiques. Il n'est donc pas
rare que les nouveaux utilisateurs de {\LaTeX} demandent: «puis-je
utiliser EndNote pour ma bibliographie?» La réponse courte est «Non»,
car {\LaTeX} ne peut traiter directement les données bibliographiques
de EndNote. La réponse plus longue est «Oui, mais pas directement»,
car EndNote possède un filtre pour exporter ses données en format
{\BibTeX}.

Il est hors de la portée de ce document de traiter de la conversion
des données bibliographiques de EndNote. Une simple recherche dans
Internet sur «EndNote BibTeX» devrait fournir toute l'information
nécessaire pour réaliser la conversion.



\section{Création d'une base de données}
\label{sec:bibliographie:bib}

% Il est tout à fait possible de citer des références et de construire
% une bibliographie avec {\LaTeX} sans avoir recours à une base de
% données bibliographiques et à {\BibTeX} pour traiter celles-ci.
% L'investissement requis en temps et en efforts pour adopter l'approche
% {\BibTeX} demeure toutefois faible, surtout au regard des avantages:
% \begin{itemize}
% \item
% \end{itemize}

L'utilisation d'une base de données bibliographiques offre
d'importants avantages par rapport à un traitement manuel:
\begin{itemize}
\item on y entre les informations une seule fois pour ensuite les
  utiliser à l'envi;
\item le traitement automatisé des informations assure une
  présentation uniforme de celles-ci;
\item on peut changer le style de présentation de la bibliographie
  sans pour autant toucher aux informations bibliographiques.
\end{itemize}

La base de données n'est en fait qu'un simple fichier texte dans
lequel sont regroupées dans un format précis les informations
bibliographiques. Le nom du fichier doit nécessairement comporter
l'extension \code{.bib}.

La base de données est composée d'entrées de divers \emph{types}:
livre, article scientifique, thèse, etc. Chaque entrée comporte un
certain nombre de \emph{champs}: titre, nom de l'auteur, date de
publication, etc. Pour un type d'entrée donné, certains champs sont
obligatoires, d'autres optionnels et d'autres simplement ignorés ou
inactifs.

La structure générale d'une entrée de base de données est la suivante:
\begin{lstlisting}
@`\meta{type\_entree}'{`\meta{clé}',
  `\marg{champs}' = `\marg{valeur}',
  ...,
  `\marg{champs}' = `\marg{valeur}'
}
\end{lstlisting}
Ci-dessus, \meta{clé} est un identifiant arbitraire, mais unique ---
et idéalement mnémonique --- de l'entrée. C'est cette clé qui sera
utilisée pour faire référence à l'entrée dans le code source du
document.

Voyons immédiatement quelques exemples pour mieux comprendre ce dont
il est question. On trouvera ci-dessous les entrées bibliographiques
d'un livre \citep{Kopka:latex:4e}, d'un article scientifique
\citep{Mori:bibliographies:2009} et d'un manuel générique, en
l'occurence la documentation d'un paquetage \citep{natbib}. Pour
faciliter la comparaison, chaque entrée est immédiatement suivie du
texte tel qu'il apparait dans la bibliographie.

\begin{lstlisting}
@Book{Kopka:latex:4e,
  author = 	 {Kopka, Helmut and Daly, Patrick W.},
  title = 	 {Guide to {\LaTeX}},
  publisher = 	 {Addison-Wesley},
  year = 	 2003,
  edition = 	 4,
  isbn = 	 {978-0321173850},
  language = 	 {english}
}
\end{lstlisting}

\begin{framed}
  \noindent \nolink{\bibentry{Kopka:latex:4e}}.
\end{framed}

\begin{lstlisting}
@Article{Mori:bibliographies:2009,
  author = 	 {Lapo F. Mori},
  title = 	 {Managing bibliographies with {\LaTeX}},
  journal = 	 {{TUG}boat},
  year = 	 2009,
  volume = 	 30,
  number = 	 1,
  pages = 	 {36-48},
  url =		 {https://www.tug.org/TUGboat/tb30-1/tb94mori.pdf},
  language = 	 {english}
}
\end{lstlisting}

\begin{framed}
\noindent \nolink{\bibentry{Mori:bibliographies:2009}}.
\end{framed}

\begin{lstlisting}
@Manual{natbib,
  author = 	 {Patrick W. Daly},
  title = 	 {Natural Sciences Citations and References},
  year = 	 2010,
  url = 	 {http://www.ctan.org/pkg/natbib/},
  language = 	 {english}
}
\end{lstlisting}

\begin{framed}
\noindent \nolink{\bibentry{natbib}}.
\end{framed}

\begin{itemize}
\item Les types d'entrées dans les exemples ci-dessus sont
  \code{Manual}, \code{Article} et \code{Manual}\footnote{%
    Les identifiants des types d'entrée et des champs sont insensibles
    à la casse. Par exemple, on pourrait tout aussi bien débuter une
    entrée par \verb=@Manual=, \verb=@manual= ou \verb=@MANUAL=.}. %
  On remarquera que les champs utilisés sont différents d'un type à un
  autre. On trouvera la liste de tous les types d'entrées et es champs
  obligatoires et optionnels pour chacun dans \citet{wikipedia:bibtex}.
\item On entre le nom d'un auteur soit sous la forme \code{Prénom
    Nom}, soit sous le forme \code{Nom, Prénom}. La seconde forme est
  surtout utile pour distinguer explicitement le nom du prénom, par
  exemple dans le cas de prénoms ou de noms multiples.
\item Lorsqu'un ouvrage compte plus d'un auteur, on sépare le nom
  complet de chacun des auteurs par \code{and}.
\item {\BibTeX} gère normalement automatiquement les hauts et bas de
  casse (majuscules et minuscules), en particulier dans les titres
  d'ouvrages. Pour préserver une casse particulière, il suffit de
  placer les lettres entre accolades; voir le  dans le titre de
  journal TUGboat ci-dessus.
\item accents
\item
\item champs isbn, url et language propres à natbib
\item
\end{itemize}




\begin{conseil}
  Tenter d'entretenir une grande base de données bibliographiques
  unique peut rapidement devenir pénible.

  Vaut mieux scinder ses références dans plusieurs fichiers par
  thématique, un peu comme dans une bibliothèque: informatique,
  mathématiques, statistique, droit, etc.

  Nommer ensuite les fichiers du nom de la thématique:
  \code{informatique.bib}, \code{droit.bib}, etc.
\end{conseil}

3. base de données
   - concepts de base et lien vers ressources complètes
   - style fr
4. besoins particuliers




\section{Insertion de références dans le texte}
\label{sec:bibliographie:cite}

La raison première d'une bibliographie, c'est évidemment d'y colliger
les informations relatives aux ouvrages auxquels un document fait
référence. Avant de penser créer une bibliographie, il faut donc
savoir comment insérer des références dans le texte.

D'ici la \autoref{sec:bibliographie:bib}, il faut prendre pour acquis
que l'on dispose d'une base de données bibliographiques et qu'à chaque
entrée de cette base de données correspond une \emph{clé} arbitraire,
mais unique (et idéalement mnémonique). Par exemple, dans la base de
données employée pour ce document, la référence bibliographique de
\citet{Mori:bibliographies:2009} est identifiée par la clé
\code{Mori:bibliographies:2009}.

La commande de base pour insérer une référence au fil du texte dans
{\LaTeX} est
\begin{lstlisting}
\cite`\marg{clé}'
\end{lstlisting}
L'effet de la commande est double:
\begin{enumerate}
\item insérer une référence comme «\nolink{\citet{Mori:bibliographies:2009}}»
  dans le texte;
\item ajouter le document dans la bibliographie.
\end{enumerate}
En somme, outre la phase de compilation \textbf{****}, c'est tout ce
qu'il y a à faire pour construire sa bibliographie.

Avec \pkg{natbib}, on utilisera plutôt les commandes
\begin{lstlisting}
\citet`\marg{clé}'
\citep`\marg{clé}'
\end{lstlisting}
Dans le style de citation auteur-année, ces commandes permettent
respectivement d'insérer une référence au fil de la phrase ou en
aparté:
\begin{demo}
  \begin{texample}[0.55\linewidth]
\begin{lstlisting}
\citet{Mori:bibliographies:2009}
en offre un bon survol.
\end{lstlisting}
    \producing
    \citet{Mori:bibliographies:2009},
    en offre un bon survol.
  \end{texample}
  \begin{texample}[0.55\linewidth]
\begin{lstlisting}
TUGboat a publié un bon survol
\citep{Mori:bibliographies:2009}.
\end{lstlisting}
    \producing
    TUGboat a publié un bon survol
    \citep{Mori:bibliographies:2009}.
  \end{texample}
\end{demo}

On ne devrait \emph{jamais} entrer directement dans le texte des
informations bibliographiques. Par exemple, pour insérer dans le texte
le nom d'un auteur ou l'année de publication d'un ouvrage, on devrait
utiliser les commandes de \pkg{natbib}
\begin{lstlisting}
\citeauthor`\marg{clé}'
\citeyear`\marg{clé}'
\end{lstlisting}
Ainsi, pas de risque de mal orthographier un nom par inadvertance, ou
d'oublier de modifier dans le texte une année de publication qui aura
été changée dans la base de données bibliographique.

Le paquetage fournit plusieurs autres commandes pour manipuler les
informations bibliographiques et contrôler leur présentation. Il est
donc fortement recommandé de consulter la %
\doc{natbib}{http://www.texdoc.org/pkg/natbib} %
de \pkg{natbib}. On y trouvera également des informations sur
l'utilisation de styles de citation autres que auteur-année.

En terminant, il arrive que l'on souhaite inclure dans la
bibliographie un ou plusieurs documents qui ne sont pas cités dans le
texte. Pour ce faire, insérer dans le corps du document la commande
\begin{lstlisting}
\nocite`\marg{clé1,clé2,...}'
\end{lstlisting}
où \meta{clé1}, \meta{clé2}, \dots, sont les clés des documents à
inclure dans la bibliographie.

\begin{conseil}
  Lorsque chargé dans le document, le paquetage \pkg{hyperref} fait
  automatiquement d'une référence bibliographique un hyperlien vers
  l'entrée dans la bibliographie. C'est le cas dans le présent
  document.

  Il peut arriver que l'hyperlien soit superflu ou indésirable. Pour
  le supprimer pour une référence particulière, on utilise
  l'environnement \Ie{NoHyper}:
\begin{lstlisting}
\begin{NoHyper} \cite`\marg{clé}' \end{NoHyper}
\end{lstlisting}
  Pour usage fréquent, définir une nouvelle commande. Par exemple,
  avec dans le préambule
\begin{lstlisting}
\newcommand{\nolink}[1]{%
  \begin{NoHyper}#1\end{NoHyper}}
\end{lstlisting}
  on pourra utiliser dans le texte
\begin{lstlisting}
\nolink{\cite`\marg{clé}'}
\end{lstlisting}
\end{conseil}



\section{Création de la bibliographie}
\label{sec:bibliographie:bibtex}

Les commandes de la section précédente servent à indiquer à {\LaTeX}
les ouvrages à inclure dans la bibliographie. C'est toutefois l'outil
externe {\BibTeX} qui se chargera de fournir à {\LaTeX} le texte des
références ainsi que le contenu de la bibliographie.

On l'a vu dans la première partie de cette formation
\citep{UL:latex:1}, le processus de création d'un document avec
pdf{\LaTeX} ou {\XeLaTeX} se représente schématiquement ainsi:
\begin{center}
  \sffamily
  \begin{minipage}[t]{0.12\linewidth}
    \centering
    {\LARGE\faFileTextO} \\ \medskip
    code source
  \end{minipage}
  \quad\faArrowRight\quad
  \begin{minipage}[t]{0.12\linewidth}
    \centering
    {\LARGE\faCogs} \\ \medskip
    \code{pdflatex} \\ \code{xelatex}
  \end{minipage}
  \quad\faArrowRight\quad
  \begin{minipage}[t]{0.12\linewidth}
    \centering
    {\LARGE\faFilePdfO} \\ \medskip
    fichier PDF
  \end{minipage}
\end{center}

Pour créer ou mettre à jour la bibliographie, il s'ajoute au processus
une étape de compilation du document avec {\BibTeX}:
\begin{center}
  \sffamily
  \begin{minipage}[t]{0.12\linewidth}
    \centering
    {\LARGE\faFileTextO} \\ \medskip
    code source
  \end{minipage}
  \quad\faArrowRight\quad
  \begin{minipage}[t]{0.12\linewidth}
    \centering
    {\LARGE\faCogs} \\ \medskip
    \code{pdflatex} \\ \code{xelatex}
  \end{minipage}
  \quad\faArrowRight\quad
  \begin{minipage}[t]{0.12\linewidth}
    \centering
    {\LARGE\faCogs} \\ \medskip
    \code{bibtex}
  \end{minipage}
  \quad\faArrowRight\quad
  \begin{minipage}[t]{0.12\linewidth}
    \centering
    {\LARGE\faCogs}\;
    \raisebox{-2pt}{\parbox[b]{1em}{\centering\large\faRepeat\tiny\\ $\times 2$}} \\ \medskip
    \code{pdflatex} \\ \code{xelatex}
  \end{minipage}
  \quad\faArrowRight\quad
  \begin{minipage}[t]{0.12\linewidth}
    \centering
    {\LARGE\faFilePdfO} \\ \medskip
    fichier PDF
  \end{minipage}
\end{center}

Plus en détails, le processus de préparation d'un document comprenant
une bibliographie est le suivant.

\begin{enumerate}
\item Composer le texte et y insérer des références avec les commandes
  de la section précédente.
\item Ajouter dans le texte la commande
\begin{lstlisting}
\bibliography`\marg{base\_donnees1, base\_donnees2, ...}'
\end{lstlisting}
  à l'endroit où l'on veut qu'apparaisse la bibliographie
  (généralement à la fin du document). Les arguments
  \meta{base\_donnees1}, \meta{base\_donnees2}, séparés par des
  virgules, sont les noms (sans l'extension \code{.bib}) des fichiers
  de données bibliographiques.
\item Ajouter dans le préambule ou près de la commande
  \cmdprint{\bibliography} ci-dessus la commande
\begin{lstlisting}
\bibliographystyle`\marg{style}'
\end{lstlisting}
  où \meta{style} est un nom de style bibliographique.
\item \label{item:bibliographie:1} Compiler le document une première
  fois avec un moteur {\TeX} afin que {\LaTeX} détecte les ouvrages à
  insérer dans la bibliographie. À cette étape, les références dans le
  texte apparaissent sous forme d'un point d'interrogation
  «\textbf{?}».
\item Compiler le document avec {\BibTeX} afin de préparer le texte
  des références et composer la bibliographie.
\item \label{item:bibliographie:2} Compiler à nouveau le document au
  moins deux fois avec un moteur {\TeX} afin d'y insérer d'abord la
  bibliographie, puis le texte des références.
\end{enumerate}

Il faut répéter les étapes
\ref*{item:bibliographie:1}--\ref*{item:bibliographie:2} chaque fois
qu'une \emph{nouvelle} référence est ajoutée dans le document.
Autrement, tant que la bibliographie demeure inchangée, une
compilation standard avec seulement le moteur {\TeX} suffit.

Les éditeurs de texte spécialisés pour la préparation de documents
{\LaTeX} offrent généralement des raccourcis pour exécuter la
compilation avec {\BibTeX}. À titre d'exemple:
\begin{itemize}
\item dans TeXShop, on sélectionne un autre programme dans le menu à
  côté du bouton «Composition»;
\item dans Texmaker, on choisit le programme approprié dans le menu de
  composition rapide.
\end{itemize}
Ces deux interfaces sont représentées à la
\autoref{fig:bibliographie:editeurs}.

\begin{figure}
  \centering
  \includegraphics[height=2.2cm]{bibtex-texshop}
  \qquad
  \includegraphics[height=2.2cm]{bibtex-texmaker}
  \caption{Interfaces de sélection du programme {\BibTeX} dans TeXShop
    (à gauche)
    et Texmaker (à droite)}
  \label{fig:bibliographie:editeurs}
\end{figure}

\begin{conseil}
  Aux toutes dernières étapes avant de rendre un document, s'assurer
  d'exécuter {\BibTeX} une dernière fois et de compiler avec
  pdf{\LaTeX} ou {\XeLaTeX} au moins deux fois. Le journal de la
  compilation (\emph{log file}) ne devrait pas rapporter de références
  manquantes (\emph{undefined references}).
\end{conseil}

Infos additionnelles dans section \href{Gestion de la
  bibliographie}{https://fr.wikibooks.org/wiki/LaTeX/Gestion_de_la_bibliographie}
de \citet{wikilivres:latex}


%%%
%%% Exercices
%%%

\section{Exercices}
\label{sec:bibliographie:exercices}

\Opensolutionfile{solutions}[solutions-bibliographie]

\begin{Filesave}{solutions}
\section*{Chapitre \ref*{chap:bibliographie}}
\addcontentsline{toc}{section}{Chapitre \protect\ref*{chap:bibliographie}}

\end{Filesave}

\Closesolutionfile{solutions}

%%% Local Variables:
%%% mode: latex
%%% TeX-engine: xetex
%%% TeX-master: "formation_latex-partie_2"
%%% coding: utf-8
%%% End:

\chapter{Commandes et environnements définis par l'usager}
\label{chap:commandes}

{\LaTeX} est un ensemble de macro commandes conçu pour faciliter
l'utilisation du système {\TeX}. Dès lors, il est assez naturel de
permettre à l'usager de définir à son tour ses propres commandes. Il
suffit généralement d'avoir rédigé quelques documents --- ou quelques
chapitres d'un long document --- avec {\LaTeX} pour réaliser combien
cette possibilité est de nature à faciliter le travail.

La définition de nouvelles commandes et de nouveaux environnements
peut servir à créer des extensions à {\LaTeX} --- c'est d'ailleurs ce
que font plusieurs paquetages. Cependant, en usage courant, on fera
principalement appel à ces fonctionnalités pour l'une ou l'autre des
trois raisons suivantes:

\begin{enumerate}
\item créer des raccourcis pour de longues commandes utilisées
  fréquemment;
\item créer des commandes sémantiques afin d'uniformiser la
  présentation du texte;
\item modifier le comportement de commandes existantes, puisqu'il est
  également possible de redéfinir une commande existante.
\end{enumerate}

\begin{exemple}
  \label{ex:commandes:intro}
  Nous avons défini ou redéfini des commandes pour chacune des raisons
  ci-dessus dans la préparation du présent document.
  \begin{enumerate}
  \item Une nouvelle commande \cmdprint{\doc} facilite et systématise
    l'insertion de liens vers la documentation. D'un seul appel, elle
    crée un hyperlien dans le texte suivi de l'icône {\faExternalLink}
    et elle ajoute le nom du fichier de documentation dans la marge
    suivi de l'icône {\faBook}.
  \item Une nouvelle commande sémantique \cmdprint{\pkg} sert pour la
    mise en forme des noms de paquetages. Ainsi, leur présentation est
    toujours la même et si nous souhaitons en changer, il suffit de
    modifier la définition de la commande.
  \item La redéfinition de la commande \cmd{\chaptitlefont} de la
    classe \class{memoir} permet de modifier la police de caractère et
    la mise en forme des titres de chapitres.
  \end{enumerate}
  Nous reviendrons sur les détails de ces exemples dans la suite du
  chapitre. %
  \qed
\end{exemple}



\section{Nouvelles commandes}
\label{sec:commandes:commandes}

Les commandes \cmd{\newcommand} et \cmd{\renewcommand} permettent,
dans l'ordre, de définir une nouvelle commande ou de redéfinir ---
c'est-à-dire de modifier la définition --- d'une commande existante.
Ces deux commandes s'utilisent exclusivement dans le préambule du
document.

\subsection{Commandes sans arguments}
\label{sec:commandes:commandes:sans_arg}

Certaines commandes ne requièrent pas d'argument; pensons à
\cmdprint{\LaTeX} ou \cmdprint{\bfseries}. Ce sont les commandes les
plus simples à créer. La syntaxe des commandes \cmd{\newcommand} et
\cmd{\renewcommand} pour de tels cas est la suivante:
\begin{lstlisting}
\newcommand{`\bs\meta{nom\_commande}'}`\marg{définition}'
\renewcommand{`\bs\meta{nom\_commande}'}`\marg{définition}'
\end{lstlisting}
Le premier argument, \bs\meta{nom\_commmande}, est le nom, avec le
caractère \bs, de la commande. Ce nom doit être différent de celui de
toute commande active\footnote{%
  Les commandes actives dans un document sont les commandes de base de
  {\TeX} et {\LaTeX} ainsi que les commandes de tous les paquetages
  chargés dans le préambule.} %
dans le document dans le cas de \cmdprint{\newcommand}. À l'inverse,
une commande active doit nécessairement porter le même nom dans le cas
de \cmdprint{\renewcommand}.

Le second argument, \marg{définition}, contient la définition de la
commande. Il peut s'agir de caractères à insérer dans le texte, de
commandes à exécuter ou une combinaison de tout cela.

\begin{exemple}
  La commande \cmd{\mathbb}, présentée à la
  \autoref{sec:math::symboles:mathcal}, permet de créer une lettre
  majuscule ajourée pour représenter un ensemble de nombres en
  mathématiques. Plutôt que de l'utiliser à divers endroits dans un
  document, il est préférable de définir une commande sémantique comme
  \cmdprint{\R} pour représenter l'ensemble des nombres réels:
\begin{lstlisting}
\newcommand{\R}{\mathbb{R}}
\end{lstlisting}
  Ainsi, si l'on souhaite modifier la représentation de l'ensemble des
  nombres réels pour une raison quelconque, il suffit de modifier la
  définition de la commande \cmdprint{\R} pour que le changement
  prenne effet dans tout le document. %
  \qed
\end{exemple}

\begin{exemple}
  Tel que mentionné à l'\autoref{ex:commandes:intro}, nous avons
  modifié la police de caractère des titres de chapitres dans ce
  document en redéfinissant la commande \cmdprint{\chaptitlefont} de
  \class{memoir}. Pour obtenir des titres de chapitres sans
  empattements, en caractères gras, de dimension \cmdprint{\Huge} et
  alignés à gauche, on trouve dans le préambule du document la
  déclaration
\begin{lstlisting}
\renewcommand{\chaptitlefont}{\normalfont%
  \sffamily\bfseries\Huge\raggedright}
\end{lstlisting}
  La commande \cmd{\normalfont} est placée au début de la définition à
  titre préventif afin de «remettre à zéro» toutes les
  caractéristiques de police. %
  \qed
\end{exemple}


\subsection{Commandes avec arguments}
\label{sec:commandes:commandes:avec_arg}

Les commandes \cmdprint{\newcommand} et \cmdprint{\renewcommand} ont
d'autres tours dans leur sac. Leur syntaxe étendue permet également de
définir ou de redéfinir des commandes acceptant un ou plusieurs
arguments:
\begin{lstlisting}
\newcommand{`\bs\meta{nom\_commande}'}`\oarg{narg}\marg{définition}'
\renewcommand{`\bs\meta{nom\_commande}'}`\oarg{narg}\marg{définition}'
\end{lstlisting}
où \meta{narg} est un nombre entre $1$ et $9$ spécifiant le nombre
d'arguments de la commande. Dès lors, la \meta{définition} de la
commande devra contenir des jetons \code{\#1}, \code{\#2},
\dots\ pour identifier les endroits où les arguments $1$, $2$, \dots\
doivent apparaître.

\begin{exemple}
  La nouvelle commande \cmdprint{\pkg} mentionnée à
  l'\autoref{ex:commandes:intro} affiche les noms de paquetages en
  caractères gras. La commande prend en argument le nom du paquetage.
  Sa définition est donc
\begin{lstlisting}
\newcommand{\pkg}[1]{\textbf{#1}}
\end{lstlisting}
  Il s'agit encore d'une commande sémantique permettant de changer
  aisément la mise en forme en modifiant une seule définition dans le
  préambule du document. %
  \qed
\end{exemple}

\begin{exemple}
  La commande \cmdprint{\doc} mentionnée à
  l'\autoref{ex:commandes:intro} requiert trois arguments:
  \begin{enumerate}
  \item le texte de l'hyperlien qui sera placé au fil du texte;
  \item le nom du fichier de documentation à placer dans la marge dans
    une police de caractère non proportionnelle;
  \item l'URL vers le fichier de documentation en ligne.
  \end{enumerate}
  Une version simplifiée de la définition de la commande est la
  suivante:
\begin{lstlisting}
\newcommand{\doc}[3]{%
  \href{#3}{#1~\raisebox{-0.2ex}{\faExternalLink}}%
  \marginpar{\faBook~\texttt{#2}}}
\end{lstlisting}
  La commande \cmd{\href} qui permet d'insérer un hyperlien dans le
  texte provient du paquetage \pkg{hyperref} \citep{hyperref}. Les
  commandes \cmdprint{\faExternalLink} et \cmdprint{\faBook}
  insèrent dans le texte des icônes de la police libre Font~Awesome;
  elles proviennent du paquetage du même nom \citep{fontawesome}.

  Avec la définition ci-dessus, la déclaration
\begin{lstlisting}
\doc{documentation}{hyperref}{http://texdoc.net/pkg/hyperref}
\end{lstlisting}
  produit: \doc{hyperref}{http://texdoc.net/pkg/hyperref}. %
  \qed
\end{exemple}


\section{Nouveaux environnements}
\label{sec:commandes:environnements}

Tel que mentionné en introduction du chapitre, {\LaTeX} permet
également à l'utilisateur de définir ou de modifier des
environnements. La mécanique est similaire à celle de la définition de
commandes, sauf qu'un environnement compte trois parties: le début,
marqué par la déclaration \verb=\begin{\dots}=; le contenu en tant que
  tel; la fin, marquée par la déclaration \verb=\end{\dots}=.

On crée ou modifie des environnements avec les commandes
\begin{lstlisting}
\newenvironment`\marg{nom\_env}\oarg{narg}\marg{début\_déf}\marg{fin\_déf}'
\renewenvironment`\marg{nom\_env}\oarg{narg}\marg{début\_déf}\marg{fin\_déf}'
\end{lstlisting}
Les nombreux arguments sont les suivants:
\begin{description}
\item[\meta{nom\_env}] nom de l'environnement à créer ou à modifier.
  Il est fortement recommandé de ne pas modifier les environnements de
  base de {\LaTeX};
\item[\meta{narg}] un nombre entre $1$ et $9$ représentant le nombre
  d'arguments de l'environnement, lorsqu'il y en a. Les arguments sont
  utilisés de la même manière que dans les définitions de commandes;
\item[\meta{début\_déf}] commandes et texte à insérer au début de
  l'environnement, lors de l'appel
  \cs{begin\{}\meta{nom\_env}\code{\}}. C'est dans ce bloc que doivent
  se trouver les jetons \code{\#1}, \dots, \code{\#}\meta{narg}
  lorsque l'environnement a des arguments.
\item[\meta{début\_déf}] commandes et texte à insérer à la fin de
  l'environnement, lors de l'appel \cs{end\{}\meta{nom\_env}\code{\}}.
\end{description}

\begin{exemple}
  On souhaite mettre en forme les citations hors paragraphe de la
  manière suivante:
  \begin{quote}
    \small\itshape%
    en italique, dans une police de taille inférieure au texte
    normal et en retrait des marges gauche et droite.
  \end{quote}
  Pour ce faire, on construit un nouvel environnement \Ie{citation} à
  partir de l'environnement standard \Ie{quote} avec la commande
\begin{lstlisting}
\newenvironment{citation}%
  {\begin{citation}\small\itshape}%
  {\end{quote}}
\end{lstlisting}
  \qed
\end{exemple}

\begin{exemple}
  Nous avons créé pour les fins du présent document un environnement
  \Ie{conseil} servant à mettre en forme les rubriques «conseils du
  {\TeX}pert». La définition --- relativement élaborée --- de
  l'environnement est la suivante:
\begin{lstlisting}
\newenvironment{conseil}
  {\begin{table}[h]
     \begin{framed}
       \noindent
       \begin{minipage}{0.1\linewidth}
         \raisebox{-1.5em}[0em][0em]{\HUGE\faThumbsOUp}
       \end{minipage}
       \begin{minipage}[t]{0.88\linewidth}
         {\bfseries\sffamily Conseil du {\TeX}pert} \newline}%
  {\end{minipage}\end{framed}\end{table}}
\end{lstlisting}
  Dans le second argument:
  \begin{enumerate}
  \item on ouvre un élément flottant pour disposer la rubrique sur la
    page;
  \item on ouvre un environnement \Ie{framed} de la classe
    \class{memoir} pour encadrer la rubrique;
  \item on crée une première \Ie{minipage} pour disposer le symbole
    {\faThumbsOUp} tiré de la police Font~Awesome;
  \item on ouvre une seconde \Ie{minipage} à côté de la première, on y
    dispose le titre de la rubrique en police grasse et sans
    empattements, puis on insére un retour à la ligne pour préparer la
    place du texte de la rubrique.
  \end{enumerate}
  Le troisième argument sert à refermer tous les environnements
  ouverts dans le second argument et qui n'ont pas déjà été fermés.

  Une fois ce travail accompli, créer une nouvelle rubrique est très
  simple:
\begin{lstlisting}
\begin{conseil}
  N'hésitez pas à créer des nouvelles commandes...
\end{conseil}
\end{lstlisting}
  On trouvera le résultat ci-dessous. %
  \qed
\end{exemple}

\begin{conseil}
  N'hésitez pas à créer des nouvelles commandes et des nouveaux
  environnements dès lors qu'une mise en forme particulière revient
  plus d'une ou deux fois dans un document.
\end{conseil}


%%% Local Variables:
%%% mode: latex
%%% TeX-engine: xetex
%%% TeX-master: "formation_latex-partie_2"
%%% coding: utf-8
%%% End:

\chapter{Trucs et astuces divers}
\label{chap:trucs}


\section{Polices de caractères: au-delà de Computer Modern}
\label{sec:trucs:police}

{\fontfamily{cmr}\selectfont%
  Les documents {\LaTeX} standards sont facilement reconnaissables par
  leur police de caractères par défaut, celle utilisée dans ce
  paragraphe, Computer Modern. Pour qui souhaitait briser la relative
  monotonie induite par cette uniformité, il a longtemps été difficile
  d'utiliser une autre police de caractères. Fort heureusement, la
  situation a beaucoup évolué et il est aujourd'hui assez simple de de
  produire des documents {\LaTeX} utilisant des polices de caractères
  variées.}

Avant d'aller plus loin, une mise en garde: si votre document contient
plus que quelques équations mathématiques très simples, le choix de
police devient très restreint. La raison: peu de polices de caractères
comprennent des symboles mathématiques et les informations nécessaires
pour réaliser une mise en page des mathématiques satisfaisant les
standards de haute qualité usuels de {\LaTeX}.

Cela dit, pour qui souhaite aller au-delà de la police Computer Modern
sans trop se compliquer la vie, il existe deux solutions principales.

\begin{enumerate}
\item Utiliser l'une ou l'autre des polices PostScript standards.
  C'est très simple avec toute distribution {\LaTeX} moderne: il
  suffit de charger le paquetage approprié. Consulter la %
  \doc{psnfss2e}{http://texdoc.net/pkg/psnfss/} %
  de l'ensemble de paquetages PSNFSS pour connaître les choix
  disponibles.
\item Utiliser une police OpenType présente sur son système avec le
  moteur {\XeLaTeX}. Seule une poignée de ces polices offrent
  toutefois un support approprié pour les mathématiques. La gestion
  des polices de caractères avec {\XeLaTeX} se fait avec le paquetage
  standard \pkg{fontspec}; consulter sa %
  \doc{fontspec}{http://texdoc.net/pkg/fontspec/}.
\end{enumerate}

Pour les thèses et mémoires de l'Université Laval, la Faculté des
études supérieures et postdoctorales accepte les polices %
{\fontfamily{cmr}\selectfont Computer Modern}, %
{\fontfamily{ptm}\selectfont Times} et %
{\fontfamily{ppl}\selectfont Palatino}. Pour utiliser ces deux
dernières avec {\LaTeX}, on charge respectivement les paquetages
\pkg{mathptmx} ou \pkg{mathpazo}. Avec {\XeLaTeX}, on utilisera les
polices {\fontfamily{qtm}\selectfont Termes} et
{\fontfamily{qpl}\selectfont Pagella} du projet %
\link{http://www.gust.org.pl/projects/e-foundry/tex-gyre/}{TeX Gyre}.
Ce sont des polices très similaires à Times et Palatino, disponibles
en version OpenType et qui fournissent un bon support pour les
mathématiques via le projet frère %
\link{http://www.gust.org.pl/projects/e-foundry/tg-math/}{TeX Gyre Math}.

Le texte principal du présent document est en %
\link{http://tug.org/store/lucida/}{Lucida Bright~OT}, %
une police commerciale de très haute qualité, offrant un excellent
support pour les équations mathématiques et dont les auteurs ont
toujours été des «amis» de la communauté {\LaTeX}. La Bibilothèque de
l'Université Laval détient une licence d'utilisation de cette police.
Les membres de la communauté --- étudiants et personnel --- peuvent
s'en produrer une copie en écrivant à
\href{mailto:lucida@bibl.ulaval.ca}{lucida@bibl.ulaval.ca}.

Hyperliens: configuration, infos du document

Présentation de code informatique

Analyse et rapport intégrés [de type Sweave]

Diapositives

Contrôle de version

%%% Local Variables:
%%% mode: latex
%%% TeX-engine: xetex
%%% TeX-master: "formation_latex-partie_2"
%%% coding: utf-8
%%% End:


\appendix
\chapter{Solutions des exercices}
\label{chap:solutions}

\input{solutions-boites}
\input{solutions-tableaux+figures}
\input{solutions-mathematiques}

%%% Local Variables:
%%% mode: latex
%%% TeX-master: "formation_latex-partie_2"
%%% coding: utf-8
%%% End:


\bibliography{formation_latex_UL}

\cleardoublepage
\printindex

\cleartoverso

\input{colophon-partie_2}

\cleartoverso

%% Page couverture arrière.
\AddToShipoutPictureFG*{%
  \AtStockLowerLeft{\raisebox{45mm}{\hspace{5mm}\includegraphics{codebarre_\ISBN}}}
}
\includepdf[pages=2]{couvertures-partie_2}

\end{document}

%%% Local Variables:
%%% mode: latex
%%% TeX-engine: xetex
%%% TeX-master: t
%%% coding: utf-8
%%% End:
