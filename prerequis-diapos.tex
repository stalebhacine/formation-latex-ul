%%% Copyright (C) 2018 Vincent Goulet
%%%
%%% Ce fichier fait partie du projet
%%% «Rédaction avec LaTeX»
%%% http://github.com/vigou3/formation-latex-ul
%%%
%%% Cette création est mise à disposition selon le contrat
%%% Attribution-Partage dans les mêmes conditions 4.0
%%% International de Creative Commons.
%%% http://creativecommons.org/licenses/by-sa/4.0/

\begin{frame}
  \frametitle{Fichiers d'accompagnement}

  \begin{adjustwidth}{10mm}{10mm}
    Ce document devrait être accompagné des fichiers nécessaires pour
    compléter les exercices.

    Si vous n'avez pas obtenu ces fichiers avec le document, vous
    pouvez les récupérer dans le site \emph{Comprehensive TeX Archive
      Network} (CTAN).

    \begin{center}
      \link{\ctanurl}{Accéder aux fichiers dans CTAN}
    \end{center}
  \end{adjustwidth}
\end{frame}

\begin{frame}
  \frametitle{Prérequis à cette formation}

  \begin{adjustwidth}{8mm}{10mm}
    \begin{enumerate}
    \item Installer une distribution {\LaTeX} sur votre poste de
      travail; nous recommandons la distribution {\TeX}~Live
      \begin{itemize}
      \item \link{https://youtu.be/kA53EQ3Q47w}{%
          Vidéo d'installation sur macOS}
      \item \link{https://youtu.be/7MfodhaghUk}{%
          Vidéo d'installation sur Windows}
      \end{itemize}
    \item Composer un premier document très simple de type \emph{Hello
        World!}
      \begin{itemize}
        \normalsize
      \item \link{https://youtu.be/vZfiZUSsP68}{%
          Démonstration vidéo sur macOS avec TeXShop}
      \item \link{https://youtu.be/mMgFVQhZbiM}{%
          Démonstration vidéo sur Windows avec Texmaker}
      \end{itemize}
    \end{enumerate}
  \end{adjustwidth}
\end{frame}

%%% Local Variables:
%%% TeX-master: "formation-latex-ul-diapos"
%%% TeX-engine: xetex
%%% coding: utf-8
%%% End:
