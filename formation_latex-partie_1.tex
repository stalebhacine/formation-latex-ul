\documentclass[aspectratio=54,10pt,xcolor=x11names]{beamer}
  \usepackage[francais]{babel}
  \usepackage[autolanguage]{numprint}
  \usepackage{amsmath,icomma}
  \usepackage{booktabs,tabularx}
  \usepackage{pgfpages}
  \usepackage{metalogo}                % \XeLaTeX logo
  \usepackage{fontawesome}
  \usepackage{listings}
  \usepackage[overlay,absolute]{textpos}

  %%% Polices de caractères
  \usepackage{fontspec}
  \usepackage{unicode-math}
  \defaultfontfeatures{Ligatures=TeX}
  \setmainfont{Lucida Bright OT}
  \setsansfont{Lucida Sans OT}
  \setmathfont{Lucida Bright Math OT}
  \setmonofont[Scale=0.95]{Bitstream Vera Sans Mono}
  \newfontfamily\titles{Myriad Pro}
  \usepackage{microtype}

  %% Définition de nouveaux symboles de la police Font Awesome pas
  %% encore définis dans le package fontawesome (v. 3.1.1)
  \def\faFileText{{\FA \symbol{"F0F6}}}
  \def\faFilePDF{{\FA \symbol{"F1C1}}}
  \def\faYouTubePlay{{\FA \symbol{"F16A}}}

  %%% Titre
  \title[]{Rédaction de thèses et de mémoire avec {\LaTeX}: premiers pas}
  \author[]{Université Laval}
  \date{}
  \renewcommand{\year}{2015}    % pour notice de copyright

  %%% Définition de couleurs
  \definecolor{comments}{rgb}{0.7,0,0}
  \definecolor{emphasis}{named}{orange}
  \hypersetup{colorlinks,allcolors=emphasis}

  %%% Paramètres de beamer
  \useinnertheme{default}
  \useoutertheme[width=10mm,height=2.2\baselineskip]{sidebar}
  \usefonttheme[onlylarge]{structurebold}
  \usefonttheme{professionalfonts}
  \addtobeamertemplate{headline}{\rule{0pt}{12pt}\par}{}
  \setbeamercolor{frametitle}{fg=white,bg=black}
  \setbeamerfont{frametitle}{family=\titles}
  \setbeamercolor{structure}{fg=orange,bg=white}
  \setbeamercolor{block title}{fg=white,bg=black}
  \setbeamercolor{alerted text}{fg=black}
  \setbeamerfont{alerted text}{series=\bfseries}
  \setbeamercolor{block title alerted}{fg=white,bg=comments}
  \setbeamertemplate{theorems}[numbered]
  \setbeamertemplate{navigation symbols}{}
  \setbeamertemplate{section in sidebar}{}
  \setbeamertemplate{section in sidebar shaded}{}
  \AtBeginSection[]
  {
    \begin{frame}
      \frametitle{Sommaire}
      \small\tableofcontents[currentsection]
    \end{frame}
  }

  %%% Nouvelles commandes et nouveaux environnements
  \newcommand{\fichier}[1]{\texttt{#1}}
  \renewenvironment{quote}{%
    \begin{beamercolorbox}[wd=\linewidth,sep=6pt]{block body example}}
    {\end{beamercolorbox}}
  \newenvironment{texoutput}[1]{%
    \begin{minipage}[t]{#1} \small}{%
    \end{minipage}}

  \theoremstyle{definition}
  \newtheorem{exercice}[theorem]{Exercice}

  %%% Paramètres de listings
  \lstloadlanguages{[LaTeX]TeX}
  \lstset{language=[LaTeX]TeX,
    escapeinside=`',
    extendedchars=true,
    inputencoding=utf8/latin1,
    basicstyle=\small\ttfamily\NoAutoSpacing,
    commentstyle=\color{comments}\slshape,
    keywordstyle=\mdseries,
    emphstyle=\color{emphasis}\bfseries,
    backgroundcolor=\color{LightYellow1},
    frame=single,
    framerule=0pt,
    showstringspaces=false}

\begin{document}

%% Le code de la page couverture est conservé dans un fichier séparé
%% Avec l'option aspectratio=54, la taille des diapos est 125mm x 100mm

\begin{textblock*}{100mm}(-8mm,29mm)
  \includegraphics[width=100mm,keepaspectratio]{TeXFZoneColor}
\end{textblock*}

\begin{textblock*}{107mm}(0mm,10mm)
  \rule[0mm]{107mm}{13mm}
\end{textblock*}

\begin{textblock*}{98mm}(7mm,14mm)
  \color{white}\fontspec{Myriad Pro}
  \fontsize{17pt}{18pt}\selectfont\bfseries%
  Rédaction de thèses et de mémoires
\end{textblock*}

\begin{textblock*}{70mm}(55mm,32mm)
  \color{black}\fontspec{Lucida Bright OT}
  \fontsize{17pt}{18pt}\selectfont%
  avec\;\;
  \fontsize{60pt}{60pt}\selectfont%
  \raisebox{-1.07ex}{\LaTeX}
\end{textblock*}

\begin{textblock*}{30mm}(85mm,63mm)
  \fontspec{Myriad Pro}
  \small\bfseries%
  1. PREMIERS PAS
\end{textblock*}

\begin{textblock*}{25mm}(93.5mm,80mm)
  \includegraphics[width=25mm,keepaspectratio]{ul_p}
\end{textblock*}


%% Le texte du contrat de licence est conservé dans un fichier séparé
\input{licence}

\begin{frame}
  \frametitle{Sommaire}
  \small\tableofcontents
\end{frame}

\begin{frame}
  \frametitle{Pré-requis à cette formation}
  \begin{enumerate}
  \item Installer une distribution {\LaTeX} sur votre poste de
    travail; nous recommandons la distribution {\TeX}~Live
    \begin{itemize}
    \item[] {\faYouTubePlay} installation sur Mac OS X
    \item[] {\faYouTubePlay} installation sur Windows
    \end{itemize}
  \item Compiler un premier un document très simple de type
    \emph{Hello World!}
    \begin{itemize}
    \item[] {\faYouTubePlay} démonstration sur Mac OS X avec TeXShop
    \item[] {\faYouTubePlay} démonstration sur Windows avec TeXMaker
    \end{itemize}
  \end{enumerate}
\end{frame}


\section{{\TeX}, {\LaTeX} et consorts: ce que c'est et ce que ce n'est pas}

\begin{frame}
  \frametitle{Ce que c'est}
  \begin{itemize}
  \item Un système de mise en page (\emph{typesetting}) ou de préparation de
    documents
  \item {\LaTeX} est un ensemble de macro commandes pour faciliter
    l'utilisation de {\TeX}
  \item Langage de balisage (\emph{Markup Language}) pour indiquer la
    mise en forme du texte
  \item Accent mis sur la production de documents de grande qualité à
    la typographie soignée (surtout pour les mathématiques)
  \end{itemize}
\end{frame}

\begin{frame}
  \frametitle{Exemples de typographie soignée}
  \begin{itemize}
  \item Ligatures
    \begin{quote}
      \begin{minipage}{0.45\linewidth}
        \vspace{-12pt}
        \begin{block}{\small Word}
          \rmfamily f\/f \quad f\/i \quad f\/l \quad f\/f\/i \quad
          f\/f\/l
        \end{block}
      \end{minipage}
      \hfill
      \begin{minipage}{0.4\linewidth}
        \vspace{-12pt}
        \begin{block}{\small \LaTeX}
          \rmfamily ff \quad fi \quad fl \quad ffi \quad ffl
        \end{block}
      \end{minipage}
    \end{quote}
  \item Espacement des lettres
    \begin{quote}
      \begin{minipage}{0.45\linewidth}
        \vspace{-12pt}
        \begin{block}{\small texte}
          \rmfamily xy \quad \emph{xy}
        \end{block}
      \end{minipage}
      \hfill
      \begin{minipage}{0.4\linewidth}
        \vspace{-12pt}
        \begin{block}{\small mathématiques}
          $xy$
        \end{block}
      \end{minipage}
    \end{quote}
  \end{itemize}
\end{frame}

\begin{frame}
  \frametitle{Ce que ce n'est pas}
  \begin{itemize}
  \item Un traitement de texte
  \item WYSIWYG
  \item Incompatible
  \item Instable
  \end{itemize}
\end{frame}

\begin{frame}
  \frametitle{Processus de création d'un document {\LaTeX}}
  \Huge
  \begin{minipage}[t]{0.25\linewidth}
    \centering
    \faFileText \\ \bigskip
    \footnotesize
    rédaction du texte et balisage avec un \emph{éditeur de texte}
  \end{minipage}
  \hfill\faArrowRight\hfill
  \begin{minipage}[t]{0.25\linewidth}
    \centering
    \faCogs \\  \bigskip
    \footnotesize
    compilation avec un \emph{moteur} {\TeX} depuis la ligne de commande
  \end{minipage}
  \hfill\faArrowRight\hfill
  \begin{minipage}[t]{0.25\linewidth}
    \centering
    \faFilePDF \\  \bigskip
    \footnotesize
    visualisation avec visionneuse externe (Aperçu,
    SumatraPDF, etc.)
  \end{minipage}
  \newline\pause
  \begin{picture}(0,0)
    \thicklines\color{blue}
    \put(-5,-2){\dashbox{2}(200,100){}}
    \put(45,-5){
      \begin{minipage}[t]{105\unitlength}
        \footnotesize\centering
        \mbox{} \\ facilité par l'utilisation
        d'un logiciel intégré de rédaction
      \end{minipage}}
  \end{picture}
\end{frame}

\begin{frame}
  \frametitle{Quelques choses simples à réaliser avec {\LaTeX}}
  \framesubtitle{(et pas nécessairement avec un traitement de texte)}
  \begin{itemize}
  \item Page titre
  \item Table des matières
  \item Numérotation des pages
  \item Numérotation des équations et renvois
  \item Bibliographie et renvois
  \item Figures et tableaux: disposition sur la page, numérotation, renvois
  \item Coupure de mots
  \item Document recto-verso
  \end{itemize}
\end{frame}

\begin{frame}
  \frametitle{Moteurs et formats}
  \visible<2->{
    \begin{tabular}{r}
      \\ \addlinespace[8pt] \\ \\
      \color{emphasis}\faArrowRight \\
      \color{emphasis}\faArrowRight
    \end{tabular}
    \hspace{-5mm}
  }
  \begin{tabularx}{0.9\linewidth}{Xlc}
    \toprule[2pt]
    Moteur & Format & Fichier de sortie \\
    \midrule
    \texttt{tex} & plain \TeX & DVI \\
    \texttt{tex} (\texttt{latex}) & \LaTeX & DVI \\
    \texttt{pdftex} (\texttt{pdflatex}) & pdf\LaTeX & PDF \\
    \texttt{xetex} (\texttt{xelatex}) & \XeLaTeX & PDF \\
    \bottomrule[2pt]
  \end{tabularx}
\end{frame}

\begin{frame}
  \frametitle{Distributions}

  Le système {\LaTeX} est rendu disponible sous forme de \emph{distributions}

  \begin{itemize}
  \item Windows: {\TeX}~Live et MiK{\TeX}
  \item OS~X: Mac{\TeX} (dérivée de {\TeX}~Live)
  \item Linux: {\TeX}~Live
  \end{itemize}
  La Bibliothèque et la Faculté des études supérieures et
  post-doctorales recommandent {\TeX}~Live
\end{frame}

\begin{frame}[fragile=singleslide]
  \frametitle{Faits amusants}
  \begin{itemize}
  \item {\TeX} est aujourd'hui considéré essentiellement exempt de bogue
  \item Récompense si vous en trouvez un!
  \item Numéro de version de {\TeX} converge vers $\pi$:
\begin{lstlisting}
$ tex --version
TeX `\textbf{3.14159265}' (TeX Live 2014)
kpathsea version 6.2.0
Copyright 2014 D.E. Knuth.
[...]
\end{lstlisting}
  \item Pour en savoir plus:
    \begin{itemize}
    \item \href{http://www.tug.org/whatis.html}{Histoire de \TeX} (anglais)
    \item {\TeX} sur Wikipedia
      (\href{http://fr.wikipedia.org/wiki/TeX}{français};
      \href{http://en.wikipedia.org/wiki/TeX}{anglais}, plus complet)
    \end{itemize}
  \end{itemize}
\end{frame}


\section{Principes de base}

\begin{frame}[fragile=singleslide]
  \frametitle{Rédaction}
  \begin{itemize}
  \item On se concentre sur le contenu et la \alert{structure} du
    document, pas sur son \alert{apparence}
      \bigskip
      \begin{tabbing}
        \verb=\textbf{titre}= \qquad\= \faArrowRight \qquad\= \verb|\section{titre}| \\[6pt]
        \verb|\textit{texte}| \> \faArrowRight \> \verb|\emph{texte}|
      \end{tabbing}
      \bigskip
  \item Apparence prise en charge par {\LaTeX} et généralement préférable de ne
    pas la modifier
  \item Mots séparés par une ou plusieurs \alert{espaces}
  \item Paragraphes séparés par une ou plusieurs \alert{lignes blanches}
  \item Utilisation de \alert{commandes} pour indiquer la structure du texte
  \end{itemize}
\end{frame}

\begin{frame}[fragile=singleslide]
  \frametitle{Structure d'un document {\LaTeX}}
  Un fichier source {\LaTeX} est toujours composé de deux
  parties:
  \begin{enumerate}
  \item le \alert{préambule}
    \begin{itemize}
    \item suite de commandes spécifiant la mise en forme \emph{globale} du
      document (format du papier, marges, entête et pied de page, etc.)
    \item au minimum \verb=\documentclass=
    \end{itemize}
  \item le \alert{corps} du document
    \begin{itemize}
    \item débute par \verb=\begin{document}=
    \item texte du document
    \item commandes à effet \emph{local}
    \item termine par \verb=\end{document}=
    \end{itemize}
  \end{enumerate}
\end{frame}

%%% >>>
\begin{frame}[plain,fragile=singleslide]
  \begin{exercice}
    \begin{enumerate}
    \item Compiler le document \fichier{exercice\_minimal.tex}.
    \item Changer la classe \textbf{article} pour la classe \textbf{book}
      et observer le résultat.
    \item Ajouter du texte en français (avec accents) et observer le
      résultat.
    \item Question de voir ce que {\LaTeX} peut faire, compiler le
      document élaboré \fichier{exercice\_demo.tex} de la manière suivante:
      \begin{enumerate}[i)]
      \item une fois avec \texttt{LaTeX};
      \item une fois avec \texttt{BibTeX};
      \item deux à trois fois avec \texttt{LaTeX}.
      \end{enumerate}
    \end{enumerate}
  \end{exercice}
\end{frame}
%%% <<<

\begin{frame}[fragile=singleslide]
  \frametitle{Commandes}
  \begin{itemize}
  \item Débutent toujours par \verb=\=
  \item Nom se termine par tout caractère qui n'est pas une lettre (y
    compris l'espace!)
  \item Arguments obligatoires entre \verb={ }=
  \item Arguments optionnels entre \verb=[ ]=
  \item Formes générales:
\begin{lstlisting}
\`\textit{nomcommande}'[`\textit{arg\_optionnel}']{`\textit{arg\_obligatoire}'}
\`\textit{nomcommande}'*[`\textit{arg\_optionnel}']{`\textit{arg\_obligatoire}'}
\end{lstlisting}
  \item Portée d'une commande limitée à la zone entre \verb={ }=
  \end{itemize}
\end{frame}

\begin{frame}[fragile=singleslide]
  \frametitle{Environnements}
  \begin{itemize}
  \item Délimités par
\begin{lstlisting}
\begin{`\textit{environnement}'}

\end{`\textit{environnement}'}
    \end{lstlisting}
  \item Contenu de l'environnement traité différemment du reste du texte
  \item Changements s'appliquent uniquement à l'intérieur de
    l'environnement
  \end{itemize}
\end{frame}

\begin{frame}[plain,fragile=singleslide]
  \begin{exercice}
    Modifier le fichier \fichier{exercice\_commandes.tex} afin
    de produire le texte ci-dessous.
  \end{exercice}
  \begin{center}
    \fbox{\includegraphics[width=\linewidth]{exercice_commandes-output}}
  \end{center}
\end{frame}

\begin{frame}[fragile=singleslide]
  \frametitle{Caractères spéciaux}
  \begin{itemize}
  \item Caractères réservés par {\TeX}:
    \begin{quote}
      \verb=# $ & ~ _ ^ % { }=
    \end{quote}
  \item Pour les utiliser, précéder par \verb=\=:
   \begin{quote}
     \begin{minipage}[t]{0.07\linewidth}
\begin{lstlisting}[aboveskip=-0.5\medskipamount]
\#
\end{lstlisting}
      \end{minipage}
      \quad
      \begin{texoutput}{0.07\linewidth}
        \#
      \end{texoutput}
      \hfill
     \begin{minipage}[t]{0.07\linewidth}
\begin{lstlisting}[aboveskip=-0.5\medskipamount]
\$
\end{lstlisting}
     \end{minipage}
     \quad
     \begin{texoutput}{0.07\linewidth}
       \$
     \end{texoutput}
     \hfill
     \begin{minipage}[t]{0.07\linewidth}
\begin{lstlisting}[aboveskip=-0.5\medskipamount]
\%
\end{lstlisting}
     \end{minipage}
     \quad
     \begin{texoutput}{0.07\linewidth}
       \%
     \end{texoutput}
     \\
     \begin{minipage}[t]{0.07\linewidth}
\begin{lstlisting}[aboveskip=-0.6\medskipamount]
\_
\end{lstlisting}
     \end{minipage}
     \quad
     \begin{texoutput}{0.07\linewidth}
       \_
     \end{texoutput}
     \hfill
     \begin{minipage}[t]{0.07\linewidth}
\begin{lstlisting}[aboveskip=-0.5\medskipamount]
\{
\end{lstlisting}
     \end{minipage}
     \quad
     \begin{texoutput}{0.07\linewidth}
       \}
     \end{texoutput}
     \hfill
     \begin{minipage}[t]{0.07\linewidth}
\begin{lstlisting}[aboveskip=-0.5\medskipamount]
\}
\end{lstlisting}
     \end{minipage}
     \quad
     \begin{texoutput}{0.07\linewidth}
       \}
     \end{texoutput}
    \end{quote}
  \item Guillemets:
    \begin{quote}
     \begin{minipage}[t]{0.45\linewidth}
\begin{lstlisting}[aboveskip=-0.6\medskipamount,escapeinside={}]
``guillemets anglais''
\end{lstlisting}
     \end{minipage}
     \quad
     \begin{texoutput}{0.45\linewidth}
       ``guillemets anglais''
     \end{texoutput}
     \\
     \begin{minipage}[t]{0.45\linewidth}
\begin{lstlisting}[aboveskip=-0.6\medskipamount]
«guillemets français»
\end{lstlisting}
     \end{minipage}
     \quad
     \begin{texoutput}{0.45\linewidth}
       «guillemets français»
     \end{texoutput}
   \end{quote}
  \item Tiret, tiret demi-cadratin, tiret cadratin:
    \begin{quote}
     \begin{minipage}[t]{0.07\linewidth}
\begin{lstlisting}[aboveskip=-0.2\medskipamount]
-
\end{lstlisting}
     \end{minipage}
     \quad
     \begin{texoutput}{0.07\linewidth}
       -
     \end{texoutput}
     \hfill
     \begin{minipage}[t]{0.07\linewidth}
\begin{lstlisting}[aboveskip=-0.2\medskipamount]
--
\end{lstlisting}
     \end{minipage}
     \quad
     \begin{texoutput}{0.07\linewidth}
       --
     \end{texoutput}
     \hfill
          \begin{minipage}[t]{0.07\linewidth}
\begin{lstlisting}[aboveskip=-0.2\medskipamount]
---
\end{lstlisting}
     \end{minipage}
     \quad
     \begin{texoutput}{0.07\linewidth}
       ---
     \end{texoutput}
   \end{quote}
 \end{itemize}
\end{frame}

\begin{frame}[fragile]
  \frametitle{Classe de document}
  \begin{itemize}
  \item La première commande du préambule est normalement la
    déclaration de la classe de la forme
\begin{lstlisting}
\documentclass[`\textit{options}']{`\textit{classe}'}
\end{lstlisting}
  \item<2-> Principales classes:
    \begin{quote}
      \textbf{article, report, book, letter} \\
      {\color<3->{emphasis} \textbf{memoir}} \\
      {\color<3->{emphasis} \textbf{ulthese}}
    \end{quote}
  \item<4-> Principales options:
    \begin{quote}
      \texttt{10pt, {\color<5->{emphasis} 11pt}, 12pt} \\
      \texttt{oneside, twoside} \\
      \texttt{openright, openany} \\
      {\color<5->{emphasis} \texttt{article}} (classe \textbf{memoir})
    \end{quote}
  \end{itemize}
\end{frame}

\begin{frame}[fragile]
  \frametitle{Paquetages}
  \begin{itemize}
  \item Permettent de modifier des commandes ou d'ajouter des
    fonctionnalités au système
  \item Chargés dans le préambule avec
    \begin{lstlisting}
\usepackage{`\textit{paquetage}'}
\usepackage[`\textit{options}']{`\textit{paquetage}'}
\usepackage{`\textit{paquetage1,paquetage2,...}'}
    \end{lstlisting}
  \item<2-> Les incontournables:
    \begin{description}
      \small
      %% \hfill = PMLA (Poor Man's Left Align) !
    \item[babel*\hfill] typographie multilingue
    \item[inputenc*\hfill] composition en français (\LaTeX)
    \item[fontspec*\hfill] contrôle des polices (\XeLaTeX)
    \item[amsmath\hfill] extensions mathématiques
    \item[booktabs*\hfill] amélioration des tableaux
    \item[hyperref*\hfill] hyperliens dans PDF
    \end{description}
    {\footnotesize * = chargé par défaut dans \textbf{ulthese}}
  \end{itemize}
\end{frame}

\begin{frame}
  \frametitle{{\LaTeX} en français}
    \small
  \begin{tabularx}{1.0\linewidth}{Xl}
    \toprule
    \textbf{Enjeu} & \textbf{Solution} \\
    \midrule
    \addlinespace[6pt]
    traduction des mots-clés prédéfinis & \textbf{babel} \\
    \addlinespace[6pt]
    coupure de mots & \textbf{babel} \\
    \addlinespace[6pt]
    typographie française & \textbf{babel} \\
    \addlinespace[6pt]
    lettres accentuées dans source & \textbf{inputenc} (\LaTeX) \\
                                   & source en UTF-8 (\XeLaTeX) \\
    \addlinespace[6pt]
    virgule comme séparateur décimal & \textbf{icomma} \\
    \addlinespace[6pt]
    espace comme séparateur des milliers & \textbf{numprint}
  \end{tabularx}
\end{frame}

%%% >>>
\begin{frame}[plain,fragile=singleslide]
  \begin{exercice}
    \begin{enumerate}
    \item Compiler tel que fourni le fichier
      \fichier{exercice\_classe+paquetages.tex}.
    \item Changer la police de caractère du document pour 11~points,
      puis 12~points. Changer la classe du document pour
      \textbf{memoir}. Observer l'effet sur les marges et sur la
      coupure automatique des mots.
    \item Charger le paquetage \textbf{icomma} et observer l'effet sur
      la formule mathématique.
    \item Charger le paquetage \textbf{numprint} avec l'option
      \verb=autolanguage= (\emph{après} le paquetage \textbf{babel}).
      Dans le code source de la formule mathématique, changer
\begin{lstlisting}
10 000
\end{lstlisting}
      pour
\begin{lstlisting}
\nombre{10000}
\end{lstlisting}
      et observer le résultat.
    \end{enumerate}
  \end{exercice}
\end{frame}
%%% <<<

\section{Parties d'un document}

\begin{frame}[plain]
  \begin{alertblock}{Conseil du {\TeX}pert}
    Utiliser impérativement les commandes {\LaTeX} pour identifier les
    différentes parties (la structure) d'un document
  \end{alertblock}
\end{frame}

\begin{frame}[fragile]
  \frametitle{Titre et page titre}
  \begin{itemize}
  \item Mise en forme automatique
    \begin{lstlisting}[frame=single]
%% préambule
\title{`\textit{Titre du document}'}
\author{`\textit{Prénom Nom}'}
\date{`\textit{31 octobre 2014}'} % automatique si omis

%% corps du document
\maketitle
    \end{lstlisting}
  \item Mise en forme libre \\[6pt]
    \begin{minipage}{0.45\linewidth}
      \begin{block}{\small classes standards}
\begin{lstlisting}
\begin{titlepage}
  ...
\end{titlepage}
\end{lstlisting}
      \end{block}

    \end{minipage}
    \hfill
    \begin{minipage}{0.45\linewidth}
      \begin{block}{\small classe \textbf{memoir}}
\begin{lstlisting}
\begin{titlingpage}
  ...
\end{titlingpage}
\end{lstlisting}
      \end{block}
    \end{minipage}
  \end{itemize}
\end{frame}

\begin{frame}[fragile=singleslide]
  \frametitle{Résumé}
  \begin{itemize}
  \item Classes \textbf{article}, \textbf{report} ou \textbf{memoir}:
    résumé créé avec l'environnement
\begin{lstlisting}
\begin{abstract}

\end{abstract}
\end{lstlisting}
  \item Classe \textbf{ulthese}: résumés français et anglais traités
    comme des chapitres normaux (non numérotés)
  \end{itemize}
\end{frame}

\begin{frame}[fragile=singleslide]
  \frametitle{Sections}
  \begin{itemize}
  \item Découpage du document en sections avec les commandes
    \begin{quote}
      \begin{tabularx}{1.0\linewidth}{Xl}
        \verb=\part= \\
        \verb=\chapter= \\
        \verb=\section= \\
        \verb=\subsection= \\
        \verb=\subsubsection= &
          \color{emphasis} \faArrowLeft\ à éviter dans un livre! \\
        \verb=\paragraph=  &
          \color{emphasis} \faArrowLeft\ jamais (?) utilisé \\
      \end{tabularx}
    \end{quote}
  \item Prennent le titre en argument
  \item Numérotation automatique
  \item Commande suivie d'une \verb=*= = section non numérotée
  \end{itemize}
\end{frame}

%%% >>>
\begin{frame}[plain,fragile=singleslide]
  \begin{exercice}
    Utiliser le fichier \fichier{exercice\_sections.tex}.

    \begin{enumerate}
    \item Ajouter un titre et un auteur au document.
    \item Insérer deux ou trois titres de sections de différents niveaux
      dans le document.
    \end{enumerate}
  \end{exercice}
\end{frame}
%%% <<<

\begin{frame}[fragile=singleslide]
  \frametitle{Renvois automatiques}
  \begin{itemize}
  \item Ne \alert{jamais} renvoyer manuellement à un numéro de
    section, d'équation, de tableau, etc.
  \item «Nommer» un élément avec \verb=\label=
  \item Faire référence par son nom avec \verb=\ref=
  \item Requiert 2 à 3 compilations
  \end{itemize}
\end{frame}

\begin{frame}[plain,fragile=singleslide]
  \begin{lstlisting}[emph={\label,\ref}]
\section{Définitions}
\label{sec:definitions}

Lorem ipsum dolor sit amet, consectetur
adipiscing elit. Duis in auctor dui. Vestibulum

\section{Historique}

Tel que vu à la section \ref{sec:definitions},
on a...
\end{lstlisting}
  \fbox{\includegraphics[width=\linewidth]{renvoi}}
\end{frame}

\begin{frame}[plain,fragile=singleslide]
  \begin{alertblock}{Conseil du {\TeX}pert}
    Adopter une manière systématique et mnémotechnique de nommer les
    éléments dans un long document afin de vous y retrouver.

    \bigskip %
    Exemple:
\begin{lstlisting}
\label{chap:`\textit{chapitre}'}         % chapitre
\label{sec:`\textit{chapitre}':`\textit{section}'}  % section
\label{tab:`\textit{chapitre}':`\textit{tableau}'}  % tableau
\label{eq:`\textit{chapitre}':`\textit{equation}'}  % équation
\end{lstlisting}
  \end{alertblock}
\end{frame}

\begin{frame}[fragile]
  \frametitle{Renvois automatiques++}
  \begin{itemize}
  \item Paquetage \textbf{hyperref} insère des hyperliens vers les
    renvois dans les fichiers PDF
\begin{lstlisting}
Tel que vu à la section \ref{sec:definitions},
on a...
\end{lstlisting}
  \fbox{\includegraphics[width=0.7\linewidth]{renvoi_avec_ref}}
  \vfill
  \item<2-> Commande \verb=\autoref= permet de
    \begin{enumerate}
    \item nommer automatiquement le type de renvoi (section, équation,
      tableau, etc.)
    \item transformer en hyperlien le texte \textbf{et} le numéro
    \end{enumerate}
\begin{lstlisting}
Tel que vu à la \autoref{sec:definitions},
on a...
\end{lstlisting}
  \fbox{\includegraphics[width=0.7\linewidth]{renvoi_avec_autoref}}
  \end{itemize}
\end{frame}

%%% >>>
\begin{frame}[plain]
  \begin{exercice}
    Utiliser le fichier \fichier{exercice\_renvois.tex}.
    \begin{enumerate}
    \item Insérer dans le texte un renvoi au numéro d'une section.
    \item Activer le paquetage \textbf{hyperref} avec l'option
      \texttt{colorlinks} et comparer l'effet d'utiliser
      \texttt{{\textbackslash}ref} ou \texttt{{\textbackslash}autoref}
      pour le renvoi.
    \end{enumerate}
  \end{exercice}
\end{frame}
%%% <<<

\begin{frame}[fragile=singleslide]
  \frametitle{Annexes}
  \begin{itemize}
  \item Annexes sont des sections ou chapitres avec une numérotation
    alphanumérique (A, A.1, ...)
  \item Prochaines sections identifiées comme des annexes par la
    commande
\begin{lstlisting}
\appendix
\end{lstlisting}
  \item Dans le titre, «Chapitre» changé pour «Annexe» le cas échéant
  \end{itemize}
\end{frame}

\begin{frame}[fragile]
  \frametitle{Structure logique d'un livre}
  \framesubtitle{(classes \textbf{book}, \textbf{memoir}, \textbf{ulthese})}
\begin{lstlisting}
\frontmatter
\end{lstlisting}
  \begin{itemize}
    \small
  \item préface, table des matières, etc.
  \item numérotation des pages en chiffres romains (i, ii, ...)
  \item chapitres non numérotés
  \end{itemize}
  \vfill

\begin{lstlisting}
\mainmatter
\end{lstlisting}
  \begin{itemize}
    \small
  \item le contenu à proprement parler
  \item numérotation des pages à partir de 1 en chiffres arabes
  \item chapitres numérotés
  \end{itemize}
  \vfill

\begin{lstlisting}
\backmatter
\end{lstlisting}
  \begin{itemize}
    \small
  \item tout le reste (bibliographie, index, etc.)
  \item numérotation des pages se poursuit
  \item chapitres non numérotés
  \end{itemize}
\end{frame}

\begin{frame}[fragile]
  \frametitle{Table des matières}
  \begin{itemize}
  \item Table des matières produite automatiquement avec
\begin{lstlisting}
\tableofcontents
\end{lstlisting}
  \item Requiert plusieurs compilations
  \item Sections non numérotées pas incluses
  \item Avec \textbf{hyperref}, produit également la table des
    matières du fichier PDF
  \item<2-> Classe \textbf{memoir} fournit également
\begin{lstlisting}
\tableofcontents*
\end{lstlisting}
    qui n'insère pas la table des matières dans la table des matières
  \item<3-> Aussi disponibles:
\begin{lstlisting}
\listoffigures
\listoftables
\end{lstlisting}
    (et leurs versions \verb=*= dans \textbf{memoir})
  \end{itemize}
\end{frame}

%%% >>>
\begin{frame}[plain]
  \begin{exercice}
    Utiliser le fichier \fichier{exercice\_tdm+annexe.tex}.
    \begin{enumerate}
    \item Étudier la structure du document dans le code source.
    \item Créer la table des matières du document en le compilant 2 à
      3 fois.
    \item Ajouter une annexe au document.
    \end{enumerate}
  \end{exercice}
\end{frame}
%%% <<<


\section{Contrôle du texte}

\begin{frame}[fragile]
  \frametitle{Changement d'attribut de la police de caractères}
  \begin{block}{famille}
    \vspace{-12pt}
    \begin{tabbing}
      \textsf{sans empattements} \qquad\= \verb=\sffamily= \qquad\=
      \verb=\textsf{texte}= \kill
      \small
      \textrm{romain} \> \verb=\rmfamily= \> \verb=\textrm{=\textit{texte}\verb=}= \\
      \texttt{largeur fixe} \> \verb=\ttfamily= \> \verb=\texttt{=\textit{texte}\verb=}= \\
      \textsf{sans empattements} \> \verb=\sffamily= \> \verb=\textsf{=\textit{texte}\verb=}=
    \end{tabbing}
  \end{block}
  \vfill
  \begin{block}{forme}
    \vspace{-12pt}
    \begin{tabbing}
      \textsf{sans empattements} \qquad\= \verb=\sffamily= \qquad\=
      \verb=\textsf{texte}= \kill
      \small
      \textup{\rmfamily droit} \> \verb=\upshape= \> \verb=\textup{=\textit{texte}\verb=}= \\
      \textit{\rmfamily italique} \> \verb=\itshape= \> \verb=\textit{=\textit{texte}\verb=}= \\
      \textsl{penché} \> \verb=\slshape= \> \verb=\textsl{=\textit{texte}\verb=}= \\
      \textsc{\rmfamily petites capitales} \> \verb=\scshape= \> \verb=\textsc{=\textit{texte}\verb=}=
    \end{tabbing}
  \end{block}
  \vfill
  \begin{block}{série}
    \vspace{-12pt}
    \begin{tabbing}
      \textsf{sans empattements} \qquad\= \verb=\sffamily= \qquad\=
      \verb=\textsf{texte}= \kill
      \rmfamily\small
      \textmd{\rmfamily moyen} \> \verb=\mdseries= \> \verb=\textmd{=\textit{texte}\verb=}= \\
      \textbf{\rmfamily gras} \> \verb=\bfseries= \> \verb=\textbf{=\textit{texte}\verb=}= \\
    \end{tabbing}
    \vfill
  \end{block}
  \pause
  \begin{picture}(0,0)
    \thicklines\color{blue}
    \put(113,30){\dashbox{2}(62,190){}}
    \put(100,20){
      \begin{minipage}[t]{75\unitlength}
        \footnotesize\centering
        s'applique à tout le texte qui suit
      \end{minipage}}
  \end{picture}
  \begin{picture}(0,0)
    \thicklines\color{blue}
    \put(185,30){\dashbox{2}(82,190){}}
    \put(180,20){
      \begin{minipage}[t]{85\unitlength}
        \footnotesize\centering
        s'applique au texte en argument
      \end{minipage}}
  \end{picture}

\end{frame}

\begin{frame}[fragile]
  \frametitle{Taille de la police}
  \vspace{-2pt}
  \begin{block}{commandes standards}
    \vspace{-10pt}
    \begin{tabbing}
      \verb=\footnotesize= \quad\= \kill
      \verb=\tiny= \> {\tiny minuscule} \\
      \verb=\scriptsize= \> {\scriptsize très petit} \\
      \verb=\footnotesize= \> {\footnotesize plus petit} \\
      \verb=\small= \> {\small petit} \\
      \verb=\normalsize= \> {\normalsize normal} \\
      \verb=\large= \> {\large grand} \\
      \verb=\Large= \> {\Large plus grand} \\
      \verb=\LARGE= \> {\LARGE un peu plus grand} \\
      \verb=\huge= \> {\huge encore plus grand} \\
      \verb=\Huge= \> {\Huge énorme}
    \end{tabbing}
  \end{block}
  \vspace{-10pt}
  \pause
  \begin{block}{ajouts de \textbf{memoir} (et donc \textbf{ulthese})}
    \vspace{-10pt}
    \begin{tabbing}
      \verb=\footnotesize= \quad\= \kill
      \verb=\miniscule= \quad\> [$<$ \verb=\tiny=] \\
      \verb=\HUGE= \> [$>$ \verb=\Huge=] \\
    \end{tabbing}
  \end{block}
\end{frame}

\begin{frame}[fragile=singleslide]
  \frametitle{Autres changements de police}
  \begin{itemize}
  \item Attributs par défaut
\begin{lstlisting}
\textnormal{`\textit{texte}'}
\end{lstlisting}
  \item Emphase (par défaut italique dans texte droit et vice versa)
\begin{lstlisting}
\emph{`\textit{texte}'}
\end{lstlisting}
  \end{itemize}
\end{frame}

\begin{frame}[fragile=singleslide]
  \frametitle{Sauts de ligne}
  \begin{itemize}
  \item Rarement nécessaire de forcer les retours à la ligne
  \item Lorsque requis utiliser
    \begin{quote}
      \begin{minipage}{0.3\linewidth}
\begin{lstlisting}[aboveskip=1.5\medskipamount]
  \\
\end{lstlisting}
      \end{minipage}
      \quad ou \quad
      \begin{minipage}{0.3\linewidth}
\begin{lstlisting}[aboveskip=1.5\medskipamount]
  \newline
\end{lstlisting}
      \end{minipage}
    \end{quote}
  \item Aussi pour délimiter
    \begin{itemize}
    \item les lignes dans les tableaux
    \item les lignes d'une suite d'équations
    \end{itemize}
  \item On peut suivre un saut de ligne d'un espace vertical
    arbitraire avec
\begin{lstlisting}
  \\[`\textit{longueur}']
\end{lstlisting}
  \item Espace insécable: \verb= ~ =
\begin{lstlisting}
M.~Tremblay
\end{lstlisting}

  \end{itemize}
\end{frame}

\begin{frame}[fragile]
  \frametitle{Sauts de page}
  \begin{itemize}
  \item Parfois nécessaires lors de coupures malheureuses
  \item Aussi pour placer des éléments où l'on veut
  \item Garder l'édition des sauts de page pour la toute fin de la
    rédaction
  \item<2-> Commandes
\begin{lstlisting}
\newpage
\clearpage
\cleartorecto              % memoir seulement
\cleartoverso              % memoir seulement
\end{lstlisting}
  \item<3-> Suggestions
\begin{lstlisting}
\pagebreak[`\textit{n}']              % n = 0, 1, 2, 3, 4
\enlargethispage{`\textit{longueur}'}
\end{lstlisting}
  \end{itemize}
\end{frame}

\begin{frame}[fragile]
  \frametitle{Longueurs}
  \begin{itemize}[<+->]
  \item Nombre positif, négatif ou nul \alert{obligatoirement} et
    \alert{immédiatement} suivi d'une unité de longueur (sans espace)
  \item Principales unités
    \begin{quote}
      \begin{tabularx}{\linewidth}{XcX}
        millimètre & \texttt{mm} \\
        centimètre & \texttt{cm} & (10~mm) \\
        pouce      & \texttt{in} & (2,54~cm) \\
        point      & \texttt{pt} & (1/72,27~pouce) \\
        largeur de la lettre M & \texttt{em} & (variable) \\
        hauteur de la lettre x & \texttt{ex} & (variable)
      \end{tabularx}
    \end{quote}
  \item Longueurs utiles prédéfinies
\begin{lstlisting}
\linewidth
\textwidth
\end{lstlisting}
  \end{itemize}
\end{frame}

\begin{frame}[fragile=singleslide]
  \frametitle{Coupure de mots}
  \begin{itemize}
  \item Coupure de mots en fin de ligne automatique avec \LaTeX
  \item Important d'indiquer à {\LaTeX} dans quelle langue est le texte!
    \begin{itemize}
    \item en anglais par défaut
    \item autrement spécifié au chargement de \textbf{babel}
    \end{itemize}
  \item Suggestions pour un mot individuel
\begin{lstlisting}
vrai\-sem\-blance
\end{lstlisting}
  \item Ajout d'exceptions ou de mots inconnus dans le préambule
\begin{lstlisting}
\hyphenation{puis-que,cons-tante}
\end{lstlisting}
  \end{itemize}
\end{frame}


\section{Portions de texte spéciales}

\begin{frame}[fragile=singleslide]
  \frametitle{Listes}
  \begin{itemize}
  \item Deux principales sortes de listes:
    \begin{enumerate}
    \item \alert{à puce} avec environnement \verb=itemize=
    \item \alert{numérotée} avec environnement \verb=enumerate=
    \end{enumerate}
  \item Possible de les imbriquer les unes dans les autres
  \item Marqueurs alors adaptés automatiquement
  \end{itemize}
\end{frame}

\begin{frame}[fragile]
  \frametitle{Code de la diapositive précédente}
\begin{lstlisting}
\begin{itemize}
\item Deux principales sortes de listes:
  \begin{enumerate}
  \item à puce avec environnement \verb=itemize=
  \item numérotée avec environnement \verb=enumerate=
  \end{enumerate}
\item Possible de les imbriquer les unes
  dans les autres
\item Marqueurs adaptés automatiquement
\end{itemize}
\end{lstlisting}
\end{frame}

\begin{frame}[fragile]
  \frametitle{Puce par défaut en français}
  \begin{itemize}
  \item Mode français de \textbf{babel} redéfinit la puce de 1{\ier}
    niveau par défaut de {\textbullet} à {\textemdash}
  \item Pour changer, utiliser dans le préambule
\begin{lstlisting}
\frenchbsetup{
  ItemLabeli=\`\textit{commande}',
  ItemLabelii=\`\textit{commande}'}
\end{lstlisting}
  \item Voir les ressources pour une vaste sélection de symboles
  \end{itemize}
\end{frame}

\begin{frame}[fragile=singleslide]
  \frametitle{Texte centré}
  \begin{center}
    Pour obtenir du texte centré on utilise l'environnement
    \verb=center=
  \end{center}

\begin{lstlisting}
\begin{center}
  Pour obtenir du texte centré on utilise
  l'environnement \verb=center=
\end{center}
\end{lstlisting}

\centering ou encore la commande \verb=\centering=

\begin{lstlisting}
\centering ou encore la commande \verb=\centering=
\end{lstlisting}
\end{frame}

\begin{frame}[fragile=singleslide]
  \frametitle{Citations}
  Deux environnements de citation dans {\LaTeX} (et \textbf{ulthese})
  \begin{enumerate}
  \item \verb=quote= pour les citations courtes, quelques lignes seulement
    \begin{itemize}
    \item retrait à gauche et à droite
    \end{itemize}
  \item \verb=quotation= pour les citations plus longues se comptant
    en paragraphes
    \begin{itemize}
    \item retrait à gauche et à droite
    \item gestion des marques de paragraphes
    \end{itemize}
  \end{enumerate}
\end{frame}

\begin{frame}[fragile]
  \frametitle{Notes de bas de page}
  \begin{itemize}
  \item Note de bas de page insérée avec la commande
\begin{lstlisting}
\footnote{`\textit{texte de la note}'}
\end{lstlisting}
  \item Commande doit suivre immédiatement le texte à annoter
  \item Méthode recommandée
\begin{lstlisting}[emph=footnote]
... fera remarquer que Pierre Lasou\footnote{%
  Spécialiste en ressources documentaires} %
fut d'une grande aide dans la préparation de ...
\end{lstlisting}
  \item Numérotation et disposition automatiques
  \end{itemize}
\end{frame}

\begin{frame}[fragile=singleslide]
  \frametitle{Code source}
  \begin{itemize}
  \item Environnement \verb=verbatim=
\begin{lstlisting}
\begin{verbatim}
Texte disposé exactement tel qu'il est tapé
dans une police à largeur fixe
\end{verbatim}
\end{lstlisting}
  \item Commande \verb=\verb= dont la syntaxe est
\begin{lstlisting}
\verb`\textit{c}' `\textit{source}' `\textit{c}'
\end{lstlisting}
    où \textit{c} est un caractère quelconque ne se trouvant pas dans
    \textit{source}
  \item Pour usage plus intensif, voir le paquetage \textbf{listings}
  \end{itemize}
\end{frame}

%%% >>>
\begin{frame}[plain]
  \begin{exercice}
    \begin{enumerate}
    \item Ouvrir le fichier \fichier{exercice\_complet.tex} et en
      étudier le code source, puis le compiler.
    \item En comparant le résultat avec le fichier produit avec le
      fichier \fichier{exercice\_tdm+annexes.tex}, déterminer l'effet
      de l'option \texttt{article} dans la classe.
    \item Effectuer les modifications suivantes au document.
      \begin{enumerate}[a)]
      \item Dernier paragraphe de la première section, placer toute la
        phrase débutant par \texttt{«De simple dérivé»} à l'intérieur
        d'une commande \texttt{{\textbackslash}emph}.
      \item Changer la puce des listes pour le caractère
        \texttt{\$>\$}.
      \end{enumerate}
    \end{enumerate}
  \end{exercice}
\end{frame}
%%% <<<



\section{B.a.-ba du mode mathématique}

\begin{frame}[fragile=singleslide]
  \frametitle{Préliminaires}
  \begin{itemize}
  \item Décrire des équations mathématiques requiert un «langage» spécial
    \begin{itemize}
    \item il faut informer {\LaTeX} que l'on passe à ce langage
    \item par le biais de modes mathématiques
    \end{itemize}
  \item Important d'utiliser un mode mathématique
    \begin{itemize}
    \item règles de typographie spéciales (constantes vs variables,
      disposition des équations, numérotation, etc.)
    \item espaces entre les symboles et autour des opérateurs gérées
      automatiquement
    \end{itemize}
  \item Vous voulez utiliser le paquetage \textbf{amsmath}
\begin{lstlisting}
\usepackage{amsmath}
\end{lstlisting}
    \begin{itemize}
    \item lire la documentation de ce paquetage pour connaître toutes
      ses fonctionnalités
    \end{itemize}
  \end{itemize}
\end{frame}

\begin{frame}[fragile]
  \frametitle{Modes mathématiques}
  \begin{enumerate}[<+->]
  \item «En ligne» directement dans le texte comme $(a + b)^2 = a^2 +
    2ab + b^2$ en plaçant l'équation entre \verb=$ $=
\begin{lstlisting}
«En ligne» directement dans le texte
comme $(a + b)^2 = a^2 + 2ab + b^2$
\end{lstlisting}
  \item «Hors paragraphe» séparé du texte principal comme
    \begin{displaymath}
      \int_0^\infty f(x)\, dx = \sum_{i = 1}^n \alpha_i e^{x_i} f(x_i)
    \end{displaymath}
    en utilisant divers types d'environnements
\begin{lstlisting}
«Hors paragraphe» séparé du texte principal comme
\begin{displaymath}
  \int_0^\infty f(x)\, dx =
  \sum_{i = 1}^n \alpha_i e^{x_i} f(x_i)
\end{displaymath}
\end{lstlisting}
  \end{enumerate}
\end{frame}

\begin{frame}[plain]
  \begin{alertblock}{Conseil du {\TeX}pert}
    Les équations, en ligne ou hors paragraphe, font partie intégrante
    de la phrase.

    \bigskip %
    Les règles de ponctuation usuelles s'appliquent donc aux
    équations.
  \end{alertblock}
  \vspace{18pt}
  \fbox{\includegraphics[width=\linewidth]{ponctuation}}
\end{frame}

\begin{frame}[fragile]
  \frametitle{Quelques règles de base}
  \begin{itemize}
  \item En mode mathématique, {\TeX} respecte automatiquement la
    convention d'écrire les constantes en \rmfamily{romain} et les
    variables en \textit{italique}
    \begin{quote}
      \begin{minipage}{0.45\linewidth}
\begin{lstlisting}[aboveskip=2.3\medskipamount]
$z = 2a + 3y$
\end{lstlisting}
      \end{minipage}
      \hfill
      \begin{texoutput}{0.45\linewidth}
        $z = 2a + 3y$
      \end{texoutput}
    \end{quote}
  \item Espace entre les éléments géré automatiquement, peu importe le
    code source
\begin{quote}
      \begin{minipage}{0.45\linewidth}
\begin{lstlisting}[aboveskip=2.3\medskipamount]
$z=2 a+3 y$
\end{lstlisting}
      \end{minipage}
      \hfill
      \begin{texoutput}{0.45\linewidth}
        $z=2 a+3 y$
      \end{texoutput}
    \end{quote}
  \end{itemize}
\end{frame}

\begin{frame}[fragile]
  \frametitle{Quelques règles de base (suite)}
  \begin{itemize}
  \item \alert{Ne pas} utiliser le mode mathématique pour obtenir du
    texte en italique!
    \begin{quote}
      \begin{minipage}{0.45\linewidth}
\begin{lstlisting}[aboveskip=2.3\medskipamount]
\emph{xyz}
\end{lstlisting}
      \end{minipage}
      \hfill
      \begin{texoutput}{0.45\linewidth}
        \rmfamily\emph{xyz}
      \end{texoutput} \\
      \begin{minipage}{0.45\linewidth}
\begin{lstlisting}[aboveskip=2.3\medskipamount]
$xyz$
\end{lstlisting}
      \end{minipage}
      \hfill
      \begin{texoutput}{0.45\linewidth}
        $xyz$
      \end{texoutput}
    \end{quote}
  \item Utiliser la commande \verb=\text{}= de \textbf{amsmath} pour
    obtenir du texte à l'intérieur du mode mathématique
\begin{quote}
      \begin{minipage}{0.55\linewidth}
\begin{lstlisting}[aboveskip=2.3\medskipamount]
$x = 0 \text{ si } y < 2$
\end{lstlisting}
      \end{minipage}
      \hfill
      \begin{texoutput}{0.35\linewidth}
          $x = 0 \text{ si } y < 2$
      \end{texoutput}
    \end{quote}
  \end{itemize}
\end{frame}

% \begin{frame}[fragile]
%   \frametitle{Exposants et indices}
%   \begin{itemize}
%   \item Utiliser \verb=^= pour mettre le caractère suivant en exposant
%   \item Utiliser \verb=_= pour mettre le caractère suivant en indice
%   \item Pour plus d'un caractère, regrouper entre \verb={ }=
%   \item Toutes les combinaisons possibles
%   \end{itemize}
%   \pause
%   \begin{quote}
%     \begin{minipage}{0.2\linewidth}
% \begin{lstlisting}[aboveskip=2.3\medskipamount]
% $x^2$
% \end{lstlisting}
%     \end{minipage}
%     \quad
%     \begin{texoutput}{0.2\linewidth}
%         $x^2$
%       \end{texoutput}
%     \hfill
%     \begin{minipage}{0.2\linewidth}
% \begin{lstlisting}[aboveskip=2.3\medskipamount]
% $x_4$
% \end{lstlisting}
%     \end{minipage}
%     \quad
%     \begin{texoutput}{0.2\linewidth}
%       $x_4$
%     \end{texoutput}
%     \\
%     \begin{minipage}{0.2\linewidth}
% \begin{lstlisting}[aboveskip=2.3\medskipamount]
% $x^{2n}$
% \end{lstlisting}
%     \end{minipage}
%     \quad
%     \begin{texoutput}{0.2\linewidth}
%       $x^{2n}$
%     \end{texoutput}
%     \hfill
%     \begin{minipage}{0.2\linewidth}
% \begin{lstlisting}[aboveskip=2.3\medskipamount]
% $x^{y^2}$
% \end{lstlisting}
%     \end{minipage}
%     \quad
%     \begin{texoutput}{0.2\linewidth}
%       $x^{y^2}$
%     \end{texoutput}
%     \\
%     \begin{minipage}{0.5\linewidth}
% \begin{lstlisting}[aboveskip=2.3\medskipamount]
% $A_{j_{(k, l)}}^{x_i^2}$
% \end{lstlisting}
%     \end{minipage}
%     \quad
%     \begin{texoutput}{0.3\linewidth}
%       $A_{j_{(k, l)}}^{x_i^2}$
%     \end{texoutput}
%   \end{quote}
% \end{frame}

% \begin{frame}[fragile]
%   \frametitle{Fractions}
%   \begin{itemize}
%   \item Pour les équations en ligne, utiliser tout simplement la barre
%     oblique \verb=/=
%     \begin{quote}
%       \begin{minipage}{0.5\linewidth}
% \begin{lstlisting}[aboveskip=2.3\medskipamount]
% On a $y = (n + m)/2$.
% \end{lstlisting}
%       \end{minipage}
%       \quad
%       \begin{texoutput}{0.4\linewidth}
%         \rmfamily On a $y = (n + m)/2$.
%       \end{texoutput}
%     \end{quote}
%   \item Pour écrire le numérateur au-dessus du dénominateur, utiliser
%     la commande \verb=\frac{}{}=
%     \begin{quote}
%       \begin{minipage}{0.75\linewidth}
% \begin{lstlisting}[aboveskip=2.3\medskipamount]
% $\frac{x}{2}$
% \end{lstlisting}
%       \end{minipage}
%       \quad
%       \begin{texoutput}{0.15\linewidth}
%         $\dfrac{x}{2}$
%       \end{texoutput}
%       \\
%       \begin{minipage}{0.75\linewidth}
% \begin{lstlisting}[aboveskip=2.3\medskipamount]
% $\frac{x_i + \frac{1}{2}}{y^2 + k}$
% \end{lstlisting}
%       \end{minipage}
%       \quad
%       \begin{texoutput}{0.15\linewidth}
%         $\dfrac{x_i + \frac{1}{2}}{y^2 + k}$
%       \end{texoutput}
%     \end{quote}
%   \end{itemize}
% \end{frame}

% \begin{frame}[fragile]
%   \frametitle{Lettres grecques \\ et autres symboles mathématiques}

%   La plupart des symboles mathématiques sont invoqués par une commande
%   dont le nom correspond au nom ou à la signification mathématique du
%   symbole

%   \begin{quote}
%     \begin{minipage}{0.55\linewidth}
% \begin{lstlisting}[aboveskip=2.3\medskipamount]
% $\mu + \sigma^2 + \gamma$
% \end{lstlisting}
%     \end{minipage}
%     \quad
%     \begin{texoutput}{0.35\linewidth}
%       $\mu + \sigma^2 + \gamma$
%     \end{texoutput}
%     \\
%     \begin{minipage}{0.55\linewidth}
% \begin{lstlisting}[aboveskip=2.3\medskipamount]
% $\Lambda + \Delta + \Gamma$
% \end{lstlisting}
%     \end{minipage}
%     \quad
%     \begin{texoutput}{0.35\linewidth}
%       $\Lambda + \Delta + \Gamma$
%     \end{texoutput}
%     \\
%     \begin{minipage}{0.55\linewidth}
% \begin{lstlisting}[aboveskip=2.3\medskipamount]
% $\times \div \leq \geq \neq$
% \end{lstlisting}
%     \end{minipage}
%     \quad
%     \begin{texoutput}{0.35\linewidth}
%       $\times \div \leq \geq \neq$
%     \end{texoutput}
%     \\
%     \begin{minipage}{0.55\linewidth}
% \begin{lstlisting}[aboveskip=2.3\medskipamount]
% $\dots \cdots \cdot$
% \end{lstlisting}
%     \end{minipage}
%     \quad
%     \begin{texoutput}{0.35\linewidth}
%       $\cdot \dots \cdots$
%     \end{texoutput}
%     \\
%     \begin{minipage}{0.71\linewidth}
% \begin{lstlisting}[aboveskip=2.3\medskipamount]
% $\sum_{x = 0}^\infty \int_0^x y\, dy$
% \end{lstlisting}
%     \end{minipage}
%     \quad
%     \begin{texoutput}{0.15\linewidth}
%       $\displaystyle \sum_{x = 0}^\infty \int_0^x y\, dy$
%     \end{texoutput}
%   \end{quote}
% \end{frame}

\begin{frame}[fragile=singleslide]
  \frametitle{Environnements pour les équations hors paragraphe}
  \begin{itemize}
  \item Équations d'une seule ligne \newline
    \begin{quote}
      \begin{minipage}[t]{0.4\linewidth}
        \begin{block}{numérotées}
\begin{lstlisting}
equation
\end{lstlisting}
        \end{block}
      \end{minipage}
      \hfill
      \begin{minipage}[t]{0.4\linewidth}
        \begin{block}{non numérotées}
\begin{lstlisting}
displaymath
equation*
\end{lstlisting}
        \end{block}
      \end{minipage}
    \end{quote}
  \item Séries d'équations alignées, généralement sur \verb|=|
    \begin{quote}
      \begin{minipage}[t]{0.4\linewidth}
        \begin{block}{numérotées}
\begin{lstlisting}
align
\end{lstlisting}
        \end{block}
      \end{minipage}
      \hfill
      \begin{minipage}[t]{0.4\linewidth}
        \begin{block}{non numérotées}
\begin{lstlisting}
align*
\end{lstlisting}
        \end{block}
      \end{minipage}
    \end{quote}
  \end{itemize}
\end{frame}

\begin{frame}[fragile]
  \frametitle{Avant-goût}

  Pouvez-vous interpréter ce code?
\begin{lstlisting}
\begin{displaymath}
  \Gamma(\alpha) =
  \sum_{j = 0}^\infty \int_j^{j + 1}
    x^{\alpha - 1} e^{-x}\, dx
\end{displaymath}
\end{lstlisting}
  \vspace{18pt}
  \pause

  Fort probablement!
  \begin{displaymath}
    \Gamma(\alpha) =
    \sum_{j = 0}^\infty \int_j^{j + 1} x^{\alpha - 1} e^{-x}\, dx
  \end{displaymath}
\end{frame}


\section{Classe ulthese}

\begin{frame}
  \frametitle{Un document conforme en un tournemain}
  \begin{itemize}
  \item \textbf{ulthese} livrée dans {\TeX}~Live donc déjà
    installée sur votre ordinateur
  \item Mise en forme conforme aux règles de présentation de la FESP
  \item Basée sur la classe \textbf{memoir}, donc les fonctionnalités
    de celle-ci sont disponibles dans \textbf{ulthese}
  \item Quelques nouvelles commandes pour la création de la page titre
  \item Partir d'un gabarit (classés avec la documentation dans
    {\TeX}~Live)
  \item Utiliser des fichiers séparés pour chaque chapitre du mémoire
    ou de la thèse
  \end{itemize}
\end{frame}

%%% >>>
\begin{frame}[plain,fragile=singleslide]
  \begin{exercice}
    Utiliser le fichier \fichier{exercice\_ulthese.tex} --- qui est
    basé sur le gabarit \fichier{gabarit-doctorat.tex} livré avec
    \textbf{ulthese}.
    \begin{enumerate}
    \item Étudier le code source du fichier.

      Remarquer que le \fichier{mathematiques.tex} est inséré dans le
      document avec la commande \verb=\include=. Étudier brièvement le
      code source de ce fichier.
    \item Activer les paquetages \textbf{amsmath} et \textbf{icomma},
      puis compiler \fichier{exercice\_ulthese.tex}.
    \item Modifier un environnement \texttt{align*} pour
      \texttt{align} dans \fichier{mathematiques.tex} et observer le
      résultat dans la compilation de \fichier{exercice\_ulthese.tex}.
    \item Compiler de nouveau le fichier en utilisant une police de
      caractères différente.
    \end{enumerate}
  \end{exercice}
\end{frame}
%%% <<<


\section{Ressources}

\begin{frame}
  \frametitle{Quelques essentiels}
  \begin{itemize}
  \item Un bon livre de référence
    \begin{quote}
      \href{http://ariane.ulaval.ca/cgi-bin/recherche.cgi?qu=a1982705}{%
        \emph{Guide to {\LaTeX}}, 4{\ieme} éd., de H.~Kopka et P.W.~Daly}
    \end{quote}
  \item Foire aux questions bien garnie
    \begin{quote}
      \href{http://www.tex.ac.uk/cgi-bin/texfaq2html}{%
        \emph{UK List of {\TeX} Frequently Asked Questions}}
    \end{quote}
  \item Forum de discussion très actif
    \begin{quote}
      \href{http://tex.stackexchange.com}{%
        {\TeX}--{\LaTeX} Stack Exchange}
    \end{quote}
  \item Liste de symboles disponibles dans {\LaTeX}
    \begin{quote}
      \href{http://www.ctan.org/tex-archive/info/symbols/comprehensive/}{%
        \emph{The Comprehensive {\LaTeX} Symbol List}} \\
      (aussi fournie avec {\TeX}~Live)
    \end{quote}
  \end{itemize}
\end{frame}

\end{document}

%%% Local Variables:
%%% TeX-engine: xetex
%%% mode: latex
%%% TeX-master: t
%%% End:
