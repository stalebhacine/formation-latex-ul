\chapter{Trucs et astuces divers}
\label{chap:trucs}


\section{Polices de caractères: au-delà de Computer Modern}
\label{sec:trucs:police}

{\CM%
  Les documents {\LaTeX} standards sont facilement reconnaissables par
  leur police de caractères par défaut --- celle utilisée dans ce
  paragraphe ---, Computer Modern. Pour qui souhaitait briser la relative
  monotonie induite par cette uniformité, il a longtemps été difficile
  d'utiliser une autre police de caractères. Fort heureusement, la
  situation a beaucoup évolué et il est aujourd'hui assez simple de
  produire des documents {\LaTeX} utilisant des polices de caractères
  variées.}

Avant d'aller plus loin, une mise en garde: si votre document contient
plus que quelques équations mathématiques très simples, le choix de
police devient très restreint. La raison: peu de polices de caractères
comprennent des symboles mathématiques et les informations nécessaires
pour les assembler dans une mise en page satisfaisant les standards de
haute qualité usuels de {\LaTeX}.

Cela dit, pour qui souhaite aller au-delà de la police Computer Modern
sans trop se compliquer la vie, il existe deux solutions principales.

\begin{enumerate}
\item Utiliser l'une ou l'autre des polices PostScript standards.
  C'est très simple avec toute distribution {\LaTeX} moderne: il
  suffit de charger le paquetage approprié. Consulter la %
  \doc{psnfss2e}{http://texdoc.net/pkg/psnfss/} %
  de l'ensemble de paquetages PSNFSS pour connaître les choix
  disponibles.
\item Utiliser une police OpenType présente sur son système avec le
  moteur {\XeLaTeX}. Seule une poignée de ces polices offrent
  toutefois un support approprié pour les mathématiques. La gestion
  des polices de caractères avec {\XeLaTeX} se fait avec le paquetage
  standard \pkg{fontspec}; consulter sa %
  \doc{fontspec}{http://texdoc.net/pkg/fontspec/}.
\end{enumerate}

Pour les thèses et mémoires de l'Université Laval, la Faculté des
études supérieures et postdoctorales accepte les polices %
\begin{quote}
  \begin{tabbing}
    Computer Modern \qquad \=  ABCDEF abcdef 1234567890 \kill \\
    {\CM Computer Modern} \> {\CM ABCDEF abcdef 1234567890} \\
    {\Times Times} \> {\Times ABCDEF abcdef 1234567890} \\
    {\Palatino Palatino} \> {\Palatino ABCDEF abcdef 1234567890}
  \end{tabbing}
\end{quote}
Pour utiliser ces deux dernières avec {\LaTeX}, on charge
respectivement les paquetages \pkg{mathptmx} ou \pkg{mathpazo}. Avec
{\XeLaTeX}, on utilisera les polices Termes et Pagella du projet %
\link{http://www.gust.org.pl/projects/e-foundry/tex-gyre/}{TeX~Gyre}.
Ce sont des polices très similaires à Times et Palatino, disponibles
en version OpenType et qui fournissent un bon support pour les
mathématiques via le projet frère %
\link{http://www.gust.org.pl/projects/e-foundry/tg-math/}{TeX~Gyre Math}.

\begin{exemple}
  Pour utiliser la police PostScript classique Palatino avec {\LaTeX}
  tant pour le texte que pour les mathématiques, il suffit d'insérer
  dans le préambule de son document la commande
\begin{lstlisting}
\usepackage{mathpazo}
\end{lstlisting}

  Avec le moteur {\XeLaTeX}, il est possible d'utiliser n'importe
  quelle police de caractères OpenType et TrueType installée dans le
  système d'exploitation de l'ordinateur. Pour obtenir un résultat
  équivalent à celui de \pkg{mathpazo}, on installe les polices
  TeX~Gyre dans le système, puis on insère dans le préambule les
  commandes
\begin{lstlisting}
\usepackage{fontspec}
\setmainfont{TeX Gyre Pagella}
\setmathfont{TeX Gyre Pagella Math}
\end{lstlisting}
  \qed
\end{exemple}

Le texte principal du présent document est en %
\link{http://tug.org/store/lucida/}{Lucida Bright~OT}, %
une police commerciale de très haute qualité offrant également un
excellent support pour les mathématiques. Ses auteurs ont toujours été
proches de la communauté {\LaTeX}. La Bibliothèque de l'Université
Laval détient une licence d'utilisation de cette police. Les étudiants
et le personnel de l'Université peuvent s'en produrer une copie
gratuitement en écrivant à
\href{mailto:lucida@bibl.ulaval.ca}{lucida@bibl.ulaval.ca}.



\section{Couleurs}
\label{sec:trucs:couleurs}

L'utilisation de couleur dans un document {\LaTeX} requiert de charger
le paquetage \pkg{xcolor} \citep{xcolor}. Celui-ci définit d'abord
plusieurs couleurs que l'on peut utiliser directement; consulter la %
\doc{xcolor}{http://texdoc.net/pkg/xcolor} %
pour en connaître les différentes listes. Le
\autoref{tab:trucs:couleurs} fournit celle des couleurs toujours
disponibles.

\begin{table}
  \centering
  \caption{Couleurs toujours disponibles quand le paquetage
    \pkg{xcolor} est chargé}
  \label{tab:trucs:couleurs}
  \begin{tabularx}{1.0\linewidth}{XlXll}
    \toprule
    \fcolorbox{black}{black}{\phantom{xx}}\, black &
    \fcolorbox{black}{blue}{\phantom{xx}}\, blue &
    \fcolorbox{black}{brown}{\phantom{xx}}\, brown &
    \fcolorbox{black}{cyan}{\phantom{xx}}\, cyan &
    \fcolorbox{black}{darkgray}{\phantom{xx}}\, darkgray \\
    \addlinespace[3pt]
    \fcolorbox{black}{gray}{\phantom{xx}}\, gray &
    \fcolorbox{black}{green}{\phantom{xx}}\, green &
    \fcolorbox{black}{lightgray}{\phantom{xx}}\, lightgray &
    \fcolorbox{black}{lime}{\phantom{xx}}\, lime &
    \fcolorbox{black}{magenta}{\phantom{xx}}\, magenta \\
    \addlinespace[3pt]
    \fcolorbox{black}{olive}{\phantom{xx}}\, olive &
    \fcolorbox{black}{orange}{\phantom{xx}}\, orange &
    \fcolorbox{black}{pink}{\phantom{xx}}\, pink &
    \fcolorbox{black}{purple}{\phantom{xx}}\, purple &
    \fcolorbox{black}{red}{\phantom{xx}}\, red \\
    \addlinespace[3pt]
    \fcolorbox{black}{teal}{\phantom{xx}}\, teal &
    \fcolorbox{black}{violet}{\phantom{xx}}\, violet &
    \fcolorbox{black}{white}{\phantom{xx}}\, white &
    \fcolorbox{black}{yellow}{\phantom{xx}}\, yellow \\
    \bottomrule
  \end{tabularx}
\end{table}

On modifie la couleur du texte avec la commande
\begin{lstlisting}
\color`\marg{nom}'
\end{lstlisting}
où \meta{nom} est le nom d'une couleur. À moins de vouloir modifier la
couleur de tout ce qui suit, on limitera la portée de la commande par
des accolades.
\begin{demo}
  \begin{texample}
\begin{lstlisting}
texte {\color{red} en rouge}
et {\color{blue} en bleu}
\end{lstlisting}
  \producing
  texte {\color{red} en rouge} et {\color{blue} en bleu}
  \end{texample}
\end{demo}

La commande \cmd{\definecolor} permet de définir de nouvelles couleurs
selon plusieurs systèmes de codage. Le plus usuel demeure \emph{Rouge,
  vert, bleu} (RVB ou RGB, en anglais) où une couleur est représentée
par une combinaison de teintes --- exprimées par un nombre entre $0$
et $1$ --- de rouge, de vert et de bleu. Dans ce cas, la syntaxe de
\cmd{\definecolor} est
\begin{lstlisting}
\definecolor`\marg{nom}'{rgb}`\marg{valeur\_r,valeur\_v,valeur\_b}
\end{lstlisting}
où \meta{valeur\_r}, \meta{valeur\_v} et \meta{valeur\_b} sont
respectivement les teintes de rouge, de vert et de bleu.

\begin{exemple}
  La commande
\begin{lstlisting}
\definecolor{acier}{rgb}{0.1,0.4,0.6}
\end{lstlisting}
  définit une nouvelle couleur nommée \meta{acier} composée de rouge
  30~\%, de vert 40~\% et de bleu 60~\%: %
  \fcolorbox{black}[rgb]{0.3,0,0}{\phantom{xx}} $+$ %
  \fcolorbox{black}[rgb]{0,0.4,0}{\phantom{xx}} $+$ %
  \fcolorbox{black}[rgb]{0,0,0.6}{\phantom{xx}} $=$ %
  \fcolorbox{black}[rgb]{0.3,0.4,0.6}{\phantom{xx}}. %
  On pourra utiliser la couleur \meta{acier} directement dans la
  commande \cmdprint{\color}. %
  \qed
\end{exemple}

La commande \cmd{\colorlet}, dont la syntaxe simplifiée est
\begin{lstlisting}
\colorlet`\marg{nom}\marg{couleur}'
\end{lstlisting}
permet de faire référence à la \meta{couleur} déjà existante par
\meta{nom}. C'est pratique pour assigner un nom sémantique à une
couleur.



\section{Hyperliens et métadonnées de documents PDF}
\label{sec:trucs:hyperliens}

Nous en avons déjà traité à quelques reprises, notamment à la
section~4 de \citet{UL:latex:1}, le paquetage \pkg{hyperref}
\citep{hyperref} permet de transformer toutes les références dans le
texte en hyperliens cliquables lorsque le document est produit avec
pdf{\LaTeX} ou {\XeLaTeX}. C'est très pratique lors de la consultation
électronique d'un document.

Le paquetage offre une multitudes d'options de configuration; nous
n'en présenterons que quelques unes. On accède aux options de
configuration de \pkg{hyperref} via la commande \cmd{\hypersetup} dans
le préambule. Celle-ci prend en arguments des paires
\code{option=valeur} séparées par des virgules.

Une des principales choses que l'on pourra souhaiter configurer, c'est
le comportement et la couleur des divers types d'hyperliens. On
trouvera ci-dessous les options de configuration pertinentes, leur
valeur (avec en gras la valeur par défaut) ainsi qu'une brève
explication de chacune.

\begin{table}[h]
  \begin{tabularx}{1.0\linewidth}{@{}p{6em}p{6em}X@{}}
    \code{colorlinks} & \code{true}|\code{\textbf{false}} & colorer les
                                                            liens selon
                                                            leur type \\
    \code{linktocpage} & \code{true}|\code{\textbf{false}} & faire du
                                                             numéro de
                                                             page
                                                             plutôt que
                                                             du titre l'hyperlien dans
                                                             la table
                                                             des
                                                             matières \\
    \code{linkcolor} & \meta{couleur} & couleur des liens internes \\
    \code{urlcolor}  & \meta{couleur} & couleur des URL externes \\
    \code{citecolor} & \meta{couleur} & couleur des citations \\
    \code{allcolor}  & \meta{couleur} & couleur pour tous les types d'hyperliens
  \end{tabularx}
\end{table}

Lorsque la valeur admissible d'une option est \code{true} ou
\code{false}, sa seule mention équivaut à \code{true}. La valeur
\meta{couleur} est le nom d'une couleur telle que définie par
\pkg{xcolor} (\autoref{sec:trucs:couleurs}).

Les fichiers PDF peuvent contenir diverses métadonnées sur leur
contenu. Le paquetage \pkg{hyperref} permet de définir certaines
catégories de celles-ci, notamment les informations de document comme
le titre ou l'auteur. Nous ne mentionnons ici que deux options de
configutation; la section~3.7 de la %
\doc{hyperref}{http://texdoc.net/pkg/hyperref} %
contient la liste complète.

\begin{table}[h]
  \begin{tabularx}{1.0\linewidth}{@{}p{6em}p{6em}X@{}}
    \code{pdftitle}  & texte & titre du document PDF \\
    \code{pdfauthor} & texte & auteur du document PDF
  \end{tabularx}
\end{table}

\begin{exemple}
  Le présent document a été préparé avec les définitions de couleurs
  et les options de configuration de \pkg{hyperref} suivantes:
\begin{lstlisting}
\definecolor{link}{rgb}{0,0.4,0.6}   % ~RoyalBlue de dvips
\definecolor{url}{rgb}{0.6,0,0}      % rouge-brun
\definecolor{citation}{rgb}{0,0.5,0} % vert foncé
\hypersetup{colorlinks, linktocpage,
  linkcolor=link, urlcolor=url, citecolor=citation,
  pdftitle={Rédaction de thèses et mémoires avec LaTeX -
            Partie II},
  pdfauthor={Université Laval}}
\end{lstlisting}
  \qed
\end{exemple}

\begin{exemple}
  Les gabarits de thèses et de mémoires livrés avec la classe
  \class{ulthese} contiennent dans le préambule la ligne suivante:
\begin{lstlisting}
\hypersetup{colorlinks,allcolors=ULlinkcolor}
\end{lstlisting}
  Tous les liens de la thèse ou du mémoire seront donc de la même
  couleur, \code{ULlinkcolor} \fcolorbox{black}{ULlinkcolor}{\phantom{xx}},
  une couleur définie par la classe. %
  \qed
\end{exemple}

\begin{conseil}
  L'interaction du paquetage \pkg{hyperref} avec les autres est
  parfois (voire souvent) délicate. Pour cette raison, il est
  fortement recommandé de charger \pkg{hyperref} en tout dernier
  dans dans le préambule.
\end{conseil}



\section{Présentation de code informatique}
\label{sec:trucs:listings}

L'environnement standard \Ie{verbatim} de {\LaTeX} permet de présenter
du texte exactement tel qu'il est entré dans le code source du
document. C'est un environnement particulièrement utile pour afficher
du code informatique puisque le texte est composé en police non
proportionnelle et que sa disposition exacte est respectée.

\begin{demo}
  \begin{texample}
\begin{lstlisting}
\begin{verbatim}
/* Hello World en C */
#include <stdio.h>

int main()
{
    printf("Hello world\n");
    return 0;
}
\end{verbatim}
\end{lstlisting}
    \producing
\begin{verbatim}
/* Hello World en C */
#include <stdio.h>

int main()
{
    printf("Hello world\n");
    return 0;
}
\end{verbatim}
  \end{texample}
\end{demo}

Si un document doit contenir beaucoup de code informatique et que l'on
souhaite exercer un fin contrôle sur sa disposition et sa mise en
forme, il vaut mieux se tourner vers un paquetage spécialisé comme
\pkg{listings} \citep{listings}. La %
\doc{listings}{http://texdoc.net/pkg/listings} %
du paquetage compare ses fonctionnalités à celles de plusieurs autres
paquetages similaires.

Le paquetage \pkg{listings} peut effectuer automatiquement le marquage
des mots-clés de nombreux langages de programmation, ajouter des
numéros de ligne, importer du code de fichiers externes ou même
indexer les mots-clés des extraits de code. À titre d'illustration, en
utilisant l'environnement \Ie{lstlisting} de \pkg{listings} plutôt que
\Ie{verbatim}, l'extrait de code C ci-dessus pourrait se présenter
ainsi:
\begin{demo}
  \begin{texample}
    \begin{vglisting}
\begin{lstlisting}
/* Hello World en C */
#include <stdio.h>

int main()
{
    printf("Hello world\n");
    return 0;
}
\end{lstlisting}
    \end{vglisting}
    \producing
\begin{lstlisting}[language=C,%
  frame=single,%
  backgroundcolor=\color{Azure2},%
  rulecolor=\color{black},%
  numbers=left,%
  numberstyle=\tiny\sffamily,%
  stringstyle=\color{orange}\itshape,%
  identifierstyle=\color{cyan}\mdseries,%
  xleftmargin=12pt,%
  keywordstyle=\color{blue}\bfseries]
/* Hello World en C */
#include <stdio.h>

int main()
{
    printf("Hello world\n");
    return 0;
}
\end{lstlisting}
  \end{texample}
\end{demo}

Il serait trop long et nettement hors du mandat du présent document
d'expliquer les nombreuses fonctionnalités de \pkg{listings}.
Précisons simplement que le paquetage a servi pour en composer les
extraits de code {\LaTeX} et pour construire une grande partie de
l'index.

\begin{exemple}
  Pour parvenir à la présentation des extraits de code source {\LaTeX}
  de ce document, le paquetage \pkg{listings} est configuré dans le
  préambule de la manière suivante:
\begin{lstlisting}
%% Couleurs
\definecolor{comments}{rgb}{0.7,0,0}

%% Configuration de listings
\lstset{language=[LaTeX]TeX,
  extendedchars=true,
  commentstyle=\color{comments}\slshape,
  showstringspaces=false,
  index=[1][keywords],
  backgroundcolor=\color{LightYellow1},
  frame=lr,
  rulecolor=\color{LightYellow1},
  xleftmargin=3.4pt,
  xrightmargin=3.4pt}
\end{lstlisting}
  (La couleur \code{LightYellow1} est définie par \pkg{xcolor} lorsque
  le paquetage est chargé avec l'option \code{x11names}.)
  \qed
\end{exemple}

% Analyse et rapport intégrés [de type Sweave]

% Diapositives

% Contrôle de version

%%% Local Variables:
%%% mode: latex
%%% TeX-engine: xetex
%%% TeX-master: "formation_latex-partie_2"
%%% coding: utf-8
%%% End:
