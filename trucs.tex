\chapter{Trucs et astuces divers}
\label{chap:trucs}


\section{Polices de caractères: au-delà de Computer Modern}
\label{sec:trucs:police}

{\CM%
  Les documents {\LaTeX} standards sont facilement reconnaissables par
  leur police de caractères par défaut --- celle utilisée dans ce
  paragraphe ---, Computer Modern. Pour qui souhaitait briser la relative
  monotonie induite par cette uniformité, il a longtemps été difficile
  d'utiliser une autre police de caractères. Fort heureusement, la
  situation a beaucoup évolué et il est aujourd'hui assez simple de
  produire des documents {\LaTeX} utilisant des polices de caractères
  variées.}

Avant d'aller plus loin, une mise en garde: si votre document contient
plus que quelques équations mathématiques très simples, le choix de
police devient très restreint. La raison: peu de polices de caractères
comprennent des symboles mathématiques et les informations nécessaires
pour les assembler dans une mise en page satisfaisant les standards de
haute qualité usuels de {\LaTeX}.

Cela dit, pour qui souhaite aller au-delà de la police Computer Modern
sans trop se compliquer la vie, il existe deux solutions principales.

\begin{enumerate}
\item Utiliser l'une ou l'autre des polices PostScript standards.
  C'est très simple avec toute distribution {\LaTeX} moderne: il
  suffit de charger le paquetage approprié. Consulter la %
  \doc{psnfss2e}{http://texdoc.net/pkg/psnfss/} %
  de l'ensemble de paquetages PSNFSS pour connaître les choix
  disponibles.
\item Utiliser une police OpenType présente sur son système avec le
  moteur {\XeLaTeX}. Seule une poignée de ces polices offrent
  toutefois un support approprié pour les mathématiques. La gestion
  des polices de caractères avec {\XeLaTeX} se fait avec le paquetage
  standard \pkg{fontspec}; consulter sa %
  \doc{fontspec}{http://texdoc.net/pkg/fontspec/}.
\end{enumerate}

Pour les thèses et mémoires de l'Université Laval, la Faculté des
études supérieures et postdoctorales accepte les polices %
\begin{quote}
  \begin{tabbing}
    Computer Modern \qquad \=  ABCDEF abcdef 1234567890 \kill \\
    {\CM Computer Modern} \> {\CM ABCDEF abcdef 1234567890} \\
    {\Times Times} \> {\Times ABCDEF abcdef 1234567890} \\
    {\Palatino Palatino} \> {\Palatino ABCDEF abcdef 1234567890}
  \end{tabbing}
\end{quote}
Pour utiliser ces deux dernières avec {\LaTeX}, on charge
respectivement les paquetages \pkg{mathptmx} ou \pkg{mathpazo}. Avec
{\XeLaTeX}, on utilisera les polices Termes et Pagella du projet %
\link{http://www.gust.org.pl/projects/e-foundry/tex-gyre/}{TeX~Gyre}.
Ce sont des polices très similaires à Times et Palatino, disponibles
en version OpenType et qui fournissent un bon support pour les
mathématiques via le projet frère %
\link{http://www.gust.org.pl/projects/e-foundry/tg-math/}{TeX~Gyre Math}.

\begin{exemple}
  Pour utiliser la police PostScript classique Palatino avec {\LaTeX}
  tant pour le texte que pour les mathématiques, il suffit d'insérer
  dans le préambule de son document la commande
\begin{lstlisting}
\usepackage{mathpazo}
\end{lstlisting}

  Avec le moteur {\XeLaTeX}, il est possible d'utiliser n'importe
  quelle police de caractères OpenType et TrueType installée dans le
  système d'exploitation de l'ordinateur. Pour obtenir un résultat
  équivalent à celui de \pkg{mathpazo}, on installe les polices
  TeX~Gyre dans le système, puis on insère dans le préambule les
  commandes
\begin{lstlisting}
\usepackage{fontspec}
\setmainfont{TeX Gyre Pagella}
\setmathfont{TeX Gyre Pagella Math}
\end{lstlisting}
  \qed
\end{exemple}

Le texte principal du présent document est en %
\link{http://tug.org/store/lucida/}{Lucida Bright~OT}, %
une police commerciale de très haute qualité offrant également un
excellent support pour les mathématiques. Ses auteurs ont toujours été
proches de la communauté {\LaTeX}. La Bibliothèque de l'Université
Laval détient une licence d'utilisation de cette police. Les étudiants
et le personnel de l'Université peuvent s'en produrer une copie
gratuitement en écrivant à
\href{mailto:lucida@bibl.ulaval.ca}{lucida@bibl.ulaval.ca}.



\section{Couleurs}

L'utilisation de couleur dans un document {\LaTeX} requiert de charger
le paquetage \pkg{xcolor} \citep{xcolor}. Celui-ci définit d'abord
plusieurs couleurs que l'on peut utiliser directement; consulter la %
\doc{xcolor}{http://texdoc.net/pkg/xcolor} %
pour en connaître les différentes listes. On peut ensuite modifier la
couleur du texte avec la commande
\begin{lstlisting}
\color`\marg{couleur}'
\end{lstlisting}
où \meta{couleur} est le nom d'une couleur. À moins de vouloir
modifier la couleur de tout ce qui suit, on limitera la portée de la
commande par des accolades.
\begin{demo}
  \begin{texample}
\begin{lstlisting}
texte {\color{red} en rouge}
et {\color{blue} en bleu}
\end{lstlisting}
  \producing
  texte {\color{red} en rouge} et {\color{blue} en bleu}
  \end{texample}
\end{demo}

La commande \cmd{\definecolor} permet de définir de nouvelles
couleurs. Le paquetage \pkg{xcolor} supporte plusieurs méthodes pour
ce faire, mais la plus usuelle demeure par une combinaison de
proportions (entre $0$ et $1$) de rouge, de vert et de bleu. Par
exemple,
\begin{lstlisting}
\definecolor`\marg{couleur}'{rgb}{0.1,0.4,0.6}
\end{lstlisting}
définit une nouvelle \meta{couleur} composée de 30~\% rouge, de 40~\%
vert et de 60~\% bleu: %
\fcolorbox{black}[rgb]{0.3,0,0}{\phantom{xx}} $+$ %
\fcolorbox{black}[rgb]{0,0.4,0}{\phantom{xx}} $+$ %
\fcolorbox{black}[rgb]{0,0,0.6}{\phantom{xx}} $=$ %
\fcolorbox{black}[rgb]{0.3,0.4,0.6}{\phantom{xx}}.

La commande \cmd{\definecolor} sert aussi à donner un nom sémantique à
une couleur déjà existante. Par exemple, insérer dans le préambule de
son document
\begin{lstlisting}
\definecolor{important}{named}{red}
\end{lstlisting}
permet d'utiliser la couleur \meta{important} dans le code source.




\section{Hyperliens: en tirer le meilleur parti}
\label{sec:trucs:hyperliens}

Nous en avons déjà traité à quelques reprises, notamment à la
section~4 de \citet{UL:latex:1}, le paquetage \pkg{hyperref} permet de
transformer toutes les références dans le texte en hyperliens
cliquables lorsque le document est produit avec pdf{\LaTeX} ou
{\XeLaTeX}. C'est très pratique lors de la consultation électronique
des documents.

Le paquetage offre une multitudes d'options de configuration. Une des
principales choses que l'on pourra souhaiter configurer, c'est la
couleur des divers types d'hyperliens.


% \begin{conseil}
%   L'interaction du paquetage \pkg{hyperref} avec les autres est
%   parfois (voire souvent) délicate. Pour cette raison, il est
%   fortement recommandé de charger \pkg{hyperref} en tout dernier
%   dans son document.
% \end{conseil}



% Hyperliens: configuration, infos du document

% Présentation de code informatique

% Analyse et rapport intégrés [de type Sweave]

% Diapositives

% Contrôle de version

%%% Local Variables:
%%% mode: latex
%%% TeX-engine: xetex
%%% TeX-master: "formation_latex-partie_2"
%%% coding: utf-8
%%% End:
