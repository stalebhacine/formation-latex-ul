\chapter{Trucs et astuces divers}
\label{chap:trucs}


\section{Contrôle de la disposition du texte}
\label{sec:trucs:controle}

Des algorithmes élaborés permettent à {\LaTeX} de généralement
disposer le texte harmonieusement sur la page. Néanmoins, il
est parfois nécessaire  d'effectuer soi-même de menus ajustements,
notamment pour les sauts de page et la coupure de mots.

\subsection{Sauts de ligne et de page}
\label{sec:trucs:controle:sauts}

Il est assez rarement nécessaire avec {\LaTeX} de devoir forcer les
retours à la ligne. Chose certaine, l'on devrait toujours utiliser une
ligne blanche dans le code source pour identifier un changement de
paragraphe.

Cela dit, les commandes suivantes permettent d'insérer un saut de
ligne manuellement lorsque requis:
\begin{lstlisting}
\\
\\`\oarg{longueur}'
\newline
\end{lstlisting}
La commande {\pixbsbs} est connue: elle sert aussi à délimiter les
lignes dans les tableaux (\autoref{sec:tableaux:tableaux}) et les lignes
d'une suite d'équations (\autoref{sec:math:align}). L'argument optionnel
\oarg{longueur} permet d'insérer un blanc entre les deux lignes; la
\autoref{sec:boites:longueurs} explique comment spécifier une
longueur.

Généralement équivalente à {\bs\bs}, la commande \cmd{\newline} est
parfois nécessaire, notamment pour insérer un changement de ligne à
l'intérieur d'une cellule d'un tableau ou à l'intérieur d'un titre de
section. Quand {\bs\bs} ne fonctionne pas, essayer
\cmdprint{\newline}.

\begin{exemple}
  La commande {\bs\bs} est particulièrement utile --- voire nécessaire
  --- pour disposer des boîtes à l'intérieur d'une figure.
  L'utilisation de l'argument \meta{longueur} permet alors de
  contrôler l'espacement vertical entre les éléments. Comparer les
  deux exemples ci-dessous.
  \begin{demo}
    \begin{texample}[0.55\linewidth]
\begin{lstlisting}
\begin{minipage}{1.0\linewidth}
  \framebox[\linewidth]{texte}
\end{minipage} \\
\begin{minipage}{1.0\linewidth}
  \framebox[\linewidth]{texte}
\end{minipage}
\end{lstlisting}
      \producing
      \begin{minipage}{1.0\linewidth}
        \framebox[\linewidth]{texte}
      \end{minipage}
      \begin{minipage}{1.0\linewidth}
        \framebox[\linewidth]{texte}
      \end{minipage}
    \end{texample}

    \begin{texample}[0.55\linewidth]
\begin{lstlisting}
\begin{minipage}{1.0\linewidth}
  \framebox[\linewidth]{texte}
\end{minipage} \\`\textbf{[6pt]}'
\begin{minipage}{1.0\linewidth}
  \framebox[\linewidth]{texte}
\end{minipage}
\end{lstlisting}
      \producing
      \begin{minipage}{1.0\linewidth}
        \framebox[\linewidth]{texte}
      \end{minipage} \\[6pt]
      \begin{minipage}{1.0\linewidth}
        \framebox[\linewidth]{texte}
      \end{minipage}
    \end{texample}
  \end{demo}
  \qed
\end{exemple}

Les commandes
\begin{lstlisting}
\newpage
\clearpage
\cleartorecto              % classe memoir seulement
\cleartoverso              % classe memoir seulement
\end{lstlisting}
permettent d'insérer manuellement un saut de page pour éviter une
coupure malheureuse. La commande de base pour insérer un saut
n'importe où dans la page est \cmd{\newpage}. La commande
\cmd{\clearpage}, quant à elle, va également s'assurer d'afficher tous
les éléments flottants (\autoref{sec:tableaux:floats}) en attente de
disposition.

Les commandes \cmd{\cleartorecto} et \cmd{\cleartoverso}, propres à la
classe \class{memoir} permettent respectivement de passer
automatiquement à une page recto ou une page verso. Évidemment, elles
n'ont d'utilité que dans les documents recto-verso.

Moins directives, les commandes
\begin{lstlisting}
\pagebreak`\oarg{n}'
\enlargethispage`\marg{longueur}'
\end{lstlisting}
permettent de seulement aider {\LaTeX} à gérer les sauts de page à un
endroit précis. La commande \cmd{\pagebreak} est intéressante
lorsqu'utilisée avec son argument optionnel \meta{n}: celui-ci
indique, par le biais d'un nombre entier entre 0 et 4, à quel point
nous \emph{recommandons} à {\LaTeX} d'insérer un saut de page à
l'endroit où la commande apparaît (0 étant une faible recommandation
et 4, une forte).

La commande \cmd{\enlargethispage}, comme son nom l'indique, permet
d'allonger une page de \meta{longueur} pour y faire tenir plus de
texte. C'est une commande particulièrement utile pour éviter que la
toute dernière ligne d'un chapitre ou d'un document se retrouve seule sur
une page.

\subsection{Coupure de mots}
\label{sec:trucs:controle:coupure}

La coupure automatique des mots en fin de ligne est toujours active
avec {\LaTeX}. C'est d'ailleurs pourquoi il importe d'indiquer à
{\LaTeX} dans quelle langue est le texte (lorsque ce n'est pas en
anglais) avec les commandes du paquetage \pkg{babel}.

Il existe deux façons de contrôler la coupure de mots. La première,
principalement utilisée lorsque {\LaTeX} refuse de couper un mot en
fin de ligne, consiste à insérer des \emph{suggestions} d'endroits où
couper le mot avec la commande \cmd{\-}. Par exemple, en écrivant
\verb=vrai\-sem\-blance=, on indique à {\LaTeX} qu'il est possible de
diviser le mot en \emph{vrai-semblance} ou \emph{vraisem-blance}.

La seconde méthode, celle-là surtout utilisée lorsque {\LaTeX} ne
reconnaît pas des mots qui reviennent souvent dans le document,
consiste à fournir dans le préambule une liste d'exceptions avec la
commande
\begin{lstlisting}
\hyphenation`\marg{liste}'
\end{lstlisting}
La \meta{liste} est une suite de mots, séparés par des virgules, des
blancs ou des retours à la ligne, dans lesquels les points de coupure
sont identifiés par un trait d'union.

\begin{exemple}
  La commande suivante, insérée dans le préambule, permet d'ajouter
  des points de coupure aux mots «puisque», «constante» et
  «vraisemblance» pour l'ensemble du document.
\begin{lstlisting}
\hyphenation{puis-que,cons-tante,vrai-sem-blance}
\end{lstlisting}
  \qed
\end{exemple}

\begin{important}
  Règle générale, garder les opérations d'ajustements de la mise en
  page --- position des éléments flottants, sauts de page, lignes trop
  longues, etc. --- pour la toute fin de la rédaction.
\end{important}


\section{Polices de caractères: au-delà de Computer Modern}
\label{sec:trucs:police}

{\CM%
  Les documents {\LaTeX} standards sont facilement reconnaissables par
  leur police de caractères par défaut, Computer Modern --- celle
  utilisée dans ce paragraphe. Pour qui souhaitait briser la relative
  monotonie induite par cette uniformité, il a longtemps été difficile
  d'utiliser une autre police de caractères. Fort heureusement, la
  situation a beaucoup évolué et il est aujourd'hui assez simple de
  produire des documents {\LaTeX} utilisant des polices de caractères
  variées.}

Avant d'aller plus loin, une mise en garde: si votre document contient
plus que quelques équations mathématiques très simples, le choix de
police devient très restreint. La raison: peu de polices de caractères
comprennent des symboles mathématiques et les informations nécessaires
pour les assembler dans une mise en page satisfaisant les standards de
haute qualité usuels de {\LaTeX}.

Cela dit, pour qui souhaite aller au-delà de la police Computer Modern
sans trop se compliquer la vie, il existe deux solutions principales.

\begin{enumerate}
\item Utiliser l'une ou l'autre des polices PostScript standards.
  C'est très simple avec toute distribution {\LaTeX} moderne: il
  suffit de charger le paquetage approprié. Consulter la %
  \doc{psnfss2e}{http://texdoc.net/pkg/psnfss/} %
  de l'ensemble de paquetages PSNFSS pour connaître les choix
  disponibles.
\item Utiliser une police OpenType présente sur son système avec le
  moteur {\XeLaTeX}. Seule une poignée de ces polices offrent
  toutefois un support approprié pour les mathématiques. La gestion
  des polices de caractères avec {\XeLaTeX} se fait avec le paquetage
  standard \pkg{fontspec}; consulter sa %
  \doc{fontspec}{http://texdoc.net/pkg/fontspec/}.
\end{enumerate}

Pour les thèses et mémoires de l'Université Laval, la Faculté des
études supérieures et postdoctorales accepte les polices %
\begin{quote}
  \begin{tabbing}
    Computer Modern \qquad \=  ABCDEF abcdef 1234567890 \kill \\
    {\CM Computer Modern} \> {\CM ABCDEF abcdef 1234567890} \\
    {\Times Times} \> {\Times ABCDEF abcdef 1234567890} \\
    {\Palatino Palatino} \> {\Palatino ABCDEF abcdef 1234567890}
  \end{tabbing}
\end{quote}
Pour utiliser ces deux dernières avec {\LaTeX}, on charge
respectivement les paquetages \pkg{mathptmx} ou \pkg{mathpazo}. Avec
{\XeLaTeX}, on utilisera les polices Termes et Pagella du projet %
\link{http://www.gust.org.pl/projects/e-foundry/tex-gyre/}{TeX~Gyre}.
Ce sont des polices très similaires à Times et Palatino, disponibles
en version OpenType et qui fournissent un bon support pour les
mathématiques via le projet frère %
\link{http://www.gust.org.pl/projects/e-foundry/tg-math/}{TeX~Gyre Math}.

\begin{exemple}
  \label{ex:trucs:palatino}
  Pour utiliser la police PostScript classique Palatino avec {\LaTeX}
  tant pour le texte que pour les mathématiques, il suffit d'insérer
  dans le préambule de son document la commande
\begin{lstlisting}
\usepackage{mathpazo}
\end{lstlisting}

  Avec le moteur {\XeLaTeX}, il est possible d'utiliser n'importe
  quelle police de caractères OpenType et TrueType installée dans le
  système d'exploitation de l'ordinateur. Pour obtenir un résultat
  équivalent à celui de \pkg{mathpazo}, on installe les polices
  TeX~Gyre dans le système, puis on insère dans le préambule les
  commandes
\begin{lstlisting}
\usepackage{fontspec}
\setmainfont{TeX Gyre Pagella}
\setmathfont{TeX Gyre Pagella Math}
\end{lstlisting}
  \qed
\end{exemple}

Le texte principal du présent document est en %
\link{http://tug.org/store/lucida/}{Lucida Bright~OT}, %
une police commerciale de très haute qualité offrant également un
excellent support pour les mathématiques. Ses auteurs ont toujours été
proches de la communauté {\LaTeX}. La Bibliothèque de l'Université
Laval détient une licence d'utilisation de cette police. Les étudiants
et le personnel de l'Université peuvent s'en produrer une copie
gratuitement en écrivant à
\href{mailto:lucida@bibl.ulaval.ca}{lucida@bibl.ulaval.ca}.



\section{Couleurs}
\label{sec:trucs:couleurs}

L'utilisation de couleur dans un document {\LaTeX} requiert de charger
le paquetage \pkg{xcolor} \citep{xcolor}. Celui-ci définit d'abord
plusieurs couleurs que l'on peut utiliser directement; consulter la %
\doc{xcolor}{http://texdoc.net/pkg/xcolor} %
pour en connaître les différentes listes. Le
\autoref{tab:trucs:couleurs} fournit celle des couleurs toujours
disponibles.

\begin{table}
  \centering
  \caption{Couleurs toujours disponibles quand le paquetage
    \pkg{xcolor} est chargé}
  \label{tab:trucs:couleurs}
  \begin{tabularx}{1.0\linewidth}{XlXll}
    \toprule
    \fcolorbox{black}{black}{\phantom{xx}}\, black &
    \fcolorbox{black}{blue}{\phantom{xx}}\, blue &
    \fcolorbox{black}{brown}{\phantom{xx}}\, brown &
    \fcolorbox{black}{cyan}{\phantom{xx}}\, cyan &
    \fcolorbox{black}{darkgray}{\phantom{xx}}\, darkgray \\
    \addlinespace[3pt]
    \fcolorbox{black}{gray}{\phantom{xx}}\, gray &
    \fcolorbox{black}{green}{\phantom{xx}}\, green &
    \fcolorbox{black}{lightgray}{\phantom{xx}}\, lightgray &
    \fcolorbox{black}{lime}{\phantom{xx}}\, lime &
    \fcolorbox{black}{magenta}{\phantom{xx}}\, magenta \\
    \addlinespace[3pt]
    \fcolorbox{black}{olive}{\phantom{xx}}\, olive &
    \fcolorbox{black}{orange}{\phantom{xx}}\, orange &
    \fcolorbox{black}{pink}{\phantom{xx}}\, pink &
    \fcolorbox{black}{purple}{\phantom{xx}}\, purple &
    \fcolorbox{black}{red}{\phantom{xx}}\, red \\
    \addlinespace[3pt]
    \fcolorbox{black}{teal}{\phantom{xx}}\, teal &
    \fcolorbox{black}{violet}{\phantom{xx}}\, violet &
    \fcolorbox{black}{white}{\phantom{xx}}\, white &
    \fcolorbox{black}{yellow}{\phantom{xx}}\, yellow \\
    \bottomrule
  \end{tabularx}
\end{table}

On modifie la couleur du texte avec la commande
\begin{lstlisting}
\color`\marg{nom}'
\end{lstlisting}
où \meta{nom} est le nom d'une couleur. À moins de vouloir modifier la
couleur de tout ce qui suit, on limitera la portée de la commande par
des accolades.
\begin{demo}
  \begin{texample}
\begin{lstlisting}
texte {\color{red} en rouge}
et {\color{blue} en bleu}
\end{lstlisting}
  \producing
  texte {\color{red} en rouge} et {\color{blue} en bleu}
  \end{texample}
\end{demo}

La commande \cmd{\definecolor} permet de définir de nouvelles couleurs
selon plusieurs systèmes de codage. Le plus usuel demeure \emph{Rouge,
  vert, bleu} (RVB ou RGB, en anglais) où une couleur est représentée
par une combinaison de teintes --- exprimées par un nombre entre $0$
et $1$ --- de rouge, de vert et de bleu. Dans ce cas, la syntaxe de
\cmd{\definecolor} est
\begin{lstlisting}
\definecolor`\marg{nom}'{rgb}`\marg{valeur\_r,valeur\_v,valeur\_b}
\end{lstlisting}
où \meta{valeur\_r}, \meta{valeur\_v} et \meta{valeur\_b} sont
respectivement les teintes de rouge, de vert et de bleu.

\begin{exemple}
  La commande
\begin{lstlisting}
\definecolor{acier}{rgb}{0.1,0.4,0.6}
\end{lstlisting}
  définit une nouvelle couleur nommée \code{acier} composée de rouge
  30~\%, de vert 40~\% et de bleu 60~\%: %
  \fcolorbox{black}[rgb]{0.3,0,0}{\phantom{xx}} $+$ %
  \fcolorbox{black}[rgb]{0,0.4,0}{\phantom{xx}} $+$ %
  \fcolorbox{black}[rgb]{0,0,0.6}{\phantom{xx}} $=$ %
  \fcolorbox{black}[rgb]{0.3,0.4,0.6}{\phantom{xx}}. %
  On pourra utiliser la couleur \code{acier} directement dans la
  commande \cmdprint{\color}. %
  \qed
\end{exemple}

La commande \cmd{\colorlet}, dont la syntaxe simplifiée est
\begin{lstlisting}
\colorlet`\marg{nom}\marg{couleur}'
\end{lstlisting}
permet de faire référence à la \meta{couleur} déjà existante par
\meta{nom}. C'est pratique pour assigner un nom sémantique à une
couleur.



\section{Hyperliens et métadonnées de documents PDF}
\label{sec:trucs:hyperliens}

Nous en avons déjà traité à quelques reprises, notamment à la
section~4 de \citet{UL:latex:1}: le paquetage \pkg{hyperref}
\citep{hyperref} permet de transformer toutes les références dans le
texte en hyperliens cliquables lorsque le document est produit avec
pdf{\LaTeX} ou {\XeLaTeX}. C'est très pratique lors de la consultation
électronique d'un document.

Le paquetage offre une multitudes d'options de configuration; nous
n'en présenterons que quelques unes. On accède aux options de
configuration de \pkg{hyperref} via la commande \cmd{\hypersetup} dans
le préambule. Celle-ci prend en arguments des paires
\code{option=valeur} séparées par des virgules.

Une des principales choses que l'on pourra souhaiter configurer, c'est
le comportement et la couleur des divers types d'hyperliens. On
trouvera ci-dessous les options de configuration pertinentes, leur
valeur (avec en gras la valeur par défaut) ainsi qu'une brève
explication de chacune.

\begin{table}[h]
  \begin{tabularx}{1.0\linewidth}{@{}p{6em}p{6em}X@{}}
    \code{colorlinks} & \code{true}|\code{\textbf{false}} & colorer les
                                                            liens selon
                                                            leur type \\
    \code{linktocpage} & \code{true}|\code{\textbf{false}} & faire du
                                                             numéro de
                                                             page
                                                             plutôt que
                                                             du titre l'hyperlien dans
                                                             la table
                                                             des
                                                             matières \\
    \code{linkcolor} & \meta{couleur} & couleur des liens internes \\
    \code{urlcolor}  & \meta{couleur} & couleur des URL externes \\
    \code{citecolor} & \meta{couleur} & couleur des citations \\
    \code{allcolor}  & \meta{couleur} & couleur pour tous les types d'hyperliens
  \end{tabularx}
\end{table}

Lorsque la valeur admissible d'une option est \code{true} ou
\code{false}, sa seule mention équivaut à \code{true}. La valeur
\meta{couleur} est le nom d'une couleur telle que définie par
\pkg{xcolor} (\autoref{sec:trucs:couleurs}).

Les fichiers PDF peuvent contenir diverses métadonnées sur leur
contenu. Le paquetage \pkg{hyperref} permet de définir certaines
catégories de celles-ci, notamment les informations de document comme
le titre ou l'auteur. Nous ne mentionnons ici que deux options de
configutation; la section~3.7 de la %
\doc{hyperref}{http://texdoc.net/pkg/hyperref} %
contient la liste complète.

\begin{table}[h]
  \begin{tabularx}{1.0\linewidth}{@{}p{6em}p{6em}X@{}}
    \code{pdftitle}  & texte & titre du document PDF \\
    \code{pdfauthor} & texte & auteur du document PDF
  \end{tabularx}
\end{table}

\begin{exemple}
  \label{ex:trucs:couleurs}
  Le présent document fait appel aux définitions de couleurs
  et aux options de configuration de \pkg{hyperref} suivantes:
\begin{lstlisting}
\definecolor{link}{rgb}{0,0.4,0.6}   % ~RoyalBlue de dvips
\definecolor{url}{rgb}{0.6,0,0}      % rouge-brun
\definecolor{citation}{rgb}{0,0.5,0} % vert foncé
\hypersetup{colorlinks, linktocpage,
  linkcolor=link, urlcolor=url, citecolor=citation,
  pdftitle={Rédaction de thèses et mémoires avec LaTeX -
            Partie II},
  pdfauthor={Université Laval}}
\end{lstlisting}
  \qed
\end{exemple}

\begin{exemple}
  Les gabarits de thèses et de mémoires livrés avec la classe
  \class{ulthese} contiennent dans le préambule la ligne suivante:
\begin{lstlisting}
\hypersetup{colorlinks,allcolors=ULlinkcolor}
\end{lstlisting}
  Tous les liens de la thèse ou du mémoire seront donc de la même
  couleur, \code{ULlinkcolor} \fcolorbox{black}{ULlinkcolor}{\phantom{xx}},
  une couleur définie par la classe. %
  \qed
\end{exemple}

\begin{conseil}
  L'interaction du paquetage \pkg{hyperref} avec les autres est
  parfois (voire souvent) délicate. Pour cette raison, il est
  fortement recommandé de charger \pkg{hyperref} en tout dernier
  dans dans le préambule.
\end{conseil}



\section{Présentation de code informatique}
\label{sec:trucs:listings}

L'environnement standard \Ie{verbatim} de {\LaTeX} permet de présenter
du texte exactement tel qu'il est entré dans le code source du
document. C'est un environnement particulièrement utile pour afficher
du code informatique puisque le texte est composé en police non
proportionnelle et que sa disposition exacte est respectée.

\begin{demo}
  \begin{texample}
\begin{lstlisting}[deletetexcs={int,include}]
\begin{verbatim}
/* Hello World en C */
#include <stdio.h>

int main()
{
    printf("Hello world\n");
    return 0;
}
\end{verbatim}
\end{lstlisting}
    \producing
\begin{verbatim}
/* Hello World en C */
#include <stdio.h>

int main()
{
    printf("Hello world\n");
    return 0;
}
\end{verbatim}
  \end{texample}
\end{demo}

Si un document doit contenir beaucoup de code informatique et que l'on
souhaite exercer un fin contrôle sur sa disposition et sa mise en
forme, il vaut mieux se tourner vers un paquetage spécialisé comme
\pkg{listings} \citep{listings}. La %
\doc{listings}{http://texdoc.net/pkg/listings} %
du paquetage compare ses fonctionnalités à celles de plusieurs autres
paquetages similaires.

Le paquetage \pkg{listings} peut effectuer automatiquement le marquage
des mots-clés de nombreux langages de programmation, ajouter des
numéros de ligne, importer du code de fichiers externes ou même
indexer les mots-clés des extraits de code. À titre d'illustration, en
utilisant l'environnement \Ie{lstlisting} de \pkg{listings} plutôt que
\Ie{verbatim}, l'extrait de code C ci-dessus pourrait se présenter
ainsi:
\begin{demo}
  \begin{texample}
    \begin{vglisting}
\begin{lstlisting}
/* Hello World en C */
#include <stdio.h>

int main()
{
  printf("Hello world\n");
  return 0;
}
\end{lstlisting}
    \end{vglisting}
    \producing
\begin{lstlisting}[language=C,%
  deletetexcs={int,include},
  frame=single,%
  backgroundcolor=\color{Azure2},%
  rulecolor=\color{black},%
  numbers=left,%
  numberstyle=\tiny\sffamily,%
  stringstyle=\color{orange}\itshape,%
  identifierstyle=\color{cyan}\mdseries,%
  xleftmargin=12pt,%
  keywordstyle=\color{blue}\bfseries,
  index=]
/* Hello World en C */
#include <stdio.h>

int main()
{
  printf("Hello world\n");
  return 0;
}
\end{lstlisting}
  \end{texample}
\end{demo}

Il serait trop long et nettement hors de la portée du présent document
d'expliquer les nombreuses fonctionnalités de \pkg{listings}.
Précisons simplement que le paquetage a servi pour en composer les
extraits de code {\LaTeX} et pour construire une grande partie de
l'index.

\begin{exemple}
  \label{ex:trucs:listings}
  Pour parvenir à la présentation des extraits de code source {\LaTeX}
  de ce document, le paquetage \pkg{listings} est configuré dans le
  préambule de la manière suivante:
\begin{lstlisting}
%% Couleurs
\definecolor{comments}{rgb}{0.7,0,0}

%% Configuration de listings
\lstset{language=[LaTeX]TeX,
  basicstyle=\ttfamily\NoAutoSpacing,
  keywordstyle=\mdseries,
  commentstyle=\color{comments}\slshape,
  extendedchars=true,
  showstringspaces=false,
  backgroundcolor=\color{LightYellow1},
  frame=lr, rulecolor=\color{LightYellow1},
  xleftmargin=3.4pt, xrightmargin=3.4pt}
\end{lstlisting}
  (La couleur \code{LightYellow1} est définie par \pkg{xcolor} lorsque
  le paquetage est chargé avec l'option \code{x11names}.)
  \qed
\end{exemple}

%%% Réactivation des mots-clés.
%\lstset{language=[LaTeX]TeX, moretexcs={int,include}}


\section{Production de rapports avec l'analyse intégrée}
\label{sec:trucs:sweave}

Les publications scientifiques reposent souvent sur une forme ou une
autre d'analyse numérique ou statistique, la production de code
informatique, une simulation stochastique, etc. La portion
développement et analyse est alors produite avec un certain outil
et la publication, avec un outil d'édition séparé --- {\LaTeX} pour ce
qui nous occupe. Or, tous les auteurs ont vécu cette situation: les
résultats de l'analyse changent et il faut modifier le rapport en
conséquence, refaire les tableaux et les graphiques, retracer cette
valeur isolée au fil du texte directement tirée de l'analyse\dots\
Seule la quantité de temps perdu rivalise avec le risque d'erreur.

Il existe pourtant une meilleure façon de travailler.

Cette meilleure façon de faire, tirée du concept de
\emph{programmation lettrée}, consiste à combiner dans un seul et même
document l'analyse et le rapport, puis de produire automatiquement une
partie du second à partir de la première.

Les utilisateurs du système statistique R bénéficient ici d'une mise
en œuvre simple et élégante du concept ci-dessus avec l'outil Sweave
\citep{Sweave}. Un fichier Sweave est à la base un document {\LaTeX}
dans lequel on a inséré du code R à l'intérieur de balises spéciales
utilisant la syntaxe \code{noweb} \citep{noweb}, comme ceci:
\begin{lstlisting}
\section{Commandes R}

L'utilisateur de R interagit avec l'interprète en entrant
des commandes à l'invite de commande:
<<>>=
2 + 3
pi
cos(pi/4)
@
La commande \verb=exp(1)= donne \Sexpr{exp(1)},
la valeur du nombre $e$.
\end{lstlisting}

Par convention, on enregistre un tel document sous un nom se terminant
par l'extension \code{.Rnw}. Sa compilation s'effectue
en deux étapes:
\begin{enumerate}
\item Le fichier \code{.Rnw} est passé à la commande \code{Sweave()}
  de R. Celle-ci retrace les extraits de code placés entre les balises
  \verb|<<>>=| et \verb|@| et les remplace par des environnements
  {\LaTeX} contenant, par défaut, les expressions R et leur résultat
  dans des environnements \Ie{Sinput} et \Ie{Soutput}. Elle évalue
  également les expressions R se trouvant dans les commandes
  \cmd{\Sexpr} pour les remplacer par leur résultat. Cela produit un
  fichier \code{.tex}:
\begin{lstlisting}
\section{Commandes R}

L'utilisateur de R interagit avec l'interprète en entrant
des commandes à l'invite de commande:
\begin{Schunk}
\begin{Sinput}
> 2 + 3
\end{Sinput}
\begin{Soutput}
[1] 5
\end{Soutput}
\begin{Sinput}
> pi
\end{Sinput}
\begin{Soutput}
[1] 3.141593
\end{Soutput}
\begin{Sinput}
> cos(pi/4)
\end{Sinput}
\begin{Soutput}
[1] 0.7071068
\end{Soutput}
\end{Schunk}
La commande \verb=exp(1)= donne 2.71828182845905,
la valeur du nombre $e$.
\end{lstlisting}
\item On compile le fichier \code{.tex} comme d'habitude.
\end{enumerate}

Sweave se révèle particulièrement utile pour créer des graphiques à
partir de R: tout ce que l'on doit conserver dans son fichier
\code{.Rnw}, c'est le code pour créer le graphique. Il est également
possible de contrôler l'exécution des blocs de code et l'affichage du
code source et des résultats par le biais d'options placées à
l'intérieur de la balise d'ouverture \verb|<<>>=|. Consulter la %
\link{http://leisch.userweb.mwn.de/Sweave/Sweave-manual.pdf}{documentation} %
de Sweave pour les détails.

\link{http://mpastell.com/pweave/}{Pweave} est un système similaire à
Sweave pour le langage Python.

Inspiré de Sweave, knitr \citep{knitr} permet également d'entrelacer
du code {\LaTeX} et du code R. Cet outil offre plus d'options de
traitement que Sweave, mais au prix d'une complexité accrue.

En terminant, rappelons que Sweave n'est qu'un exemple de système de
programmation lettrée. Il en existe plusieurs autres. Il appartient
dès lors au lecteur de trouver le système qui correspond le mieux à
ses besoins.

\begin{information}
  On doit le concept de programmation lettrée au créateur de {\TeX},
  Donald Knuth.  En fait, tout le code source de {\TeX} est écrit
  en programmation lettrée! La %
  \link{https://fr.wikipedia.org/wiki/Programmation_lettrée}{page Wikipedia} %
  consacrée au sujet offre un très bon survol de l'historique et de la
  nature du concept.

  Le lecteur intéressé pourra consulter, par exemple, le %
  \link{http://www.ctan.org/tex-archive/macros/latex/contrib/ulthese}{code source} %
  de la classe \class{ulthese}. La documentation de la classe, le
  code {\LaTeX}, les gabarits, tout se trouve entrelacé
  dans le seul fichier \code{ulthese.dtx}.
\end{information}



\section{Diapositives}
\label{sec:trucs:diapositives}

Il n'est pas rare qu'une publication scientifique fasse l'objet d'une
présentation dans le cadre d'un colloque ou d'un séminaire. Lorsque le
texte a été rédigé avec {\LaTeX}, il est tout naturel de souhaiter le
réutiliser pour la préparation de diapositives --- surtout si le texte
comporte de nombreuses équations mathématiques qu'il serait
extrêmement long de retranscrire dans un logiciel de présentation
comme PowerPoint.

Fort heureusement, il est tout à fait possible de composer ses
diapositives avec {\LaTeX}. La classe standard \class{slides} produit
des diapositives élégantes, quoique minimalistes, qui conviendront à
l'auteur qui ne recherche rien de plus que du texte noir sur fond
blanc.

Cependant, l'outil devenu le standard \emph{de facto} pour la
production de diapositives est la classe \class{beamer}
\citep{beamer}. Celle-ci compte un grand nombre de thèmes et de
gabarits élaborés, rend très simple l'insertion d'animations d'une
diapositive à l'autre, gère automatiquement la table des matières
et\dots\ produit des diapositives en couleur!

Quelle que soit la classe utilisée, les diapositives produites avec
{\LaTeX} se présentent sous forme de fichier PDF. On les projète avec
une liseuse PDF en mode plein écran.

Nous n'irons pas plus loin sur le sujet dans ce document. Consulter
la %
\doc{beamer}{http://texdoc.net/pkg/beameruserguide} %
de \class{beamer} pour apprendre à utiliser la classe. Produire des
diapositives de grande qualité avec les gabarits fournis avec la
classe est simple et rapide.

Les diapositives de la première partie de la formation
\citep{UL:latex:1} ont été produites avec \class{beamer}.



\section{Gestion des versions et travail collaboratif}
\label{sec:trucs:cvs}

Vous travaillez à plusieurs sur un même fichier ou seul, mais de
plusieurs postes de travail différents. Quelle est la plus récente
version du fichier? Mon ajout fait hier dans le fichier a-t-il été
pris en compte par ma collègue aujourd'hui? Une modification apportée
au fichier n'est plus nécessaire; comment retourner en arrière
facilement?

Les informaticiens ont résolu ce genre de problèmes il y a des
dizaines d'années avec les systèmes de gestion de versions. Les
systèmes les plus populaires en ce moment sont Subversion
\citep{subversion} et Git \citep{git}.

Bien que développés à l'origine pour la gestion du code source de
logiciels, les systèmes de contrôle de versions conviennent
parfaitement pour les sources {\LaTeX}. L'utilisation d'un tel système
permet de:
\begin{itemize}
\item toujours savoir quelle est la plus récente version d'un fichier;
\item travailler à plusieurs personnes simultanément sur un même
  fichier;
\item revenir aisément à une version antérieure d'un fichier;
\item comparer deux versions d'un fichier pour connaître les
  modifications qui y ont été apportées;
\item gérer automatiquement les éventuels conflits de modification
  d'un fichier;
\item disposer en tout temps d'une copie de secours de son travail
  lorsque l'on a recours à un serveur central (obligatoire avec
  Subversion; optionnel avec Git\footnote{%
    Si ce n'est que pour le volet de sauvegarde, nous recommandons
    d'utiliser avec Git un dépôt central tel que
    \link{https://github.com}{GitHub}, un serveur public qui héberge
    déjà des millions de projets.}.)
\end{itemize}
Un système de gestion de versions est un outil qui permet d'augmenter
considérablement sa productivité ou celle de son équipe de
travail au moment de rédiger un ouvrage scientifique.

La %
\link{https://fr.wikipedia.org/wiki/Gestion_de_versions}{page Wikipedia} %
sur le sujet offre une bonne introduction aux systèmes de gestion de
versions. Les membres de la communauté de l'Université Laval qui
souhaitent mettre sur pied un dépôt pour un projet pourront
consulter leur équipe de soutien informatique facultaire.

\begin{information}
  La présente documentation est conservée dans un dépôt Subversion
  public d'où l'on peut toujours obtenir la plus récente version.
  Consulter la page des notices de copyright au début du document pour
  accéder au dépôt.
\end{information}



%%%
%%% Exercices
%%%

\section{Exercices}
\label{sec:trucs:exercices}

\Opensolutionfile{solutions}[solutions-trucs]

\begin{Filesave}{solutions}
\section*{Chapitre \ref*{chap:trucs}}
\addcontentsline{toc}{section}{Chapitre \protect\ref*{chap:trucs}}

\end{Filesave}

\noindent%
Pour les exercices
\nolink{\ref{exercice:trucs:1}}--\nolink{\ref{exercice:trucs:n}},
utiliser le fichier \fichier{exercices\_trucs.tex}. Celui-ci reprend
une partie de la documentation de la classe \class{ulthese}. De plus,
on remarquera que le document:
\begin{itemize}
\item doit être compilé avec pdf{\LaTeX} puisqu'il charge les
  paquetages \pkg{inputenc} et \pkg{fontenc};
\item définit des nouvelles commandes \cmdprint{\class} et
  \cmdprint{\fichier} pour composer, respectivement, les noms de
  classe et les noms de fichier;
\item utilise la commande \cmdprint{\doc} de
  l'\autoref{ex:commandes:doc};
\item charge le paquetage \pkg{hyperref}, ce qui transforme les titres
  de la table des matières, les renvois aux notes de bas de page et
  les liens externes en hyperliens.
\end{itemize}
\medskip

\begin{exercice}
  \label{exercice:trucs:1}
  Compiler et visualiser le fichier sans aucune modification. Le texte
  est composé dans la police par défaut Computer Modern.

  Ensuite, modifier le préambule du document pour composer le document
  dans la police Palatino, tel qu'expliqué à
  l'\autoref{ex:trucs:palatino}. Charger également le paquetage
  \pkg{helvet} afin d'utiliser Helvetica pour le texte en police sans
  empattements (\cmdprint{\textsf}).
  \begin{sol}
    On doit ajouter dans le préambule les commandes
\begin{lstlisting}
\usepackage{mathpazo}
\usepackage{helvet}
\end{lstlisting}
    La police Helvetica est très grande. Tel que mentionné dans la %
    \doc{psnfss2e}{http://texdoc.net/pkg/psnfss/} %
    de PSNFSS (section~4), il est généralement préférable de charger
    le paquetage \pkg{helvet} avec, par exemple,
\begin{lstlisting}
\usepackage[scaled=0.92]{helvet}
\end{lstlisting}
    pour que le texte en Helvetica se marie mieux à celui dans une
    autre police.
  \end{sol}
\end{exercice}

\begin{exercice}
  Configurer le paquetage \pkg{hyperref} pour que les hyperliens dans
  la table des matières soient ancrés aux numéros de page plutôt
  qu'aux titres de section.
  \begin{sol}
    Tel qu'expliqué à la \autoref{sec:trucs:hyperliens}, on doit
    insérer dans le préambule du document la commande
\begin{lstlisting}
\hypersetup{linktocpage=true}
\end{lstlisting}
    ou, plus simplement,
\begin{lstlisting}
\hypersetup{linktocpage}
\end{lstlisting}
  \end{sol}
\end{exercice}

\begin{exercice}
  Charger le paquetage \pkg{xcolor} et ajouter l'option
  \code{colorlinks} à \pkg{hyperref}, puis recompiler le document.
  Observer les changements.
  \begin{sol}
    En conservant l'ajout de l'exercice précédent, on a dans le
    préambule la commande
\begin{lstlisting}
\hypersetup{colorlinks, linktocpage}
\end{lstlisting}
    Les hyperliens se présentent maintenant en couleur selon les
    paramètres par défaut de \pkg{hyperref}.
  \end{sol}
\end{exercice}

\begin{exercice}
  En s'inspirant de l'\autoref{ex:trucs:couleurs}, modifier la couleur
  des liens internes et externes.
  \begin{sol}
    On peut soit utiliser des couleurs prédéfinies de \pkg{xcolor}
    (\autoref{tab:trucs:couleurs}), soit en  définir de nouvelles avec
    \cmd{\definecolor}. On fait ensuite appel aux couleurs choisies
    pour les options \code{linkcolor} (liens internes) et
    \code{urlcolor} (liens externes) de \pkg{hyperref}.

    Exemple utilisant des couleurs prédéfinies:
\begin{lstlisting}
\hypersetup{colorlinks, linktocpage,
  linkcolor=brown, urlcolor=blue}
\end{lstlisting}

    Exemple avec de nouvelles couleurs:
\begin{lstlisting}
\definecolor{link}{rgb}{0,0.4,0.6}
\definecolor{url}{rgb}{0.6,0,0}
\hypersetup{colorlinks, linktocpage,
  urlcolor=url, linkcolor=link}
\end{lstlisting}
  \end{sol}
\end{exercice}

\begin{exercice}
  \label{exercice:trucs:n}
  Charger le paquetage \pkg{listings} et modifier l'environnement
  \Pe{verbatim} que l'on trouve dans le document pour un environnement
  \Pe{lstlisting}. En s'inspirant de l'\autoref{ex:trucs:listings},
  configurer la présentation des extraits de code pour utiliser une
  police non proportionnelle (\cmdprint{\ttfamily}) et un arrière-plan
  de la couleur standard «lightgray».
  \begin{sol}
    S'il y avait plusieurs extraits de code dans le document, mieux
    vaudrait les configurer tous à l'identique dans le préambule du
    document avec
\begin{lstlisting}
\lstset{basicstyle=\ttfamily,
  backgroundcolor=\color{lightgray}}
\end{lstlisting}
    Ensuite,
\begin{vglisting}
\begin{lstlisting}
latex ulthese.ins
\end{lstlisting}
\end{vglisting}
    donne le résultat demandé.

    Pour un seul extrait, il est également possible de simplement
    charger le paquetage dans le préambule et d'effectuer la
    configuration à l'ouverture de l'environnement, comme ceci:
\begin{vglisting}
\begin{lstlisting}[basicstyle=\ttfamily,
  backgroundcolor=\color{lightgray}]
latex ulthese.ins
\end{lstlisting}
\end{vglisting}
  \end{sol}
\end{exercice}

\begin{exercice}[nosol]
  Accéder au dépôt qui héberge le code source de cette formation
  {\LaTeX} en suivant le lien à la page des notices de copyright.
  Télécharger le fichier maître de la première partie de la formation
  et ses fichiers auxiliaires:

  \medskip
  \begin{minipage}{\linewidth}
    \dirtree{%
      .1 formation\_latex-partie\_1.tex.
      .2 licence-partie\_1.tex.
      .2 colophon-partie\_1.tex.
      .2 exercice\_commandes-output.pdf.
      .2 renvoi.pdf.
      .2 renvoi\_avec\_ref.pdf.
      .2 renvoi\_avec\_autoref.pdf.
      .2 ponctuation.pdf.
    }
  \end{minipage}
  \medskip

  (Les lecteurs curieux pourront aussi installer sur leur poste de
  travail un %
  \link{http://subversion.apache.org/packages.html}{client Subversion} %
  et l'utiliser pour extraire l'ensemble des fichiers du dépôt.)

  Une fois les fichiers en main, examiner le code source du fichier
  maître. Il s'agit d'un document utilisant la classe \class{beamer}
  pour créer des diapositives.

  Désactiver (effacer ou mettre en commentaire) les lignes suivantes
  du préambule:
\begin{lstlisting}
\defaultfontfeatures{Ligatures=TeX,Scale=0.92}
\setmainfont{Lucida Bright OT}
\setsansfont{Lucida Sans OT}
\setmathfont{Lucida Bright Math OT}
\setmonofont{Lucida Grande Mono DK}
\newfontfamily\titles{Myriad Pro}
\end{lstlisting}
  et, plus loin dans le fichier,
\begin{lstlisting}
\setbeamerfont{frametitle}{family=\titles}
\end{lstlisting}
  Compiler ensuite le document avec {\XeLaTeX}. En comparant le code
  source et les diapositives, observer comment \class{beamer} peut
  afficher des listes une puce à la fois à partir d'un environnement
  \Pe{itemize} usuel une fois qu'on lui a ajouté une option
  \verb=[<+->]=.
\end{exercice}

\Closesolutionfile{solutions}


%%% Local Variables:
%%% mode: latex
%%% TeX-engine: xetex
%%% TeX-master: "formation_latex_UL-partie_2"
%%% coding: utf-8
%%% End:
