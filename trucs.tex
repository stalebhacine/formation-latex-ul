\chapter{Trucs et astuces divers}
\label{chap:trucs}


\section{Polices de caractères: au-delà de Computer Modern}
\label{sec:trucs:police}

{\CM%
  Les documents {\LaTeX} standards sont facilement reconnaissables par
  leur police de caractères par défaut --- celle utilisée dans ce
  paragraphe ---, Computer Modern. Pour qui souhaitait briser la relative
  monotonie induite par cette uniformité, il a longtemps été difficile
  d'utiliser une autre police de caractères. Fort heureusement, la
  situation a beaucoup évolué et il est aujourd'hui assez simple de
  produire des documents {\LaTeX} utilisant des polices de caractères
  variées.}

Avant d'aller plus loin, une mise en garde: si votre document contient
plus que quelques équations mathématiques très simples, le choix de
police devient très restreint. La raison: peu de polices de caractères
comprennent des symboles mathématiques et les informations nécessaires
pour les assembler dans une mise en page satisfaisant les standards de
haute qualité usuels de {\LaTeX}.

Cela dit, pour qui souhaite aller au-delà de la police Computer Modern
sans trop se compliquer la vie, il existe deux solutions principales.

\begin{enumerate}
\item Utiliser l'une ou l'autre des polices PostScript standards.
  C'est très simple avec toute distribution {\LaTeX} moderne: il
  suffit de charger le paquetage approprié. Consulter la %
  \doc{psnfss2e}{http://texdoc.net/pkg/psnfss/} %
  de l'ensemble de paquetages PSNFSS pour connaître les choix
  disponibles.
\item Utiliser une police OpenType présente sur son système avec le
  moteur {\XeLaTeX}. Seule une poignée de ces polices offrent
  toutefois un support approprié pour les mathématiques. La gestion
  des polices de caractères avec {\XeLaTeX} se fait avec le paquetage
  standard \pkg{fontspec}; consulter sa %
  \doc{fontspec}{http://texdoc.net/pkg/fontspec/}.
\end{enumerate}

Pour les thèses et mémoires de l'Université Laval, la Faculté des
études supérieures et postdoctorales accepte les polices %
\begin{quote}
  \begin{tabbing}
    Computer Modern \qquad \=  ABCDEF abcdef 1234567890 \kill \\
    {\CM Computer Modern} \> {\CM ABCDEF abcdef 1234567890} \\
    {\Times Times} \> {\Times ABCDEF abcdef 1234567890} \\
    {\Palatino Palatino} \> {\Palatino ABCDEF abcdef 1234567890}
  \end{tabbing}
\end{quote}
Pour utiliser ces deux dernières avec {\LaTeX}, on charge
respectivement les paquetages \pkg{mathptmx} ou \pkg{mathpazo}. Avec
{\XeLaTeX}, on utilisera les polices Termes et Pagella du projet %
\link{http://www.gust.org.pl/projects/e-foundry/tex-gyre/}{TeX~Gyre}.
Ce sont des polices très similaires à Times et Palatino, disponibles
en version OpenType et qui fournissent un bon support pour les
mathématiques via le projet frère %
\link{http://www.gust.org.pl/projects/e-foundry/tg-math/}{TeX~Gyre Math}.

\begin{exemple}
  Pour utiliser la police PostScript classique Palatino avec {\LaTeX}
  tant pour le texte que pour les mathématiques, il suffit d'insérer
  dans le préambule de son document la commande
\begin{lstlisting}
\usepackage{mathpazo}
\end{lstlisting}

  Avec le moteur {\XeLaTeX}, il est possible d'utiliser n'importe
  quelle police de caractères OpenType et TrueType installée dans le
  système d'exploitation de l'ordinateur. Pour obtenir un résultat
  équivalent à celui de \pkg{mathpazo}, on installe les polices
  TeX~Gyre dans le système, puis on insère dans le préambule les
  commandes
\begin{lstlisting}
\usepackage{fontspec}
\setmainfont{TeX Gyre Pagella}
\setmathfont{TeX Gyre Pagella Math}
\end{lstlisting}
  \qed
\end{exemple}

Le texte principal du présent document est en %
\link{http://tug.org/store/lucida/}{Lucida Bright~OT}, %
une police commerciale de très haute qualité offrant également un
excellent support pour les mathématiques. Ses auteurs ont toujours été
proches de la communauté {\LaTeX}. La Bibliothèque de l'Université
Laval détient une licence d'utilisation de cette police. Les étudiants
et le personnel de l'Université peuvent s'en produrer une copie
gratuitement en écrivant à
\href{mailto:lucida@bibl.ulaval.ca}{lucida@bibl.ulaval.ca}.



\section{Couleurs}
\label{sec:trucs:couleurs}

L'utilisation de couleur dans un document {\LaTeX} requiert de charger
le paquetage \pkg{xcolor} \citep{xcolor}. Celui-ci définit d'abord
plusieurs couleurs que l'on peut utiliser directement; consulter la %
\doc{xcolor}{http://texdoc.net/pkg/xcolor} %
pour en connaître les différentes listes. Le
\autoref{tab:trucs:couleurs} fournit celle des couleurs toujours
disponibles.

\begin{table}
  \centering
  \caption{Couleurs toujours disponibles quand le paquetage
    \pkg{xcolor} est chargé}
  \label{tab:trucs:couleurs}
  \begin{tabularx}{1.0\linewidth}{XlXll}
    \toprule
    \fcolorbox{black}{red}{\phantom{xx}}\, red &
    \fcolorbox{black}{green}{\phantom{xx}}\, green &
    \fcolorbox{black}{blue}{\phantom{xx}}\, blue &
    \fcolorbox{black}{cyan}{\phantom{xx}}\, cyan &
    \fcolorbox{black}{magenta}{\phantom{xx}}\, magenta \\
    \addlinespace[3pt]
    \fcolorbox{black}{black}{\phantom{xx}}\, black &
    \fcolorbox{black}{darkgray}{\phantom{xx}}\, darkgray &
    \fcolorbox{black}{gray}{\phantom{xx}}\, gray &
    \fcolorbox{black}{lightgray}{\phantom{xx}}\, lightgray &
    \fcolorbox{black}{white}{\phantom{xx}}\, white \\
    \addlinespace[3pt]
    \fcolorbox{black}{yellow}{\phantom{xx}}\, yellow &
    \fcolorbox{black}{brown}{\phantom{xx}}\, brown &
    \fcolorbox{black}{lime}{\phantom{xx}}\, lime &
    \fcolorbox{black}{olive}{\phantom{xx}}\, olive &
    \fcolorbox{black}{orange}{\phantom{xx}}\, orange \\
    \addlinespace[3pt]
    \fcolorbox{black}{pink}{\phantom{xx}}\, pink &
    \fcolorbox{black}{purple}{\phantom{xx}}\, purple &
    \fcolorbox{black}{teal}{\phantom{xx}}\, teal &
    \fcolorbox{black}{violet}{\phantom{xx}}\, violet \\
    \bottomrule
  \end{tabularx}
\end{table}

On modifie la couleur du texte avec la commande
\begin{lstlisting}
\color`\marg{nom}'
\end{lstlisting}
où \meta{nom} est le nom d'une couleur. À moins de vouloir modifier la
couleur de tout ce qui suit, on limitera la portée de la commande par
des accolades.
\begin{demo}
  \begin{texample}
\begin{lstlisting}
texte {\color{red} en rouge}
et {\color{blue} en bleu}
\end{lstlisting}
  \producing
  texte {\color{red} en rouge} et {\color{blue} en bleu}
  \end{texample}
\end{demo}

La commande \cmd{\definecolor} permet de définir de nouvelles couleurs
selon plusieurs systèmes de codage. Le plus usuel demeure \emph{Rouge,
  vert, bleu} (RVB ou RGB, en anglais) où une couleur est représentée
par une combinaison de teintes --- exprimées par un nombre entre $0$
et $1$ --- de rouge, de vert et de bleu. Dans ce cas, la syntaxe de
\cmd{\definecolor} est
\begin{lstlisting}
\definecolor`\marg{nom}'{rgb}`\marg{valeur\_r,valeur\_v,valeur\_b}
\end{lstlisting}
où \meta{valeur\_r}, \meta{valeur\_v} et \meta{valeur\_b} sont
respectivement les teintes de rouge, de vert et de bleu.

\begin{exemple}
  La commande
\begin{lstlisting}
\definecolor{acier}{rgb}{0.1,0.4,0.6}
\end{lstlisting}
  définit une nouvelle couleur nommée \meta{acier} composée de rouge
  30~\%, de vert 40~\% et de bleu 60~\%: %
  \fcolorbox{black}[rgb]{0.3,0,0}{\phantom{xx}} $+$ %
  \fcolorbox{black}[rgb]{0,0.4,0}{\phantom{xx}} $+$ %
  \fcolorbox{black}[rgb]{0,0,0.6}{\phantom{xx}} $=$ %
  \fcolorbox{black}[rgb]{0.3,0.4,0.6}{\phantom{xx}}. %
  On pourra utiliser la couleur \meta{acier} directement dans la
  commande \cmdprint{\color}. %
  \qed
\end{exemple}

La commande \cmd{\colorlet}, dont la syntaxe simplifiée est
\begin{lstlisting}
\colorlet`\marg{nom}\marg{couleur}'
\end{lstlisting}
permet de faire référence à la \meta{couleur} déjà existante par
\meta{nom}. C'est pratique pour assigner un nom sémantique à une
couleur.



\section{Hyperliens: en tirer le meilleur parti}
\label{sec:trucs:hyperliens}

Nous en avons déjà traité à quelques reprises, notamment à la
section~4 de \citet{UL:latex:1}, le paquetage \pkg{hyperref}
\citep{hyperref} permet de transformer toutes les références dans le
texte en hyperliens cliquables lorsque le document est produit avec
pdf{\LaTeX} ou {\XeLaTeX}. C'est très pratique lors de la consultation
électronique d'un document.

Le paquetage offre une multitudes d'options de configuration; nous
n'en présenterons que quelques unes. On accède aux options de
configuration de \pkg{hyperref} via la commande \cmd{\hypersetup} dans
le préambule. Celle-ci prend en arguments des paires
\code{option=valeur} séparées par des virgules.

Une des principales choses que l'on pourra souhaiter configurer, c'est
le comportement et la couleur des divers types d'hyperliens. On
trouvera ci-dessous les options de configuration pertinentes, leur
valeur (avec en gras la valeur par défaut) ainsi qu'une brève
explication de chacune.

\begin{table}[h]
  \begin{tabularx}{1.0\linewidth}{@{}p{6em}p{6em}X@{}}
    \code{colorlinks} & \code{true}|\code{\textbf{false}} & colorer les
                                                            liens selon
                                                            leur type \\
    \code{linktocpage} & \code{true}|\code{\textbf{false}} & faire du
                                                             numéro de
                                                             page
                                                             plutôt que
                                                             du titre l'hyperlien dans
                                                             la table
                                                             des
                                                             matières \\
    \code{linkcolor} & \meta{couleur} & couleur des liens internes \\
    \code{urlcolor}  & \meta{couleur} & couleur des URL externes \\
    \code{citecolor} & \meta{couleur} & couleur des citations \\
    \code{allcolor}  & \meta{couleur} & couleur pour tous les types d'hyperliens
  \end{tabularx}
\end{table}

La valeur \meta{couleur} est le nom d'une couleur telle que définie
par \pkg{xcolor} (\autoref{sec:trucs:couleurs}). Lorsque la valeur
admissible d'une option est \code{true} ou \code{false}, sa seule
mention équivaut à \code{true}. Par conséquent,
\begin{lstlisting}
\hypersetup{colorlinks}
\end{lstlisting}
est plus court et équivalent à
\begin{lstlisting}
\hypersetup{colorlinks=true}
\end{lstlisting}

Le paquetage \pkg{hyperref} permet également de définir des
informations du fichier PDF. Celles-ci seront ou non prises en compte
par la visionneuse. Mentionnons au moins les deux options ci-dessous.

\begin{table}[h]
  \begin{tabularx}{1.0\linewidth}{@{}p{6em}p{6em}X@{}}
    \code{pdftitle}  & texte & titre du document PDF \\
    \code{pdfauthor} & texte & auteur du document PDF
  \end{tabularx}
\end{table}

\begin{exemple}
  Le présent document a été préparé avec les définitions de couleurs
  et les options de configuration de \pkg{hyperref} suivantes:
\begin{lstlisting}
\definecolor{link}{rgb}{0,0.4,0.6}   % ~RoyalBlue de dvips
\definecolor{url}{rgb}{0.6,0,0}      % rouge-brun
\definecolor{citation}{rgb}{0,0.5,0} % vert foncé
\hypersetup{colorlinks, linktocpage,
  linkcolor=link, urlcolor=url, citecolor=citation,
  pdftitle={Rédaction de thèses et mémoires avec LaTeX -
            Partie II},
  pdfauthor={Université Laval}}
\end{lstlisting}
  \qed
\end{exemple}

\begin{exemple}
  Les gabarits de thèses et de mémoires livrés avec la classe
  \class{ulthese} contiennent dans le préambule la ligne suivante:
\begin{lstlisting}
\hypersetup{colorlinks,allcolors=ULlinkcolor}
\end{lstlisting}
  Tous les liens de la thèse ou du mémoire seront donc de la même
  couleur, \code{ULlinkcolor} \fcolorbox{black}{ULlinkcolor}{\phantom{xx}},
  une couleur définie par la classe. %
  \qed
\end{exemple}

\begin{conseil}
  L'interaction du paquetage \pkg{hyperref} avec les autres est
  parfois (voire souvent) délicate. Pour cette raison, il est
  fortement recommandé de charger \pkg{hyperref} en tout dernier
  dans dans le préambule.
\end{conseil}



% Hyperliens: configuration, infos du document

% Présentation de code informatique

% Analyse et rapport intégrés [de type Sweave]

% Diapositives

% Contrôle de version

%%% Local Variables:
%%% mode: latex
%%% TeX-engine: xetex
%%% TeX-master: "formation_latex-partie_2"
%%% coding: utf-8
%%% End:
