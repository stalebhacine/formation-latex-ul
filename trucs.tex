\chapter{Trucs et astuces divers}
\label{chap:trucs}


\section{Polices de caractères: au-delà de Computer Modern}
\label{sec:trucs:police}

{\fontfamily{cmr}\selectfont%
  Les documents {\LaTeX} standards sont facilement reconnaissables par
  leur police de caractères par défaut, celle utilisée dans ce
  paragraphe, Computer Modern. Pour qui souhaitait briser la relative
  monotonie induite par cette uniformité, il a longtemps été difficile
  d'utiliser une autre police de caractères. Fort heureusement, la
  situation a beaucoup évolué et il est aujourd'hui assez simple de de
  produire des documents {\LaTeX} utilisant des polices de caractères
  variées.}

Avant d'aller plus loin, une mise en garde: si votre document contient
plus que quelques équations mathématiques très simples, le choix de
police devient très restreint. La raison: peu de polices de caractères
comprennent des symboles mathématiques et les informations nécessaires
pour réaliser une mise en page des mathématiques satisfaisant les
standards de haute qualité usuels de {\LaTeX}.

Cela dit, pour qui souhaite aller au-delà de la police Computer Modern
sans trop se compliquer la vie, il existe deux solutions principales.

\begin{enumerate}
\item Utiliser l'une ou l'autre des polices PostScript standards.
  C'est très simple avec toute distribution {\LaTeX} moderne: il
  suffit de charger le paquetage approprié. Consulter la %
  \doc{psnfss2e}{http://texdoc.net/pkg/psnfss/} %
  de l'ensemble de paquetages PSNFSS pour connaître les choix
  disponibles.
\item Utiliser une police OpenType présente sur son système avec le
  moteur {\XeLaTeX}. Seule une poignée de ces polices offrent
  toutefois un support approprié pour les mathématiques. La gestion
  des polices de caractères avec {\XeLaTeX} se fait avec le paquetage
  standard \pkg{fontspec}; consulter sa %
  \doc{fontspec}{http://texdoc.net/pkg/fontspec/}.
\end{enumerate}

Pour les thèses et mémoires de l'Université Laval, la Faculté des
études supérieures et postdoctorales accepte les polices %
{\fontfamily{cmr}\selectfont Computer Modern}, %
{\fontfamily{ptm}\selectfont Times} et %
{\fontfamily{ppl}\selectfont Palatino}. Pour utiliser ces deux
dernières avec {\LaTeX}, on charge respectivement les paquetages
\pkg{mathptmx} ou \pkg{mathpazo}. Avec {\XeLaTeX}, on utilisera les
polices {\fontfamily{qtm}\selectfont Termes} et
{\fontfamily{qpl}\selectfont Pagella} du projet %
\link{http://www.gust.org.pl/projects/e-foundry/tex-gyre/}{TeX Gyre}.
Ce sont des polices très similaires à Times et Palatino, disponibles
en version OpenType et qui fournissent un bon support pour les
mathématiques via le projet frère %
\link{http://www.gust.org.pl/projects/e-foundry/tg-math/}{TeX Gyre Math}.

Le texte principal du présent document est en %
\link{http://tug.org/store/lucida/}{Lucida Bright~OT}, %
une police commerciale de très haute qualité, offrant un excellent
support pour les équations mathématiques et dont les auteurs ont
toujours été des «amis» de la communauté {\LaTeX}. La Bibilothèque de
l'Université Laval détient une licence d'utilisation de cette police.
Les membres de la communauté --- étudiants et personnel --- peuvent
s'en produrer une copie en écrivant à
\href{mailto:lucida@bibl.ulaval.ca}{lucida@bibl.ulaval.ca}.

Hyperliens: configuration, infos du document

Présentation de code informatique

Analyse et rapport intégrés [de type Sweave]

Diapositives

Contrôle de version

%%% Local Variables:
%%% mode: latex
%%% TeX-engine: xetex
%%% TeX-master: "formation_latex-partie_2"
%%% coding: utf-8
%%% End:
