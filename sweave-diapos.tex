%%% Copyright (C) 2018 Vincent Goulet
%%%
%%% Ce fichier fait partie du projet
%%% «Rédaction avec LaTeX»
%%% http://github.com/vigou3/formation-latex-ul
%%%
%%% Cette création est mise à disposition selon le contrat
%%% Attribution-Partage dans les mêmes conditions 4.0
%%% International de Creative Commons.
%%% http://creativecommons.org/licenses/by-sa/4.0/

\section{Programmation lettrée}

\begin{frame}[fragile=singleslide]
  \frametitle{Document source combinant {\LaTeX} et code R}

  \fichier{fichier.Rnw}
\begin{lstlisting}[emph={Sexpr}]
...

L'utilisateur de R interagit avec l'interprète en entrant
des commandes à l'invite de commande:
<<echo=TRUE>>=
2 + 3
@
La commande \verb=exp(1)= donne \Sexpr{exp(1)},
la valeur du nombre $e$.

...
\end{lstlisting}
\end{frame}

\begin{frame}[fragile=singleslide]
  \frametitle{Après traitement par Sweave dans R}

  \texttt{Sweave("fichier.Rnw")}
  $\rightarrow$ \fichier{fichier.tex}
\begin{lstlisting}[emph={Schunk,Sinput,Soutput}]
...

L'utilisateur de R interagit avec l'interprète en entrant
des commandes à l'invite de commande:
\begin{Schunk}
\begin{Sinput}
> 2 + 3
\end{Sinput}
\begin{Soutput}
[1] 5
\end{Soutput}
\end{Schunk}
La commande \verb=exp(1)= donne 2.71828182845905,
la valeur du nombre $e$.

...
\end{lstlisting}

\end{frame}

%%% Local Variables:
%%% TeX-master: "formation-latex-ul-diapos"
%%% TeX-engine: xetex
%%% coding: utf-8
%%% End:
