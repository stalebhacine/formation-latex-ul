\section{Apparence du texte}

\subsection{Police de caractères}

\begin{frame}
  \frametitle{Police de caractères}
  \begin{itemize}
  \item Par défaut, tous les documents {\LaTeX} utilisent la même
    police, {\fontfamily{cmr}\selectfont Computer Modern}
  \item Aujourd'hui plus facile d'utiliser d'autres polices, surtout
    avec {\XeLaTeX}
    \begin{itemize}
    \item voir les fichiers d'exercices et les gabarits de
      \class{ulthese} pour des exemples
    \end{itemize}
  \item Privilégier les polices de grande qualité et très complètes
    (lettres accentuées, grand choix de symboles)
    \begin{itemize}
    \item polices Postscript standards ou leurs clones du projet
      TeX~Gyre
    \end{itemize}
  \item Peu de polices sont adaptées pour les mathématiques
    \begin{itemize}
    \item {\fontfamily{ppl}\selectfont Palatino},
      {\fontfamily{ptm}\selectfont Times}, \textrm{Lucida} (\$) sont des choix sûrs
    \end{itemize}
  \end{itemize}
\end{frame}

\begin{frame}[fragile]
  \frametitle{Changement d'attribut de la police}
  \begin{block}{famille}
    \vspace{-12pt}
    \begin{tabbing}
      \textsf{sans empattements} \qquad\= \verb=\sffamily= \qquad\=
      \verb=\textsf{texte}= \kill
      \small
      \textrm{romain} \> \verb=\rmfamily= \> \verb=\textrm{=\textit{texte}\verb=}= \\
      \texttt{largeur fixe} \> \verb=\ttfamily= \> \verb=\texttt{=\textit{texte}\verb=}= \\
      \textsf{sans empattements} \> \verb=\sffamily= \> \verb=\textsf{=\textit{texte}\verb=}=
    \end{tabbing}
  \end{block}
  \vfill
  \begin{block}{forme}
    \vspace{-12pt}
    \begin{tabbing}
      \textsf{sans empattements} \qquad\= \verb=\sffamily= \qquad\=
      \verb=\textsf{texte}= \kill
      \small
      \textup{\rmfamily droit} \> \verb=\upshape= \> \verb=\textup{=\textit{texte}\verb=}= \\
      \textit{\rmfamily italique} \> \verb=\itshape= \> \verb=\textit{=\textit{texte}\verb=}= \\
      \textsl{penché} \> \verb=\slshape= \> \verb=\textsl{=\textit{texte}\verb=}= \\
      \textsc{\rmfamily petites capitales} \> \verb=\scshape= \> \verb=\textsc{=\textit{texte}\verb=}=
    \end{tabbing}
  \end{block}
  \vfill
  \begin{block}{série}
    \vspace{-12pt}
    \begin{tabbing}
      \textsf{sans empattements} \qquad\= \verb=\sffamily= \qquad\=
      \verb=\textsf{texte}= \kill
      \rmfamily\small
      \textmd{\rmfamily moyen} \> \verb=\mdseries= \> \verb=\textmd{=\textit{texte}\verb=}= \\
      \textbf{\rmfamily gras} \> \verb=\bfseries= \> \verb=\textbf{=\textit{texte}\verb=}= \\
    \end{tabbing}
    \vfill
  \end{block}
  \begin{picture}(0,0)
    \thicklines\color{blue}
    \put(98,30){\dashbox{2}(62,190){}}
    \put(88,20){
      \begin{minipage}[t]{75\unitlength}
        \footnotesize\centering
        s'applique à tout le texte qui suit
      \end{minipage}}
  \end{picture}
  \begin{picture}(0,0)
    \thicklines\color{blue}
    \put(162,30){\dashbox{2}(82,190){}}
    \put(158,20){
      \begin{minipage}[t]{85\unitlength}
        \footnotesize\centering
        s'applique au texte en argument
      \end{minipage}}
  \end{picture}
\end{frame}

\subsection{Taille de la police}

\begin{frame}[fragile]
  \frametitle{Taille de la police}
  \vspace{-2pt}
  \begin{block}{commandes standards}
    \vspace{-10pt}
    \begin{tabbing}
      \verb=\footnotesize= \quad\= \kill
      \verb=\tiny= \> {\tiny vraiment petit} \\
      \verb=\scriptsize= \> {\scriptsize encore plus petit} \\
      \verb=\footnotesize= \> {\footnotesize plus petit} \\
      \verb=\small= \> {\small petit} \\
      \verb=\normalsize= \> {\normalsize normal} \\
      \verb=\large= \> {\large grand} \\
      \verb=\Large= \> {\Large plus grand} \\
      \verb=\LARGE= \> {\LARGE encore plus grand} \\
      \verb=\huge= \> {\huge énorme} \\
      \verb=\Huge= \> {\Huge encore plus énorme}
    \end{tabbing}
  \end{block}
  \vspace{-10pt}
  \pause
  \begin{block}{ajouts de \class{memoir} (et donc \class{ulthese})}
    \vspace{-10pt}
    \begin{tabbing}
      \verb=\footnotesize= \quad\= \kill
      \verb=\miniscule= \quad\> [$<$ \verb=\tiny=] \\
      \verb=\HUGE= \> [$>$ \verb=\Huge=] \\
    \end{tabbing}
  \end{block}
\end{frame}

\subsection{Italique}

\begin{frame}[fragile]
  \frametitle{Italique}
  \begin{itemize}
  \item<1-> Une des propriétés les \emph{plus utilisées} dans le texte
    \vfill
  \item<1-> Commande sémantique:
\begin{lstlisting}
\emph{`\textit{texte}'}
\end{lstlisting}
    \vfill
  \item<2-> Par défaut: texte en italique dans texte droit et vice versa
    \begin{demo}
      \small
      \begin{texample}[0.48\linewidth]
\begin{lstlisting}
C'était un peu \emph{rough}
par moments
\end{lstlisting}
        \producing
        C'était un peu \emph{rough} par moments
      \end{texample}

      \begin{texample}[0.48\linewidth]
\begin{lstlisting}
Il m'a dit: «\emph{Enough
\emph{poutine} for the
week!}»
\end{lstlisting}
        \producing
        Il m'a dit: «\emph{Enough \emph{poutine} for the week!}»
      \end{texample}
    \end{demo}
  \item<3-> Pas de commande pour souligner en {\LaTeX\dots} et ce n'est
    pas une omission!
  \end{itemize}
\end{frame}

\subsection{Listes}

\begin{frame}[fragile]
  \frametitle{Listes}
  \begin{itemize}
  \item Deux principales sortes de listes:
    \begin{enumerate}
    \item \alert{à puce} avec environnement \verb=itemize=
    \item \alert{numérotée} avec environnement \verb=enumerate=
    \end{enumerate}
  \item Possible de les imbriquer les unes dans les autres
  \item Marqueurs adaptés automatiquement jusqu'à 4 niveaux
  \end{itemize}
  \pause

\begin{lstlisting}
\begin{itemize}
\item Deux principales sortes de listes:
  \begin{enumerate}
  \item à puce avec environnement \verb=itemize=
  \item numérotée avec environnement \verb=enumerate=
  \end{enumerate}
\item Possible de les imbriquer les unes
  dans les autres
\item Marqueurs adaptés automatiquement jusqu'à 4 niveaux
\end{itemize}
\end{lstlisting}
\end{frame}

\begin{frame}[plain]
  \begin{conseil}
    \begin{itemize}
    \item {\LaTeX} permet de configurer à peu près toutes les facettes
      de la présentation des listes (puces, alignement, espacement).
    \item Plusieurs paquetages facilitent la configuration.
    \item Nous suggérons \pkg{enumitem} pour une configuration simple.
    \end{itemize}
  \end{conseil}
\end{frame}

\subsection{Notes de bas de page}

\begin{frame}[fragile]
  \frametitle{Notes de bas de page}
  \begin{itemize}
  \item Note de bas de page insérée avec la commande
\begin{lstlisting}
\footnote{`\textit{texte de la note}'}
\end{lstlisting}
  \item Commande doit suivre immédiatement le texte à annoter
  \item Méthode recommandée
\begin{lstlisting}[emph=footnote]
... fera remarquer que Pierre Lasou\footnote{%
  Spécialiste en ressources documentaires} %
fut d'une grande aide dans la préparation de ...
\end{lstlisting}
  \item Numérotation et disposition automatiques
  \end{itemize}
\end{frame}

\subsection{Code source}

\begin{frame}[fragile=singleslide]
  \frametitle{Code source}
  \begin{itemize}
  \item Environnement \verb=verbatim=
\begin{lstlisting}
\begin{verbatim}
Texte disposé exactement tel qu'il est tapé
dans une police à largeur fixe
\end{verbatim}
\end{lstlisting}
  \item Commande \verb=\verb= dont la syntaxe est
\begin{lstlisting}
\verb`\textit{c}'`\textit{source}'`\textit{c}'
\end{lstlisting}
    où \textit{c} est un caractère quelconque ne se trouvant pas dans
    \textit{source}
  \item Pour usage plus intensif, voir le paquetage \pkg{listings}
  \end{itemize}
\end{frame}

%%% >>>
\stepcounter{exerciceref}
\subsection{[~Exercice \theexerciceref~]}

\begin{frame}[plain,fragile=singleslide]
  \begin{exercice}
    \begin{enumerate}
    \item Ouvrir le fichier \fichier{exercice\_complet.tex} et en
      étudier le code source, puis le compiler.
    \item Supprimer l'option \texttt{article} au chargement de la
      classe et compiler de nouveau le document. Observer l'effet de
      cette option.
    \item Effectuer les modifications suivantes au document.
      \begin{enumerate}[a)]
      \item Dernier paragraphe de la première section, placer toute la
        phrase débutant par \texttt{«De simple dérivé»} à l'intérieur
        d'une commande \texttt{{\textbackslash}emph}.
      \item Changer la puce des listes en spécifiant le symbole
        \texttt{\$>\$} pour \verb=ItemLabeli= dans la commande
        \verb=\frenchbsetup= du préambule.
      \end{enumerate}
    \end{enumerate}
  \end{exercice}
\end{frame}
%%% <<<


%%% Local Variables:
%%% mode: latex
%%% TeX-engine: xetex
%%% TeX-master: "formation-latex-ul-diapos"
%%% End:
