\section{Apparence et disposition du texte}

\subsection{Police et style}

\begin{frame}
  \frametitle{Police de caractères}

  Par défaut, {\LaTeX} compose les documents dans la police
  {\CM Computer Modern}.

  \begin{itemize}
  \item Aujourd'hui plus facile d'utiliser d'autres polices, surtout
    avec {\XeLaTeX}
  \item \alert{Attention}: peu de polices adaptées pour les
    mathématiques
  \item Commandes pour modifier les \alert{attributs} de la police
    (famille, forme, graisse)
  \item Commandes pour modifier la \alert{taille} du texte (de
    \cs{tiny} à \cs{Huge})
  \end{itemize}
\end{frame}

\begin{frame}[fragile]
  \frametitle{Italique}
  \begin{itemize}
  \item<1-> Une des propriétés les \emph{plus utilisées} dans le texte
    \vfill
  \item<1-> Commande sémantique:
\begin{lstlisting}
\emph`\marg{texte}'
\end{lstlisting}
    \vfill
  \item<2-> Par défaut: texte en italique dans texte droit et vice versa
    \begin{demo}
      \small
      \begin{texample}[0.48\linewidth]
\begin{lstlisting}
C'était un peu \emph{rough}
par moments
\end{lstlisting}
        \producing
        C'était un peu \emph{rough} par moments
      \end{texample}

      \begin{texample}[0.48\linewidth]
\begin{lstlisting}
Il m'a dit: «\emph{Enough
\emph{poutine} for the
week!}»
\end{lstlisting}
        \producing
        Il m'a dit: «\emph{Enough \emph{poutine} for the week!}»
      \end{texample}
    \end{demo}
  \item<3-> Pas de commande pour souligner en {\LaTeX\dots} et ce n'est
    pas une omission!
  \end{itemize}
\end{frame}

\subsection{Disposition sur la page}

\begin{frame}[fragile]
  \frametitle{Listes}
  \begin{itemize}
  \item Deux principales sortes de listes:
    \begin{enumerate}
    \item \alert{à puce} avec environnement \texttt{itemize}
    \item \alert{numérotée} avec environnement \texttt{enumerate}
    \end{enumerate}
  \item Possible de les imbriquer les unes dans les autres
  \item Marqueurs adaptés automatiquement jusqu'à 4 niveaux
  \end{itemize}
  \pause

\begin{lstlisting}
\begin{itemize}
\item Deux principales sortes de listes:
  \begin{enumerate}
  \item à puce avec environnement \texttt{itemize}
  \item numérotée avec environnement \texttt{enumerate}
  \end{enumerate}
\item Possible de les imbriquer les unes dans les autres
\item Marqueurs adaptés automatiquement jusqu'à 4 niveaux
\end{itemize}
\end{lstlisting}
\end{frame}

\begin{frame}[fragile=singleslide]
  \frametitle{Texte centré}

  \begin{itemize}
  \item Environnement \texttt{center} pour centrer un bloc de texte
    \begin{demo}
      \begin{eqxample}
\begin{lstlisting}
\begin{center}
  Centrer un mot ou une
  expression les met en
  évidence.
\end{center}
\end{lstlisting}
        \producing
        \begin{center}
          Centrer un mot ou une expression
          les met en évidence.
        \end{center}
      \end{eqxample}
    \end{demo}
  \item Commande \cs{centering} pour centrer tout le texte qui suit
    \begin{itemize}
    \item surtout utilisée pour centrer les tableaux et les figures
    \end{itemize}
  \end{itemize}
\end{frame}

\begin{frame}[fragile]
  \frametitle{Notes de bas de page}
  \begin{itemize}
  \item Note de bas de page insérée avec la commande
\begin{lstlisting}
\footnote`\marg{texte de la note}'
\end{lstlisting}
  \item Commande doit suivre immédiatement le texte à annoter
  \item Numérotation et disposition automatiques
  \end{itemize}
\end{frame}

\begin{frame}[fragile=singleslide]
  \frametitle{Code source}
  \begin{itemize}
  \item Environnement \texttt{verbatim}
\begin{lstlisting}
\begin{verbatim}
Texte disposé exactement tel qu'il est tapé
dans une police à largeur fixe
\end{verbatim}
\end{lstlisting}
  \item Pour usage plus intensif, utiliser le paquetage \pkg{listings}
  \end{itemize}
\end{frame}

\subsection{[~Exercice~]}

\begin{exercice}
  Utiliser le fichier \fichier{exercice\_complet.tex}.

  \begin{enumerate}
  \item Étudier le code source du fichier, puis le compiler.
  \item Supprimer l'option \texttt{article} au chargement de la classe
    et compiler de nouveau le document. Observer l'effet de cette
    option.
  \item Effectuer les modifications suivantes au document.
    \begin{enumerate}[a)]
    \item Dernier paragraphe de la première section, placer toute la
      phrase débutant par \texttt{«De simple dérivé»} à l'intérieur
      d'une commande \cs{emph}.
    \item Changer la puce des listes en spécifiant le symbole
      \texttt{\$>\$} pour \texttt{ItemLabeli} dans la commande
      \cs{frenchbsetup} du préambule.
    \end{enumerate}
  \end{enumerate}
\end{exercice}

%%% Local Variables:
%%% TeX-master: "formation-latex-ul-diapos"
%%% TeX-engine: xetex
%%% coding: utf-8
%%% End:
