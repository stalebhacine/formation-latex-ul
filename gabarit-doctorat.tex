%% GABARIT POUR THÈSE STANDARD
%%
%% Consulter la documentation de la classe ulthese pour une
%% description détaillée de la classe, de ce gabarit et des options
%% disponibles.
%%
%% [Ne pas hésiter à supprimer les commentaires après les avoir lus.]
%%
%% Déclaration de la classe avec le type de grade
%%   [l'un de PhD, LLD, DPsy, DThP]
%% et les langues les plus courantes. Le français sera la langue par
%% défaut du document.
\documentclass[PhD,english,francais]{ulthese}
  %% Encodage utilisé pour les caractères accentués dans les fichiers
  %% source du document. Les gabarits sont encodés en UTF-8. Inutile avec
  %% XeLaTeX, qui gère Unicode nativement.
  \ifxetex\else \usepackage[utf8]{inputenc} \fi

  %% Charger ici les autres paquetages nécessaires pour le document.
  %% Quelques exemples; décommenter au besoin.
  %\usepackage{amsmath}          % recommandé pour les mathématiques
  %\usepackage{icomma}           % gestion de la virgule dans les nombres

  %% Utilisation d'une autre police de caractères pour le document.
  %% - Sous LaTeX
  %\usepackage{mathpazo}         % texte et mathématiques en Palatino
  %\usepackage{mathptmx}         % texte et mathématiques en Times
  %% - Sous XeLaTeX
  %\setmainfont{TeX Gyre Pagella}      % texte en Pagella (Palatino)
  %\setmathfont{TeX Gyre Pagella Math} % mathématiques en Pagella (Palatino)
  %\setmainfont{TeX Gyre Termes}       % texte en Termes (Times)
  %\setmathfont{TeX Gyre Termes Math}  % mathématiques en Termes (Times)

  %% Gestion des hyperliens dans le document. S'assurer que hyperref
  %% est le dernier paquetage chargé.
  \usepackage{hyperref}
  \hypersetup{colorlinks,allcolors=ULlinkcolor}

  %% Options de mise en forme du mode français de babel. Consulter la
  %% documentation du paquetage babel pour les options disponibles.
  %% Désactiver (effacer ou mettre en commentaire) si l'option
  %% 'nobabel' est spécifiée au chargement de la classe.
  \frenchbsetup{%
    CompactItemize=false,         % ne pas compacter les listes
    ThinSpaceInFrenchNumbers=true % espace fine dans les nombres
  }

  %% Style de la bibliographie.
  %\bibliographystyle{}

  %% Déclarations de la page titre. Remplacer les éléments entre < >.
  %% Supprimer les caractères < >. Couper un long titre ou un long
  %% sous-titre manuellement avec \\.
  \titre{<Titre principal>}
  % \titre{Ceci est un exemple de long titre \\
  %   avec saut de ligne manuel}
  % \soustitre{Sous-titre le cas échéant}
  % \soustitre{Ceci est un exemple de long sous-titre \\
  %   avec saut de ligne manuel}
  \auteur{<Prénom Nom>}
  \programme{Doctorat en <discipline> -- <majeure, s'il y a lieu>}
  \annee{<20xx>}

\begin{document}

\frontmatter                    % pages liminaires

\pagetitre                      % production de la page titre

\tableofcontents                % production de la TdM
\cleardoublepage

\listoftables                   % production de la liste des tableaux
\cleardoublepage

\listoffigures                  % production de la liste des figures
\cleardoublepage

\mainmatter                     % corps du document

\chapter{B.a.-ba du mode mathématique}

De manière générale, soit $x$ un nombre (entier pour le moment) dans
la base de numération $b$ composé de $m$ chiffres ou symboles, c'est-à-dire
\begin{equation*}
  x = x_{m-1}x_{m-2} \cdots x_1x_0,
\end{equation*}
où $0 \leq x_i \leq b - 1$. On a donc
\begin{equation}
  \label{eq:ordinateurs:def_base}
  x = \sum_{i = 0}^{m - 1} x_i b^i.
\end{equation}

Lorsque le contexte ne permet pas de déterminer avec certitude la base
d'un nombre, celle-ci est identifiée en indice du nombre par un nombre
décimal. Par exemple, $10011_2$ est le nombre binaire $10011$.

Soit le nombre décimal $348$. Selon la notation ci-dessus, on a $x_0 =
8$, $x_1 = 4$, $x_2 = 3$ et $b = 10$. En effet, et en conformité avec
\eqref{eq:ordinateurs:def_base},
\begin{displaymath}
  348 = 3 \times 10^2 + 4 \times 10^1 + 8 \times 10^0.
\end{displaymath}
Ce nombre a les représentations suivantes dans d'autres bases. En binaire:
\begin{align*}
  101011100_{2}
  &= 1 \times 2^8 + 0 \times 2^7 + 1 \times 2^6 \\
  &\phantom{=} + 0 \times 2^5 + 1 \times 2^4 + 1 \times 2^3 \\
  &\phantom{=} + 1 \times 2^2 + 0 \times 2^1 + 0 \times 2^0.
\end{align*}
En octal:
\begin{equation*}
  534_{8} = 5 \times 8^2 + 3 \times 8^1 + 4 \times 8^0.
\end{equation*}
En hexadécimal:
\begin{equation*}
  15\mathrm{C}_{16} = 1 \times 16^2 + 5 \times 16^1 + 12 \times 16^0.
\end{equation*}
Des représentations ci-dessus, l'hexadécimale est la plus compacte:
elle permet de représenter avec un seul symbole un nombre binaire
comptant jusqu'à quatre chiffres. C'est, entre autres, pourquoi
c'est une représentation populaire en informatique.%

Dans un ordinateur réel (par opposition à théorique), l'espace
disponible pour stocker un nombre est fini, c'est-à-dire que $m <
\infty$. Le plus grand nombre que l'on peut représenter avec $m$
chiffres ou symboles en base $b$ est
\begin{align*}
  x_{\text{max}}
  &= \sum_{i = 0}^{m - 1} (b - 1) b^i \\
  &= (b - 1) \sum_{i = 0}^{m - 1} b^i \\
  &= (b - 1)
  \left(
    \frac{b^m - 1}{b - 1}
  \right) \\
  &= b^m - 1.
\end{align*}


\end{document}
