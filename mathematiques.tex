\chapter{Mathématiques}
\label{chap:math}


S'il est un domaine où {\LaTeX} brille particulièrement, c'est bien
dans la préparation et la présentation d'équations mathématiques ---
des plus simples aux plus complexes. Après tout, l'amélioration de la
qualité typographique des équations mathématiques dans son ouvrage
phare \emph{The Art of Computer Programming} figurait parmi les
objectifs premiers de Knuth lorsqu'il a développé {\TeX}.

La première partie de cette formation aborde le sujet des
mathématiques, mais en s'en tenant qu'aux principes de base
\citep[section~7]{UL:latex:1}. [...]

Trop de symboles, consulter \citet{comprehensive} %
\doc[???]{comprehensive}{http://texdoc.net/pkg/comprehensive}


\section{Rappel des principes de base}
\label{sec:math:rappel}

La mise en forme d'équations mathématiques requiert d'indiquer à
l'ordinateur, dans un langage spécial, le contenu des dites équations
et la position des symboles: en exposant, en indice, sous forme de
fraction, etc. L'ordinateur peut ensuite assembler le tout à partir de
règles typographiques portant, par exemple, sur la représentation des
variables et des constantes, l'espacement entre les symboles ou la
disposition des équations selon qu'elles apparaissent au fil du texte
ou hors d'un paragraphe.

On indique à {\LaTeX} que l'on change de «langage», par l'utilisation
d'un mode mathématique. Il y a deux grandes manière d'activer le mode
mathématique:
\begin{enumerate}
\item en insérant le code entre les symboles \verb=$ $= pour générer
  une équation «en ligne», ou au fil du texte;
  \begin{trivlist}
  \item
    \begin{texinput}{0.48\linewidth}
\begin{lstlisting}
on sait que $(a + b)^2 = a^2
+ b^2$, d'où on obtient...
\end{lstlisting}
    \end{texinput}
    \hfill
    \begin{texoutput}[c]{0.48\linewidth}
      on sait que $(a + b)^2 = a^2 + b^2$, d'où on obtient...
    \end{texoutput}
  \end{trivlist}
\item en utilisant un environnement servant à créer une équation hors
  paragraphe;
\begin{trivlist}
  \item
    \begin{texinput}{0.48\linewidth}
\begin{lstlisting}
on sait que
\begin{equation*}
  (a + b)^2 = a^2 + b^2,
\end{equation*}
d'où on obtient...
\end{lstlisting}
    \end{texinput}
    \hfill
    \begin{texoutput}[c]{0.48\linewidth}
      on sait que
      \begin{equation*}
        (a + b)^2 = a^2 + b^2,
      \end{equation*}
      d'où on obtient...
    \end{texoutput}
  \end{trivlist}
\end{enumerate}

En mode mathématique, les chiffres sont automatiquement considérés
comme des constantes, les lettres comme des variables et une suite de
lettres comme un produit de variables (nous verrons plus loin comment
représenter des fonctions mathématiques comme $\sin$, $\log$ ou
$\lim$). Ceci a trois conséquences principales:
\begin{enumerate}
\item conformément aux conventions typographiques, les chiffres sont
  représentés en caractère \textrm{romain} et les variables en
  \emph{italique};
  \begin{trivlist}
  \item
    \begin{texinput}{0.48\linewidth}
\begin{lstlisting}
123xyz
\end{lstlisting}
    \end{texinput}
    \hfill
    \begin{texoutput}{0.48\linewidth}
      $123xyz$
    \end{texoutput}
  \end{trivlist}
\item l'espace entre les constantes, les variables et les opérateurs
  mathématiques est géré automatiquement;
  \begin{trivlist}
  \item
    \begin{texinput}{0.48\linewidth}
\begin{lstlisting}
z = 2 x + 3 x y
\end{lstlisting}
    \end{texinput}
    \hfill
    \begin{texoutput}{0.48\linewidth}
        $z = 2 x + 3 x y$
    \end{texoutput}
  \end{trivlist}
\item les espaces dans le code source n'ont aucun impact sur la
  disposition d'une équation.
  \begin{trivlist}
  \item
    \begin{texinput}{0.48\linewidth}
\begin{lstlisting}
z=2x + 3xy
\end{lstlisting}
    \end{texinput}
    \hfill
    \begin{texoutput}{0.48\linewidth}
      $z=2x + 3xy$
    \end{texoutput}
  \end{trivlist}
\end{enumerate}


% Pensez simplement
% à ce que pourrait être en mots la représentation de l'équation
% \begin{equation*}
%   \int_0^\infty \int_0^1 x_1^{\alpha + 1} x_2^{\beta + 1}\, dx_1 dx_2.
% \end{equation*}
% Cela ressemblerait sans doute à ceci (avec entre parenthèses les mots
% habituellement omis et en gras les symboles qu'il faut pouvoir décrire
% autrement que part un caractère disponible sur le clavier):
% \begin{quote}
%   \textbf{intégrale} de (indice) zéro à (exposant) l'\textbf{infini} \\
%   \textbf{intégrale} de (indice) zéro à (exposant) à un \\
%   $x$ (indice) un, puissance \textbf{alpha} plus un \\
%   $x$ (indice) deux, puissance \textbf{beta} plus un \\
%   $d$ $x$ (indice) un \\
%   $d$ $x$ (indice) deux.
% \end{quote}

Quant au langage retenu par {\LaTeX} pour décrire les équations
mathématiques, il est très similaire à celui que l'on utiliserait pour
le faire à voix haute; nous y reviendrons. Il faut simplement avoir
recours à des commandes pour identifier les symboles mathématiques que
l'on ne retrouve pas sur un clavier usuel, comme les lettres grecques,
les opérateurs d'inégalité ou les symboles de sommes et d'intégrales.


\section{Un paquetage incontournable}
\label{sec:math:amsmath}

Le paquetage \pkg{amsmath} produit par la prestigieuse \emph{American
  Mathematical Society} fournit diverses extensions à {\LaTeX} pour
faciliter encore davantage la saisie d'équations mathématiques
complexes et en améliorer le rendu. L'utilisation de ce paquetage doit
être considérée incontournable pour tout document contenant plus que
quelques équations très simples.

Au chapitre des améliorations fournies par \pkg{amsmath}, notons
particulièrement:
\begin{itemize}
\item plusieurs environnements pour les équations hors paragraphe, en
  particulier pour les équations multi-lignes;
\item une meilleure gestion de l'espacement autour des signes
  d'égalité dans les équations multi-lignes;
\item une commande pour faciliter l'entrée de texte à l'intérieur du
  mode mathématique;
\item un environnement pour la saisie des matrices et des coefficients
  binomiaux;
\item des commandes pour les intégrales multiples;
\item la possibilité de définir de nouveaux opérateurs mathématiques.
\end{itemize}
Nous décrivons certaines de ces fonctionnalités dans la suite, mais
l'utilisateur le moindrement avancé devrait impérativement consulter
la %
\doc[documentation complète]{amsldoc}{http://texdoc.net/pkg/amsmath}
du paquetage.


\section{Principaux éléments du mode mathématique}
\label{sec:math:bases}

\begin{equation*}
  \frac{\Gamma(\alpha)}{\lambda^\alpha} =
  \sum_{j = 0}^\infty \int_j^{j + 1} x^{\alpha - 1} e^{-\lambda x}\,
  dx,
  \quad
  \alpha > 0 \text{ et } \lambda > 0.
\end{equation*}

\subsection{Exposants et indices}
\label{sec:math:bases:exposants}

{\LaTeX} permet de créer facilement et avec la bonne taille de
symboles n'importe quelle combinaison d'exposants et d'indices.

\begin{itemize}
\item On place un caractère en exposant d'un autre avec la commande
  \verb=^= et en indice avec la commande \verb=_=. Les indices et
  exposants se combinent naturellement.
  \begin{trivlist}
  \item
    \begin{texinput}{0.2\linewidth}
\begin{lstlisting}
x^2
\end{lstlisting}
    \end{texinput}
    \quad
    \begin{texoutput}{0.1\linewidth}
      $x^2$
    \end{texoutput}
    \hfill
    \begin{texinput}{0.2\linewidth}
\begin{lstlisting}
a_n
\end{lstlisting}
    \end{texinput}
    \quad
    \begin{texoutput}{0.1\linewidth}
      $a_n$
    \end{texoutput}
    \hfill
    \begin{texinput}{0.2\linewidth}
\begin{lstlisting}
x_i^k
\end{lstlisting}
    \end{texinput}
    \quad
    \begin{texoutput}{0.1\linewidth}
      $x_i^k$
    \end{texoutput}
  \end{trivlist}
  (L'ordre de saisie n'a pas d'importance; le troisième exemple
  donnerait le même résultat avec \verb=x^k_i=.)
\item Si l'exposant ou l'indice compte plus d'un caractère, il faut
  regrouper le tout entre accolades \verb={ }=.
  \begin{trivlist}
  \item
    \begin{texinput}{0.2\linewidth}
\begin{lstlisting}
x^{2k + 1}
\end{lstlisting}
    \end{texinput}
    \quad
    \begin{texoutput}{0.1\linewidth}
      $x^{2k + 1}$
    \end{texoutput}
    \hfill
    \begin{texinput}{0.2\linewidth}
\begin{lstlisting}
x_{i,j}
\end{lstlisting}
    \end{texinput}
    \quad
    \begin{texoutput}{0.1\linewidth}
      $x_{i,j}$
    \end{texoutput}
    \hfill
    \begin{texinput}{0.2\linewidth}
\begin{lstlisting}
x_{ij}^{2n}
\end{lstlisting}
    \end{texinput}
    \quad
    \begin{texoutput}{0.1\linewidth}
      $x_{ij}^{2n}$
    \end{texoutput}
  \end{trivlist}
\item Toutes les combinaisons d'exposants et d'indices sont possibles,
  y compris les puissances de puissances ou les indices d'indices.
  \begin{trivlist}
  \item
    \begin{texinput}{0.2\linewidth}
\begin{lstlisting}
e^{-x^2}
\end{lstlisting}
    \end{texinput}
    \quad
    \begin{texoutput}{0.1\linewidth}
      $e^{-x^2}$
    \end{texoutput}
    \hfill
    \begin{texinput}{0.4\linewidth}
\begin{lstlisting}
A_{i_s, k^n}^{y_i}
\end{lstlisting}
    \end{texinput}
    \quad
    \begin{texoutput}{0.2\linewidth}
      $A_{i_s,k^n}^{y_i}$
    \end{texoutput}
    \quad \mbox{}     % pour alignement avec bloc d'exemples ci-dessus
  \end{trivlist}
\end{itemize}

\begin{important}
  Les commandes \verb=^= et \verb=_= sont permises dans le mode
  mathématique seulement. En fait, si {\TeX} rencontre l'une de ces
  commandes en mode texte, il tentera automatiquement de passer au
  mode mathématique après avoir émis l'avertissement
\begin{verbatim}
! Missing $ inserted.
\end{verbatim}
  Il est assez rare que le résultat soit celui souhaité.
\end{important}

\subsection{Fractions}
\label{sec:math:bases:fractions}

Il y a plusieurs façons de représenter une fraction selon qu'elle se
trouve au fil du texte, dans une équation hors paragraphe ou à
l'intérieur d'une autre fraction.
\begin{itemize}
\item Pour les fractions au fil du texte, il vaut souvent mieux
  utiliser simplement la barre oblique \verb=/= pour séparer le
  numérateur du dénominateur, quitte à utiliser des parenthèses.
  Ainsi, on utilise \verb=$(n + 1)/2$= pour obtenir $(n + 1)/2$.
\item De manière plus générale, la commande
\begin{lstlisting}
\frac`\marg{numérateur}\marg{dénominateur}'
\end{lstlisting}
  dispose \meta{numérateur} au-dessus de \meta{dénominateur}, séparé
  par une ligne horizontale. La taille des caractères s'ajuste
  automatiquement selon que la fraction se trouve au fil
  du texte ou dans une équation hors paragraphe, ainsi que selon la
  position de la fraction dans l'équation.
  \begin{trivlist}
  \item
    \begin{texinput}{0.48\linewidth}
\begin{lstlisting}
% taille au fil du texte
On a $z_1 = \frac{x}{y}$ et
$z_2 = xy$.
\end{lstlisting}
    \end{texinput}
    \hfill
    \begin{texoutput}[c]{0.48\linewidth}
      On a $z_1 = \frac{x}{y}$ et $z_2 = xy$.
    \end{texoutput}
  \item
    \begin{texinput}{0.48\linewidth}
\begin{lstlisting}
% taille hors paragraphe
On a
\begin{equation*}
  z_1 = \frac{x}{y}
\end{equation*}
et $z_2 = xy$.
\end{lstlisting}
    \end{texinput}
    \hfill
    \begin{texoutput}[c]{0.48\linewidth}
      On a
      \begin{equation*}
        z_1 = \frac{x}{y}
      \end{equation*}
      et $z_2 = xy$.
    \end{texoutput}
  \item
    \begin{texinput}{0.48\linewidth}
\begin{lstlisting}
% deux tailles combinées
Soit
\begin{equation*}
  z = \frac{\frac{x}{2}
    + 1}{y}.
\end{equation*}
\end{lstlisting}
    \end{texinput}
    \hfill
    \begin{texoutput}[c]{0.48\linewidth}
      Soit
      \begin{equation*}
        z = \frac{\frac{x}{2} + 1}{y}.
      \end{equation*}
    \end{texoutput}
  \end{trivlist}
\item Les commandes
\begin{lstlisting}
\dfrac`\marg{numérateur}\marg{dénominateur}'
\tfrac`\marg{numérateur}\marg{dénominateur}'
\end{lstlisting}
  de \pkg{amsmath} permettent de forcer une fraction à adopter la
  taille d'une fraction hors paragraphe (\emph{displayed}) dans le cas
  de \cmd{\dfrac} ou de celle d'une fraction au fil du texte
  (\emph{text}) dans le cas de \cmd{\tfrac}.
  %% quelque chose à cet effet dans exercices ou dans présentation de 'cases'
\item Il est parfois visuellement plus intéressant, surtout au fil du
  texte, d'écrire une fraction comme $1/x$ sous le forme $x^{-1}$.
\end{itemize}

\subsection{Racines}
\label{sec:math:bases:racines}

La commande
\begin{lstlisting}
\sqrt`\oarg{n}\marg{radicande}'
\end{lstlisting}
construit un symbole de radical autour de \meta{radicande}, par défaut
la racine carrée. Si l'argument optionnel \meta{n} est spécifié, c'est
plutôt un symbole de racine d'ordre $n$ qui est tracé. La longueur et
la hauteur du radical s'adapte toujours à celles du radicande.
\begin{trivlist}
\item
  \begin{texinput}{0.2\linewidth}
\begin{lstlisting}
\sqrt{2}
\end{lstlisting}
  \end{texinput}
  \quad
  \begin{texoutput}{0.1\linewidth}
    $\sqrt{2}$
  \end{texoutput}
  \hfill
  \begin{texinput}{0.2\linewidth}
\begin{lstlisting}
\sqrt{625}
\end{lstlisting}
  \end{texinput}
  \quad
  \begin{texoutput}{0.1\linewidth}
    $\sqrt{625}$
  \end{texoutput}
  \hfill
  \begin{texinput}{0.2\linewidth}
\begin{lstlisting}
\sqrt[3]{8}
\end{lstlisting}
  \end{texinput}
  \quad
  \begin{texoutput}{0.1\linewidth}
    $\sqrt[3]{8}$
  \end{texoutput}
\item
  \begin{texinput}{0.48\linewidth}
\begin{lstlisting}
\sqrt[n]{x + y + z}
\end{lstlisting}
  \end{texinput}
  \hfill
  \begin{texoutput}{0.48\linewidth}
    $\sqrt[n]{x + y + z}$
  \end{texoutput}
\item
  \begin{texinput}{0.48\linewidth}
\begin{lstlisting}
\sqrt{\frac{x + y}{x^2 - y^2}}
\end{lstlisting}
  \end{texinput}
  \hfill
  \begin{texoutput}[c]{0.48\linewidth}
    $\displaystyle \sqrt{\frac{x + y}{x^2 - y^2}}$
  \end{texoutput}
\end{trivlist}

\subsection{Sommes et intégrales}
\label{sec:math:bases:sommes-et-integrales}

Les sommes et intégrales requièrent un symbole spécial ainsi que des
limites inférieures et supérieures, le cas échéant.
\begin{itemize}
\item Les commandes \cmd{\sum} et \cmd{\int} servent respectivement à tracer les
  symboles de somme $\sum$ et d'intégrale $\int$.
\item On entre les éventuelles limites inférieures et supérieures
  comme des indices et des exposants.
  \begin{trivlist}
  \item
    \begin{texinput}{0.48\linewidth}
\begin{lstlisting}
\sum_{i = 0}^n x_i
\end{lstlisting}
    \end{texinput}
    \hfill
    \begin{texoutput}{0.48\linewidth}
      $\displaystyle \sum_{i = 0}^n x_i$
    \end{texoutput}
  \item
    \begin{texinput}{0.48\linewidth}
\begin{lstlisting}
\int_0^{10} f(x)\, dx
\end{lstlisting}
    \end{texinput}
    \hfill
    \begin{texoutput}{0.48\linewidth}
      $\displaystyle \int_0^{10} f(x)\, dx$
    \end{texoutput}
  \end{trivlist}
\item La taille des symboles et la position des limites s'ajustent
  automatiquement selon le contexte. Au fil du texte, la somme et
  l'intégrale ci-dessus apparaîtraient comme $\sum_{i = 0}^n x_i$ et
  $\int_0^{10} f(x)\, dx$.
\item Dans une intégrale il est recommandé de séparer l'intégrande de
  l'opérateur de différentiation $dx$ par une espace fine. C'est ce à
  quoi sert la commande \cmd{\,} ci-dessus.
\end{itemize}


\subsection{Texte et espaces}
\label{sec:math:bases:texte}



symboles

suites d'équations
  left ... right
  numérotation

vecteurs et matrices

%%%
%%% Exercices
%%%

\section{Exercices}
\label{sec:math:exercices}

\Opensolutionfile{solutions}[solutions-mathematiques]

\begin{Filesave}{solutions}
\section*{Chapitre \ref*{chap:math}}
\addcontentsline{toc}{section}{Chapitre \protect\ref*{chap:math}}

\end{Filesave}



\Closesolutionfile{solutions}

%%% Local Variables:
%%% mode: latex
%%% TeX-engine: xetex
%%% TeX-master: "formation_latex-partie_2"
%%% coding: utf-8
%%% End:
