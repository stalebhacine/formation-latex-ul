\chapter{Mathématiques}
\label{chap:math}


S'il est un domaine où {\LaTeX} brille particulièrement, c'est bien
dans la préparation et la présentation d'équations mathématiques ---
des plus simples aux plus complexes. Après tout, l'amélioration de la
qualité typographique des équations mathématiques dans son ouvrage
phare \emph{The Art of Computer Programming} figurait parmi les
objectifs premiers de Knuth lorsqu'il a développé {\TeX}.


\section{Rappel des principes de base}
\label{sec:math:rappel}

Par souci d'exhaustivité, nous revenons d'abord sur les quelques
principes de base présentés dans la première partie de cette formation
\citep[section~7]{UL:latex:1}.

La mise en forme d'équations mathématiques requiert d'indiquer à
l'ordinateur, dans un langage spécial, le contenu des dites équations
et la position des symboles: en exposant, en indice, sous forme de
fraction, etc. L'ordinateur peut ensuite assembler le tout à partir de
règles typographiques portant, par exemple, sur la représentation des
variables et des constantes, l'espacement entre les symboles ou la
disposition des équations selon qu'elles apparaissent au fil du texte
ou hors d'un paragraphe.

On indique à {\LaTeX} que l'on change de «langage», par l'utilisation
d'un mode mathématique. Il y a deux grandes manière d'activer le mode
mathématique:
\begin{enumerate}
\item en insérant le code entre les symboles \verb=$ $= pour générer
  une équation «en ligne», ou au fil du texte;
  \begin{demo}
    \begin{texample}
\begin{lstlisting}
on sait que $(a + b)^2 =
a^2 + 2ab + b^2$, d'où
on obtient...
\end{lstlisting}
      \producing
      on sait que $(a + b)^2 = a^2 + 2ab + b^2,$ d'où on obtient...
    \end{texample}
  \end{demo}
\item en utilisant un environnement servant à créer une équation hors
  paragraphe;
  \begin{demo}
    \begin{texample}
\begin{lstlisting}
on sait que
\begin{equation*}
  (a + b)^2
  = a^2 + 2ab + b^2,
\end{equation*}
d'où on obtient...
\end{lstlisting}
      \producing
      on sait que
      \begin{equation*}
        (a + b)^2 = a^2 + 2ab + b^2,
      \end{equation*}
      d'où on obtient...
    \end{texample}
  \end{demo}
\end{enumerate}

Dans l'exemple ci-dessus, l'environnement \Pe{equation*} (tiré du
paquetage \pkg{amsmath}; voir la section suivante) crée une
équation hors paragraphe, centrée sur la ligne et non numérotée. Avec
l'environnement \Ie{equation} (donc sans \verb=*= dans le nom),
{\LaTeX} ajoute automatiquement un numéro d'équation séquentiel aligné
sur la marge de droite:
\begin{demo}
  \begin{texample}
\begin{lstlisting}
on sait que
\begin{equation}
  (a + b)^2 = a^2 + b^2,
\end{equation}
d'où on obtient...
\end{lstlisting}
    \producing
    on sait que
    \begin{equation}
      (a + b)^2 = a^2 + b^2,
    \end{equation}
    d'où on obtient...
  \end{texample}
\end{demo}
Cette disposition est la plus usuelle dans les ouvrages mathématiques.
Le type de numérotation diffère selon qu'un document comporte des
chapitres ou non.

En mode mathématique, les chiffres sont automatiquement considérés
comme des constantes, les lettres comme des variables et une suite de
lettres comme un produit de variables (nous verrons plus loin comment
représenter des fonctions mathématiques comme $\sin$, $\log$ ou
$\lim$). Ceci a trois conséquences principales:
\begin{enumerate}
\item conformément aux conventions typographiques, les chiffres sont
  représentés en caractère \textrm{romain} et les variables en
  \emph{italique};
  \begin{demo}
    \begin{texample}
\begin{lstlisting}
$123xyz$
\end{lstlisting}
      \producing
      $123xyz$
    \end{texample}
  \end{demo}
\item l'espace entre les constantes, les variables et les opérateurs
  mathématiques est géré automatiquement;
  \begin{demo}
    \begin{texample}
\begin{lstlisting}
$z = 2 x + 3 x y$
\end{lstlisting}
    \producing
    $z = 2 x + 3 x y$
    \end{texample}
  \end{demo}
\item les espaces dans le code source n'ont aucun impact sur la
  disposition d'une équation.
  \begin{demo}
    \begin{texample}
\begin{lstlisting}
$z=2x + 3xy$
\end{lstlisting}
    \producing
    $z=2x + 3xy$
    \end{texample}
  \end{demo}
\end{enumerate}


% Pensez simplement
% à ce que pourrait être en mots la représentation de l'équation
% \begin{equation*}
%   \int_0^\infty \int_0^1 x_1^{\alpha + 1} x_2^{\beta + 1}\, dx_1 dx_2.
% \end{equation*}
% Cela ressemblerait sans doute à ceci (avec entre parenthèses les mots
% habituellement omis et en gras les symboles qu'il faut pouvoir décrire
% autrement que part un caractère disponible sur le clavier):
% \begin{quote}
%   \textbf{intégrale} de (indice) zéro à (exposant) l'\textbf{infini} \\
%   \textbf{intégrale} de (indice) zéro à (exposant) à un \\
%   $x$ (indice) un, puissance \textbf{alpha} plus un \\
%   $x$ (indice) deux, puissance \textbf{beta} plus un \\
%   $d$ $x$ (indice) un \\
%   $d$ $x$ (indice) deux.
% \end{quote}

Quant au langage retenu par {\LaTeX} pour décrire les équations
mathématiques, il est très similaire à celui que l'on utiliserait pour
le faire à voix haute (??? capsule vidéo???); nous y reviendrons. Il faut simplement avoir
recours à des commandes pour identifier les symboles mathématiques que
l'on ne retrouve pas sur un clavier usuel, comme les lettres grecques,
les opérateurs d'inégalité ou les symboles de sommes et d'intégrales.


\section{Un paquetage incontournable}
\label{sec:math:amsmath}

Le paquetage \pkg{amsmath} \citep{amsmath} produit par la prestigieuse
\emph{American Mathematical Society} fournit diverses extensions à
{\LaTeX} pour faciliter encore davantage la saisie d'équations
mathématiques complexes et en améliorer la présentation. L'utilisation
de ce paquetage doit être considérée incontournable pour tout document
contenant plus que quelques équations très simples.

Au chapitre des améliorations fournies par \pkg{amsmath}, notons
particulièrement:
\begin{itemize}
\item plusieurs environnements pour les équations hors paragraphe, en
  particulier pour les équations multilignes;
\item une meilleure gestion de l'espacement autour des signes
  d'égalité dans les équations multilignes;
\item une commande pour faciliter l'entrée de texte à l'intérieur du
  mode mathématique;
\item un environnement pour la saisie des matrices et des coefficients
  binomiaux;
\item des commandes pour les intégrales multiples;
\item la possibilité de définir de nouveaux opérateurs mathématiques.
\end{itemize}
Nous décrivons certaines de ces fonctionnalités dans la suite, mais
l'utilisateur le moindrement avancé devrait impérativement consulter
la %
\doc[documentation complète]{amsldoc}{http://texdoc.net/pkg/amsmath}
du paquetage.


\section{Principaux éléments du mode mathématique}
\label{sec:math:bases}

Cette section explique comment créer et assembler divers éléments
d'une formule mathématique: exposants, indices, fractions, texte, etc.
Les seuls symboles utilisés sont pour le moment les chiffres et les
lettres latines. La \autoref{sec:math:symboles} présente une
partie de l'éventail de symboles mathématiques offerts par {\LaTeX}.

\subsection{Exposants et indices}
\label{sec:math:bases:exposants}

{\LaTeX} permet de créer facilement et avec la bonne taille de
symboles n'importe quelle combinaison d'exposants et d'indices.

On place un caractère en \textsuperscript{exposant} d'un autre avec la
commande \verb=^= et en \textsubscript{indice} avec la commande
\verb=_=. Les indices et exposants se combinent naturellement.
\begin{demo}
  \def\strut{\rule[-0.4ex]{0pt}{2ex}}
  \begin{minipage}{0.3\linewidth}
    \begin{texample}[0.6\linewidth]
\begin{lstlisting}
x^2
\end{lstlisting}
      \producing\strut $x^2$
    \end{texample}
  \end{minipage}
  \quad
  \begin{minipage}{0.3\linewidth}
    \begin{texample}[0.6\linewidth]
\begin{lstlisting}
a_n
\end{lstlisting}
      \producing\strut $a_n$
    \end{texample}
  \end{minipage}
  \quad
  \begin{minipage}{0.3\linewidth}
    \begin{texample}[0.63\linewidth]
\begin{lstlisting}
x_i^k
\end{lstlisting}
      \producing\strut $x_i^k$
    \end{texample}
  \end{minipage}
\end{demo}
(L'ordre de saisie n'a pas d'importance; le troisième exemple
donnerait le même résultat avec \verb=x^k_i=.)

Si l'exposant ou l'indice compte plus d'un caractère, il faut
regrouper le tout entre accolades \verb={ }=.
\begin{demo}
  \begin{minipage}{0.3\linewidth}
    \begin{texample}[0.6\linewidth]
\begin{lstlisting}
x^{2k + 1}
\end{lstlisting}
      \producing\strut $x^{2k + 1}$
    \end{texample}
  \end{minipage}
  \quad
  \begin{minipage}{0.3\linewidth}
    \begin{texample}[0.6\linewidth]
\begin{lstlisting}
x_{i,j}
\end{lstlisting}
      \producing\strut $x_{i,j}$
    \end{texample}
  \end{minipage}
  \quad
  \begin{minipage}{0.3\linewidth}
    \begin{texample}[0.63\linewidth]
\begin{lstlisting}
x_{ij}^{2n}
\end{lstlisting}
      \producing\strut $x_{ij}^{2n}$
    \end{texample}
  \end{minipage}
\end{demo}

Toutes les combinaisons d'exposants et d'indices sont possibles, y
compris les puissances de puissances ou les indices d'indices.
\begin{demo}
  \begin{minipage}{0.3\linewidth}
    \begin{texample}[0.6\linewidth]
\begin{lstlisting}
e^{-x^2}
\end{lstlisting}
      \producing\strut $e^{-x^2}$
    \end{texample}
  \end{minipage}
  \quad
  \begin{minipage}{0.6\linewidth}
    \begin{texample}[0.6\linewidth]
\begin{lstlisting}
A_{i_s, k^n}^{y_i}
\end{lstlisting}
      \producing\strut $A_{i_s,k^n}^{y_i}$
    \end{texample}
  \end{minipage}
  \quad \mbox{} % pour alignement avec bloc d'exemples ci-dessus
\end{demo}

\begin{important}
  Les commandes \verb=^= et \verb=_= sont permises dans le mode
  mathématique seulement. En fait, si {\TeX} rencontre l'une de ces
  commandes en mode texte, il tentera automatiquement de passer au
  mode mathématique après avoir émis l'avertissement
\begin{verbatim}
! Missing $ inserted.
\end{verbatim}
  Il est assez rare que le résultat soit celui souhaité.
\end{important}

\subsection{Fractions}
\label{sec:math:bases:fractions}

Il y a plusieurs façons de représenter une fraction selon qu'elle se
trouve au fil du texte, dans une équation hors paragraphe ou à
l'intérieur d'une autre fraction.

Pour les fractions au fil du texte, il vaut souvent mieux utiliser
simplement la barre oblique \verb=/= pour séparer le numérateur du
dénominateur, quitte à utiliser des parenthèses. Ainsi, on utilise
\verb=$(n + 1)/2$= pour obtenir $(n + 1)/2$.

De manière plus générale, la commande
\begin{lstlisting}
\frac`\marg{numérateur}\marg{dénominateur}'
\end{lstlisting}
dispose \meta{numérateur} au-dessus de \meta{dénominateur}, séparé par
une ligne horizontale. La taille des caractères s'ajuste
automatiquement selon que la fraction se trouve au fil du texte ou
dans une équation hors paragraphe, ainsi que selon la position de la
fraction dans l'équation.
\begin{demo}
  \begin{texample}
\begin{lstlisting}
% taille au fil du texte
On a $z_1 = \frac{x}{y}$ et
$z_2 = xy$.
\end{lstlisting}
    \producing On a $z_1 = \frac{x}{y}$ et $z_2 = xy$.
  \end{texample}

  \begin{texample}
\begin{lstlisting}
% taille hors paragraphe
On a
\begin{equation*}
  z_1 = \frac{x}{y}
\end{equation*}
et $z_2 = xy$.
\end{lstlisting}
    \producing On a
    \begin{equation*}
      z_1 = \frac{x}{y}
    \end{equation*}
    et $z_2 = xy$.
  \end{texample}

  \begin{texample}
\begin{lstlisting}
% deux tailles combinées
Soit
\begin{equation*}
  z = \frac{\frac{x}{2}
    + 1}{y}.
\end{equation*}
\end{lstlisting}
    \producing Soit
    \begin{equation*}
      z = \frac{\frac{x}{2} + 1}{y}.
    \end{equation*}
  \end{texample}
\end{demo}

Les commandes
\begin{lstlisting}
\dfrac`\marg{numérateur}\marg{dénominateur}'
\tfrac`\marg{numérateur}\marg{dénominateur}'
\end{lstlisting}
de \pkg{amsmath} permettent de forcer une fraction à adopter la taille
d'une fraction hors paragraphe (\emph{displayed}) dans le cas de
\cmd{\dfrac} ou de celle d'une fraction au fil du texte (\emph{text})
dans le cas de \cmd{\tfrac}. Consulter l'\autoref{ex:math:matrices} à
la \autopageref{ex:math:matrices} pour visualiser l'effet de la
commande \cmd{\dfrac}.

\begin{conseil}
  Il est parfois visuellement plus intéressant, surtout au fil du
  texte, d'écrire une fraction comme $1/x$ sous la forme $x^{-1}$.
\end{conseil}

\subsection{Racines}
\label{sec:math:bases:racines}

La commande
\begin{lstlisting}
\sqrt`\oarg{n}\marg{radicande}'
\end{lstlisting}
construit un symbole de radical autour de \meta{radicande}, par défaut
la racine carrée. Si l'argument optionnel \meta{n} est spécifié, c'est
plutôt un symbole de racine d'ordre $n$ qui est tracé. La longueur et
la hauteur du radical s'adapte toujours à celles du radicande.
\begin{demo}
  \begin{minipage}{0.3\linewidth}
    \begin{texample}[0.6\linewidth]
\begin{lstlisting}
\sqrt{2}
\end{lstlisting}
      \producing $\sqrt{2}$
    \end{texample}
  \end{minipage}
  \quad
  \begin{minipage}{0.3\linewidth}
    \begin{texample}[0.6\linewidth]
\begin{lstlisting}
\sqrt{625}
\end{lstlisting}
      \producing
      $\sqrt{625}$
    \end{texample}
  \end{minipage}
  \quad
  \begin{minipage}{0.3\linewidth}
    \begin{texample}[0.6\linewidth]
\begin{lstlisting}
\sqrt[3]{8}
\end{lstlisting}
      \producing
      $\sqrt[3]{8}$
    \end{texample}
  \end{minipage}

  \begin{texample}
\begin{lstlisting}
\sqrt[n]{x + y + z}
\end{lstlisting}
    \producing
    $\sqrt[n]{x + y + z}$
  \end{texample}

  \begin{texample}
\begin{lstlisting}
\sqrt{\frac{x + y}{x^2 - y^2}}
\end{lstlisting}
    \producing
    $\displaystyle \sqrt{\frac{x + y}{x^2 - y^2}}$
  \end{texample}
\end{demo}

\subsection{Sommes et intégrales}
\label{sec:math:bases:sommes-et-integrales}

Les sommes et intégrales requièrent un symbole spécial ainsi que des
limites inférieures et supérieures, le cas échéant.

Les commandes \cmd{\sum} et \cmd{\int} servent respectivement à tracer
les symboles de somme $\sum$ et d'intégrale $\int$. Le paquetage
\pkg{amsmath} fournit également des commandes comme \cmd{\iint} et
\cmd{\iiint} pour obtenir des symboles d'intégrales multiples finement
disposés ($\iint$ et $\iiint$).

On entre les éventuelles limites inférieures et supérieures comme des
indices et des exposants.
\begin{demo}
  \begin{texample}
\begin{lstlisting}
\sum_{i = 0}^n x_i
\end{lstlisting}
    \producing $\displaystyle \sum_{i = 0}^n x_i$
  \end{texample}

  \begin{texample}
\begin{lstlisting}
\int_0^{10} f(x)\, dx
\end{lstlisting}
    \producing $\displaystyle \int_0^{10} f(x)\, dx$
  \end{texample}

  \begin{texample}
\begin{lstlisting}
\iint_D f(x, y)\, dx dy
\end{lstlisting}
    \producing $\displaystyle \iint_D f(x, y)\, dx dy$
  \end{texample}
\end{demo}

La taille des symboles et la position des limites s'ajustent
automatiquement selon le contexte. Au fil du texte, la somme et
l'intégrale simple ci-dessus apparaîtraient comme $\sum_{i = 0}^n x_i$
et $\int_0^{10} f(x)\, dx$.

Dans une intégrale il est recommandé de séparer l'intégrande de
l'opérateur de différentiation $dx$ par une espace fine. C'est ce à
quoi sert la commande \cmd{\,} ci-dessus; voir aussi le
\autoref{tab:math:espaces} de la \autopageref{tab:math:espaces}.

\subsection{Points de suspension}
\label{sec:math:bases:dots}

Les formules mathématiques comportent fréquemment des points de
suspension dans des suites de variables ou d'opérations. On recommande
d'éviter de les entrer comme trois points finaux consécutifs, car
l'espacement entre les points sera trop petit et le résultat, jugé
disgracieux\footnote{%
  Le résultat exact dépend de la police de caractère utilisée.} %
d'un point de vue typographique: $...$

Le \autoref{tab:math:dots} fournit les commandes {\LaTeX} servant à
générer divers types de points de suspension.

\begin{table}
  \caption{Points de suspension}
  \label{tab:math:dots}
  \centering
  \begin{tabular}{lll}
    \toprule
    commande & type de points & exemple \\
    \midrule
    \cmd{\dots} &  sélection automatique \\
    \cmd{\ldots} & points à la ligne de base & $x_1, \ldots, x_n$ \\
    \cmd{\cdots} & points centrés & $x_1 + \cdots + x_n$ \\
    \cmd{\vdots} & points verticaux & $
                                      \begin{matrix}
                                        x_1 \\ \vdots \\ x_n
                                      \end{matrix}$ \\
    \cmd{\ddots} & points diagonaux & $
                                      \begin{matrix}
                                        x_1 &&\\ &\ddots& \\ && x_n
                                      \end{matrix}$ \\
    \bottomrule
  \end{tabular}
\end{table}

Avec \pkg{amsmath}, la commande \cmd{\dots} tâche de sélectionner
automatiquement entre les points à la ligne de base ou les points
centrés selon le contexte. Comme le résultat est en général le bon,
nous recommandons d'utiliser principalement cette commande pour
insérer des points de suspension en mode mathématique.
\begin{demo}
  \begin{texample}
\begin{lstlisting}
$x_1, \dots, x_n$
\end{lstlisting}
    \producing $x_1, \dots, x_n$
  \end{texample}

  \begin{texample}
\begin{lstlisting}
$x_1, \ldots, x_n$
\end{lstlisting}
    \producing $x_1, \ldots, x_n$
  \end{texample}

  \begin{texample}
\begin{lstlisting}
$x_1 + \dots + x_n$
\end{lstlisting}
    \producing $x_1 + \dots + x_n$
  \end{texample}

  \begin{texample}
\begin{lstlisting}
$x_1 + \cdots + x_n$
\end{lstlisting}
    \producing $x_1 + \cdots + x_n$
  \end{texample}
\end{demo}
Le paquetage définit également les commandes sémantiques
\begin{itemize}
\item \cmd{\dotsc} pour des «points avec des virgules» (\emph{commas});
\item \cmd{\dotsb} pour des «points avec des opérateurs binaires»;
\item \cmd{\dotsm} pour des «points de multiplication»;
\item \cmd{\dotsi} pour des «points avec des intégrales»;
\item \cmd{\dotso} pour des «autres points» (\emph{other}).
\end{itemize}

\subsection{Texte et espaces}
\label{sec:math:bases:texte}

On l'a vu, en mode mathématique {\LaTeX} traite les lettres comme des
variables et gère automatiquement l'espacement entre les divers
symboles. Or, il n'est pas rare que des formules mathématiques
contiennent du texte (notamment des mots comme «où», «si», «quand»).
De plus, il est parfois souhaitable de pouvoir ajuster les blancs
entre des éléments.

La commande de \pkg{amsmath}
\begin{lstlisting}
\text`\marg{texte}'
\end{lstlisting}
insère \meta{texte} dans une formule mathématique. Le texte est inséré
tel quel, sans aucune gestion des espaces avant ou après le texte. Si
des espaces sont nécessaires, ils doivent faire partie de
\meta{texte}.
\begin{demo}
  \begin{texample}
\begin{lstlisting}
f(x) = a e^{-ax}
\text{ pour } x > 0
\end{lstlisting}
    \producing $f(x) = a e^{-ax} \text{ pour } x > 0$
  \end{texample}
\end{demo}

Les commandes
\begin{lstlisting}
\quad
\qquad
\end{lstlisting}
insèrent un blanc de largeur variable selon la taille de la police en
vigueur. La commande \cmd{\quad} insère un blanc de $1$~em (la largeur de
la lettre M dans la police en vigueur), alors que \cmd{\qquad}
insère le double de cette longueur.\footnote{%
  Bien qu'elles soient surtout utilisées dans le mode mathématique,
  les commandes \cmd{\quad} et \cmd{\qquad} sont également valides
  dans le mode texte.}
\begin{demo}
  \begin{texample}
\begin{lstlisting}
f(x) = a e^{-ax},
\quad x > 0
\end{lstlisting}
    \producing
    $f(x) = a e^{-ax}, \quad x > 0$
  \end{texample}
\end{demo}

Le \autoref{tab:math:espaces} répertorie et compare les différentes
commandes qui permettent d'insérer des espaces plus ou moins fines
entre des éléments dans le mode mathématique.

\begin{table}
  \caption{Espaces dans le mode mathématique}
  \label{tab:math:espaces}
  \centering
  \begin{tabular}{lll}
    \toprule
    commande & longueur & exemple \\
    \midrule
             & pas d'espace & \spx{} \\
    \cmd{\,} & $3/18$ de \cmdprint{quad} & \spx{\,} \\
    \cmd{\:} & $4/18$ de \cmdprint{quad} & \spx{\:} \\
    \cmd{\;} & $5/18$ de \cmdprint{quad} & \spx{\;} \\
    \cmd{\!} & $-3/18$ de \cmdprint{quad} & \spx{\!} \\
    \cmd{\quad} & $1$~em & \spx{\quad} \\
    \cmd{\qquad} & $2$~em & \spx{\qquad} \\
    \bottomrule
  \end{tabular}
\end{table}

\subsection{Fonctions et opérateurs}
\label{sec:math:bases:fonctions}

Les règles de typographie des équations mathématiques veulent que les
variables apparaissent en \textit{italique}, mais que les noms de
fonctions, eux, apparaissent en \textrm{romain}, comme le texte
standard. Pensons, ici, à des fonctions comme $\sin$ ou $\log$.

On sait que l'on ne peut entrer le nom d'une fonction tel quel en mode
mathématique, car {\LaTeX} interprétera la suite de lettres comme un
produit de variables:
\begin{demo}
  \begin{minipage}{0.45\linewidth}
    \begin{texample}
\begin{lstlisting}
$2 sin 30$
\end{lstlisting}
      \producing
      $2 sin 30$
    \end{texample}
  \end{minipage}
  \hfill
  \begin{minipage}{0.45\linewidth}
    \begin{texample}
\begin{lstlisting}
$3 log 2$
\end{lstlisting}
      \producing
      $3 log 2$
    \end{texample}
  \end{minipage}
\end{demo}
Or, utiliser à répétition la commande \cmdprint{\text} pour entrer des
noms de fonction se révélerait peu pratique à l'usage.

{\LaTeX} définit donc des commandes pour un grand nombre de fonctions
et d'opérateurs mathématiques standards:
\begin{lstlisting}
\arccos   \cosh    \det    \inf      \limsup   \Pr      \tan
\arcsin   \cot     \dim    \ker      \ln       \sec     \tanh
\arctan   \coth    \exp    \lg       \log      \sin
\arg      \csc     \gcd    \lim      \max      \sinh
\cos      \deg     \hom    \liminf   \min      \sup
\end{lstlisting}
L'espacement autour des fonctions et opérateurs est géré par {\LaTeX}.
\begin{demo}
  \begin{minipage}{0.45\linewidth}
    \begin{texample}
\begin{lstlisting}
$2 \sin30$
\end{lstlisting}
      \producing
      $2 \sin30$
    \end{texample}
  \end{minipage}
  \hfill
  \begin{minipage}{0.45\linewidth}
    \begin{texample}
\begin{lstlisting}
$3 \log 2$
\end{lstlisting}
      \producing
      $3 \log 2$
    \end{texample}
  \end{minipage}
\end{demo}

Certaines des fonctions ci-dessus, notamment \cmd{\lim}, acceptent des
limites comme les symboles de somme et d'intégrale.
\begin{demo}
  \begin{texample}
\begin{lstlisting}
% au fil du texte
\lim_{x \to 0} x = 0
\end{lstlisting}
    \producing $\lim_{x \to 0} x = 0$
  \end{texample}

  \begin{texample}
\begin{lstlisting}
% hors paragraphe
\lim_{x \to 0} x = 0
\end{lstlisting}
    \producing $\displaystyle \lim_{x \to 0} x = 0$
  \end{texample}
\end{demo}

Lorsque des usages particuliers requièrent de nouveaux noms de
fonctions, la commande \cmd{\DeclareMathOperator} de \pkg{amsmath}
permet de les définir; consulter la %
\doc{amsldoc}{http://texdoc.net/pkg/amsmath} %
du paquetage (section 5.1) pour les détails.


\begin{exemple}
  Le matériel passé en revue jusqu'à maintenant permet déjà
  de composer des équations élaborées --- sous réserve qu'elles
  tiennent sur une seule ligne comme dans le présent exemple.

  On présente ci-dessous, pièce par pièce, le code {\LaTeX} pour créer
  l'équation suivante:
  \begin{equation*}
    \int_x^\infty (y - x) f_{X|X > x}(y)\, dy =
    \frac{1}{1 - F_X(x)} \int_x^\infty (y - x) f_X(y)\, dy.
  \end{equation*}
  \begin{demo}
    \begin{texample}[0.58\linewidth]
\begin{lstlisting}
\begin{equation*}
\end{lstlisting}
      \producing
      \emph{équation hors paragraphe}
    \end{texample}

    \begin{texample}[0.58\linewidth]
\begin{lstlisting}
\int_x^\infty
\end{lstlisting}
      \producing
      $\displaystyle \int_x^\infty$
    \end{texample}

    \begin{texample}[0.58\linewidth]
\begin{lstlisting}
(y - x) f_{X|X > x}(y)\, dy =
\end{lstlisting}
      \producing
      $(y - x) f_{X|X > x}(y)\, dy =$
    \end{texample}

    \begin{texample}[0.58\linewidth]
\begin{lstlisting}
\frac{1}{1 - F_X(x)}
\end{lstlisting}
      \producing
      $\dfrac{1}{1 - F_X(x)}$
    \end{texample}

    \begin{texample}[0.58\linewidth]
\begin{lstlisting}
\int_x^\infty (y - x) f_X(y)\, dy
\end{lstlisting}
      \producing
      $\displaystyle \int_x^\infty (y - x) f_X(y)\, dy$
    \end{texample}

    \begin{texample}[0.58\linewidth]
\begin{lstlisting}
\end{equation*}
\end{lstlisting}
      \producing
      \emph{fin de l'environnement}
    \end{texample}
  \end{demo}
  \qed
\end{exemple}


\section{Symboles mathématiques}
\label{sec:math:symboles}

Outre les chiffres et les lettres de l'alphabet, les claviers
d'ordinateurs ne comptent normalement que les symboles
mathématiques suivants:
\begin{center}
%\begin{verbatim}
\verb0+  -  =  <  >  /  :  !  '  |  [  ]  (  )0
%\end{verbatim}
\end{center}
Ceux-ci sont utilisables directement dans les équations. Les
accolades \verb={ }= étant des symboles réservés par {\LaTeX}, il faut
les entrer avec \cmd{\{} et \cmd{\}}, comme dans du texte normal.

Pour représenter les innombrables autres symboles mathématiques, on
aura recours à des commandes qui débutent, comme d'habitude, par le
symbole {\bs} et dont le nom est habituellement dérivé de la
signification mathématique du symbole.

Si un symbole mathématique a été utilisé quelque part dans une
publication, il y a de fortes chances que sa version existe dans
{\LaTeX}. Il serait donc utopique de tenter de faire ici une recension
des symboles disponibles. Nous nous contenterons d'un avant goût des
principales catégories.

L'ouvrage de référence pour connaître les symboles disponibles dans
{\LaTeX} est la bien nommée %
\doc[Comprehensive {\LaTeX} Symbol List]{compre\-hen\-sive}{http://texdoc.net/pkg/comprehensive} %
\citep{comprehensive}. La liste comprend près de \nombre{6000}
symboles répartis sur plus de 160 pages! On y trouve de tout, des
symboles mathématiques aux pictogrammes de météo ou d'échecs, en
passant par\dots\ des figurines des Simpsons.

\begin{important}
  Les moteurs {\XeTeX} et {\LuaTeX} supportent nativement le code
  source en format Unicode UTF-8 \citep{Unicode:5.0}. Ce standard
  contient des définitions pour plusieurs symboles mathématiques
  \citep{wikipedia:unicode-math}. Cela signifie qu'il est possible
  d'entrer une partie au moins des équations mathématiques avec des
  caractères visibles à l'écran, plutôt qu'avec des commandes
  {\LaTeX}. Nous ne saurions toutefois recommander cette pratique qui
  rend les fichiers source moins compatibles d'un système à un autre.
\end{important}

\subsection{Lettres grecques}
\label{sec:math:symboles:grecques}

On obtient les lettres grecques en {\LaTeX} avec des commandes
correspondant au nom de chaque lettre. Lorsque la commande débute par
une capitale, on obtient une lettre grecque majuscule. Les commandes
de certaines lettres grecques majuscules n'existent pas lorsque celles-ci
sont identiques aux lettres romaines.

Les tableaux \ref{tab:math:grecques} et \ref{tab:math:Grecques}
présentent l'ensemble des lettres grecques disponibles dans {\LaTeX}.

\begin{table}
  \caption{Lettres grecques minuscules}
  \label{tab:math:grecques}
  \begin{tabularx}{1.0\linewidth}{lXlXlXlX}
    $\alpha$      & \cmd{\alpha}      & $\theta$    & \cmd{\theta} &
    $o$           & o                 & $\tau$      & \cmd{\tau} \\
    $\beta$       & \cmd{\beta}       & $\vartheta$ & \cmd{\vartheta} &
    $\pi$         & \cmd{\pi}         & $\upsilon$  & \cmd{\upsilon} \\
    $\gamma$      & \cmd{\gamma}      & $\iota$     & \cmd{\iota} &
    $\varpi$      & \cmd{\varpi}      & $\phi$      & \cmd{\phi} \\
    $\delta$      & \cmd{\delta}      & $\kappa$    & \cmd{\kappa} &
    $\rho$        & \cmd{\rho}        & $\varphi$   & \cmd{\varphi} \\
    $\epsilon$    & \cmd{\epsilon}    & $\lambda$   & \cmd{\lambda} &
    $\varrho$     & \cmd{\varrho}     & $\chi$      & \cmd{\chi} \\
    $\varepsilon$ & \cmd{\varepsilon} & $\mu$       & \cmd{\mu} &
    $\sigma$      & \cmd{\sigma}      & $\psi$      & \cmd{\psi} \\
    $\zeta$       & \cmd{\zeta}       & $\nu$       & \cmd{\nu} &
    $\varsigma$   & \cmd{\varsigma}   & $\omega$    & \cmd{\omega} \\
    $\eta$        & \cmd{\eta}        & $\xi$       & \cmd{\xi}
  \end{tabularx}
\end{table}

\begin{table}
  \caption{Lettres grecques majuscules}
  \label{tab:math:Grecques}
  \begin{tabularx}{1.0\linewidth}{lXlXlXlX}
    $\Gamma$    & \cmd{\Gamma}   &
    $\Lambda$   & \cmd{\Lambda}  &
    $\Sigma$    & \cmd{\Sigma}   &
    $\Psi$      & \cmd{\Psi}     \\
    $\Delta$    & \cmd{\Delta}   &
    $\Xi$       & \cmd{\Xi}      &
    $\Upsilon$  & \cmd{\Upsilon} &
    $\Omega$    & \cmd{\Omega}   \\
    $\Theta$    & \cmd{\Theta}   &
    $\Pi$       & \cmd{\Pi}      &
    $\Phi$      & \cmd{\Phi}
  \end{tabularx}
\end{table}

\subsection{Lettres modifiées}
\label{sec:math:symboles:mathcal}

Les lettres de l'alphabet, principalement en majuscule, servent
parfois en mathématiques dans des versions modifiées pour représenter
des quantités, notamment les ensembles.

La commande \cmd{\mathcal} permet de transformer un ou plusieurs
caractères en version dite «calligraphique» dans le mode mathématique.
\begin{demo}
  \begin{texample}
\begin{lstlisting}
\mathcal{ABC\, xyz}
\end{lstlisting}
    \producing
    %% pris de lucidaot.tex; apparemment lent
    \setmathfont[RawFeature=+ss04]{Lucida Bright Math OT}
    $\mathscr{ABC\, xyz}$
  \end{texample}
\end{demo}

La commande \cmd{\mathbb} fournie, entre autres, par les paquetages
\pkg{amsfonts} et \pkg{unicode-math} génère des versions majuscule
ajourée (\emph{blackboard bold}) des lettres de l'alphabet. Elles sont
principalement utilisée pour représenter les ensembles de nombres.
\begin{demo}
  \begin{texample}
\begin{lstlisting}
\mathbb{NZRC}
\end{lstlisting}
    \producing $\mathbb{NZRC}$
  \end{texample}
\end{demo}

Le tableau~213 de la %
\doc[Comprehensive {\LaTeX} Symbol List]{}{http://texdoc.net/pkg/comprehensive} %
présente plusieurs autres alphabets spéciaux disponibles en mode
mathématique.

\begin{conseil}
  Certaines polices de caractères OpenType contiennent plusieurs
  versions des symboles mathématiques. Par exemple, la police utilisée
  dans le présent document contient deux versions de la police
  calligraphique, celle présentée ci-dessus et celle-ci: %
  \setmathfont[RawFeature=-ss04]{Lucida Bright Math OT}
  $\mathscr{ABC\, xyz}$. Consultez éventuellement la documentation de
  la police pour les détails.
\end{conseil}

\subsection{Opérateurs binaires et relations}
\label{sec:math:symboles:binaires+relations}

Les opérateurs binaires combinent deux quantités pour en former une
troisième; pensons simplement aux opérateurs d'addition $+$ et de
soustraction $-$ que l'on retrouve sur un clavier d'ordinateur normal.
Les relations, quant à elles, servent pour la comparaison entre deux
quantités, comme $<$ et $>$. Le \autoref{tab:math:binaires}
présente une sélection d'opérateurs binaires et
le \autoref{tab:math:relations}  une sélection de relations.

La %
\doc[Comprehensive {\LaTeX} Symbol List]{}{http://texdoc.net/pkg/comprehensive} %
consacre plus d'une dizaine de tableaux aux opérateurs binaires et
près d'une quarantaine aux relations. C'est dire à quel point les
tableaux \ref*{tab:math:binaires} et \ref*{tab:math:relations} de la
\autopageref{tab:math:binaires} ne présentent que les principaux
éléments à titre indicatif.

Certaines relations existent directement en version opposée, ou
négative (comme $\neq$ ou $\notin$) soit dans {\LaTeX} de base, soit
avec \pkg{amsmath} ou un autre paquetage. Autrement, il est possible
de préfixer toute relation de \cmd{\not} pour y superposer une barre
oblique $/$.

\begin{table}[p]
  \caption{Quelques opérateurs binaires}
  \label{tab:math:binaires}
  \begin{tabularx}{1.0\linewidth}{lXlXlXlX}
    $\times$    & \cmd{\times} &
    $\div$      & \cmd{\div}   &
    $\pm$       & \cmd{\pm}    &
    $\cdot$     & \cmd{\cdot}    \\
    $\cup$      & \cmd{\cup} &
    $\cap$      & \cmd{\cap} &
    $\setminus$ & \cmd{\setminus} &
    $\circ$     & \cmd{\circ}  \\
    $\wedge$    & \cmd{\wedge} &
    $\vee$      & \cmd{\vee} &
    $\oplus$    & \cmd{\oplus} &
    $\otimes$   & \cmd{\otimes} \\
    $\ast$      & \cmd{\ast} &
    $\star$     & \cmd{\star} &
    $\boxplus$  & \cmd{\boxplus}$^\dagger$ &
    $\boxtimes$ & \cmd{\boxtimes}$^\dagger$ \\
    \addlinespace
  \end{tabularx}
  \hspace*{1em}{\footnotesize $^\dagger$ requiert \pkg{amsmath}}
\end{table}

\begin{table}[p]
  \caption{Quelques relations et leur négation}
  \label{tab:math:relations}
  \begin{tabularx}{1.0\linewidth}{lXlXlXlX}
    $\leq$      & \cmd{\leq} &
    $\geq$      & \cmd{\geq}   &
    $\neq$      & \cmd{\neq}    &
    $\equiv$    & \cmd{\equiv}    \\
    $\subset$   & \cmd{\subset} &
    $\subseteq$ & \cmd{\subseteq}  &
    $\in$       & \cmd{\in} &
    $\notin$    & \cmd{\notin} \\
    $\nless$    & \cmd{\nless}$^\dagger$ &
    $\ngtr$     & \cmd{\ngtr}$^\dagger$   &
    $\nleq$     & \cmd{\nleq}$^\dagger$    &
    $\ngeq$     & \cmd{\ngeq}$^\dagger$ \\
    \addlinespace
  \end{tabularx}
  \hspace*{1em}{\footnotesize $^\dagger$ requiert \pkg{amsmath}}
\end{table}

\subsection{Flèches}
\label{sec:math:symboles:fleches}

Les flèches de différents types sont souvent utilisées en notation
mathématique, notamment dans les limites ou pour les expressions
logiques. Le \autoref{tab:math:fleches} en présente une sélection.

On retrouve les flèches utilisables en notation mathématique dans les
tableaux 102 à 119 de la %
\doc[Comprehensive {\LaTeX} Symbol List]{}{http://texdoc.net/pkg/comprehensive}. %
Le document contient divers autres types de flèches, mais celles-ci ne
sont généralement pas appropriées pour les mathématiques (pensons à
{\manerrarrow} ou {\faArrowRight}).

Le paquetage \pkg{amsmath} fournit plusieurs flèches additionnelles
ainsi que la négation des plus communes. Ces dernières sont d'ailleurs
présentées dans le \autoref{tab:math:fleches}.

\begin{table}[p]
  \caption{Quelques flèches et leur négation}
  \label{tab:math:fleches}
  \begin{tabularx}{1.0\linewidth}{lXlX}
    $\gets$                & \cmd{\leftarrow}\quad \cmd{\gets} &
    $\longleftarrow$       & \cmd{\longleftarrow}              \\
    $\Leftarrow$           & \cmd{\Leftarrow}                  &
    $\Longleftarrow$       & \cmd{\Longleftarrow}              \\
    $\to$                  & \cmd{\rightarrow}\quad \cmd{\to}  &
    $\longrightarrow$      & \cmd{\longrightarrow}             \\
    $\Rightarrow$          & \cmd{\Rightarrow}                 &
    $\Longrightarrow$      & \cmd{\Longrightarrow}             \\
    $\uparrow$             & \cmd{\uparrow}                    &
    $\downarrow$           & \cmd{\downarrow}                  \\
    $\Uparrow$             & \cmd{\Uparrow}                    &
    $\Downarrow$           & \cmd{\Downarrow}                  \\
    $\updownarrow$         & \cmd{\updownarrow}                &
    $\Updownarrow$         & \cmd{\Updownarrow}                \\
    $\leftrightarrow$      & \cmd{\leftrightarrow}             &
    $\longleftrightarrow$  & \cmd{\longleftrightarrow}         \\
    $\Leftrightarrow$      & \cmd{\Leftrightarrow}             &
    $\Longleftrightarrow$  & \cmd{\Longleftrightarrow}         \\
    $\nleftarrow$          & \cmd{\nleftarrow}$^\dagger$         &
    $\nleftrightarrow$     & \cmd{\nleftrightarrow}$^\dagger$    \\
    $\nrightarrow$         & \cmd{\nrightarrow}$^\dagger$        &
    $\nLeftarrow$          & \cmd{\nLeftarrow}$^\dagger$         \\
    $\nLeftrightarrow$     & \cmd{\nLeftrightarrow}$^\dagger$    &
    $\nRightarrow$         & \cmd{\nRightarrow}$^\dagger$        \\
    \addlinespace
  \end{tabularx}
  \hspace*{1em}{\footnotesize $^\dagger$ requiert \pkg{amsmath}}
\end{table}

\subsection{Accents et autres symboles utiles}
\label{sec:math:symboles:autres}

Le \autoref{tab:math:autres} présente quelques uns des accents
disponibles dans le mode mathématique, ainsi que divers symboles
fréquemment utilisés en mathématiques.

Pour connaître l'ensemble des accents du mode mathématique de
{\LaTeX}, consulter le tableau~164 de la %
\doc[Comprehensive {\LaTeX} Symbol List]{}{http://texdoc.net/pkg/comprehensive}. %
Les versions extensibles de certains accents se trouvent au
tableau~169. Quant aux symboles mathématiques divers, on en trouve de
toutes les sortes dans les tableaux 201--212.

\begin{table}[p]
  \caption{Accents et symboles mathématiques divers}
  \label{tab:math:autres}
  \begin{tabularx}{1.0\linewidth}{lXlXlXlX}
    $\hat{a}$ & \cmd{\hat}\verb={a}= &
    $\bar{a}$ & \cmd{\bar}\verb={a}= &
    $\tilde{a}$ & \cmd{\tilde}\verb={a}= &
    $\ddot{a}$ & \cmd{\ddot}\verb={a}= \\
    $\infty$ & \cmd{\infty} &
    $\nabla$ & \cmd{\nabla} &
    $\partial$ & \cmd{\partial} &
    $\ell$ & \cmd{\ell} \\
    $\forall$ & \cmd{\forall} &
    $\exists$ & \cmd{\exists} &
    $\emptyset$ & \cmd{\emptyset} &
    $\prime$ & \cmd{\prime} \\
    $\neg$ & \cmd{\neg} &
    $\backslash$ & \cmd{\backslash} &
    $\|$ & \cmd{\|} &
    $\angle$ & \cmd{\angle}
  \end{tabularx}
\end{table}


\begin{exemple}
  L'équation suivante contient plusieurs des éléments présentés dans
  cette section et la précédente:
  \begin{equation*}
    \frac{\Gamma(\alpha)}{\lambda^\alpha} =
    \sum_{j = 0}^\infty \int_j^{j + 1} x^{\alpha - 1} e^{-\lambda x}\,
    dx,
    \quad
    \alpha > 0 \text{ et } \lambda > 0.
  \end{equation*}
  \begin{demo}
    \begin{texample}[0.58\linewidth]
\begin{lstlisting}
\begin{equation*}
\end{lstlisting}
      \producing
      \emph{équation hors paragraphe}
    \end{texample}

    \begin{texample}[0.58\linewidth]
\begin{lstlisting}
\frac{\Gamma(\alpha)}{
  \lambda^\alpha} =
\end{lstlisting}
      \producing
      $\dfrac{\Gamma(\alpha)}{\lambda^\alpha} =$
    \end{texample}

    \begin{texample}[0.58\linewidth]
\begin{lstlisting}
\sum_{j = 0}^\infty \int_j^{j + 1}
\end{lstlisting}
      \producing
      $\displaystyle \sum_{j = 0}^\infty \int_j^{j + 1}$
    \end{texample}

    \begin{texample}[0.58\linewidth]
\begin{lstlisting}
x^{\alpha - 1} e^{-\lambda x}\, dx
\end{lstlisting}
      \producing
      $\displaystyle x^{\alpha - 1} e^{-\lambda x}\, dx$
    \end{texample}

    \begin{texample}[0.58\linewidth]
\begin{lstlisting}
, \quad \alpha > 0 \text{ et }
\lambda > 0.
\end{lstlisting}
      \producing
      $, \quad \alpha > 0 \text{ et } \lambda > 0.$
    \end{texample}

    \begin{texample}[0.58\linewidth]
\begin{lstlisting}
\end{equation*}
\end{lstlisting}
      \producing
      \emph{fin de l'environnement}
    \end{texample}
  \end{demo}
  \qed
\end{exemple}


\section{Équations sur plusieurs lignes et numérotation}
\label{sec:math:align}

Dans ce qui précède, nous n'avons présenté que des équations tenant
sur une seule ligne en mode hors paragraphe. Cette section se penche
sur la manière de représenter des groupes d'équations du type
\begin{align}
  y &= 2x + 4 \\
  y &= 6x - 1
\end{align}
ou des suites d'équations comme
\begin{align*}
  x_{\text{max}}
  &= \sum_{i = 0}^{m - 1} (b - 1) b^i \\
  &= (b - 1) \sum_{i = 0}^{m - 1} b^i \\
  &= b^m - 1.
\end{align*}

Nous recommandons fortement les environnements de \pkg{amsmath} pour
les équations sur plusieurs lignes: ils sont plus polyvalents, plus
simples à utiliser et leur rendu est meilleur. Le
\autoref{tab:math:displays} --- repris presque intégralement de la
documentation de ce paquetage --- compare les différents
environnements pour les équations hors paragraphe.

\begin{table}[p]
  \caption{Comparaison des environnements pour les équations hors
    paragraphe de \pkg{amsmath} (les lignes verticales indiquent les
    marges logiques).}
  \label{tab:math:displays}
  \renewcommand{\theequation}{\arabic{equation}}
  \begin{eqxample}
\begin{lstlisting}
\begin{equation*}
  a = b
\end{equation*}
\end{lstlisting}
    \producing
    \begin{equation*}
      a = b
    \end{equation*}
  \end{eqxample}

  \begin{eqxample}
\begin{lstlisting}
\begin{equation}
  a = b
\end{equation}
\end{lstlisting}
    \producing
    \begin{equation}
      a = b
    \end{equation}
  \end{eqxample}

  \begin{eqxample}
\begin{lstlisting}
\begin{equation}
  \label{xx}
  \begin{split}
    a &= b + c - d \\
    &\phantom{=} + e - f \\
    &= g + h \\
    &= i
  \end{split}
\end{equation}
\end{lstlisting}
    \producing
    \begin{equation}\label{eq:math:xx}
      \begin{split}
        a& =b+c-d\\
        &\phantom{=} +e-f\\
        & =g+h\\
        & =i
      \end{split}
    \end{equation}
  \end{eqxample}

  \begin{eqxample}
\begin{lstlisting}
\begin{multline}
  a + b + c + d + e + f \\
  + i + j + k + l + m + n
\end{multline}
\end{lstlisting}
    \producing
    \begin{multline}
      a+b+c+d+e+f\\
      +i+j+k+l+m+n
    \end{multline}
  \end{eqxample}

  \begin{eqxample}
\begin{lstlisting}
\begin{gather}
  a_1 = b_1 + c_1 \\
  a_2 = b_2 + c_2 - d_2 + e_2
\end{gather}
\end{lstlisting}
    \producing
    \begin{gather}
      a_1=b_1+c_1\\
      a_2=b_2+c_2-d_2+e_2
    \end{gather}
  \end{eqxample}

  \begin{eqxample}
\begin{lstlisting}
\begin{align}
  a_1 &= b_1 + c_1 \\
  a_2 &= b_2 + c_2 - d_2 + e_2
\end{align}
\end{lstlisting}
    \producing
    \begin{align}
      a_1& =b_1+c_1\\
      a_2& =b_2+c_2-d_2+e_2
    \end{align}
  \end{eqxample}

  \begin{eqxample}
\begin{lstlisting}
\begin{align}
  a_{11} &= b_{11} &
  a_{12} &= b_{12} \\
  a_{21} &= b_{21} &
  a_{22} &= b_{22} + c_{22}
\end{align}
\end{lstlisting}
    \producing
    \begin{align}
      a_{11}& =b_{11}&
      a_{12}& =b_{12}\\
      a_{21}& =b_{21}&
      a_{22}& =b_{22}+c_{22}
    \end{align}
  \end{eqxample}
\end{table}

\begin{itemize}
\item L'environnement de base pour les équations alignées sur un
  symbole de relation (en une ou plusieurs colonnes) est \Ie{align}.
  C'est l'environnement le plus utilisé en mode mathématique hormis
  \Pe{equation}.
\item Dans les environnements \Pe{align} et \Ie{split}, les équations
  successives sont alignées sur le caractère se trouvant immédiatement
  après le marqueur de colonne \verb=&=.
\item Remarquer, dans le troisième exemple du
  \autoref{tab:math:displays}, comment la commande \cmd{\phantom} sert
  à insérer un blanc exactement de la largeur du symbole $=$ au début
  de la seconde ligne de la suite d'égalités.
\item Pour supprimer la numérotation d'une ligne dans une série
  d'équations numérotées, placer la commande \cmd{\notag} juste avant
  la commande de changement de ligne {\bs\bs}.
  \begin{demo}
    \begin{texample}
\begin{lstlisting}
\begin{align}
  a_1 &= b_1 + c_1 \notag \\
  a_2 &= b_2 + c_2 - d_2 + e_2
\end{align}
\end{lstlisting}
      \producing
      \begin{align}
        a_1& =b_1+c_1 \notag \\
        a_2& =b_2+c_2-d_2+e_2
      \end{align}
    \end{texample}
  \end{demo}
\item Les renvois vers des équations numérotées fonctionnent, comme
  partout ailleurs en {\LaTeX}, avec le système d'étiquettes et de
  références (\autoref{sec:tableaux:floats}). Le paquetage
  \pkg{amsmath} fournit également la pratique commande \cmd{\eqref}
  qui place automatiquement le numéro d'équation entre parenthèses.
  \begin{demo}
    \begin{texample}
\begin{lstlisting}
On voit en \eqref{xx} du
tableau `\ref*{tab:math:displays}' que...
\end{lstlisting}
      \producing
      On voit en \eqref{eq:math:xx} du tableau
      \ref*{tab:math:displays} que\dots
    \end{texample}
  \end{demo}
\item L'environnement \Ie{split} sert à apposer un seul numéro à une
  équation affichée sur plusieurs lignes. Il doit être employé à
  l'intérieur d'un autre environnement d'équations hors paragraphe.
\end{itemize}
Consulter le chapitre 3 de la %
\doc{amsldoc}{http://texdoc.net/pkg/amsmath} %
du paquetage \pkg{amsmath} pour les détails sur
l'utilisation des environnements du \autoref*{tab:math:displays}.

\begin{conseil}
  Veillez à respecter les règles suivantes pour la mise en forme des
  équations:
  \begin{enumerate}
  \item Qu'elles apparaissent en ligne ou hors paragraphe, les
    équations font partie intégrante de la phrase, aussi les règles de
    ponctuation usuelles s'appliquent-elles aux équations.
  \item Lorsqu'une équation s'étend sur plus d'une ligne, couper
    chaque ligne \emph{avant} un opérateur de sorte que chaque ligne
    constitue une expression mathématique complète (voir les troisième
    et quatrième exemples du \autoref{tab:math:displays}).
  \item Ne numéroter que les équations d'un document auxquelles le
    texte fait référence.
  \end{enumerate}
\end{conseil}

\begin{exemple}
  Les deux suites d'équations au début de la section ont été réalisées
  avec les blocs de code ci-dessous, dans l'ordre.
\begin{lstlisting}
\begin{align}
  y &= 2x + 4 \\
  y &= 6x - 1
\end{align}
\end{lstlisting}
\begin{lstlisting}
\begin{align*}
  x_{\text{max}}
  &= \sum_{i = 0}^{m - 1} (b - 1) b^i \\
  &= (b - 1) \sum_{i = 0}^{m - 1} b^i \\
  &= b^m - 1.
\end{align*}
\end{lstlisting}
  \qed
\end{exemple}


\section{Délimiteurs de taille variable}
\label{sec:math:delimiteurs}

Les délimiteurs en mathématiques sont des symboles généralement
utilisés en paire tels que les parenthèses $(~)$, les crochets $[~]$
ou les accolades $\{~\}$ et qui servent à regrouper des termes d'une
équation. La taille des délimiteurs doit s'adapter au contenu entre
ceux-ci afin d'obtenir, par exemple, non pas
\begin{equation*}
  (1 + \frac{1}{x}),
\end{equation*}
mais plutôt
\begin{equation*}
  \left( 1 + \frac{1}{x} \right).
\end{equation*}

La paire de commandes
\begin{lstlisting}
\left`\meta{delim\_g}'  ...  \right`\meta{delim\_d}'
\end{lstlisting}
définit un délimiteur gauche \meta{delim\_g} et un délimiteur droit
\meta{delim\_d} dont la taille s'ajustera automatiquement au contenu
entre les deux commandes.
\begin{demo}
  \begin{texample}
\begin{lstlisting}
\left( 1 + \frac{1}{x} \right)
\end{lstlisting}
    \producing
    \begin{equation*}
      \left( 1 + \frac{1}{x} \right)
    \end{equation*}
  \end{texample}

  \begin{texample}
\begin{lstlisting}
\left(
  \sum_{i = 1}^n x_i^2
\right)^{1/2}
\end{lstlisting}
    \producing
    \begin{equation*}
      \left(
        \sum_{i = 1}^n x_i^2
      \right)^{1/2}
    \end{equation*}
  \end{texample}
\end{demo}
Les commandes \cmd{\left} et \cmd{\right} doivent toujours former une
paire, c'est-à-dire qu'à \emph{toute} commande \cmdprint{\left} doit
  absolument correspondre une commande \cmdprint{\right}. Cette
contrainte est facile à oublier!

Il est possible d'imbriquer des paires de commandes les unes à
l'intérieur des autres, pour autant que l'expression compte toujours
autant de \cmdprint{\left} que de \cmdprint{\right}.
\begin{demo}
  \begin{texample}
\begin{lstlisting}
\left[
  \int
  \left(
    1 + \frac{x}{k}
  \right) dx
\right]
\end{lstlisting}
    \producing
    \begin{equation*}
      \left[
        \int
        \left(
          1 + \frac{x}{k}
        \right) dx
      \right]
    \end{equation*}
  \end{texample}
\end{demo}

Les symboles \meta{delim\_g} et \meta{delim\_d} n'ont pas à former une
paire logique; toute combinaison de délimiteurs est valide.
\begin{demo}
  \begin{texample}
\begin{lstlisting}
\int_0^1 x\, dx =
\left[
  \frac{x^2}{2}
\right|_0^1
\end{lstlisting}
    \producing
    \begin{equation*}
      \int_0^1 x\, dx =
      \left[
        \frac{x^2}{2}
      \right|_0^1
    \end{equation*}
  \end{texample}
\end{demo}

Il arrive qu'un seul délimiteur soit nécessaire. Pour respecter la
règle de la paire ci-dessus, on aura recours dans ce cas à un
délimiteur \emph{invisible} représenté par le caractère «\verb=.=».
\begin{demo}
  \begin{texample}
\begin{lstlisting}
f(x) =
\left\{
  \begin{aligned}
    1 - x, &\quad x < 1 \\
    x - 1, &\quad x \geq 1
  \end{aligned}
\right.
\end{lstlisting}
    \producing
    \begin{equation*}
      f(x) =
      \left\{
        \begin{aligned}
          1 - x, &\quad x < 1 \\
          x - 1, &\quad x \geq 1
        \end{aligned}
      \right.
    \end{equation*}
  \end{texample}
\end{demo}
(L'environnement \Ie{aligned} utilisé ci-dessus provient de
\pkg{amsmath}.) On notera au passage que l'environnement \Ie{cases} de
\pkg{amsmath} rend plus simple la réalisation de constructions comme
celle ci-dessus.
\begin{demo}
  \begin{texample}
\begin{lstlisting}
f(x) =
\begin{cases}
  1 - x, & x < 1 \\
  x - 1, & x \geq 1
\end{cases}
\end{lstlisting}
    \producing
    \begin{equation*}
      f(x) =
      \begin{cases}
        1 - x, & x < 1 \\
        x - 1, & x \geq 1
      \end{cases}
    \end{equation*}
  \end{texample}
\end{demo}

La règle de la paire est tout spécialement délicate dans les équations
sur plusieurs lignes car elle s'applique à chaque ligne d'une
équation. Par conséquent, si la paire de délimiteurs s'ouvre sur une
ligne et se referme sur une autre, il faudra ajouter un délimiteur
invisible à la fin de la première ligne ainsi qu'au début de la
seconde.
\begin{demo}
  \begin{texample}
\begin{lstlisting}
\begin{align*}
  a
  &= \left(
    b + \frac{c}{d}
    \right. \\
  &\phantom{=} \left.
    + \frac{e}{f} - g
    \right)
\end{align*}
\end{lstlisting}
    \producing
    \begin{align*}
      a
      &= \left(
        b + \frac{c}{d}
        \right. \\
      &\phantom{=} \left.
        + \frac{e}{f} - g
        \right)
    \end{align*}
  \end{texample}
\end{demo}

Quand les choix de taille de délimiteurs de {\LaTeX} ne conviennent
pas pour une raison ou pour une autre, on peut sélectionner soi-même
leur taille avec les commandes %
\cmd{\big}, %
\cmd{\Big}, %
\cmd{\bigg} et %
\cmd{\Bigg}. %
Ces commandes s'utilisent comme \cmdprint{\left} et \cmdprint{\right}
en les faisant immédiatement suivre d'un délimiteur. Le
\autoref{tab:math:big_et_al} contient des exemples de délimiteurs pour
chaque taille.

\begin{table}
  \centering
  \caption{Tailles des délimiteurs mathématiques}
  \label{tab:math:big_et_al}
  \begin{tabular}{ll}
    taille standard & $(~) \quad [~] \quad \{~\}$ \\
    \addlinespace[6pt]
    \cmd{\big} & $\big(~\big) \quad \big[~\big] \quad \big\{~\big\}$ \\
    \addlinespace[6pt]
    \cmd{\Big} & $\Big(~\Big) \quad \Big[~\Big] \quad \Big\{~\Big\}$ \\
    \addlinespace[6pt]
    \cmd{\bigg} & $\bigg(~\bigg) \quad \bigg[~\bigg] \quad \bigg\{~\bigg\}$ \\
    \addlinespace[6pt]
    \cmd{\Bigg} & $\Bigg(~\Bigg) \quad \Bigg[~\Bigg] \quad \Bigg\{~\Bigg\}$ \\
  \end{tabular}
\end{table}

La section~14 de la %
\doc{amsldoc}{http://texdoc.net/pkg/amsmath} %
de \pkg{amsmath} traite de divers enjeux typographiques en lien
avec les délimiteurs et on y introduit des nouvelles commandes pour
contrôler leur taille. C'est une lecture suggérée.

\begin{exemple}
  Le développement de la formule d'approximation de Simpson comporte
  plusieurs des éléments discutés jusqu'à maintenant:
  \begin{align*}
    \int_a^b f(x)\, dx
    &\approx \sum_{j = 0}^{n - 1}
      \int_{x_{2j}}^{x_{2(j + 1)}} f(x)\, dx \\
    &= \frac{h}{3} \sum_{j = 0}^{n - 1}
      \left[
      f(x_{2j}) + 4 f(x_{2j + 1}) + f(x_{2(j + 1)})
      \right]
    \displaybreak[0] \\
    &= \frac{h}{3}
      \left[
      f(x_0) +
      \sum_{j = 1}^{n - 1} f(x_{2j}) +
      4 \sum_{j = 0}^{n - 1} f(x_{2j + 1})
      \right. \displaybreak[0] \\
    &\phantom{=}  + \left.
      \sum_{j = 0}^{n - 2} f(x_{2(j + 1)}) +
      f(x_{2n})
      \right] \\
    &= \frac{h}{3}
      \left[
      f(a) +
      2 \sum_{j = 1}^{n - 1} f(x_{2j}) +
      4 \sum_{j = 0}^{n - 1} f(x_{2j + 1}) +
      f(b)
      \right].
  \end{align*}
  On compose ce bloc d'équation avec le code source ci-dessous.
\begin{lstlisting}
\begin{align*}
  \int_a^b f(x)\, dx
  &\approx \sum_{j = 0}^{n - 1}
    \int_{x_{2j}}^{x_{2(j + 1)}} f(x)\, dx \\
  &= \frac{h}{3} \sum_{j = 0}^{n - 1}
    \left[
    f(x_{2j}) + 4 f(x_{2j + 1}) + f(x_{2(j + 1)})
    \right] \\
  &= \frac{h}{3}
    \left[
    f(x_0) +
    \sum_{j = 1}^{n - 1} f(x_{2j}) +
    4 \sum_{j = 0}^{n - 1} f(x_{2j + 1})
    \right. \\
  &\phantom{=}  + \left.
    \sum_{j = 0}^{n - 2} f(x_{2(j + 1)}) +
    f(x_{2n})
    \right] \\
  &= \frac{h}{3}
    \left[
    f(a) +
    2 \sum_{j = 1}^{n - 1} f(x_{2j}) +
    4 \sum_{j = 0}^{n - 1} f(x_{2j + 1}) +
    f(b)
    \right].
\end{align*}
\end{lstlisting}
  \qed
\end{exemple}

% \begin{conseil}
%   De manière générale, il est déconseillé de scinder une suite
%   d'équations entre deux pages. Le chargement du paquetage
%   \pkg{amsmath} désactive d'ailleurs cette possibilité. Cependant,
%   c'est parfois inévitable pour les longs blocs d'équations.

%   La commande \cmd{\displaybreak}, placée immédiatement avant
%   \verb=\\= permet d'indiquer à {\LaTeX} la possibilité d'insérer un
%   saut de page après la ligne courante dans le bloc d'équations.

%   La commande accepte en argument optionnel un entier entre $0$ et $4$
%   indiquant à quel point un saut de page est désiré:
%   \cmdprint{\displaybreak[0]} signifie «il est permis de changer de
%   page ici» sans que ce ne soit obligatoire;
%   \cmdprint{\displaybreak[4]} ou, de manière équivalente,
%   \cmdprint{\displaybreak} force un saut de page.

%   À moins d'en être vraiment aux toutes dernières étapes d'édition
%   d'un document, il est conseillé d'utiliser la commande
%   \cmdprint{\displaybreak} avec parcimonie et avec un argument
%   optionnel faible.
% \end{conseil}


\section{Caractères gras en mathématiques}
\label{sec:math:gras}

Les caractères gras sont parfois utilisés en mathématiques,
particulièrement pour représenter les vecteurs et les matrices:
\begin{equation*}
  \symbf{A} \symbf{x} = \symbf{b} \Leftrightarrow
  \symbf{x} = \symbf{A}^{-1} \symbf{b}
\end{equation*}

Pourquoi consacrer une section spécialement à cette convention
typographique? Parce que la création de symboles mathématiques en
gras doit certainement figurer parmi les questions les plus
fréquemment posées par les utilisateurs de {\LaTeX}\dots\ et que la
réponse n'est pas unique!

La commande
\begin{lstlisting}
\mathbf`\marg{symbole}'
\end{lstlisting}
place \meta{symbole} en caractère gras en mode mathématique. C'est
donc l'analogue de la commande \cmd{\textbf} du mode texte. Dans
{\LaTeX} de base, la commande n'a toutefois un effet que sur les
lettres latines et, parfois, les lettres grecques majuscules.
\begin{demo}
  \begin{texample}[0.6\linewidth]
\begin{lstlisting}
\theta \mathbf{\theta} +
\Gamma \mathbf{\Gamma} \mathbf{+}
A \mathbf{A}
\end{lstlisting}
    \producing
    %% ok, ici il faut vraiment tricher pour reproduire avec
    %% unicode-math ce qui se produirait avec pdflatex standard
    $\theta {\theta} + \Gamma \symbf{\Gamma} {+} A \symbfup{A}$
  \end{texample}
\end{demo}
On remarquera aussi que \verb=\mathbf{A}= produit une lettre
majuscule droite plutôt qu'en italique.

La manière la plus standard et robuste d'obtenir des symboles
mathématiques (autres que les lettres) en gras semble être, au moment
d'écrire ces lignes, via la commande
\begin{lstlisting}
\bm`\marg{symbole}'
\end{lstlisting}
fournie par le paquetage \pkg{bm} \citep{bm}.
\begin{demo}
  \begin{texample}[0.6\linewidth]
\begin{lstlisting}
\theta \bm{\theta} +
\Gamma \bm{\Gamma} \bm{+}
A \bm{A}
\end{lstlisting}
    \producing
    %% ok, ici il faut vraiment tricher pour reproduire avec
    %% unicode-math ce qui se produirait avec pdflatex standard
    $\theta \symbf{\theta} + \Gamma \symbfup{\Gamma} \symbf{+} A
    \symbf{A}$
  \end{texample}
\end{demo}

Les utilisateurs de {\XeLaTeX} devraient charger le paquetage
\pkg{unicode-math} \citep{unicode-math} pour sélectionner leur police
de caractère pour les mathématiques. Ce paquetage fournit la commande
\begin{lstlisting}
\symbf`\marg{symbole}'
\end{lstlisting}
pour placer un \meta{symbole} mathématique en gras. Le paquetage offre
différentes combinaisons de lettres latines et grecques droites ou
italiques en gras selon la valeur de l'option \code{bold-style};
consulter la section~5 de la %
\doc{unicode-math}{http://texdoc.net/pkg/unicode-math}. %
\begin{demo}
  \begin{texample}[0.6\linewidth]
\begin{lstlisting}
% XeLaTeX + paquetage unicode-math
% avec l'option bold-style=ISO
\theta \symbf{\theta} +
\Gamma \symbf{\Gamma} \symbf{+}
A \symbf{A}
\end{lstlisting}
    \producing
    $\theta \symbf{\theta} + \Gamma \symbf{\Gamma} \symbf{+} A
    \symbf{A}$
  \end{texample}
\end{demo}

\begin{conseil}
  Si le gras est fréquemment utilisé dans un document pour une
  notation particulière, il est fortement recommandé de définir une
  nouvelle commande\footnotemark\
  sémantique plutôt que d'utiliser à répétition l'une ou l'autre des
  commandes ci-dessus.

  Par exemple, si le gras est utilisé pour les vecteurs et matrices,
  on pourrait définir une nouvelle commande \cmdprint{\mat} en
  insérant dans le préambule du document
\begin{lstlisting}
\newcommand[1]{\mat}{\bm{#1}}
\end{lstlisting}
\end{conseil}
%
\footnotetext{%
  La définition de nouvelles commandes est couvert plus en détail au
  \autoref{chap:commandes}.}


\begin{exemple}
  \label{ex:math:matrices}
  Le paquetage \pkg{amsmath} fournit quelques environnements qui
  facilitent la mise en forme de matrices; ils diffèrent simplement
  par le type de délimiteur autour de la matrice.

  Supposons que la commande \cmdprint{\mat} ci-dessus est définie dans
  le préambule du document. On obtient l'équation
  \begin{align*}
    \mat{J}(\mat{\theta})
    &=
      \begin{bmatrix}
        \dfrac{\partial f_1(\mat{\theta})}{\partial \theta_1} &
        \dfrac{\partial f_1(\mat{\theta})}{\partial \theta_2} \\[12pt]
        \dfrac{\partial f_2(\mat{\theta})}{\partial \theta_1} &
        \dfrac{\partial f_2(\mat{\theta})}{\partial \theta_2}
      \end{bmatrix} \\
    &=
      \left[
      \frac{\partial f_i(\mat{\theta})}{\partial \theta_j}
      \right]_{2 \times 2}, \quad i, j = 1, 2.
  \end{align*}
  avec le code source ci-dessous.
\begin{lstlisting}
\begin{align*}
  \mat{J}(\mat{\theta})
  &=
    \begin{bmatrix}
      \dfrac{\partial f_1(\mat{\theta})}{\partial \theta_1} &
      \dfrac{\partial f_1(\mat{\theta})}{\partial \theta_2}
      \\[12pt] % augmenter l'espace entre les lignes
      \dfrac{\partial f_2(\mat{\theta})}{\partial \theta_1} &
      \dfrac{\partial f_2(\mat{\theta})}{\partial \theta_2}
    \end{bmatrix} \\
  &=
    \left[
    \frac{\partial f_i(\mat{\theta})}{\partial \theta_j}
    \right]_{2 \times 2}, \quad i, j = 1, 2.
\end{align*}
\end{lstlisting}
  On remarquera l'utilisation de la commande \cmd{\dfrac}
  (\autoref{sec:math:bases:fractions}) pour composer des grandes
  fractions à l'intérieur des matrices. %
  \qed
\end{exemple}



%%%
%%% Exercices
%%%

\section{Exercices}
\label{sec:math:exercices}

\Opensolutionfile{solutions}[solutions-mathematiques]

\begin{Filesave}{solutions}
\section*{Chapitre \ref*{chap:math}}
\addcontentsline{toc}{section}{Chapitre \protect\ref*{chap:math}}

\end{Filesave}

\begin{exercice}
  Utiliser le gabarit de document \fichier{exercice\_gabarit.tex} pour
  reproduire le texte suivant:
  \begin{quote}
    La dérivée de la fonction composée $f \circ g(x) = f[g(x)]$ est
    $\{f[g(x)]\}^\prime = f^\prime[g(x)] g^\prime(x)$. La dérivée
    seconde du produit des fonctions $f$ et $g$ est $[f(x)
    g(x)]^{\prime\prime} = f^{\prime\prime}(x) g(x) + 2 f^\prime(x)
    g^\prime(x) + f(x) g^{\prime\prime}(x)$.
  \end{quote}
  \begin{sol}
    On trouve la commande pour produire le symbole $\circ$ dans le
    \autoref{tab:math:binaires}. On peut produire les symboles de
    dérivée $\prime$ avec la commande \cmd{\prime}
    (\autoref{tab:math:autres}) ou simplement avec le caractère
    \verb='=.
\begin{lstlisting}
La dérivée de la fonction composée $f \circ g(x) = f[g(x)]$
est $\{f[g(x)]\}^\prime = f^\prime[g(x)] g^\prime(x)$. La
dérivée seconde du produit des fonctions $f$ et $g$ est
$[f(x) g(x)]^{\prime\prime} = f^{\prime\prime}(x) g(x) +
2 f^\prime(x) g^\prime(x) + f(x) g^{\prime\prime}(x)$.
\end{lstlisting}
  \end{sol}
\end{exercice}

\begin{exercice}
  Composer l'équation suivante avec l'environnement \Pe{align*}:
  \begin{align*}
    f(x +& h, y + k)
    = f(x, y) +
      \left\{
      \frac{\partial f(x, y)}{\partial x} h +
      \frac{\partial f(x, y)}{\partial y} k
      \right\} \\
    &+
      \frac{1}{2}
      \left\{
      \frac{\partial^2 f(x, y)}{\partial x^2} h^2 +
      \frac{\partial^2 f(x, y)}{\partial x \partial y} kh +
      \frac{\partial^2 f(x, y)}{\partial y^2} k^2
      \right\} \\
    &+
      \frac{1}{6} \{\cdots\} + \dots + \frac{1}{n!} \{\cdots\} + R_n.
  \end{align*}
  Aligner les deuxième et troisième lignes de l'équation sur divers
  caractères de la première ligne afin que l'équation ne dépasse pas
  les marges du document.
  \begin{sol}
    Nous avons aligné les lignes de l'équation juste à droite du
    premier symbole $+$ à la première ligne. Remarquer l'usage des
    commandes \cmd{\cdots} et \cmd{\dots} dans la dernière ligne:
    {\LaTeX} choisit correctement la position centrée des points entre les
    opérateurs d'addition, mais pas entre les accolades.
\begin{lstlisting}
\begin{align*}
  f(x +& h, y + k) = f(x, y) +
    \left\{
    \frac{\partial f(x, y)}{\partial x} h +
    \frac{\partial f(x, y)}{\partial y} k
    \right\} \\
  &+
    \frac{1}{2}
    \left\{
    \frac{\partial^2 f(x, y)}{\partial x^2} h^2 +
    \frac{\partial^2 f(x, y)}{\partial x \partial y} kh +
    \frac{\partial^2 f(x, y)}{\partial y^2} k^2
    \right\} \\
  &+
    \frac{1}{6} \{\cdots\} + \dots +
    \frac{1}{n!} \{\cdots\} + R_n.
\end{align*}
\end{lstlisting}
  \end{sol}
\end{exercice}

\begin{exercice}
  Composer à l'aide de l'environnement \Ie{cases}
  (\autoref{sec:math:delimiteurs}) la définition de la fonction
  $\tilde{f}(x)$:
  \begin{equation*}
    \tilde{f}(x) =
    \begin{cases}
      0, & x \leq c_0\\
      \dfrac{F_n(c_j) - F_n(c_{j-1})}{c_j - c_{j-1}} =
      \dfrac{n_j}{n (c_j - c_{j - 1})}, &  c_{j-1} < x \leq c_j\\
      0, & x > c_r.
    \end{cases}
  \end{equation*}
  Il est nécessaire d'imposer la taille des fractions dans la seconde
  branche de la définition à l'aide des fonctions de la
  \autoref{sec:math:bases:fractions}.
  \begin{sol}
    Il faut utiliser \cmd{\dfrac} pour obtenir des fractions dans une
    branche de \Pe{cases} de la même taille que dans une équation hors
    paragraphe:
\begin{lstlisting}
\begin{equation*}
  \tilde{f}(x) =
  \begin{cases}
    0, & x \leq c_0\\
    \dfrac{F_n(c_j) - F_n(c_{j-1})}{c_j - c_{j-1}} =
    \dfrac{n_j}{n (c_j - c_{j - 1})}, &
      c_{j-1} < x \leq c_j \\
    0, & x > c_r.
  \end{cases}
\end{equation*}
\end{lstlisting}
  \end{sol}
\end{exercice}

\begin{exercice}[nosol]
  Le fichier \fichier{exercice\_mathematiques.tex} contient un exemple
  complet de développement mathématique. Étudier le contenu du fichier
  puis compiler celui-ci tel quel avec {pdf\LaTeX} ou {\XeLaTeX}.
  Effectuer ensuite les modifications suivantes.
  \begin{enumerate}
  \item Charger le paquetage \pkg{amsfonts} dans le préambule, puis
    remplacer \verb=$R^+$= par \verb=$\mathbb{R}^+$= à la ligne
    débutant par «Le domaine».
  \item Dans l'équation du Jacobien de la transformation, remplacer
    successivement l'environnement \Ie{vmatrix} par %
    \Ie{pmatrix}, %
    \Ie{bmatrix}, %
    \Ie{Bmatrix} et %
    \Ie{Vmatrix}. Observer l'effet sur les délimiteurs de la matrice.
  \item Toujours dans la même matrice, composer successivement les
    deux fractions avec les commandes \cmd{\frac}, \cmd{\tfrac} et
    \cmd{\dfrac}. Observer le résultat.
  \item Réduire l'espacement de part et d'autre du symbole
    $\Leftrightarrow$ dans la seconde équation hors paragraphe.
  \item À l'aide de la fonction Rechercher et remplacer de l'éditeur
    de texte, remplacer toutes les occurrences du symbole $\theta$ par
    $\lambda$.
  \end{enumerate}
\end{exercice}

\Closesolutionfile{solutions}

%%% Local Variables:
%%% mode: latex
%%% TeX-engine: xetex
%%% TeX-master: "formation_latex-partie_2"
%%% coding: utf-8
%%% End:
