\section{Classe ulthese}

\begin{frame}
  \frametitle{Un document conforme en un tournemain}
  \begin{itemize}
  \item \class{ulthese} livrée dans {\TeX}~Live donc déjà
    installée sur votre ordinateur
  \item Mise en page conforme aux règles de présentation de la FESP
  \item Basée sur la classe \class{memoir}
  \item Quelques nouvelles commandes pour la création de la page de
    titre
  \item Partir d'un gabarit (classés avec la documentation dans
    {\TeX}~Live)
  \item Utiliser des fichiers séparés pour chaque chapitre de la thèse
    ou du mémoire
  \end{itemize}
\end{frame}

%%% >>>
\stepcounter{exerciceref}
\subsection{[~Exercice \theexerciceref~]}

\begin{frame}[plain,fragile=singleslide]
  \begin{exercice}
    Utiliser le fichier \fichier{exercice\_ulthese.tex} --- qui est
    basé sur le gabarit \texttt{gabarit-doctorat.tex} livré avec
    \class{ulthese}.
    \medskip

    Ce fichier insère \fichier{b-a-ba-math.tex} dans le document
    avec la commande \verb=\include=.
    \begin{enumerate}
    \item Étudier le code source des deux fichiers et identifier à
      quel endroit \texttt{b-a-ba-math.tex} est chargé dans le
      document.
    \item Activer les paquetages \pkg{amsmath} et \pkg{icomma},
      puis compiler \texttt{exercice\_ulthese.tex}.
    \item Modifier un environnement \texttt{align*} pour
      \texttt{align} dans \texttt{b-a-ba-math.tex} et observer le
      résultat dans la compilation de \texttt{exercice\_ulthese.tex}.
    \item Compiler de nouveau le fichier en utilisant une police
      différente.
    \end{enumerate}
  \end{exercice}
\end{frame}
%%% <<<

%%% Local Variables:
%%% mode: latex
%%% TeX-engine: xetex
%%% TeX-master: "formation-latex-ul-diapos"
%%% End:
