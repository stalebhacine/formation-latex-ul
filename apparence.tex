\chapter{Apparence et disposition du texte}
\label{chap:apparence}


Les bonnes pratiques dictent de séparer le contenu du texte de son
apparence lorsque l'on utilise un système de mise en page comme
{\LaTeX}. Néanmoins, vient un moment où l'on peut souhaiter modifier
l'apparence générale du document ou disposer le texte d'une manière
particulière. {\LaTeX} offre tous les outils nécessaires pour
contrôler les attributs et la taille des polices de caractères,
disposer du texte sous forme de liste numérotée ou à puce, centrer du
texte ou créer des notes de base de page et des citations.

\section{Police de caractère}
\label{sec:apparence:police}

Par défaut, tous les documents {\LaTeX} utilisent la même police de
caractère\footnote{%
  Donald~Knuth a créé la police en même temps que {\TeX}.}, %
{\fontfamily{cmr}\selectfont Computer Modern}. Chose qui ne manquera
pas de surprendre les utilisateurs débutants: le système n'a pas été
conçu pour changer facilement la police de caractère du document.

Cela dit, il est aujourd'hui devenu assez simple d'utiliser d'autres
polices de caractère pour son document, surtout avec les moteurs
{\TeX} modernes comme {\XeTeX}. La \autoref{sec:trucs:police} traite
du sujet plus en détail. On pourra également consulter les fichiers
d'exercices et les gabarits de la classe \class{ulthese} pour des
exemples.

Cette section se concentre plutôt sur le changement d'\emph{attribut}
de la police de caractère du document, qu'il s'agisse de la famille
(avec ou sans empattements, à largeur fixe ou variable), de la forme
(droit, italique, penché) ou de la graisse (normal, gras). On trouvera
au \autoref{tab:apparence:police} les commandes de changement
d'attribut de la police de caractères.

\begin{table}
  \centering
  \caption{Commandes de changement d'attribut de la police de
    caractères. Les commandes de la deuxième colonne s'appliquent à
    tout le texte qui suit. Celles de la troisième colonne
    s'appliquent uniquement au texte en argument.}
  \label{tab:apparence:police}
  \begin{tabularx}{0.9\linewidth}{p{0.35\linewidth}Xl}
    \toprule
    \textbf{famille} \\
    \textrm{romain} & \cmd{\rmfamily} & \cmdprint{\textrm{\meta{texte}}} \\
    \texttt{largeur fixe} & \cmd{\ttfamily} & \cmd{\texttt{\meta{texte}}} \\
    \textsf{sans empattements} & \cmd{\sffamily} & \cmdprint{\textsf{\meta{texte}}} \\
    \addlinespace[6pt]
    \textbf{forme} \\
    \textup{\rmfamily droit} & \cmd{\upshape} & \cmdprint{\textup{\meta{texte}}} \\
    \textit{\rmfamily italique} & \cmd{\itshape} & \cmdprint{\textit{\meta{texte}}} \\
    \textsl{penché}$^\dagger$ & \cmd{\slshape} & \cmdprint{\textsl{\meta{texte}}} \\
    \textsc{\rmfamily petites capitales} & \cmd{\scshape} & \cmdprint{\textsc{\meta{texte}}} \\

    \addlinespace[6pt]
    \textbf{série} \\
    \textmd{\rmfamily moyen} & \cmd{\mdseries} & \cmdprint{\textmd{\meta{texte}}} \\
    \textbf{\rmfamily gras} & \cmd{\bfseries} & \cmdprint{\textbf{\meta{texte}}} \\
    \bottomrule
  \end{tabularx} \\
  \raggedright
  \hspace*{3em}{\footnotesize $^\dagger$ diffère de l'italique selon
    la police de caractère utilisée}
\end{table}

\begin{exemple}
  \label{ex:apparence:police}
  La commande \cmd{\setsecheadstyle} de la classe \class{memoir}
  permet de modifier facilement le style des titres de section pour
  tout le document. Pour obtenir des titres de section en gras sans
  empattements, on placera dans le préambule du document la commande
  suivante:
\begin{lstlisting}
\setsecheadstyle{\normalfont\sffamily\bfseries}
\end{lstlisting}
  L'utilisation de \cmd{\normalfont} au début de la série de commandes
  permet de réinitialiser le style des titres, question d'éviter
  d'éventuels conflits avec une configuration antérieure. %
  \qed
\end{exemple}

\section{Taille du texte}
\label{sec:apparence:taille}

Les commandes du \autoref{tab:apparence:taille} permettent de réduire
ou d'agrandir la taille des caractères. On se souviendra que l'on
règle la taille de base du texte au chargement de la classe du
document, tel qu'expliqué à la \autoref{sec:bases:classes}. Ces
commandes ne devraient donc servir que pour modifier temporairement la
taille des caractères pour une section du texte, ou encore pour la
configuration de l'apparence générale du document dans le préambule.

\begin{table}
  \centering
  \caption{Commandes de changement de la taille de la police de
    caractère. Les tailles sont déterminées en fonction de
    celle de la police de base du document.}
  \label{tab:apparence:taille}
  \begin{tabularx}{0.9\linewidth}{Xl}
    \toprule
    \cmd{\miniscule}$^\dagger$ & {\miniscule minuscule} \\
    \cmd{\tiny} & {\tiny vraiment petit} \\
    \cmd{\scriptsize} & {\scriptsize encore plus petit} \\
    \cmd{\footnotesize} & {\footnotesize plus petit} \\
    \cmd{\small} & {\small petit} \\
    \cmd{\normalsize} & {\normalsize taille normale} \\
    \cmd{\large} & {\large grand} \\
    \cmd{\Large} & {\Large plus grand} \\
    \cmd{\LARGE} & {\LARGE encore plus grand} \\
    \cmd{\huge} & {\huge énorme} \\
    \cmd{\Huge} & {\Huge encore plus énorme} \\
    \cmd{\HUGE}$^\dagger$ & {\HUGE vraiment énorme} \\
    \bottomrule
  \end{tabularx} \\
  \raggedright
  \hspace*{3em}{\footnotesize $^\dagger$ ajout de la classe
    \class{memoir} (et donc aussi de \class{ulthese})}
\end{table}

\begin{exemple}
  \label{ex:taille-du-texte}
  Les titres de sections sont généralement composés dans une taille
  supérieure à celle du texte. La commande suivante permet d'augmenter
  de deux échelons la taille des titres de
  l'\autoref{ex:apparence:police}:
\begin{lstlisting}
\setsecheadstyle{\normalfont\Large\sffamily\bfseries}
\end{lstlisting}
  \qed
\end{exemple}

\section{Italique}

L'italique est l'un des attributs de police de caractères les plus
fréquemment employés dans le texte. Il sert, notamment, à insister sur
des mots, à composer les expressions et locutions en langue étrangère
ou à détacher des titres d'{\oe}uvres du fil du texte.

Évidemment, il est possible d'obtenir de l'italique avec la commande
\cmd{\textit} du \autoref{tab:apparence:police}. Cependant, nous
recommandons de plutôt utiliser une commande spécifiquement dédiée à
mettre en évidence une portion de texte:
\begin{lstlisting}
\emph`\marg{texte}'
\end{lstlisting}
Par défaut, la commande \cmd{\emph} (pour \emph{emphasis}, «emphase»)
placera \meta{texte} en italique dans du texte en romain (droit), ou
encore en romain dans du texte déjà en italique.
\begin{demo}
  \begin{texample}
\begin{lstlisting}
C'était un peu \emph{rough}
par moments.
\end{lstlisting}
    \producing
    C'était un peu \emph{rough} par moments.
  \end{texample}
  \begin{texample}
\begin{lstlisting}
Il m'a dit: «\emph{Enough
\emph{poutine} for the week!»
\end{lstlisting}
    \producing
    Il m'a dit: «\emph{Enough \emph{poutine} for the week!}»
  \end{texample}
\end{demo}

\begin{conseil}
  Le soulignement\index{soulignement} servait pour remplacer
  l'italique à l'ère des dactylos. C'est aujourd'hui une marque de
  typographie très rarement utilisée, voire à éviter. Ce n'est donc
  pas pour rien qu'il n'existe pas de commande de soulignement dans
  les classes {\LaTeX} standards.
\end{conseil}


\section{Listes}
\label{sec:apparence:listes}

{\LaTeX} offre deux environnements pour la composition de listes ou
d'énumérations:
\begin{demo}
  \begin{minipage}{0.48\linewidth}
\begin{lstlisting}
\begin{itemize}
\item `\meta{texte}'
\item `\meta{texte}'
...
\end{itemize}
\end{lstlisting}
  \end{minipage}
  \hfill
  \begin{minipage}{0.48\linewidth}
\begin{lstlisting}
\begin{enumerate}
\item `\meta{texte}'
\item `\meta{texte}'
...
\end{enumerate}
\end{lstlisting}
  \end{minipage}
\end{demo}
L'environnement \Ie{itemize} crée une liste à puce, alors que
l'environnement \Ie{enumerate} crée une énumération. Il est possible
d'imbriquer les listes les unes dans les autres, peu importe leur
type. {\LaTeX} se chargera d'adapter les marqueurs ou la numérotation
de chaque niveau.

\begin{exemple}
  Les étapes de création d'une liste sont les suivantes.
  \begin{enumerate}
  \item Décider s'il s'agit d'une liste à puce ou d'une énumération;
    \begin{itemize}
    \item pour une liste à puce utiliser environnement \Ie{itemize};
      \begin{itemize}
      \item chaque niveau d'une liste à puce possède un marqueur
        différent;
      \end{itemize}
    \item pour une énumération utiliser environnement \Ie{enumerate};
    \end{itemize}
  \item Débuter chaque élément de la liste par la commande
    \cmdprint{\item}.
    \begin{enumerate}
    \item utiliser simplement un autre environnement \Pe{itemize} ou
      \Pe{enumerate} comme texte d'un élément pour créer des listes
      imbriquées;
    \item {\LaTeX} ajustera automatiquement les marqueurs jusqu'à
      quatre niveaux de profondeur;
    \end{enumerate}
  \item S'assurer de fermer tous les environnements dans le bon ordre
    pour retourner au texte normal.
  \end{enumerate}

  L'énumération ci-dessus constitue évidemment un exemple de listes
  mixtes imbriquées. On l'a composée avec le texte suivant.
  \begin{demo}
\begin{lstlisting}
\begin{enumerate}
\item Décider s'il s'agit d'une liste à puce ou ...
  \begin{itemize}
  \item pour une liste à puce utiliser ...
    \begin{itemize}
    \item chaque niveau d'une liste à puce ...
    \end{itemize}
  \item pour une énumération utiliser ...
  \end{itemize}
\item Débuter chaque élément de la liste par la commande
  \verb=\item=.
  \begin{enumerate}
  \item utiliser simplement un autre environnement ...
  \item {\LaTeX} ajustera automatiquement les marqueurs ...
  \end{enumerate}
\item S'assurer de fermer tous les environnements ...
\end{enumerate}
\end{lstlisting}
  \end{demo}
  \qed
\end{exemple}

{\LaTeX} permet de configurer à peu près toutes les facettes de la
présentation des listes: puces, folios, alignement, espacement entre
les éléments, etc. Une telle flexibilité implique aussi une certaine
complexité. Heureusement, plusieurs paquetages visent à faciliter la
configuration des listes. Nous recommandons à ce titre le paquetage
\pkg{enumitem} \citep{enumitem}.

\begin{conseil}
  Le mode français de \pkg{babel} remplace les puces par défaut en
  anglais ({\textbullet}, {\textendash}, $\ast$ et $.$ pour chacun
  des quatre niveaux de l'environnement \Ie{itemize}) par le tiret
  quadratin {\textemdash}. On peut désactiver cette fonctionnalité de
  \pkg{babel} en insérant dans le préambule:
\begin{lstlisting}
\frenchbsetup{StardardItemLabels=true}
\end{lstlisting}
  Il est également possible de modifier le symbole utilisé comme
  puce pour l'un, l'autre ou chacun des quatre niveaux avec:
\begin{lstlisting}
\frenchbsetup{
  ItemLabeli=\`\meta{commande}',
  ItemLabelii=\`\meta{commande}',
  ItemLabeliii=\`\meta{commande}',
  ItemLabeliv=\`\meta{commande}'}
\end{lstlisting}
  La \emph{Comprehensive {\LaTeX} Symbol List} \citep{comprehensive}
  propose une immense sélection de symboles pouvant faire office de
  puces.
\end{conseil}



\section{Texte centré}

\begin{center}
  Pour obtenir du texte centré on utilise l'environnement
  \Ie{center}
\end{center}

\begin{lstlisting}
\begin{center}
  Pour obtenir du texte centré on utilise
  l'environnement \verb=center=
\end{center}
\end{lstlisting}

{\centering ou encore la commande \cmd{\centering}}

\begin{lstlisting}
\centering ou encore la commande \verb=\centering=
\end{lstlisting}


\section{Citations}

Deux environnements de citation dans {\LaTeX} (et \class{ulthese})
\begin{enumerate}
\item \Ie{quote} pour les citations courtes, quelques lignes seulement
  \begin{itemize}
  \item retrait à gauche et à droite
  \end{itemize}
\item \Ie{quotation} pour les citations plus longues se comptant
  en paragraphes
  \begin{itemize}
  \item retrait à gauche et à droite
  \item gestion des marques de paragraphes
  \end{itemize}
\end{enumerate}


\section{Notes de bas de page}

\begin{itemize}
\item Note de bas de page insérée avec la commande
\begin{lstlisting}
\footnote`\marg{texte de la note}'
\end{lstlisting}
\item Commande doit suivre immédiatement le texte à annoter
\item Méthode recommandée
\begin{lstlisting}[emph=footnote]
... fera remarquer que Pierre Lasou\footnote{%
  Spécialiste en ressources documentaires} %
fut d'une grande aide dans la préparation de ...
\end{lstlisting}
\item Numérotation et disposition automatiques
\end{itemize}


\section{Code source}

\begin{itemize}
\item Environnement \Ie{verbatim}
\begin{lstlisting}
\begin{verbatim}
Texte disposé exactement tel qu'il est tapé
dans une police à largeur fixe
\end{verbatim}
\end{lstlisting}
\item Commande \cmd{\verb} dont la syntaxe est
\begin{lstlisting}
\verb`\meta{c}' `\meta{source}' `\meta{c}'
\end{lstlisting}
  où \meta{c} est un caractère quelconque ne se trouvant pas dans
  \meta{source}
\item Pour usage plus intensif, voir le paquetage \pkg{listings}.
\item Plus de détails à la \autoref{sec:trucs:listings}.
\end{itemize}




%%%
%%% Exercices
%%%

\section{Exercices}
\label{sec:apparence:exercices}

\begin{exercice}[nosol]
  Utiliser le fichier \fichier{exercice\_parties.tex}.
  \begin{enumerate}
  \item Étudier la structure du document dans le code source.
  \item Ajouter un titre et un auteur au document.
  \item Créer la table des matières du document en le compilant 2 à 3
    fois.
  \item Insérer deux ou trois titres de sections de différents niveaux
    dans le document.
  \item Vous remarquerez que la numérotation cesse à partir des
    sous-sections. C'est une particularité de la classe
    \class{memoir}.

    Recompiler le document après avoir ajouté au préambule la commande
\begin{lstlisting}
\maxsecnumdepth{subsection}
\end{lstlisting}
  \item Ajouter une annexe au document.
  \end{enumerate}
\end{exercice}

\begin{exercice}[nosol]
  Utiliser le fichier \fichier{exercice\_renvois.tex}.
  \begin{enumerate}
  \item Insérer dans le texte un renvoi au numéro d'une section.
  \item Activer le paquetage \pkg{hyperref} avec l'option
    \code{colorlinks} et comparer l'effet d'utiliser \cmd{\ref} ou
    \cmd{\autoref} pour le renvoi.
  \end{enumerate}
\end{exercice}

\begin{exercice}
  \begin{enumerate}
  \item Ouvrir le fichier \fichier{exercice\_complet.tex} et en
    étudier le code source, puis le compiler.
  \item Supprimer l'option \code{article} au chargement de la classe
    et compilier de nouveau le document. Observer l'effet de cette
    option.
  \item Effectuer les modifications suivantes au document.
    \begin{enumerate}[a)]
    \item Dernier paragraphe de la première section, placer toute la
      phrase débutant par \code{«De simple dérivé»} à l'intérieur
      d'une commande \cmd{\emph}.
    \item Changer la puce des listes pour le caractère \code{\$>\$}.
    \end{enumerate}
  \end{enumerate}
\end{exercice}

%%% Local Variables:
%%% mode: latex
%%% TeX-engine: xetex
%%% TeX-master: "formation-latex-ul"
%%% coding: utf-8
%%% End:
