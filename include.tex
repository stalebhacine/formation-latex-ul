\chapter{Document contenu dans plusieurs fichiers}
\label{chap:include}

Un document {\LaTeX} comporte toujours un préambule suivi du corps du
texte. Lorsque ceux-ci sont relativement courts (peu de commandes
spéciales et moins d'une vingtaine de pages de texte), il demeure
assez simple et convivial d'en faire l'édition dans un seul fichier à
l'aide de son éditeur de texte favori.

Cependant, si le préambule devient long et complexe ou, surtout,
lorsque l'ampleur du document augmente jusqu'à compter un grand nombre
de pages sur plusieurs chapitres, il convient de répartir les divers
éléments du document dans des fichiers séparés.

La segmentation en plusieurs fichiers rend l'édition du texte plus
simple et plus efficace. De plus, durant la phase de rédaction, elle
peut significativement accélérer, la compilation des documents très
longs ou comptant plusieurs images.


\section{Insertion du contenu d'un autre fichier}
\label{sec:include:input}

La commande \cmd{\input} permet d'insérer le contenu d'un autre
fichier dans un document {\LaTeX}. La syntaxe de la commande est
\begin{lstlisting}
\input`\marg{fichier}'
\end{lstlisting}
où le nom du fichier à insérer est \meta{fichier}\code{.tex}. On
laisse donc tomber l'extension \code{.tex}, qui est implicite. Le
contenu du fichier est inséré tel quel dans le document, comme s'il
avait été tapé dans le fichier qui contient l'appel à \cmd{\input}.

Le procédé est surtout utile pour sauvegarder séparément des bouts de
code qui pourraient nuire à l'édition du texte (figures, longs
tableaux) ou qui sont communs entre plusieurs documents (licence
d'utilisation, auteur et affiliation).

La commande peut aussi être utilisée dans le préambule pour charger
une partie ou l'ensemble de celui-ci. Cela permet de composer un même
préambule pour plusieurs documents. Il suffit alors de faire
d'éventuelles modifications à un seul endroit pour les voir prendre
effet dans tous les documents.


\section{Insertion de parties d'un document}
\label{sec:include:include}

Extrait de la %
\doc{ulthese}{http://texdoc.net/pkg/ulthese/} %
de la classe \class{ulthese}:
\begin{quote}
  «Il est recommandé de segmenter tout document d'une certaine ampleur
  dans des fichiers \verb=.tex= distincts pour chaque partie ---
  habituellement un fichier par chapitre. Le document complet est
  composé à l'aide d'un fichier maître qui contient le préambule
  {\LaTeX} et un ensemble de commandes \verb=\include= pour réunir les
  parties dans un tout.»
\end{quote}

Comme \cmd{\input}, la commande \cmd{\include} insère le contenu
d'un autre fichier dans un document {\LaTeX}. Son effet est cependant
différent et c'est son utilisation qui permet d'accélérer la
compilation d'un long document.

L'insertion d'un fichier avec \cmd{\include} débute toujours une
nouvelle page. On utilisera donc \cmd{\include} principalement pour
insérer des chapitres entiers plutôt que seulement des portions de
texte. De plus, un fichier inséré avec \cmd{\include} peut contenir
des appels à \cmd{\input}, mais pas à \cmd{\include}.

La syntaxe de la commande \cmd{\include} est
\begin{lstlisting}
\include`\marg{fichier}'
\end{lstlisting}
où le nom du fichier à insérer est \meta{fichier}\code{.tex}. Ici
aussi on laisse tomber l'extension \code{.tex} qui est implicite.

 La structure type d'un fichier maître est la suivante:
\begin{lstlisting}
\documentclass{ulthese}
  [...]

\begin{document}

\frontmatter

\chapter{Introduction}
\label{chap:introduction}

Ce document constitue la seconde partie d'une formation sur la
rédaction de thèses et de mémoires avec {\LaTeX} développée pour la
Bibliothèque de l'Université Laval. La première partie de
la formation se déroulant en classe, la documentation qui
l'accompagne consiste en une série de diapositives
\citep{UL:latex:1}.

Nous reprenons ici la formation une fois présentés les concepts de
base de {\LaTeX} pour un nouvel utilisateur: processus d'édition,
compilation, visualisation; séparation du contenu et de l'apparence du
texte; mise en forme du texte; séparation du document en parties;
rudiments du mode mathématique. Avec cette seconde partie, une
personne devrait être en mesure de composer des documents relativement
complexes comportant des tableaux, des figures, des équations
mathématiques élaborées, une bibliographie, etc.

Le présent ouvrage n'a aucune prétention d'exhaustivité. La
consultation de documentation additionnelle peut s'avérer nécessaire
pour réaliser des mises en page plus élaborées. À cet égard, nous
recommandons chaudement le livre de \citet{Kopka:latex:4e} --- il a
servi d'inspiration pour ce document à maints endroits. La très
complète documentation (plus de 600 pages!) de la classe
\class{memoir} \citep{memoir}, sur laquelle se base la classe
\class{ulthese} pour les thèses et mémoires de l'Université Laval,
constitue une autre référence de choix. Nous recommandons également:
\begin{itemize}
\item \link{http://fr.wikibooks.org/wiki/LaTeX}{\emph{LaTeX} dans
    Wikilivre} pour de la documentation en ligne, en français et
  libre;
\item le très actif forum de discussion
  \link{http://tex.stackexchange.com}{{\TeX}--{\LaTeX} Stack Exchange}
  (avant de penser y poser une question, vérifier que la réponse ne se trouve
  pas déjà dans le forum\dots\ ce qui risque fort d'être le cas);
\item la très complète
  \link{http://www.tex.ac.uk/cgi-bin/texfaq2html}{%
    \emph{foire aux questions}} (en anglais) du groupe des
  utilisateurs de {\LaTeX} du Royaume-Uni.
\end{itemize}

%%% Local Variables:
%%% mode: latex
%%% TeX-engine: xetex
%%% TeX-master: "formation_latex_UL-partie_2"
%%% encoding: utf-8
%%% End:

\tableofcontents*

\mainmatter
\include{historique}            % premier chapitre
\include{modele}                % deuxième chapitre
[...]

\end{document}
\end{lstlisting}

\begin{conseil}
  Utiliser des noms des fichiers qui permettent de facilement
  identifier leur contenu. Par exemple, un nom comme
  \fichier{rappels.tex} est plus parlant et résiste mieux aux
  changements à l'ordre des chapitres que \fichier{chapitre1.tex}.
\end{conseil}

Le principal avantage de \cmd{\include} par rapport à \cmd{\input}
réside dans le fait que {\LaTeX} peut préserver entre les compilations
les informations telles que les numéros de pages, de sections ou
d'équations, ainsi que les références. Cela permet, par exemple, de
compiler le texte d'un seul chapitre --- plutôt que le document entier
--- et néanmoins obtenir une image représentative du chapitre.
Procéder ainsi accélère significativement la compilation des documents
longs ou complexes.

La commande \cmd{\includeonly}, que l'on utilise exclusivement dans le
préambule, sert à spécifier le ou les fichiers à compiler tout en
préservant la numérotation et les références. Sa syntaxe est
\begin{lstlisting}
\includeonly`\marg{liste\_fichiers}'
\end{lstlisting}
où \meta{liste\_fichiers} contient les noms des fichiers à
inclure dans la compilation, séparés par des virgules et sans
l'extension \code{.tex}.

Lors de l'utilisation de la commande \cmd{\includeonly}, toute la
numérotation dans les fichiers \meta{liste\_fichiers} suivra celle
établie lors de la compilation précédente. Si l'édition des fichiers
de \meta{liste\_fichiers} cause des changements dans la numérotation
et les références dans les autres parties du document, une nouvelle
compilation de l'ensemble ou d'une partie de celui-ci s'avérera
nécessaire.

\begin{exemple}
  Un document est composé en plusieurs parties avec les commandes
  suivantes:
\begin{lstlisting}
\include{historique}            % chapitre 1
\include{rappels}               % chapitre 2
\include{modele}                % chapitre 3
\end{lstlisting}
  Les chapitres débutent respectivement aux pages~1, 23 et 41.
  \begin{itemize}
  \item Si on ajoute au préambule du document la commande
\begin{lstlisting}
\includeonly{rappels}
\end{lstlisting}
    le numéro du chapitre sera toujours 2 et le folio de
    la première page sera toujours 23, même si les 22 pages
    précédentes ne se trouvent pas dans le document.
  \item Si l'on modifie le fichier \fichier{rappels.tex} de telle
    sorte que le chapitre se termine maintenant à la page 46, il
    faudra recompiler le document avec au moins les fichiers
    \fichier{rappels.tex} et \code{modele.tex} pour que les pages du
    chapitre~3 soient renumérotées à partir de 47.
  \end{itemize}
  \qed
\end{exemple}

L'\autoref{ex:include} illustre mieux le cycle typique
d'utilisation des commandes \cmd{\include} et \cmd{\includeonly}.



%%%
%%% Exercices
%%%

\section{Exercices}
\label{sec:include:exercices}

\begin{exercice}[nosol]
  \label{ex:include}
  Cet exercice fait appel au fichier maître
  \fichier{exercice\_include.tex} et à plusieurs fichiers auxiliaires.
  Schématiquement, le document est composé ainsi:

  \medskip
  \begin{minipage}{\linewidth}
    \dirtree{%
      .1 exercice\_include.tex.
      .2 {\textbackslash}input pagetitre.tex.
      .2 {\textbackslash}include presentation.tex.
      .3 {\textbackslash}includegraphics console-screenshot.pdf.
      .2 {\textbackslash}include emacs.tex.
    }
  \end{minipage}
  \medskip

  La commande \cmd{\includegraphics} permet d'insérer une image dans
  un document {\LaTeX}. Elle provient du paquetage \pkg{graphicx}.

  \begin{enumerate}
  \item Étudier le code source du fichier maître
    \fichier{exercice\_include.tex}, puis le compiler deux à trois
    fois jusqu'à ce que toutes les références internes soient à jour.
    Il est normal à ce stade que la figure~1 du document soit vide.
  \item Ajouter dans le préambule du fichier maître la commande
\begin{lstlisting}
\includeonly{emacs}
\end{lstlisting}
    puis compiler le document.

    Observer que, malgré l'absence du chapitre~1, la numérotation et
    les références demeurent à jour, notamment la table des matières.
  \item Remplacer la commande ajoutée en b) dans le préambule du
    fichier maître par la commande
\begin{lstlisting}
\includeonly{presentation}
\end{lstlisting}
    Vers la fin du fichier \fichier{presentation.tex}, activer la
    commande
\begin{lstlisting}
\includegraphics[width=\textwidth]{console-screenshot}
\end{lstlisting}
    en supprimant le symbole \% au début de la ligne. Compiler de
    nouveau le document deux fois.

    Les modifications ont eu pour effet d'ajouter une page au
    chapitre~1. Observer que selon la table des matières, le
    chapitre~2 débute toujours à la page~3 alors que celle-ci est
    maintenant occupée par la figure~1.
  \item Afin de corriger la table des matières, désactiver dans le
    préambule du fichier maître la commande \cmd{\includeonly}, puis
    compiler de nouveau le document quelques fois.
  \end{enumerate}
\end{exercice}

\begin{exercice}[nosol]
  Déplacer dans un fichier \fichier{preambule.tex} toutes les lignes
  du préambule du fichier \fichier{exercice\_include.tex} utilisé à
  l'exercice précédent, à l'exception de celles relatives à la page
  titre (titre, auteur, date). Insérer le préambule dans
  \fichier{exercice\_include.tex} avec la commande \cmd{\input}.
\end{exercice}

%%% Local Variables:
%%% mode: latex
%%% TeX-engine: xetex
%%% TeX-master: "formation_latex-partie_2"
%%% coding: utf-8
%%% End:
