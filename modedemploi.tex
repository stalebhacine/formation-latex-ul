\chapter{Mode d'emploi}
\label{chap:modedemploi}

\section*{Hyperliens vers la documentation}

À plusieurs endroits dans le document nous renvoyons le lecteur vers
la documentation d'un paquetage ou d'une classe, par exemple vers la %
\doc{ulthese}{http://texdoc.net/pkg/ulthese/} %
de la classe \class{ulthese}. Le format du renvoi est toujours tel
qu'illustré ici: un hyperlien mène vers la version en ligne de la
documentation dans le site %
\link{http://texdoc.net}{TeXdoc Online}; on trouve dans la marge le
nom du fichier correspondant (sans l'extension \code{.pdf}) sur un
système doté de {\TeX}~Live.

Sur la plupart des systèmes, il est possible de consulter hors ligne
le fichier de documentation \meta{fichier}\code{.pdf} en entrant à une
invite de commande
\begin{quote}
\begin{lstlisting}[backgroundcolor=\color{white}]
texdoc `\meta{fichier}'
\end{lstlisting}
\end{quote}

Plusieurs logiciels intégrés de rédaction offrent une inferface pour
accéder à cette documentation. Quelques exemples.
\begin{itemize}
\item TeXShop: menu \code{Aide|Afficher l'aide pour le
    package} (\optkey\,\cmdkey\, I);
\item Texmaker: menu \code{Aide|TeXDoc [selection]};
\item GNU Emacs: commande \code{TeX-doc} (\code{C-c ?}).
\end{itemize}
Consulter l'aide de votre éditeur pour savoir s'il offre une
interface à \code{texdoc}.

% \section{Capsules d'aide additionnelle}
%
% Un symbole de lecture vidéo dans la marge indique qu'une capsule vidéo
% est disponible dans la %
% \capsule{http://www.youtube.com/user/ULFormationLaTeX}{chaîne
%   YouTube} %
% de la formation sur le sujet en hyperlien.

\section*{Fichiers d'accompagnement}
\enlargethispage{5mm}

Ce document devrait être accompagné des fichiers nécessaires pour
compléter certains exercices figurant à la fin des chapitres, ainsi
que d'un fichier \fichier{exercice\_gabarit.tex} pouvant servir de
gabarit pour composer les solutions des autres exercices.

Vous pouvez
récupérer ces fichiers dans le site \emph{Comprehensive TeX Archive Network}
(CTAN).
\begin{center}
  \href{\ctanurl}{\ctanbutton}
\end{center}


%%% Local Variables:
%%% mode: latex
%%% TeX-engine: xetex
%%% TeX-master: "formation_latex_UL-partie_2"
%%% encoding: utf-8
%%% End:
