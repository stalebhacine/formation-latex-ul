\chapter{Classe pour les thèses et mémoires de
  l'Université Laval}
\label{chap:ulthese}

Un document conforme en un tournemain

\begin{itemize}
\item \class{ulthese} livrée dans {\TeX}~Live donc déjà installée sur
  votre ordinateur
\item Mise en page conforme aux règles de présentation de la FESP
\item Basée sur la classe \class{memoir}, donc les fonctionnalités de
  celle-ci sont disponibles dans \class{ulthese}
\item Quelques nouvelles commandes pour la création de la page titre
\item Partir d'un gabarit (classés avec la documentation dans
  {\TeX}~Live)
\item Utiliser des fichiers séparés pour chaque chapitre du mémoire ou
  de la thèse
\end{itemize}


%%%
%%% Exercices
%%%

\section{Exercice}
\label{sec:ulthese:exercices}

\begin{exercice}
  Utiliser le fichier \fichier{exercice\_ulthese.tex} --- qui est basé
  sur le gabarit \fichier{gabarit-doctorat.tex} livré avec
  \class{ulthese}.
  \begin{enumerate}
  \item Étudier le code source du fichier.

    Remarquer que le fichier \fichier{mathematiques.tex} est inséré
    dans le document avec la commande \verb=\include=. Étudier
    brièvement le code source de ce fichier.
  \item Activer les paquetages \pkg{amsmath} et \pkg{icomma}, puis
    compiler \fichier{exercice\_ulthese.tex}.
  \item Modifier un environnement \texttt{align*} pour \texttt{align}
    dans \fichier{mathematiques.tex} et observer le résultat dans la
    compilation de \fichier{exercice\_ulthese.tex}.
  \item Compiler de nouveau le fichier en utilisant une police de
    caractères différente.
  \end{enumerate}
\end{exercice}

%%% Local Variables:
%%% mode: latex
%%% TeX-engine: xetex
%%% TeX-master: "formation_latex_UL-partie_2"
%%% coding: utf-8
%%% End:
