\section{Tableaux}

\begin{frame}
  \frametitle{De la conception de beaux tableaux}

  Lequel de ces deux tableaux est le plus facile à consulter?
  \begin{center}
  \hfill
  \begin{tabular}{|>{$}c<{$}|>{$}r<{$}|>{$}r<{$}|}
    \hline\hline
    i &
    \multicolumn{1}{c|}{$v$} &
    \multicolumn{1}{c|}{$b_i$} \\
    \hline
    0 & \nombre{91492} &  60 \\
    \hline
    1 &  \nombre{1524} &  60 \\
    \hline
    2 &            25  &  24 \\
    \hline
    3 &             1  & 365 \\
    \hline\hline
  \end{tabular}
  \hfill
  \begin{tabular}{>{$}c<{$}>{$}r<{$}>{$}r<{$}}
    \toprule
    i &
    \multicolumn{1}{c}{$v$} &
    \multicolumn{1}{c}{$b_i$} \\
    \midrule
    0 & \nombre{91492} &  60 \\
    1 &  \nombre{1524} &  60 \\
    2 &            25  &  24 \\
    3 &             1  & 365 \\
    \bottomrule
  \end{tabular}
  \hspace*{\fill}
  \end{center}

  \pause
  Deux règles d'or:
  \begin{enumerate}
  \item \alert{jamais} de filets verticaux
  \item pas de filets doubles
  \end{enumerate}
\end{frame}

\begin{frame}[fragile=singleslide]
  \frametitle{Paquetage essentiel}

  \begin{itemize}
  \item Vous voulez utiliser le paquetage \pkg{booktabs}
\begin{lstlisting}
\usepackage{booktabs}
\end{lstlisting}
  \item Fonctionnalités intégrées dans la classe \class{memoir}
  \end{itemize}
\end{frame}

\begin{frame}[fragile=singleslide]
  \frametitle{Exemple de tableau}

  \begin{center}
    \begin{tabular}{lcrr}
      \toprule
      Produit & Quantité & Prix unitaire (\$) & Prix (\$) \\
      \midrule
      Vis à bois    & 2 & 9,90 & 19,80 \\
      Clous vrillés & 5 & 4,35 & 21,75 \\
      \midrule
      TOTAL         & 7 &      & 41,55 \\
      \bottomrule
    \end{tabular}
  \end{center}

\begin{lstlisting}
\begin{tabular}{lcrr}
  \toprule
  Produit & Quantité & Prix unitaire (\$) & Prix (\$) \\
  \midrule
  Vis à bois    & 2 & 9,90 & 19,80 \\
  Clous vrillés & 5 & 4,35 & 21,75 \\
  \midrule
  TOTAL         & 7 &      & 41,55 \\
  \bottomrule
\end{tabular}
\end{lstlisting}
\end{frame}

%%% Local Variables:
%%% TeX-master: "formation-latex-ul-diapos"
%%% TeX-engine: xetex
%%% coding: utf-8
%%% End:
