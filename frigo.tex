\section{Contrôle du texte}



\begin{frame}[fragile=singleslide]
  \frametitle{Autres changements de police}
  \begin{itemize}
  \item Attributs par défaut
\begin{lstlisting}
\textnormal{`\textit{texte}'}
\end{lstlisting}
  \end{itemize}
\end{frame}

\begin{frame}[fragile=singleslide]
  \frametitle{Sauts de ligne}
  \begin{itemize}
  \item Rarement nécessaire de forcer les retours à la ligne
  \item Lorsque requis utiliser
    \begin{quote}
      \begin{minipage}{0.3\linewidth}
\begin{lstlisting}[aboveskip=1.5\medskipamount]
  \\
\end{lstlisting}
      \end{minipage}
      \quad ou \quad
      \begin{minipage}{0.3\linewidth}
\begin{lstlisting}[aboveskip=1.5\medskipamount]
  \newline
\end{lstlisting}
      \end{minipage}
    \end{quote}
  \item Commande \verb=\\= aussi pour délimiter
    \begin{itemize}
    \item les lignes dans les tableaux
    \item les lignes d'une suite d'équations
    \end{itemize}
  \item On peut suivre un saut de ligne d'un espace vertical
    arbitraire avec
\begin{lstlisting}
  \\[`\textit{longueur}']
\end{lstlisting}
  \item Espace insécable: \verb= ~ =
\begin{lstlisting}
M.~Tremblay
\end{lstlisting}

  \end{itemize}
\end{frame}

\begin{frame}[fragile]
  \frametitle{Sauts de page}
  \begin{itemize}
  \item Parfois nécessaires lors de coupures malheureuses
  \item Aussi pour placer des éléments où l'on veut
  \item Garder l'édition des sauts de page pour la toute fin de la
    rédaction
  \item<2-> Commandes
\begin{lstlisting}
\newpage
\clearpage
\cleartorecto              % memoir seulement
\cleartoverso              % memoir seulement
\end{lstlisting}
  \item<3-> Suggestions
\begin{lstlisting}
\pagebreak[`\textit{n}']              % n = 0, 1, 2, 3, 4
\enlargethispage{`\textit{longueur}'}
\end{lstlisting}
  \end{itemize}
\end{frame}


\begin{frame}[fragile=singleslide]
  \frametitle{Coupure de mots}
  \begin{itemize}
  \item Coupure de mots en fin de ligne automatique avec \LaTeX
  \item Important d'indiquer à {\LaTeX} dans quelle langue est le texte!
    \begin{itemize}
    \item en anglais par défaut
    \item autrement spécifié au chargement de \textbf{babel}
    \end{itemize}
  \item Suggestions pour un mot individuel
\begin{lstlisting}
vrai\-sem\-blance
\end{lstlisting}
  \item Ajout d'exceptions ou de mots inconnus dans le préambule
\begin{lstlisting}
\hyphenation{puis-que,cons-tante}
\end{lstlisting}
  \end{itemize}
\end{frame}
