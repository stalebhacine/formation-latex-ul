\chapter{Introduction}
\label{chap:introduction}

À l'origine du présent ouvrage, il y a une formation sur la rédaction
de thèses et de mémoires avec {\LaTeX} développée pour la Bibliothèque
de l'Université Laval. La formation aborde les concepts de base pour
un nouvel utilisateur: processus d'édition, compilation,
visualisation; séparation du contenu et de l'apparence du texte; mise
en forme du texte; séparation du document en parties; rudiments du
mode mathématique. Transformée en prose, la série de diapositives qui
appuie la présentation correspond grosso modo aux quatre premiers
chapitres de l'ouvrage.

Les six autres chapitres visent à rendre l'utilisateur de {\LaTeX}
débutant ou intermédiaire autonome dans la rédaction de documents
relativement complexes comportant: tableaux, figures, équations
mathématiques élaborées, une bibliographie, etc. Nous avons émaillé le
texte de conseils et d'astuces glanés au fil de nos vingt années
d'utilisation du système de mise en page.

Les nombreuses références à la classe \class{ulthese} s'adressent au
premier public de l'ouvrage: les étudiantes et étudiants de
l'Université Laval occupés à la rédaction de leur thèse ou de leur
mémoire. Les autres lecteurs pourront sans mal escamoter ces passages.

L'ouvrage n'a aucune prétention d'exhaustivité. La consultation de
documentation additionnelle pourra s'avérer nécessaire pour réaliser
des mises en page plus élaborées. À cet égard, nous recommandons
chaudement le livre de \citet{Kopka:latex:4e} --- il a servi
d'inspiration pour ce document à maints endroits. La très complète
documentation (plus de 600~pages!) de la classe \class{memoir}
\citep{memoir} constitue une autre référence de choix. Nous
recommandons également:
\begin{itemize}
\item \link{http://fr.wikibooks.org/wiki/LaTeX}{\emph{LaTeX} dans
    Wikilivre} pour de la documentation en ligne, en français et
  libre;
\item le très actif forum de discussion
  \link{http://tex.stackexchange.com}{{\TeX}--{\LaTeX} Stack Exchange}
  (avant de penser y poser une question, vérifier que la réponse ne se trouve
  pas déjà dans le forum\dots\ ce qui risque fort d'être le cas);
\item la très complète
  \link{http://www.tex.ac.uk/cgi-bin/texfaq2html}{%
    \emph{foire aux questions}} (en anglais) du groupe des
  utilisateurs de {\LaTeX} du Royaume-Uni.
\end{itemize}

Le texte comporte plusieurs renvois vers la documentation d'un
paquetage ou d'une classe, par exemple vers la %
\doc{memoir}{http://texdoc.net/pkg/memoir/} %
de la classe \class{memoir}. L'hyperlien mène vers la version en ligne
de la documentation dans le site %
\link{http://texdoc.net}{TeXdoc Online} et la marge contient le nom du
fichier correspondant (sans l'extension \code{.pdf}) sur un système
doté de {\TeX}~Live.

Sur la plupart des systèmes, il est possible de consulter hors ligne
le fichier de documentation \meta{fichier}\code{.pdf} en entrant à une
invite de commande
\begin{quote}
\begin{lstlisting}[backgroundcolor=\color{white}]
texdoc `\meta{fichier}'
\end{lstlisting}
\end{quote}
Plusieurs logiciels intégrés de rédaction offrent une inferface pour
accéder à cette documentation.
\begin{itemize}
\item TeXShop: menu \code{Aide|Afficher l'aide pour le
    package} (\optkey\,\cmdkey\, I).
\item Texmaker: menu \code{Aide|TeXDoc [selection]}.
\item GNU~Emacs: commande \code{TeX-doc} (\code{C-c ?}) du mode
  spécialisé AUC{\TeX}.
\end{itemize}
Consulter la rubrique d'aide de votre éditeur pour savoir s'il offre
une interface à \code{texdoc}.

% \section{Capsules d'aide additionnelle}
%
% Un symbole de lecture vidéo dans la marge indique qu'une capsule vidéo
% est disponible dans la %
% \capsule{http://www.youtube.com/user/ULFormationLaTeX}{chaîne
%   YouTube} %
% de la formation sur le sujet en hyperlien.

Enfin, cet ouvrage devrait être accompagné des fichiers nécessaires
pour compléter certains exercices figurant à la fin des chapitres,
ainsi que d'un gabarit \fichier{exercice\_gabarit.tex} pour composer
les solutions des autres exercices. Si ce n'est pas le cas, il est
possible de récupérer les fichiers dans le site
\href{\ctanurl}{\emph{Comprehensive TeX Archive Network}} (CTAN).

\begin{flushright}
  Vincent Goulet \\
  Québec, novembre \year
\end{flushright}

%%% Local Variables:
%%% mode: latex
%%% TeX-engine: xetex
%%% TeX-master: "formation-latex-ul"
%%% encoding: utf-8
%%% End:
