\chapter{Introduction}
\label{chap:introduction}

Ce document constitue la seconde partie d'une formation sur la
rédaction de thèses et de mémoires avec {\LaTeX} développée pour la
Bibliothèque de l'Université Laval. La première partie de
la formation se déroulant en classe, la documentation qui
l'accompagne consiste en une série de diapositives
\citep{UL:latex:1}.

Nous reprenons ici la formation une fois présentés les concepts de
base de {\LaTeX} pour un nouvel utilisateur: processus d'édition,
compilation, visualisation; séparation du contenu et de l'apparence du
texte; mise en forme du texte; séparation du document en parties;
rudiments du mode mathématique. Avec cette seconde partie, une
personne devrait être en mesure de composer des documents relativement
complexes comportant des tableaux, des figures, des équations
mathématiques élaborées, une bibliographie, etc.

Le présent ouvrage n'a aucune prétention d'exhaustivité. La
consultation de documentation additionnelle peut s'avérer nécessaire
pour réaliser des mises en page plus élaborées. À cet égard, nous
recommandons chaudement le livre de \citet{Kopka:latex:4e} --- il a
servi d'inspiration pour ce document à maints endroits. La très
complète documentation (plus de 600 pages!) de la classe
\class{memoir} \citep{memoir}, sur laquelle se base la classe
\class{ulthese} pour les thèses et mémoires de l'Université Laval,
constitue une autre référence de choix. Nous recommandons également:
\begin{itemize}
\item \link{http://fr.wikibooks.org/wiki/LaTeX}{\emph{LaTeX} dans
    Wikilivre} pour de la documentation en ligne, en français et
  libre;
\item le très actif forum de discussion
  \link{http://tex.stackexchange.com}{{\TeX}--{\LaTeX} Stack Exchange}
  (avant de penser y poser une question, vérifier que la réponse ne se trouve
  pas déjà dans le forum\dots\ ce qui risque fort d'être le cas);
\item la très complète
  \link{http://www.tex.ac.uk/cgi-bin/texfaq2html}{%
    \emph{foire aux questions}} (en anglais) du groupe des
  utilisateurs de {\LaTeX} du Royaume-Uni.
\end{itemize}

%%% Local Variables:
%%% mode: latex
%%% TeX-engine: xetex
%%% TeX-master: "formation_latex-partie_2"
%%% encoding: utf-8
%%% End:
