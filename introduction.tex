\chapter{Introduction}
\label{chap:introduction}

Ce document constitue la seconde partie d'une formation sur
l'utilisation de {\LaTeX} pour la rédaction de thèses et de mémoires
développée pour la Bibliothèque de l'Université Laval. La première
partie de la formation se déroulant en classe, sa documentation
n'existe que sous forme de diapositives \citep{UL:latex:1}.

Ce document reprend la formation une fois présentés les concepts de
base de {\LaTeX} pour un nouvel utilisateur: processus d'édition,
compilation, visualisation; séparation du contenu et de l'apparence du
texte; mise en forme du texte; séparation du document en parties;
rudiments du mode mathématique. Avec cette seconde partie, une
personne devrait être en mesure de composer des documents relativement
complexes comportant des tableaux, des figures, des équations
mathématiques élaborées, etc.

Le présent manuel n'a aucune prétention d'exhaustivité. La
consultation de documentation additionnelle peut s'avérer nécessaire
pour réaliser des mises en page plus élaborées. À cet égard, nous
recommandons chaudement le livre de \cite{Kopka:latex:4e} --- il a
servi d'inspiration pour ce document à maints endroits. La très
complète (plus de 600 pages!) documentation de la classe
\class{memoir} \citep{memoir}, sur laquelle se base la classe
\class{ulthese}, constitue une autre référence de choix. Nous
recommandons également:
\begin{itemize}
\item \link{http://fr.wikibooks.org/wiki/LaTeX}{\emph{LaTeX} dans
    Wikilivre} pour de la documentation en ligne, en français et
  libre;
\item le très actif forum de discussion
  \link{http://tex.stackexchange.com}{{\TeX}--{\LaTeX} Stack Exchange}
  (avant de poser une question, vérifiez que la réponse ne se trouve
  pas déjà dans ce forum... ce qui risque fort d'être le cas);
\item la très complète
  \link{http://www.tex.ac.uk/cgi-bin/texfaq2html}{%
    \emph{foire aux questions}} (en anglais) du groupe des
  utilisateurs de {\LaTeX} du Royaume-Uni.
\end{itemize}

À plusieurs endroits dans le document nous renvoyons le lecteur vers
la documentation d'un paquetage ou d'une classe, par exemple vers la %
\doc{ulthese}{http://texdoc.net/pkg/ulthese/} %
de la classe \class{ulthese}. Le format du renvoi est toujours tel
qu'illustré ici: un hyperlien mène vers la version en ligne de la
documentation, telle qu'on la trouve dans la distribution {\TeX}~Live,
dans le site %
\link{http://texdoc.net}{TeXdoc Online}. De plus, on indique dans la
marge le nom du fichier (sans l'extension \code{.pdf}) correspondant
sur un système doté de {\TeX}~Live. Sur la plupart des systèmes, il
est possible de consulter hors ligne la documentation
\meta{fichier}\code{.pdf} en entrant à une invite de commande
\begin{quote}
\begin{lstlisting}[backgroundcolor=\color{white}]
texdoc `\meta{fichier}'
\end{lstlisting}
\end{quote}

% Un symbole de lecture vidéo dans la marge indique qu'une capsule vidéo
% est disponible dans la %
% \capsule{http://www.youtube.com/user/ULFormationLaTeX}{chaîne
%   YouTube} %
% de la formation sur le sujet en hyperlien.

Chaque chapitre comporte quelques exercices pour lesquels les solutions
sont fournies en annexe. Le cas échéant, le numéro d'un exercice est
un hyperlien vers sa solution, et vice versa. Afin de faciliter la
résolution de certains exercices, le document est accompagné d'un
gabarit de document nommé \fichier{exercice\_gabarit.tex}.



%%% Local Variables:
%%% mode: latex
%%% TeX-engine: xetex
%%% TeX-master: "formation_latex-partie_2"
%%% encoding: utf-8
%%% End:
