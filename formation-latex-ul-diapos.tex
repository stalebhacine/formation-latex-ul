%%% Copyright (C) 2018 Vincent Goulet
%%%
%%% Ce fichier fait partie du projet
%%% «Rédaction avec LaTeX»
%%% http://github.com/vigou3/formation-latex-ul
%%%
%%% Cette création est mise à disposition selon le contrat
%%% Attribution-Partage dans les mêmes conditions 4.0
%%% International de Creative Commons.
%%% http://creativecommons.org/licenses/by-sa/4.0/

\documentclass[aspectratio=1610,xcolor=x11names,french]{beamer}
  \usepackage{babel}
  \usepackage[autolanguage]{numprint}
  \usepackage{amsmath}
  \usepackage{changepage}                % page licence
  \usepackage{tabularx}                  % page licence
  \usepackage{booktabs}                  % beaux tableaux
  \usepackage{fontawesome}               % icônes
  \usepackage{awesomebox}                % \tipbox et autres
  \usepackage{listings}                  % code source
  \usepackage[export]{adjustbox}         % cadre autour image
  \usepackage[overlay,absolute]{textpos} % couvertures
  \usepackage{metalogo}                  % logo \XeLaTeX

  %% ==========================================
  %%  Informations de publication
  %%  (titre et al. dans couverture-avant.tex)
  %% ==========================================
  \renewcommand{\year}{2016}
  \renewcommand{\month}{11-3}
  \newcommand{\ctanurl}{https://ctan.org/pkg/formation-latex-ul/}
  \newcommand{\ghurl}{https://github.com/vigou3/formation-latex-ul/}

  %% =======================
  %%  Apparence du document
  %% =======================

  %% Thème Beamer
  \usetheme{metropolis}
  \metroset{subsectionpage=progressbar}

  %% Polices de caractères pour les titres
  \newfontfamily\myriadOS[Scale=1.0,Numbers=OldStyle]{Myriad Pro}
  \newfontfamily\myriadFC[Letters=Uppercase,Numbers=Uppercase]{Myriad Pro}
  \newfontfamily\lucida[Scale=0.92]{Lucida Bright OT}

  %% Police Computer Modern pour exemple de police par défaut
  \newfontfamily\CM{cmunrm.otf}

  %% Mathématiques en arev et ajustement de la taille des autres
  %% polices Fira; https://tex.stackexchange.com/a/405211/24355
  \usepackage{arevmath}
  \setsansfont[%
    BoldFont = {Fira Sans SemiBold},
    ItalicFont = {Fira Sans Book Italic},
    BoldItalicFont = {Fira Sans SemiBold Italic}]{Fira Sans Book}

  %% Couleurs
  \definecolor{comments}{rgb}{0.7,0,0}      % commentaires
  \definecolor{link}{rgb}{0,0.4,0.6}        % liens internes
  \definecolor{url}{rgb}{0.6,0,0}           % liens externes
  \definecolor{codebg}{named}{LightYellow1} % fond code source
  \definecolor{rouge}{rgb}{0.85,0,0.07} % rouge bandeau identitaire
  \definecolor{or}{rgb}{1,0.8,0}        % or bandeau identitaire
  \colorlet{alert}{mLightBrown} % alias pour couleur Metropolis
  \colorlet{dark}{mDarkTeal}    % alias pour couleur Metropolis
  \colorlet{shadecolor}{codebg}

  %% Hyperliens
  \hypersetup{%
    pdfauthor = {Vincent Goulet},
    pdftitle = {Rédaction avec LaTeX - Premiers pas},
    colorlinks = true,
    linktocpage = true,
    urlcolor = {url},
    linkcolor = {link},
    citecolor = {citation},
    pdfpagemode = {UseOutlines},
    pdfstartview = {Fit}}
  \setlength{\XeTeXLinkMargin}{1pt}

  %% Affichage de la table des matières du PDF
  \usepackage{bookmark}
  \bookmarksetup{%
    open = true,
    depth = 3,
    numbered = true}

  %% Paramétrage de babel pour les guillemets
  \frenchbsetup{og=«, fg=»}

  %% Sections de code source
  \lstloadlanguages{[LaTeX]TeX}
  \lstset{language=[LaTeX]TeX,
    escapeinside=`',
    extendedchars=true,
    inputencoding=utf8/latin1,
    basicstyle=\small\ttfamily\NoAutoSpacing,
    commentstyle=\color{comments},
    keywordstyle=\mdseries,
    emphstyle=\color{alert}\bfseries,
    backgroundcolor=\color{LightYellow1},
    frame=lr, rulecolor=\color{LightYellow1},
    showstringspaces=false}

  %%% =========================
  %%%  Nouveaux environnements
  %%% =========================

  %% Environnements pour les demo de code; tirés du document
  %% principal. (L'environnement 'eqxample' ajoute des filets de part
  %% et d'autre du bloc pour illustrer les marges.)
  \newenvironment{demo}{%
    \begin{beamercolorbox}[wd=\linewidth,sep=6pt]{block body example}}
    {\end{beamercolorbox}}
  \newenvironment{texample}[1][0.45\linewidth]{%
    \noindent\begin{minipage}{#1}%
      \def\producing{\end{minipage}\hfill\begin{minipage}{\dimexpr0.9\linewidth-#1}%
        \hbox\bgroup\kern-.2pt%
        \vbox\bgroup\parindent0pt\relax
        % The 3pt is to cancel the -\lineskip from \displ@y
        \abovedisplayskip3pt \abovedisplayshortskip\abovedisplayskip
        \belowdisplayskip0pt \belowdisplayshortskip\belowdisplayskip
        \noindent}
    }{%
      \par
      % Ensure that a lonely \[\] structure doesn't take up width less than
      % \hsize.
      \hrule height0pt width\hsize
      \egroup\kern-.2pt\egroup
    \end{minipage}%
    \par
  }
  \newenvironment{eqxample}{%
    \noindent\begin{minipage}{.45\linewidth}%
      \def\producing{\end{minipage}\hfill\begin{minipage}{.45\linewidth}%
        \hbox\bgroup\kern-.2pt\vrule width.2pt%
        \vbox\bgroup\parindent0pt\relax
        % The 3pt is to cancel the -\lineskip from \displ@y
        \abovedisplayskip3pt \abovedisplayshortskip\abovedisplayskip
        \belowdisplayskip0pt \belowdisplayshortskip\belowdisplayskip
        \noindent}
    }{%
      \par
      % Ensure that a lonely \[\] structure doesn't take up width less than
      % \hsize.
      \hrule height0pt width\hsize
      \egroup\vrule width.2pt\kern-.2pt\egroup
    \end{minipage}%
    \par
  }

  %% Simplfication de l'environnement 'quote' de beamer
  \renewenvironment{quote}{%
    \begin{beamercolorbox}[wd=\linewidth,sep=6pt]{block body example}}
    {\end{beamercolorbox}}

  %% Exercices
  \newenvironment{exercice}{%
    \begin{frame}[fragile=singleslide]
      \frametitle{Exercice}}{\end{frame}}

  %% Rubriques conseil du TeXpert
  \newenvironment{conseil}{%
    \begin{frame}[standout,fragile=singleslide]
      \textcolor{alert}{Conseil du {\TeX}pert}\normalsize
      \begin{block}{}}%
    {\end{block}\end{frame}}

  %% =====================
  %%  Nouvelles commandes
  %% =====================

  %% Noms de fonctions, code, environnement, etc.
  \newcommand{\fichier}[1]{\texttt{#1}}
  \newcommand{\class}[1]{\textbf{#1}}
  \newcommand{\pkg}[1]{\textbf{#1}}
  \newcommand{\link}[2]{\href{#1}{#2~\raisebox{-0.2ex}{\faExternalLink}}}

  %% Pour documenter des commandes LaTeX; dérivé de memoir.cls
  \def\bs{\texttt{\char`\\}}
  \newcommand{\meta}[1]{%
    \ensuremath\langle{\normalfont\itshape #1\/}\ensuremath\rangle}
  \newcommand{\marg}[1]{%
    {\ttfamily\char`\{}\meta{#1}{\ttfamily\char`\}}}
  \newcommand{\oarg}[1]{%
    {\ttfamily\char`\[}\meta{#1}{\ttfamily\char`\]}}
  \newcommand{\cs}[1]{\texttt{\char`\\#1}}

  %% Lien vers GitHub dans la page de notices
  \newcommand{\viewsource}[1]{%
    \href{#1}{\faGithub\ Voir sur GitHub}}

  %%% =======
  %%%  Varia
  %%% =======

  %% Longueurs pour la composition des pages couvertures avant et
  %% arrière.
  \newlength{\banderougewidth} \newlength{\banderougeheight}
  \newlength{\bandeorwidth}    \newlength{\bandeorheight}
  \newlength{\imageheight}
  \newlength{\imagewidth}
  \newlength{\logoheight}
  \newlength{\gapwidth}

\begin{document}

%%% Copyright (C) 2018 Vincent Goulet
%%%
%%% Ce fichier fait partie du projet
%%% «Rédaction avec LaTeX»
%%% http://github.com/vigou3/formation-latex-ul
%%%
%%% Cette création est mise à disposition selon le contrat
%%% Attribution-Partage dans les mêmes conditions 4.0
%%% International de Creative Commons.
%%% http://creativecommons.org/licenses/by-sa/4.0/

\begingroup

\TPGrid{3}{36}
\textblockorigin{0mm}{0mm}
\setlength{\parindent}{0mm}
\setlength{\banderougewidth}{2\TPHorizModule}
\setlength{\banderougeheight}{\TPVertModule}
\setlength{\bandeorwidth}{\TPHorizModule}
\setlength{\bandeorheight}{\banderougeheight}
\setlength{\imageheight}{29\TPVertModule}
\setlength{\imagewidth}{3\TPHorizModule}
\setlength{\logoheight}{3.5\TPVertModule}
\setlength{\gapwidth}{0.75pt}
\addtolength{\bandeorwidth}{-\gapwidth}
\addtolength{\imageheight}{-\gapwidth}

\def\titlefmt{%
  \color{black}%
  \myriadOS\bfseries\fontsize{24}{24}\selectfont%
  Rédaction avec\par\vspace*{-18pt}
  \lucida\mdseries\fontsize{27}{27}\selectfont%
  \raisebox{6pt}{{\textbackslash}title}%
  \fontsize{48}{48}\selectfont%
  \{%
  \fontsize{42}{42}\selectfont%
  \LaTeX
  \fontsize{48}{48}\selectfont%
  \}\par}
\def\subtitlefmt{%
  \color{black}%
  \myriadOS\bfseries\fontsize{11}{11}\selectfont%
  Premiers pas\par}
\def\authorfmt{%
  \color{black}%
  \myriadOS\bfseries\fontsize{14}{14}\selectfont%
  Vincent Goulet\par}
\def\affiliation{%
  \color{black}%
  \myriadOS\mdseries\fontsize{11}{12}\selectfont%
  Professeur titulaire \\
  École d'actuariat, Université Laval}
\def\edition{%
  \color{black}%
  \myriadOS\mdseries\fontsize{11}{11}\selectfont%
  Édition {\myriadFC\year}.\month}

%%%
%%% Page de titre
%%%

\begin{frame}[plain]
  %% bandeau identitaire
  \begin{textblock*}{125mm}[0,1](0mm,30\TPVertModule)
    \textcolor{rouge}{\rule{\banderougewidth}{\banderougeheight}}% % bande rouge
    \rule{\gapwidth}{0pt}%                                         % filet
    \textcolor{or}{\rule{\bandeorwidth}{\bandeorheight}}           % bande or
  \end{textblock*}

  %% logo UL
  \begin{textblock*}{\TPHorizModule}(2\TPHorizModule,31\TPVertModule)
    \rule{\gapwidth}{0pt}%                                     % filet
    \includegraphics[height=\logoheight,keepaspectratio=true]{ul_p}
  \end{textblock*}

  %% image de fond
  \begin{textblock*}{125mm}(0mm,0mm)
    \includegraphics[width=\imagewidth,%
                     keepaspectratio=true]{Suricata-diapos.jpg}
  \end{textblock*}

  %% trame (titre)
  \begin{textblock*}{2\TPHorizModule}(0mm,13\TPVertModule)
    \pgfsetfillopacity{0.5}
    \textcolor{white}{\rule{\linewidth}{11\TPVertModule}}
    \pgfsetfillopacity{1}
  \end{textblock*}

  %% titre
  \begin{textblock*}{1.6\TPHorizModule}(0.4\TPHorizModule,14.2\TPVertModule)
    \titlefmt
  \end{textblock*}

  %% trame (premiers pas)
  \begin{textblock*}{\TPHorizModule}(2\TPHorizModule,24\TPVertModule)
    \pgfsetfillopacity{0.5}
    \textcolor{white}{\rule{\linewidth}{2\TPVertModule}}
    \pgfsetfillopacity{1}
  \end{textblock*}

  %% sous-titre (premiers pas)
  \begin{textblock*}{0.8\TPHorizModule}(2.1\TPHorizModule,24.5\TPVertModule)
    \subtitlefmt
  \end{textblock*}
\end{frame}

%%%
%%% Page frontispice
%%%

\begin{frame}[plain]
  \begin{textblock*}{1.6\TPHorizModule}(0.4\TPHorizModule,4\TPVertModule)
    \authorfmt
  \end{textblock*}

  \begin{textblock*}{1.6\TPHorizModule}(0.4\TPHorizModule,6\TPVertModule)
    \affiliation
  \end{textblock*}

  \begin{textblock*}{1.6\TPHorizModule}(0.4\TPHorizModule,14.2\TPVertModule)
    \titlefmt
  \end{textblock*}

  \begin{textblock*}{1.6\TPHorizModule}(0.4\TPHorizModule,30\TPVertModule)
    \edition
  \end{textblock*}
\end{frame}
\endgroup

%%% Local Variables:
%%% TeX-master: "formation-latex-ul-diapos"
%%% TeX-engine: xetex
%%% coding: utf-8
%%% End:

%%% Copyright (C) 2018 Vincent Goulet
%%%
%%% Ce fichier fait partie du projet
%%% «Rédaction avec LaTeX»
%%% http://github.com/vigou3/formation-latex-ul
%%%
%%% Cette création est mise à disposition selon le contrat
%%% Attribution-Partage dans les mêmes conditions 4.0
%%% International de Creative Commons.
%%% http://creativecommons.org/licenses/by-sa/4.0/

\begin{frame}[t,plain,fragile=singleslide]
  \tiny
  \vspace*{10mm}

  \begin{adjustwidth}{10mm}{10mm}
    \raisebox{-1.4mm}{%
      \includegraphics[height=4mm,keepaspectratio=true]{by-sa}} %
    Vincent Goulet, {\year}

    {\textcopyright} {\year} par Vincent Goulet. «Rédaction avec
    {\LaTeX}» est mis à disposition selon le contrat
    \href{http://creativecommons.org/licenses/by-sa/4.0/deed.fr}{%
      Attribution-Partage dans les mêmes conditions 4.0 International}
    de Creative Commons. En vertu de ce contrat, vous êtes libre de:
    \begin{itemize}
    \item[\color{black}\tiny$\blacktriangleright$]%
      \textbf{partager} --- reproduire, distribuer et communiquer
      l'{\oe}uvre;
    \item[\color{black}\tiny$\blacktriangleright$]%
      \textbf{remixer} --- adapter l'{\oe}uvre;
    \item[\color{black}\tiny$\blacktriangleright$] utiliser cette
      {\oe}uvre à des fins commerciales.
    \end{itemize}
    Selon les conditions suivantes: \vspace*{2mm}

    \begin{tabularx}{\linewidth}{@{}lX@{}}
      \raisebox{-5.5mm}[0mm][10mm]{%
      \includegraphics[height=7mm,keepaspectratio=true]{by}}
      & \textbf{Attribution} --- Vous devez créditer l'{\oe}uvre, intégrer
        un lien vers le contrat et indiquer si des modifications ont été
        effectuées à l'{\oe}uvre. Vous devez indiquer ces informations par
        tous les moyens possibles, mais vous ne pouvez suggérer que
        l'Offrant vous soutient ou soutient la façon dont vous avez
        utilisé son {\oe}uvre. \\
      \raisebox{-5.5mm}{\includegraphics[height=7mm,keepaspectratio=true]{sa}}
      & \textbf{Partage dans les mêmes conditions} --- Dans le cas où
        vous modifiez, transformez ou créez à partir du matériel composant
        l'{\oe}uvre originale, vous devez diffuser l'{\oe}uvre modifiée
        dans les même conditions, c'est à dire avec le même contrat avec
        lequel l'{\oe}uvre originale a été diffusée.
    \end{tabularx}

    \textbf{Code source} \\
    \viewsource{\ghurl}

    \textbf{Crédits} \\
    Concept original de la couverture: Marie-Ève Guérard. \\
    Photo: Olaf Leillinger,
    \href{https://creativecommons.org/licenses/by-sa/3.0/deed.en}{CC
      BY-SA 3.0 Unported} via
    \href{https://commons.wikimedia.org/wiki/File:Suricata.suricatta.6861.jpg}{%
      Wikimedia Commons}. \\
  \end{adjustwidth}
\end{frame}

%%% Local Variables:
%%% TeX-master: "formation-latex-ul-diapos"
%%% TeX-engine: xetex
%%% coding: utf-8
%%% End:

\begin{frame}
  \frametitle{Fichiers d'accompagnement}

  Ce document devrait être accompagné des fichiers nécessaires pour
  compléter les exercices.
  \bigskip

  Si vous n'avez pas obtenu ces fichiers avec le document, vous pouvez
  les récupérer dans le site \emph{Comprehensive TeX Archive Network}
  (CTAN).
  \medskip
  \begin{center}
    \href{\ctanurl}{\ctanbutton}
  \end{center}
\end{frame}

\begin{frame}
  \frametitle{Pré-requis à cette formation}
  \begin{enumerate}
  \item Installer une distribution {\LaTeX} sur votre poste de
    travail; nous recommandons la distribution {\TeX}~Live
    \begin{itemize}
      \normalsize
    \item[] \capsule{https://www.youtube.com/watch?v=fjcR6lFy0c4}{%
        installation sur macOS}
    \item[] \capsule{https://www.youtube.com/watch?v=z_dq3dns-WU}{%
        installation sur Windows}
    \end{itemize}
    \bigskip
  \item Compiler un premier document très simple de type \emph{Hello World!}
    \begin{itemize}
      \normalsize
    \item[] \capsule{https://www.youtube.com/watch?v=QOUx_aOZ42o}{%
        démonstration sur macOS avec TeXShop}
    \item[] \capsule{https://www.youtube.com/watch?v=qddRMGwXnNM}{%
        démonstration sur Windows avec Texmaker}
    \end{itemize}
  \end{enumerate}
\end{frame}

%%% Local Variables:
%%% mode: latex
%%% TeX-engine: xetex
%%% TeX-master: "formation-latex-ul-diapos"
%%% End:


\include{presentation-diapos}
\section{Principes de base}

\subsection{Règles de saisie}

\begin{frame}[fragile=singleslide]
  \frametitle{Rédaction}

  L'apparence du document est prise en charge par {\LaTeX} et
  il est généralement préférable de ne pas la modifier.

  \begin{itemize}
  \item On se concentre sur le \alert{contenu} et la \alert{structure} du
    document
      \bigskip
      \begin{tabbing}
        titre de section \qquad\= \faArrowRight \qquad\= \verb|\section{titre}| \\[6pt]
        emphase \> \faArrowRight \> \verb|\emph{texte}|
      \end{tabbing}
      \bigskip
  \item Mots séparés par une ou plusieurs \alert{espaces}
  \item Paragraphes séparés par une ou plusieurs \alert{lignes blanches}
  \item Utilisation de \alert{commandes} pour indiquer la structure du texte
  \end{itemize}
\end{frame}

\subsection{Structure d'un fichier}

\begin{frame}[fragile]
  \frametitle{Structure d'un document {\LaTeX}}

  Un fichier source {\LaTeX} est toujours composé de deux parties.

  \hfill
  \begin{minipage}{0.75\linewidth}
\begin{lstlisting}[emph={documentclass,begin,end,document}]
\documentclass[11pt,french]{article}
  \usepackage{babel}
  \usepackage[autolanguage]{numprint}
  \usepackage[utf8]{inputenc}
  \usepackage[T1]{fontenc}

\begin{document}

Lorem ipsum dolor sit amet, consectetur
adipiscing elit. Donec quam nulla, bibendum
vitae ipsum vel, fermentum pellentesque orci.

\end{document}
\end{lstlisting}
  \end{minipage}

  \begin{textblock*}{\linewidth}(0mm,0mm)
    \begin{picture}(0,0)
      \thicklines\color{blue}
      \onslide<2>{\put(120,-138){\dashbox{2}(285,76){}}}
      \onslide<3>{\put(120,-248){\dashbox{2}(285,104){}}}
      \onslide<2>{\put(46,-103){préambule}}
      \onslide<3>{\put(46,-198){\parbox{25mm}{corps du\\ document}}}
    \end{picture}
  \end{textblock*}
\end{frame}

\subsection{Classes et paquetages}

\begin{frame}[fragile]
  \frametitle{Classe de document}

  La première commande du préambule est normalement la déclaration de
  la classe de la forme
\begin{lstlisting}
\documentclass[`\textit{options}']{`\textit{classe}'}
\end{lstlisting}

  \begin{itemize}
  \item Principales classes
    \begin{quote}
      \class{article, report, book, letter} \\
      {\color{alert} \class{memoir}} \\
      {\color{alert} \class{ulthese}}
    \end{quote}
  \item Principales options
    \begin{quote}
      \texttt{10pt, {\color{alert} 11pt}, 12pt} \\
      \texttt{oneside, twoside} \\
      \texttt{openright, openany} \\
      {\color{alert} \texttt{article}} (classe \class{memoir})
    \end{quote}
  \end{itemize}
\end{frame}

\begin{frame}[fragile=singleslide]
  \frametitle{Paquetages}
  \begin{itemize}
  \item Permettent de modifier des commandes ou d'ajouter des
    fonctionnalités au système
  \item Chargés dans le préambule avec
    \begin{lstlisting}
\usepackage{`\textit{paquetage}'}
\usepackage[`\textit{options}']{`\textit{paquetage}'}
\usepackage{`\textit{paquetage1,paquetage2,...}'}
    \end{lstlisting}
  \end{itemize}
\end{frame}

\subsection{[~Exercice~]}

\begin{exercice}
  Utiliser le fichier \fichier{exercice\_classe+paquetages.tex}.

  \begin{enumerate}
  \item Compiler le fichier tel que fourni.
  \item Changer la police du document pour 11~points, puis 12~points.
    Observer l'effet sur les marges et sur la coupure automatique des
    mots.
  \item Activer le paquetage \pkg{icomma} en supprimant le symbole \%
    au début de la ligne dans le préambule. Observer l'effet sur la
    formule mathématique.
  \item Charger le paquetage \pkg{numprint} avec l'option
    \verb=autolanguage= (\emph{après} le paquetage \pkg{babel}). Dans
    le code source de la formule mathématique, changer
\begin{lstlisting}
10 000
\end{lstlisting}
    pour
\begin{lstlisting}
\nombre{10000}
\end{lstlisting}
    et observer le résultat.
  \end{enumerate}
\end{exercice}

\subsection{Commandes et environnements}

\begin{frame}[fragile=singleslide]
  \frametitle{Commandes}
  \begin{itemize}
  \item Débutent toujours par \verb=\=
  \item Formes générales:
\begin{lstlisting}
\`\textit{nomcommande}'[`\textit{arg\_optionnel}']{`\textit{arg\_obligatoire}'}
\`\textit{nomcommande}'*[`\textit{arg\_optionnel}']{`\textit{arg\_obligatoire}'}
\end{lstlisting}
  \item Arguments obligatoires entre \verb={ }=
  \item Arguments optionnels entre \verb=[ ]=
  \item Commande sans argument: le nom se termine par tout
    caractère qui n'est pas une lettre (y compris l'espace!)
  \item Portée d'une commande limitée à la zone entre \verb={ }=
  \end{itemize}
\end{frame}

\begin{frame}[fragile=singleslide]
  \frametitle{Environnements}
  \begin{itemize}
  \item Délimités par
\begin{lstlisting}
\begin{`\textit{environnement}'}
   ...
\end{`\textit{environnement}'}
    \end{lstlisting}
  \item Contenu de l'environnement traité différemment du reste du texte
  \item Changements s'appliquent uniquement à l'intérieur de
    l'environnement
  \end{itemize}
\end{frame}

\subsection{Commentaires}

\begin{frame}[fragile=singleslide]
  \frametitle{Commentaires}
  \begin{itemize}
  \item Le symbole \verb=%= sert à identifier les commentaires dans
    le code source
  \item Tout ce qui suit \verb=%= sur la ligne est ignoré
    \begin{demo}
      \begin{texample}
\begin{lstlisting}
texte % ignoré par LaTeX
\end{lstlisting}
        \producing
        texte % ignoré par LaTeX
      \end{texample}
    \end{demo}
  \end{itemize}
\end{frame}

\subsection{[~Exercice~]}

\begin{exercice}
  Modifier le fichier \fichier{exercice\_commandes.tex} afin de
  produire le texte ci-dessous.

  \bigskip
  \centering
  \fbox{\includegraphics[viewport=108 551 502 665,%
    clip=true,width=0.9\linewidth]{exercice_commandes-solution}}
\end{exercice}

\subsection{Caractères spéciaux}

\begin{frame}[fragile=singleslide]
  \frametitle{Caractères réservés}

  \begin{itemize}
  \item Caractères réservés par {\TeX}:
    \begin{quote}
      \verb=# $ & ~ _ ^ % { }=
    \end{quote}
  \item Pour les utiliser, précéder par \verb=\=
  \item On écrira donc
    \begin{demo}
      \begin{texample}
\begin{lstlisting}
L'augmentation de 2~\$
représente une hausse
de 5~\%.
\end{lstlisting}
        \producing
        L'augmentation de 2~\$ représente une
        hausse de 5~\%.
      \end{texample}
    \end{demo}
  \end{itemize}
\end{frame}

\begin{frame}[fragile=singleslide]
  \frametitle{Espaces, guillemets et tirets}
  \begin{itemize}
  \item Espace insécable: \verb=~= %
\begin{lstlisting}
M.~Tremblay me doit 200~\$.
\end{lstlisting}
  \item Guillemets:
    \begin{demo}
      \begin{texample}
\begin{lstlisting}[escapeinside={}]
``guillemets anglais''
\end{lstlisting}
        \producing
        ``guillemets anglais''
      \end{texample}
      \begin{texample}
\begin{lstlisting}
«guillemets français»
\end{lstlisting}
        \producing
        «guillemets français»
      \end{texample}
   \end{demo}
  \item Tiret, tiret demi-cadratin, tiret cadratin:
    \begin{demo}
      \begin{minipage}{0.15\linewidth}
        \begin{texample}
\begin{lstlisting}
-
\end{lstlisting}
          \producing
          -
        \end{texample}
      \end{minipage}
      \hfill
      \begin{minipage}{0.15\linewidth}
        \begin{texample}
\begin{lstlisting}
--
\end{lstlisting}
          \producing
          --
        \end{texample}
      \end{minipage}
      \hfill
      \begin{minipage}{0.15\linewidth}
        \begin{texample}
\begin{lstlisting}
---
\end{lstlisting}
          \producing
          ---
        \end{texample}
      \end{minipage}
      \hfill
    \end{demo}
  \item Détails additionnels à la section~2.9 du document de référence
  \end{itemize}
\end{frame}

\begin{frame}[fragile]
  \frametitle{{\LaTeX} en français}

  Il faut charger un certain nombre de paquetages pour franciser \LaTeX.

  \begin{itemize}
  \item \pkg{babel}: traduction des mots-clés prédéfinis,
    typographie française, coupure de mots, document multilingue
  \item \pkg{inputenc} et \pkg{fontenc}: lettres accentuées dans le
    code source (pdf{\LaTeX} seulement)
  \item \pkg{icomma}: virgule comme séparateur décimal
  \item \pkg{numprint}: espace comme séparateur des milliers
  \end{itemize}
\end{frame}

%%% Local Variables:
%%% TeX-master: "formation-latex-ul-diapos"
%%% TeX-engine: xetex
%%% coding: utf-8
%%% End:

%%% Copyright (C) 2018 Vincent Goulet
%%%
%%% Ce fichier fait partie du projet
%%% «Rédaction avec LaTeX»
%%% http://github.com/vigou3/formation-latex-ul
%%%
%%% Cette création est mise à disposition selon le contrat
%%% Attribution-Partage dans les mêmes conditions 4.0
%%% International de Creative Commons.
%%% http://creativecommons.org/licenses/by-sa/4.0/

\section{Organisation d'un document}

\begin{conseil}
  Utilisez impérativement les commandes {\LaTeX} pour identifier les
  différentes parties (la structure) d'un document.
\end{conseil}

\subsection{Parties d'un document}

\begin{frame}[fragile]
  \frametitle{Titre et page de titre}
  \begin{itemize}
  \item Mise en forme automatique
    \begin{lstlisting}
%% préambule
\title{`\meta{Titre du document}'}
\author{`\meta{Prénom Nom}'}
\date{`\meta{31 octobre 2014}'} % automatique si omis

%% corps du document
\maketitle
    \end{lstlisting}
  \item Mise en forme libre \\[6pt]
    \begin{minipage}{0.45\linewidth}
      \begin{block}{\small classes standards}
\begin{lstlisting}
\begin{titlepage}
  ...
\end{titlepage}
\end{lstlisting}
      \end{block}
    \end{minipage}
    \hfill
    \begin{minipage}{0.45\linewidth}
      \begin{block}{\small classe \class{memoir}}
\begin{lstlisting}
\begin{titlingpage}
  ...
\end{titlingpage}
\end{lstlisting}
      \end{block}
    \end{minipage}
  \end{itemize}
\end{frame}

\begin{frame}[fragile=singleslide]
  \frametitle{Sections}
  \begin{itemize}
  \item Découpage du document en sections
\begin{lstlisting}
\part{`\meta{titre}'}
\chapter{`\meta{titre}'}
\section{`\meta{titre}'}
\subsection{`\meta{titre}'}
\end{lstlisting}
\begin{lstlisting}
\subsubsection{`\meta{titre}'}     % à éviter dans un livre
\end{lstlisting}
\begin{lstlisting}
\paragraph{`\meta{titre}'}         % jamais (?) utilisé
\subparagraph{`\meta{titre}'}      % idem
\end{lstlisting}
  \item Numérotation automatique
    \begin{demo}
      \begin{minipage}{0.45\linewidth}
\begin{lstlisting}
\section{Hypothèses}
\end{lstlisting}
      \end{minipage}
      \hfill
      \begin{minipage}{0.45\linewidth}
        \includegraphics[height=0.8\baselineskip,keepaspectratio]{section-num}
      \end{minipage}
    \end{demo}
  \item Sans la numérotation
    \begin{demo}
      \begin{minipage}{0.45\linewidth}
\begin{lstlisting}
\section*{Hypothèses}
\end{lstlisting}
      \end{minipage}
      \hfill
      \begin{minipage}{0.45\linewidth}
        \includegraphics[height=0.8\baselineskip,keepaspectratio]{section-non-num}
      \end{minipage}
    \end{demo}
  \end{itemize}
\end{frame}

\begin{frame}[fragile=singleslide]
  \frametitle{Annexes}

  Les annexes sont des sections ou des chapitres avec une numérotation
  alphanumérique (A, A.1, ...)
  \begin{itemize}
  \item \cs{appendix} identifie les sections suivantes comme des annexes
  \item Dans le titre, «Chapitre» changé pour «Annexe» le cas échéant
  \end{itemize}
\end{frame}

\begin{frame}[fragile=singleslide]
  \frametitle{Table des matières}

  La commande \cs{tableofcontents} produit automatiquement la table
  des matières.

  \begin{itemize}
  \item Requiert plusieurs compilations
  \item Sections non numérotées pas incluses
  \item Avec \pkg{hyperref}, produit également la table des
    matières du fichier PDF
  \item Classe \class{memoir} fournit \cs{tableofcontents*} qui
    n'insère pas la table des matières dans la table des matières
  \end{itemize}
\end{frame}

\subsection{[~Exercice~]}

\begin{exercice}
  Utiliser le fichier \fichier{exercice\_parties.tex}.

  \begin{enumerate}
  \item Étudier la structure du document dans le code source.
  \item Ajouter un titre et un auteur au document.
  \item Créer la table des matières du document en le compilant 2 à 3
    fois.
  \item Insérer deux ou trois titres de sections de différents niveaux
    dans le document.
  \item Vous remarquerez que la numérotation cesse à partir des
    sous-sections. C'est une particularité de la classe
    \class{memoir}.

    Recompiler le document après avoir ajouté au préambule la commande
\begin{lstlisting}
\maxsecnumdepth{subsection}
\end{lstlisting}
  \item Ajouter une annexe au document.
  \end{enumerate}
\end{exercice}

\subsection{Renvois automatiques}

\begin{frame}
  \frametitle{Étiquettes et renvois automatiques}

  \begin{itemize}
  \item Ne \alert{jamais} renvoyer manuellement à un numéro de
    section, d'équation, de tableau, etc.
  \item «Nommer» un élément avec \cs{label}
  \item Faire référence par son nom avec \cs{ref}
  \item Requiert 2 à 3 compilations
  \end{itemize}
\end{frame}

\begin{frame}[fragile=singleslide]
  \frametitle{Exemple (code source)}

  \begin{lstlisting}[emph={\label,\ref}]
\section{Définitions}
\label{sec:definitions}

Lorem ipsum dolor sit amet, consectetur
adipiscing elit. Duis in auctor dui. Vestibulum
ut, placerat ac, adipiscing vitae, felis.

\section{Historique}

Tel que vu à la section \ref{sec:definitions},
on a...
\end{lstlisting}
\end{frame}

\begin{frame}
  \frametitle{Exemple (résultat)}

  \fbox{\includegraphics[viewport=124 550 484 664,clip=true,width=0.98\linewidth]{exemple-renvoi}}
\end{frame}

\begin{conseil}
  Adoptez une manière systématique et mnémotechnique de nommer les
  éléments dans un long document afin de vous y retrouver.

  \bigskip %
\begin{lstlisting}[basicstyle=\normalsize\ttfamily\NoAutoSpacing\color{dark}]
\label{chap:`\meta{chapitre}'}         % chapitre
\label{sec:`\meta{chapitre}':`\meta{section}'}  % section
\label{tab:`\meta{chapitre}':`\meta{tableau}'}  % tableau
\label{eq:`\meta{chapitre}':`\meta{equation}'}  % équation
\end{lstlisting}
\end{conseil}

\subsection{[~Exercice~]}

\begin{exercice}
  Utiliser le fichier \fichier{exercice\_renvois.tex}.
  \begin{enumerate}
  \item Insérer dans le texte un renvoi au numéro d'une section.
  \item Activer le paquetage \pkg{hyperref} avec l'option
    \texttt{colorlinks} et comparer l'effet d'utiliser \cs{ref} ou
    \cs{autoref} pour le renvoi.
  \end{enumerate}
\end{exercice}

%%% Local Variables:
%%% TeX-master: "formation-latex-ul-diapos"
%%% TeX-engine: xetex
%%% coding: utf-8
%%% End:

\section{Apparence et disposition du texte}

\subsection{Police et style}

\begin{frame}
  \frametitle{Police de caractères}

  Par défaut, {\LaTeX} compose les documents dans la police
  {\CM Computer Modern}.

  \begin{itemize}
  \item Aujourd'hui plus facile d'utiliser d'autres polices, surtout
    avec {\XeLaTeX}
  \item \alert{Attention}: peu de polices adaptées pour les
    mathématiques
  \item Commandes pour modifier les \alert{attributs} de la police
    (famille, forme, graisse)
  \item Commandes pour modifier la \alert{taille} du texte (de
    \cs{tiny} à \cs{Huge})
  \end{itemize}
\end{frame}

\begin{frame}[fragile]
  \frametitle{Italique}
  \begin{itemize}
  \item<1-> Une des propriétés les \emph{plus utilisées} dans le texte
    \vfill
  \item<1-> Commande sémantique:
\begin{lstlisting}
\emph`\marg{texte}'
\end{lstlisting}
    \vfill
  \item<2-> Par défaut: texte en italique dans texte droit et vice versa
    \begin{demo}
      \small
      \begin{texample}[0.48\linewidth]
\begin{lstlisting}
C'était un peu \emph{rough}
par moments
\end{lstlisting}
        \producing
        C'était un peu \emph{rough} par moments
      \end{texample}

      \begin{texample}[0.48\linewidth]
\begin{lstlisting}
Il m'a dit: «\emph{Enough
\emph{poutine} for the
week!}»
\end{lstlisting}
        \producing
        Il m'a dit: «\emph{Enough \emph{poutine} for the week!}»
      \end{texample}
    \end{demo}
  \item<3-> Pas de commande pour souligner en {\LaTeX\dots} et ce n'est
    pas une omission!
  \end{itemize}
\end{frame}

\subsection{Disposition sur la page}

\begin{frame}[fragile]
  \frametitle{Listes}
  \begin{itemize}
  \item Deux principales sortes de listes:
    \begin{enumerate}
    \item \alert{à puce} avec environnement \texttt{itemize}
    \item \alert{numérotée} avec environnement \texttt{enumerate}
    \end{enumerate}
  \item Possible de les imbriquer les unes dans les autres
  \item Marqueurs adaptés automatiquement jusqu'à 4 niveaux
  \end{itemize}
  \pause

\begin{lstlisting}
\begin{itemize}
\item Deux principales sortes de listes:
  \begin{enumerate}
  \item à puce avec environnement \texttt{itemize}
  \item numérotée avec environnement \texttt{enumerate}
  \end{enumerate}
\item Possible de les imbriquer les unes dans les autres
\item Marqueurs adaptés automatiquement jusqu'à 4 niveaux
\end{itemize}
\end{lstlisting}
\end{frame}

\begin{frame}[fragile=singleslide]
  \frametitle{Texte centré}

  \begin{itemize}
  \item Environnement \texttt{center} pour centrer un bloc de texte
    \begin{demo}
      \begin{eqxample}
\begin{lstlisting}
\begin{center}
  Centrer un mot ou une
  expression les met en
  évidence.
\end{center}
\end{lstlisting}
        \producing
        \begin{center}
          Centrer un mot ou une expression
          les met en évidence.
        \end{center}
      \end{eqxample}
    \end{demo}
  \item Commande \cs{centering} pour centrer tout le texte qui suit
    \begin{itemize}
    \item surtout utilisée pour centrer les tableaux et les figures
    \end{itemize}
  \end{itemize}
\end{frame}

\begin{frame}[fragile]
  \frametitle{Notes de bas de page}
  \begin{itemize}
  \item Note de bas de page insérée avec la commande
\begin{lstlisting}
\footnote`\marg{texte de la note}'
\end{lstlisting}
  \item Commande doit suivre immédiatement le texte à annoter
  \item Numérotation et disposition automatiques
  \end{itemize}
\end{frame}

\begin{frame}[fragile=singleslide]
  \frametitle{Code source}
  \begin{itemize}
  \item Environnement \texttt{verbatim}
\begin{lstlisting}
\begin{verbatim}
Texte disposé exactement tel qu'il est tapé
dans une police à largeur fixe
\end{verbatim}
\end{lstlisting}
  \item Pour usage plus intensif, utiliser le paquetage \pkg{listings}
  \end{itemize}
\end{frame}

\subsection{[~Exercice~]}

\begin{exercice}
  Utiliser le fichier \fichier{exercice\_complet.tex}.

  \begin{enumerate}
  \item Étudier le code source du fichier, puis le compiler.
  \item Supprimer l'option \texttt{article} au chargement de la classe
    et compiler de nouveau le document. Observer l'effet de cette
    option.
  \item Effectuer les modifications suivantes au document.
    \begin{enumerate}[a)]
    \item Dernier paragraphe de la première section, placer toute la
      phrase débutant par \texttt{«De simple dérivé»} à l'intérieur
      d'une commande \cs{emph}.
    \item Changer la puce des listes en spécifiant le symbole
      \texttt{\$>\$} pour \texttt{ItemLabeli} dans la commande
      \cs{frenchbsetup} du préambule.
    \end{enumerate}
  \end{enumerate}
\end{exercice}

%%% Local Variables:
%%% TeX-master: "formation-latex-ul-diapos"
%%% TeX-engine: xetex
%%% coding: utf-8
%%% End:

%%% Copyright (C) 2018 Vincent Goulet
%%%
%%% Ce fichier fait partie du projet
%%% «Rédaction avec LaTeX»
%%% http://github.com/vigou3/formation-latex-ul
%%%
%%% Cette création est mise à disposition selon le contrat
%%% Attribution-Partage dans les mêmes conditions 4.0
%%% International de Creative Commons.
%%% http://creativecommons.org/licenses/by-sa/4.0/

\section{B.a.-ba des mathématiques}

\begin{frame}[fragile=singleslide]
  \frametitle{Principes de base}
  \begin{itemize}
  \item Décrire des équations mathématiques requiert un «langage» spécial
    \begin{itemize}
    \item il faut informer {\LaTeX} que l'on passe à ce langage
    \item par le biais de modes mathématiques
    \end{itemize}
  \item Important d'utiliser un mode mathématique
    \begin{itemize}
    \item règles de typographie spéciales
    \item espaces gérées automatiquement
    \end{itemize}
  \item Vous voulez utiliser le paquetage \pkg{amsmath}
\begin{lstlisting}
\usepackage{amsmath}
\end{lstlisting}
  \end{itemize}
\end{frame}

\begin{frame}[fragile]
  \frametitle{Modes mathématiques}
  \begin{enumerate}[<+->]
  \item «En ligne» directement dans le texte comme $(a + b)^2 = a^2 +
    2ab + b^2$ en plaçant l'équation entre \verb=$ $=
\begin{lstlisting}
«En ligne» directement dans le texte
comme $(a + b)^2 = a^2 + 2ab + b^2$
\end{lstlisting}
  \item «Hors paragraphe» séparé du texte principal comme
    \begin{equation*}
      \int_0^\infty f(x)\, dx = \sum_{i = 1}^n \alpha_i e^{x_i} f(x_i)
    \end{equation*}
    en utilisant divers types d'environnements
\begin{lstlisting}
«Hors paragraphe» séparé du texte principal comme
\begin{equation*}
  \int_0^\infty f(x)\, dx =
  \sum_{i = 1}^n \alpha_i e^{x_i} f(x_i)
\end{equation*}
\end{lstlisting}
  \end{enumerate}
\end{frame}


\begin{conseil}
  En ligne ou hors paragraphe, les équations font partie intégrante de
  la phrase.

  Les règles de ponctuation usuelles s'appliquent donc aux équations.

  \bigskip
  \fbox{\includegraphics[width=0.95\linewidth]{ponctuation}}
\end{conseil}

\begin{frame}[fragile]
  \frametitle{Quelques règles de base}
  \begin{itemize}
  \item En mode mathématique, {\TeX} écrit automatiquement les
    constantes en romain et les variables en italique
    \begin{demo}
      \begin{texample}
\begin{lstlisting}
$z = 2a + 3y$
\end{lstlisting}
        \producing
        $z = 2a + 3y$
      \end{texample}
    \end{demo}
  \item Espacement entre les éléments géré automatiquement, peu importe
    le code source
    \begin{demo}
      \begin{texample}
\begin{lstlisting}
$z=2 a+3 y$
\end{lstlisting}
        \producing
        $z=2 a+3 y$
      \end{texample}
    \end{demo}
  \end{itemize}
\end{frame}

\begin{frame}[fragile]
  \frametitle{Quelques règles de base (suite)}
  \begin{itemize}
  \item \alert{Ne pas} utiliser le mode mathématique pour obtenir du
    texte en italique!
    \begin{demo}
      \begin{minipage}{0.45\linewidth}
\begin{lstlisting}
\emph{xyz}
\end{lstlisting}
      \end{minipage}
      \hfill
      \begin{minipage}{0.45\linewidth}
        \includegraphics[height=0.8\baselineskip,keepaspectratio]{xyz-emph}
      \end{minipage}\par
      \begin{minipage}{0.45\linewidth}
\begin{lstlisting}
$xyz$
\end{lstlisting}
      \end{minipage}
      \hfill
      \begin{minipage}{0.45\linewidth}
        \includegraphics[height=0.8\baselineskip,keepaspectratio]{xyz-math}
      \end{minipage}
    \end{demo}
  \item Commande \cs{text} de \pkg{amsmath} pour texte à
    l'intérieur du mode mathématique
    \begin{demo}
      \begin{texample}
\begin{lstlisting}
$x = 0 \text{ si } y < 2$
\end{lstlisting}
        \producing
        $x = 0 \text{ si } y < 2$
      \end{texample}
    \end{demo}
  \end{itemize}
\end{frame}

\begin{frame}[fragile]
  \frametitle{Avant-goût}

  Pouvez-vous interpréter ce code?
\begin{lstlisting}
\begin{equation*}
  \Gamma(\alpha) =
  \sum_{j = 0}^\infty \int_j^{j + 1}
    x^{\alpha - 1} e^{-x}\, dx
\end{equation*}
\end{lstlisting}
  \vspace{18pt}
  \pause

  Fort probablement!
  \begin{equation*}
    \Gamma(\alpha) =
    \sum_{j = 0}^\infty \int_j^{j + 1} x^{\alpha - 1} e^{-x}\, dx
  \end{equation*}
\end{frame}

\subsection{[~Exercice~]}

\begin{exercice}
  Utiliser le fichier \fichier{exercice\_mathematiques.tex}.

  \begin{enumerate}
  \item Étudier le code source du fichier, puis le compiler.
  \item Charger le paquetage \pkg{amsfonts} dans le préambule, puis
    remplacer \verb=$R^+$= par \verb=$\mathbb{R}^+$= à la ligne
    débutant par «Le domaine».
  \item À l'aide de la fonction Rechercher et remplacer de l'éditeur
    de texte, remplacer toutes les occurrences du symbole $\theta$ par
    $\lambda$.
  \end{enumerate}
\end{exercice}


%%% Local Variables:
%%% TeX-master: "formation-latex-ul-diapos"
%%% TeX-engine: xetex
%%% coding: utf-8
%%% End:

%%% Copyright (C) 2018 Vincent Goulet
%%%
%%% Ce fichier fait partie du projet
%%% «Rédaction avec LaTeX»
%%% http://github.com/vigou3/formation-latex-ul
%%%
%%% Cette création est mise à disposition selon le contrat
%%% Attribution-Partage dans les mêmes conditions 4.0
%%% International de Creative Commons.
%%% http://creativecommons.org/licenses/by-sa/4.0/

\section{Tableaux}

\begin{frame}
  \frametitle{De la conception de beaux tableaux}

  Lequel de ces deux tableaux est le plus facile à consulter?
  \begin{center}
  \hfill
  \begin{tabular}{|>{$}c<{$}|>{$}r<{$}|>{$}r<{$}|}
    \hline\hline
    i &
    \multicolumn{1}{c|}{$v$} &
    \multicolumn{1}{c|}{$b_i$} \\
    \hline
    0 & \nombre{91492} &  60 \\
    \hline
    1 &  \nombre{1524} &  60 \\
    \hline
    2 &            25  &  24 \\
    \hline
    3 &             1  & 365 \\
    \hline\hline
  \end{tabular}
  \hfill
  \begin{tabular}{>{$}c<{$}>{$}r<{$}>{$}r<{$}}
    \toprule
    i &
    \multicolumn{1}{c}{$v$} &
    \multicolumn{1}{c}{$b_i$} \\
    \midrule
    0 & \nombre{91492} &  60 \\
    1 &  \nombre{1524} &  60 \\
    2 &            25  &  24 \\
    3 &             1  & 365 \\
    \bottomrule
  \end{tabular}
  \hspace*{\fill}
  \end{center}

  \pause
  Deux règles d'or:
  \begin{enumerate}
  \item \alert{jamais} de filets verticaux
  \item pas de filets doubles
  \end{enumerate}
\end{frame}

\begin{frame}[fragile=singleslide]
  \frametitle{Paquetage essentiel}

  \begin{itemize}
  \item Vous voulez utiliser le paquetage \pkg{booktabs}
\begin{lstlisting}
\usepackage{booktabs}
\end{lstlisting}
  \item Fonctionnalités intégrées dans la classe \class{memoir}
  \end{itemize}
\end{frame}

\begin{frame}[fragile=singleslide]
  \frametitle{Exemple de tableau}

  \begin{center}
    \begin{tabular}{lcrr}
      \toprule
      Produit & Quantité & Prix unitaire (\$) & Prix (\$) \\
      \midrule
      Vis à bois    & 2 & 9,90 & 19,80 \\
      Clous vrillés & 5 & 4,35 & 21,75 \\
      \midrule
      TOTAL         & 7 &      & 41,55 \\
      \bottomrule
    \end{tabular}
  \end{center}

\begin{lstlisting}
\begin{tabular}{lcrr}
  \toprule
  Produit & Quantité & Prix unitaire (\$) & Prix (\$) \\
  \midrule
  Vis à bois    & 2 & 9,90 & 19,80 \\
  Clous vrillés & 5 & 4,35 & 21,75 \\
  \midrule
  TOTAL         & 7 &      & 41,55 \\
  \bottomrule
\end{tabular}
\end{lstlisting}
\end{frame}

%%% Local Variables:
%%% TeX-master: "formation-latex-ul-diapos"
%%% TeX-engine: xetex
%%% coding: utf-8
%%% End:

%%% Copyright (C) 2018 Vincent Goulet
%%%
%%% Ce fichier fait partie du projet
%%% «Rédaction avec LaTeX»
%%% http://github.com/vigou3/formation-latex-ul
%%%
%%% Cette création est mise à disposition selon le contrat
%%% Attribution-Partage dans les mêmes conditions 4.0
%%% International de Creative Commons.
%%% http://creativecommons.org/licenses/by-sa/4.0/

\section{Programmation lettrée}

\begin{frame}[fragile=singleslide]
  \frametitle{Document source combinant {\LaTeX} et code R}

  \fichier{fichier.Rnw}
\begin{lstlisting}[emph={Sexpr}]
...

L'utilisateur de R interagit avec l'interprète en entrant
des commandes à l'invite de commande:
<<echo=TRUE>>=
2 + 3
@
La commande \verb=exp(1)= donne \Sexpr{exp(1)},
la valeur du nombre $e$.

...
\end{lstlisting}
\end{frame}

\begin{frame}[fragile=singleslide]
  \frametitle{Après traitement par Sweave dans R}

  \texttt{Sweave("fichier.Rnw")}
  $\rightarrow$ \fichier{fichier.tex}
\begin{lstlisting}[emph={Schunk,Sinput,Soutput}]
...

L'utilisateur de R interagit avec l'interprète en entrant
des commandes à l'invite de commande:
\begin{Schunk}
\begin{Sinput}
> 2 + 3
\end{Sinput}
\begin{Soutput}
[1] 5
\end{Soutput}
\end{Schunk}
La commande \verb=exp(1)= donne 2.71828182845905,
la valeur du nombre $e$.

...
\end{lstlisting}

\end{frame}

%%% Local Variables:
%%% TeX-master: "formation-latex-ul-diapos"
%%% TeX-engine: xetex
%%% coding: utf-8
%%% End:

\include{ulthese-diapos}
\section{Et la suite?}

\begin{frame}
  \frametitle{Pour en savoir plus}

  \begin{columns}
    \begin{column}{.6\textwidth}
      Le document de référence fournit des détails additionnels et
      couvre des concepts plus avancés:
      \begin{itemize}
        \small
      \item boîtes, tableaux et figures
      \item équations mathématiques élaborées
      \item bibliographie et citations
      \item commandes et environnement sur mesure
      \item changement de police
      \item diapositives
      \item rapports avec analyse intégrée
      \item etc.
      \end{itemize}
    \end{column}
    \begin{column}{.4\textwidth}
      \includegraphics[height=0.8\textheight,frame]{formation-latex-ul}
    \end{column}
  \end{columns}
\end{frame}

%%% Local Variables:
%%% TeX-master: "formation-latex-ul-diapos"
%%% TeX-engine: xetex
%%% coding: utf-8
%%% End:


%%% Copyright (C) 2018 Vincent Goulet
%%%
%%% Ce fichier fait partie du projet
%%% «Rédaction avec LaTeX»
%%% http://github.com/vigou3/formation-latex-ul
%%%
%%% Cette création est mise à disposition selon le contrat
%%% Attribution-Partage dans les mêmes conditions 4.0
%%% International de Creative Commons.
%%% http://creativecommons.org/licenses/by-sa/4.0/

\begin{frame}[plain]
  \begin{adjustwidth}{20mm}{20mm}
    \scriptsize \raggedright %
    Ce document a été produit par le système de mise en page
    {\XeLaTeX} avec la classe \textbf{beamer} et le thème Metropolis.
    Les pages de titre sont composées en Myriad~Pro, le texte
    principal en Fira~Sans, les mathématiques en Arev~Math et le code
    informatique en Fira~Mono. Les icônes proviennent de la police
    Font~Awesome.
  \end{adjustwidth}
\end{frame}

%%% Local Variables:
%%% TeX-master: "formation-latex-ul-diapos"
%%% TeX-engine: xetex
%%% coding: utf-8
%%% End:

%%% Copyright (C) 2018 Vincent Goulet
%%%
%%% Ce fichier fait partie du projet
%%% «Rédaction avec LaTeX»
%%% http://github.com/vigou3/formation-latex-ul
%%%
%%% Cette création est mise à disposition selon le contrat
%%% Attribution-Partage dans les mêmes conditions 4.0
%%% International de Creative Commons.
%%% http://creativecommons.org/licenses/by-sa/4.0/

\begingroup

\TPGrid{3}{36}
\textblockorigin{0mm}{0mm}
\setlength{\parindent}{0mm}
\setlength{\banderougewidth}{2\TPHorizModule}
\setlength{\bandeorwidth}{\TPHorizModule}
\setlength{\gapwidth}{0.75pt}
\addtolength{\bandeorwidth}{-\gapwidth}

\begin{frame}[plain]
  %% bandeau identitaire
  \begin{textblock*}{125mm}[0,1](0mm,30\TPVertModule)
    \textcolor{or}{\rule{\bandeorwidth}{\TPVertModule}}%      % bande or
    \rule{\gapwidth}{0pt}%                                    % filet
    \textcolor{rouge}{\rule{\banderougewidth}{\TPVertModule}} % bande rouge
  \end{textblock*}
\end{frame}
\endgroup

%%% Local Variables:
%%% TeX-master: "formation-latex-ul-diapos"
%%% TeX-engine: xetex
%%% coding: utf-8
%%% End:


\end{document}

%%% Local Variables:
%%% TeX-master: t
%%% TeX-engine: xetex
%%% coding: utf-8
%%% End:
