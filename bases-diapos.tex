%%% Copyright (C) 2018 Vincent Goulet
%%%
%%% Ce fichier fait partie du projet
%%% «Rédaction avec LaTeX»
%%% http://github.com/vigou3/formation-latex-ul
%%%
%%% Cette création est mise à disposition selon le contrat
%%% Attribution-Partage dans les mêmes conditions 4.0
%%% International de Creative Commons.
%%% http://creativecommons.org/licenses/by-sa/4.0/

\section{Principes de base}

\subsection{Règles de saisie}

\begin{frame}[fragile=singleslide]
  \frametitle{Rédaction}

  L'apparence du document est prise en charge par {\LaTeX} et
  il est généralement préférable de ne pas la modifier.

  \begin{itemize}
  \item On se concentre sur le \alert{contenu} et la \alert{structure} du
    document
      \bigskip
      \begin{tabbing}
        titre de section \qquad\= \faArrowRight \qquad\= \verb|\section{titre}| \\[6pt]
        emphase \> \faArrowRight \> \verb|\emph{texte}|
      \end{tabbing}
      \bigskip
  \item Mots séparés par une ou plusieurs \alert{espaces}
  \item Paragraphes séparés par une ou plusieurs \alert{lignes blanches}
  \item Utilisation de \alert{commandes} pour indiquer la structure du texte
  \end{itemize}
\end{frame}

\begin{frame}[fragile=singleslide]
  \frametitle{Caractères réservés}

  \begin{itemize}
  \item Caractères réservés par {\TeX}:
    \begin{quote}
      \verb=# $ & ~ _ ^ % { }=
    \end{quote}
  \item Pour les utiliser, précéder par \bs
  \item On écrira donc
    \begin{demo}
      \begin{texample}
\begin{lstlisting}
L'augmentation de 2~\$
représente une hausse
de 5~\%.
\end{lstlisting}
        \producing
        L'augmentation de 2~\$ représente une
        hausse de 5~\%.
      \end{texample}
    \end{demo}
  \end{itemize}
\end{frame}

\subsection{Structure d'un fichier}

\begin{frame}[fragile]
  \frametitle{Structure d'un document {\LaTeX}}

  Un fichier source {\LaTeX} est toujours composé de deux parties.

  \hfill
  \begin{minipage}{0.75\linewidth}
\begin{lstlisting}[emph={documentclass,begin,end,document}]
\documentclass[11pt,french]{article}
  \usepackage{babel}
  \usepackage[autolanguage]{numprint}
  \usepackage[utf8]{inputenc}
  \usepackage[T1]{fontenc}

\begin{document}

Lorem ipsum dolor sit amet, consectetur
adipiscing elit. Donec quam nulla, bibendum
vitae ipsum vel, fermentum pellentesque orci.

\end{document}
\end{lstlisting}
  \end{minipage}

  \begin{textblock*}{\linewidth}(0mm,0mm)
    \begin{picture}(0,0)
      \thicklines\color{blue}
      \onslide<2>{\put(120,-138){\dashbox{2}(285,76){}}}
      \onslide<3>{\put(120,-248){\dashbox{2}(285,104){}}}
      \onslide<2>{\put(46,-103){préambule}}
      \onslide<3>{\put(46,-198){\parbox{25mm}{corps du\\ document}}}
    \end{picture}
  \end{textblock*}
\end{frame}

\subsection{Commandes et environnements}

\begin{frame}[fragile=singleslide]
  \frametitle{Commandes}
  \begin{itemize}
  \item Débutent toujours par \bs
  \item Formes générales:
\begin{lstlisting}
\`\meta{nomcommande}\oarg{arg\_optionnel}\marg{arg\_obligatoire}'
\`\meta{nomcommande}'*`\oarg{arg\_optionnel}\marg{arg\_obligatoire}'
\end{lstlisting}
   \item Commande sans argument: le nom se termine par tout
    caractère qui n'est pas une lettre (y compris l'espace!)
  \item Portée d'une commande limitée à la zone entre \verb={ }=
  \end{itemize}
\end{frame}

\begin{frame}[fragile=singleslide]
  \frametitle{Environnements}
  \begin{itemize}
  \item Délimités par
\begin{lstlisting}
\begin`\marg{environnement}'
   ...
\end`\marg{environnement}'
    \end{lstlisting}
  \item Contenu de l'environnement traité différemment du reste du texte
  \item Changements s'appliquent uniquement à l'intérieur de
    l'environnement
  \end{itemize}
\end{frame}

\subsection{[~Exercice~]}

\begin{exercice}
  Modifier le fichier \fichier{exercice\_commandes.tex} afin de
  produire le texte ci-dessous.

  \bigskip
  \centering
  \fbox{\includegraphics[viewport=108 551 502 665,%
    clip=true,width=0.9\linewidth]{exercice_commandes-solution}}
\end{exercice}

\begin{frame}[fragile]
  \frametitle{{\LaTeX} en français}

  Il faut charger un certain nombre de paquetages pour franciser \LaTeX.

  \begin{itemize}
  \item \pkg{babel}: traduction des mots-clés prédéfinis,
    typographie française, coupure de mots, document multilingue
  \item \pkg{inputenc} et \pkg{fontenc}: lettres accentuées dans le
    code source (pdf{\LaTeX} seulement)
  \item \pkg{icomma}: virgule comme séparateur décimal
  \item \pkg{numprint}: espace comme séparateur des milliers
  \end{itemize}
\end{frame}

%%% Local Variables:
%%% TeX-master: "formation-latex-ul-diapos"
%%% TeX-engine: xetex
%%% coding: utf-8
%%% End:
