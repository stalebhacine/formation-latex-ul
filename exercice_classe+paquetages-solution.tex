\documentclass[12pt]{memoir}
  \usepackage[francais]{babel}
  \usepackage[autolanguage]{numprint}
  \usepackage{icomma}
  %% Activer les lignes appropriées selon le moteur utilisé
  \usepackage[utf8]{inputenc}   % LaTeX
  \usepackage[T1]{fontenc}      % LaTeX
  % \usepackage{fontspec}         % XeLaTeX

\begin{document}

La longueur des lignes en {\LaTeX} est ajustée automatiquement en
fonction de la taille de la police de caractères de manière à ce que le
nombre de caractères par ligne demeure à peu près constant. Cela a
pour but d'améliorer la lisibilité: lorsqu'une ligne de texte est trop
longue, notre œil a plus de difficulté à suivre celle-ci de gauche à
droite.

En passant à une classe de document recto-verso, vous remarquerez que
les marges gauche et droite ne sont pas identiques. C'est afin de
tenir compte de la marge de reliure.

Le paquetage \textbf{babel} fournit les commandes \verb=\ier=,
\verb=\iere=, \verb=\ieme= pour écrire «premier», «première»,
«deuxième» en chiffres: 1{\ier}, 1{\iere}, 36{\ieme}.

En typographie française, on doit utiliser la virgule comme séparateur
décimal et l'espace fine comme séparateur des milliers.
\begin{displaymath}
  y = 1,2x + 5, \quad x = 0, 1, \dots, \nombre{10 000}.
\end{displaymath}

\end{document}
