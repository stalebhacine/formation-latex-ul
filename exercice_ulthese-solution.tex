\documentclass[PhD,english,french]{ulthese}
  %% Encodage utilisé pour les caractères accentués dans les fichiers
  %% source du document. Les gabarits sont encodés en UTF-8. Inutile avec
  %% XeLaTeX, qui gère Unicode nativement.
  \ifxetex\else \usepackage[utf8]{inputenc} \fi

  %% Charger ici les autres paquetages nécessaires pour le document.
  %% Quelques exemples; décommenter au besoin.
  \usepackage{amsmath}          % recommandé pour les mathématiques
  \usepackage{icomma}           % gestion de la virgule dans les nombres

  %% Utilisation d'une autre police de caractères pour le document.
  %% - Sous LaTeX
  \usepackage{mathpazo}         % texte et mathématiques en Palatino
  %\usepackage{mathptmx}         % texte et mathématiques en Times
  %% - Sous XeLaTeX
  %\setmainfont{TeX Gyre Pagella}      % texte en Pagella (Palatino)
  %\setmathfont{TeX Gyre Pagella Math} % mathématiques en Pagella (Palatino)
  %\setmainfont{TeX Gyre Termes}       % texte en Termes (Times)
  %\setmathfont{TeX Gyre Termes Math}  % mathématiques en Termes (Times)

  %% Gestion des hyperliens dans le document. S'assurer que hyperref
  %% est le dernier paquetage chargé.
  \usepackage{hyperref}
  \hypersetup{colorlinks,allcolors=ULlinkcolor}

  %% Options de mise en forme du mode français de babel. Consulter la
  %% documentation du paquetage babel pour les options disponibles.
  %% Désactiver (effacer ou mettre en commentaire) si l'option
  %% 'nobabel' est spécifiée au chargement de la classe.
  \frenchbsetup{%
    CompactItemize=false,         % ne pas compacter les listes
    ThinSpaceInFrenchNumbers=true % espace fine dans les nombres
  }

  %% Déclarations de la page titre. Remplacer les éléments entre < >.
  %% Supprimer les caractères < >. Couper un long titre ou un long
  %% sous-titre manuellement avec \\.
  \titre{<Titre principal>}
  % \titre{Ceci est un exemple de long titre \\
  %   avec saut de ligne manuel}
  % \soustitre{Sous-titre le cas échéant}
  % \soustitre{Ceci est un exemple de long sous-titre \\
  %   avec saut de ligne manuel}
  \auteur{<Prénom Nom>}
  \programme{Doctorat en <discipline> -- <majeure, s'il y a lieu>}
  \annee{<20xx>}

\begin{document}

\frontmatter                    % pages liminaires

\pagetitre                      % production de la page titre

\tableofcontents                % production de la TdM
\cleardoublepage

\listoftables                   % production de la liste des tableaux
\cleardoublepage

\listoffigures                  % production de la liste des figures
\cleardoublepage

\mainmatter                     % corps du document

\chapter{Mathématiques}
\label{chap:math}


S'il est un domaine où {\LaTeX} brille particulièrement, c'est bien
dans la préparation et la présentation d'équations mathématiques ---
des plus simples aux plus complexes. Après tout, l'amélioration de la
qualité typographique des équations mathématiques dans son ouvrage
phare \emph{The Art of Computer Programming} figurait parmi les
objectifs premiers de Knuth lorsqu'il a développé {\TeX}.

La première partie de cette formation aborde le sujet des
mathématiques, mais en s'en tenant qu'aux principes de base
\citep[section~7]{UL:latex:1}. [...]

Trop de symboles, consulter \citet{comprehensive} %
\doc[???]{comprehensive}{http://texdoc.net/pkg/comprehensive}


\section{Rappel des principes de base}
\label{sec:math:rappel}

La mise en forme d'équations mathématiques requiert d'indiquer à
l'ordinateur, dans un langage spécial, le contenu des dites équations
et la position des symboles: en exposant, en indice, sous forme de
fraction, etc. L'ordinateur peut ensuite assembler le tout à partir de
règles typographiques portant, par exemple, sur la représentation des
variables et des constantes, l'espacement entre les symboles ou la
disposition des équations selon qu'elles apparaissent au fil du texte
ou hors d'un paragraphe.

On indique à {\LaTeX} que l'on change de «langage», par l'utilisation
d'un mode mathématique. Il y a deux grandes manière d'activer le mode
mathématique:
\begin{enumerate}
\item en insérant le code entre les symboles \verb=$ $= pour générer
  une équation «en ligne», ou au fil du texte;
  \begin{demo}
    \begin{texample}
\begin{lstlisting}
on sait que $(a + b)^2 =
a^2 + 2ab + b^2$, d'où
on obtient...
\end{lstlisting}
      \producing
      on sait que $(a + b)^2 = a^2 + 2ab + b^2,$ d'où on obtient...
    \end{texample}
  \end{demo}
\item en utilisant un environnement servant à créer une équation hors
  paragraphe;
  \begin{demo}
    \begin{texample}
\begin{lstlisting}
on sait que
\begin{equation*}
  (a + b)^2
  = a^2 + 2ab + b^2,
\end{equation*}
d'où on obtient...
\end{lstlisting}
      \producing
      on sait que
      \begin{equation*}
        (a + b)^2 = a^2 + 2ab + b^2,
      \end{equation*}
      d'où on obtient...
    \end{texample}
  \end{demo}
\end{enumerate}

Dans l'exemple ci-dessus, l'environnement \Ie{equation*} crée une
équation hors paragraphe, centrée sur la ligne et non numérotée.
Avec l'environnement \Ie{equation} (donc sans \verb=*= dans le nom),
{\LaTeX} ajoute automatiquement un numéro d'équation séquentiel aligné
sur la marge de droite:
\begin{demo}
  \begin{texample}
\begin{lstlisting}
on sait que
\begin{equation}
  (a + b)^2 = a^2 + b^2,
\end{equation}
d'où on obtient...
\end{lstlisting}
    \producing
    on sait que
    \begin{equation}
      (a + b)^2 = a^2 + b^2,
    \end{equation}
    d'où on obtient...
  \end{texample}
\end{demo}
Cette disposition est la plus usuelle dans les ouvrages mathématiques.
Le type de numérotation diffère selon qu'un document comporte des
chapitres ou non.

En mode mathématique, les chiffres sont automatiquement considérés
comme des constantes, les lettres comme des variables et une suite de
lettres comme un produit de variables (nous verrons plus loin comment
représenter des fonctions mathématiques comme $\sin$, $\log$ ou
$\lim$). Ceci a trois conséquences principales:
\begin{enumerate}
\item conformément aux conventions typographiques, les chiffres sont
  représentés en caractère \textrm{romain} et les variables en
  \emph{italique};
  \begin{demo}
    \begin{texample}
\begin{lstlisting}
123xyz
\end{lstlisting}
      \producing
      $123xyz$
    \end{texample}
  \end{demo}
\item l'espace entre les constantes, les variables et les opérateurs
  mathématiques est géré automatiquement;
  \begin{demo}
    \begin{texample}
\begin{lstlisting}
z = 2 x + 3 x y
\end{lstlisting}
    \producing
    $z = 2 x + 3 x y$
    \end{texample}
  \end{demo}
\item les espaces dans le code source n'ont aucun impact sur la
  disposition d'une équation.
  \begin{demo}
    \begin{texample}
\begin{lstlisting}
z=2x + 3xy
\end{lstlisting}
    \producing
    $z=2x + 3xy$
    \end{texample}
  \end{demo}
\end{enumerate}


% Pensez simplement
% à ce que pourrait être en mots la représentation de l'équation
% \begin{equation*}
%   \int_0^\infty \int_0^1 x_1^{\alpha + 1} x_2^{\beta + 1}\, dx_1 dx_2.
% \end{equation*}
% Cela ressemblerait sans doute à ceci (avec entre parenthèses les mots
% habituellement omis et en gras les symboles qu'il faut pouvoir décrire
% autrement que part un caractère disponible sur le clavier):
% \begin{quote}
%   \textbf{intégrale} de (indice) zéro à (exposant) l'\textbf{infini} \\
%   \textbf{intégrale} de (indice) zéro à (exposant) à un \\
%   $x$ (indice) un, puissance \textbf{alpha} plus un \\
%   $x$ (indice) deux, puissance \textbf{beta} plus un \\
%   $d$ $x$ (indice) un \\
%   $d$ $x$ (indice) deux.
% \end{quote}

Quant au langage retenu par {\LaTeX} pour décrire les équations
mathématiques, il est très similaire à celui que l'on utiliserait pour
le faire à voix haute (??? capsule vidéo???); nous y reviendrons. Il faut simplement avoir
recours à des commandes pour identifier les symboles mathématiques que
l'on ne retrouve pas sur un clavier usuel, comme les lettres grecques,
les opérateurs d'inégalité ou les symboles de sommes et d'intégrales.


\section{Un paquetage incontournable}
\label{sec:math:amsmath}

Le paquetage \pkg{amsmath} produit par la prestigieuse \emph{American
  Mathematical Society} fournit diverses extensions à {\LaTeX} pour
faciliter encore davantage la saisie d'équations mathématiques
complexes et en améliorer le rendu. L'utilisation de ce paquetage doit
être considérée incontournable pour tout document contenant plus que
quelques équations très simples.

Au chapitre des améliorations fournies par \pkg{amsmath}, notons
particulièrement:
\begin{itemize}
\item plusieurs environnements pour les équations hors paragraphe, en
  particulier pour les équations multi-lignes;
\item une meilleure gestion de l'espacement autour des signes
  d'égalité dans les équations multi-lignes;
\item une commande pour faciliter l'entrée de texte à l'intérieur du
  mode mathématique;
\item un environnement pour la saisie des matrices et des coefficients
  binomiaux;
\item des commandes pour les intégrales multiples;
\item la possibilité de définir de nouveaux opérateurs mathématiques.
\end{itemize}
Nous décrivons certaines de ces fonctionnalités dans la suite, mais
l'utilisateur le moindrement avancé devrait impérativement consulter
la %
\doc[documentation complète]{amsldoc}{http://texdoc.net/pkg/amsmath}
du paquetage.


\section{Principaux éléments du mode mathématique}
\label{sec:math:bases}

\subsection{Exposants et indices}
\label{sec:math:bases:exposants}

{\LaTeX} permet de créer facilement et avec la bonne taille de
symboles n'importe quelle combinaison d'exposants et d'indices.

\begin{itemize}
\item On place un caractère en exposant d'un autre avec la commande
  \verb=^= et en indice avec la commande \verb=_=. Les indices et
  exposants se combinent naturellement.
  \begin{demo}
    \begin{minipage}{0.3\linewidth}
      \begin{texample}[0.6\linewidth]
\begin{lstlisting}
x^2
\end{lstlisting}
        \producing $x^2$
      \end{texample}
    \end{minipage}
    \quad
    \begin{minipage}{0.3\linewidth}
      \begin{texample}[0.6\linewidth]
\begin{lstlisting}
a_n
\end{lstlisting}
        \producing
        $a_n$
      \end{texample}
    \end{minipage}
    \quad
    \begin{minipage}{0.3\linewidth}
      \begin{texample}[0.63\linewidth]
\begin{lstlisting}
x_i^k
\end{lstlisting}
        \producing
        $x_i^k$
      \end{texample}
    \end{minipage}
  \end{demo}
  (L'ordre de saisie n'a pas d'importance; le troisième exemple
  donnerait le même résultat avec \verb=x^k_i=.)
\item Si l'exposant ou l'indice compte plus d'un caractère, il faut
  regrouper le tout entre accolades \verb={ }=.
  \begin{demo}
    \begin{minipage}{0.3\linewidth}
      \begin{texample}[0.6\linewidth]
\begin{lstlisting}
x^{2k + 1}
\end{lstlisting}
        \producing
        $x^{2k + 1}$
      \end{texample}
    \end{minipage}
    \quad
    \begin{minipage}{0.3\linewidth}
      \begin{texample}[0.6\linewidth]
\begin{lstlisting}
x_{i,j}
\end{lstlisting}
        \producing
        $x_{i,j}$
      \end{texample}
    \end{minipage}
    \quad
    \begin{minipage}{0.3\linewidth}
      \begin{texample}[0.63\linewidth]
\begin{lstlisting}
x_{ij}^{2n}
\end{lstlisting}
        \producing
        $x_{ij}^{2n}$
      \end{texample}
    \end{minipage}
  \end{demo}
\item Toutes les combinaisons d'exposants et d'indices sont possibles,
  y compris les puissances de puissances ou les indices d'indices.
  \begin{demo}
    \begin{minipage}{0.3\linewidth}
      \begin{texample}[0.6\linewidth]
\begin{lstlisting}
e^{-x^2}
\end{lstlisting}
        \producing
        $e^{-x^2}$
      \end{texample}
    \end{minipage}
    \quad
    \begin{minipage}{0.6\linewidth}
      \begin{texample}[0.6\linewidth]
\begin{lstlisting}
A_{i_s, k^n}^{y_i}
\end{lstlisting}
        \producing
        $A_{i_s,k^n}^{y_i}$
      \end{texample}
    \end{minipage}
    \quad \mbox{}     % pour alignement avec bloc d'exemples ci-dessus
  \end{demo}
\end{itemize}

\begin{important}
  Les commandes \verb=^= et \verb=_= sont permises dans le mode
  mathématique seulement. En fait, si {\TeX} rencontre l'une de ces
  commandes en mode texte, il tentera automatiquement de passer au
  mode mathématique après avoir émis l'avertissement
\begin{verbatim}
! Missing $ inserted.
\end{verbatim}
  Il est assez rare que le résultat soit celui souhaité.
\end{important}

\subsection{Fractions}
\label{sec:math:bases:fractions}

Il y a plusieurs façons de représenter une fraction selon qu'elle se
trouve au fil du texte, dans une équation hors paragraphe ou à
l'intérieur d'une autre fraction.
\begin{itemize}
\item Pour les fractions au fil du texte, il vaut souvent mieux
  utiliser simplement la barre oblique \verb=/= pour séparer le
  numérateur du dénominateur, quitte à utiliser des parenthèses.
  Ainsi, on utilise \verb=$(n + 1)/2$= pour obtenir $(n + 1)/2$.
\item De manière plus générale, la commande
\begin{lstlisting}
\frac`\marg{numérateur}\marg{dénominateur}'
\end{lstlisting}
  dispose \meta{numérateur} au-dessus de \meta{dénominateur}, séparé
  par une ligne horizontale. La taille des caractères s'ajuste
  automatiquement selon que la fraction se trouve au fil
  du texte ou dans une équation hors paragraphe, ainsi que selon la
  position de la fraction dans l'équation.
  \begin{demo}
    \begin{texample}
\begin{lstlisting}
% taille au fil du texte
On a $z_1 = \frac{x}{y}$ et
$z_2 = xy$.
\end{lstlisting}
      \producing
      On a $z_1 = \frac{x}{y}$ et $z_2 = xy$.
    \end{texample}

    \begin{texample}
\begin{lstlisting}
% taille hors paragraphe
On a
\begin{equation*}
  z_1 = \frac{x}{y}
\end{equation*}
et $z_2 = xy$.
\end{lstlisting}
      \producing
      On a
      \begin{equation*}
        z_1 = \frac{x}{y}
      \end{equation*}
      et $z_2 = xy$.
    \end{texample}

    \begin{texample}
\begin{lstlisting}
% deux tailles combinées
Soit
\begin{equation*}
  z = \frac{\frac{x}{2}
    + 1}{y}.
\end{equation*}
\end{lstlisting}
      \producing
      Soit
      \begin{equation*}
        z = \frac{\frac{x}{2} + 1}{y}.
      \end{equation*}
    \end{texample}
  \end{demo}

\item Les commandes
\begin{lstlisting}
\dfrac`\marg{numérateur}\marg{dénominateur}'
\tfrac`\marg{numérateur}\marg{dénominateur}'
\end{lstlisting}
  de \pkg{amsmath} permettent de forcer une fraction à adopter la
  taille d'une fraction hors paragraphe (\emph{displayed}) dans le cas
  de \cmd{\dfrac} ou de celle d'une fraction au fil du texte
  (\emph{text}) dans le cas de \cmd{\tfrac}.
  %% quelque chose à cet effet dans exercices ou dans présentation de 'cases'
\item Il est parfois visuellement plus intéressant, surtout au fil du
  texte, d'écrire une fraction comme $1/x$ sous la forme $x^{-1}$.
\end{itemize}

\subsection{Racines}
\label{sec:math:bases:racines}

La commande
\begin{lstlisting}
\sqrt`\oarg{n}\marg{radicande}'
\end{lstlisting}
construit un symbole de radical autour de \meta{radicande}, par défaut
la racine carrée. Si l'argument optionnel \meta{n} est spécifié, c'est
plutôt un symbole de racine d'ordre $n$ qui est tracé. La longueur et
la hauteur du radical s'adapte toujours à celles du radicande.
\begin{demo}
  \begin{minipage}{0.3\linewidth}
    \begin{texample}[0.6\linewidth]
\begin{lstlisting}
\sqrt{2}
\end{lstlisting}
      \producing $\sqrt{2}$
    \end{texample}
  \end{minipage}
  \quad
  \begin{minipage}{0.3\linewidth}
    \begin{texample}[0.6\linewidth]
\begin{lstlisting}
\sqrt{625}
\end{lstlisting}
      \producing
      $\sqrt{625}$
    \end{texample}
  \end{minipage}
  \quad
  \begin{minipage}{0.3\linewidth}
    \begin{texample}[0.6\linewidth]
\begin{lstlisting}
\sqrt[3]{8}
\end{lstlisting}
      \producing
      $\sqrt[3]{8}$
    \end{texample}
  \end{minipage}

  \begin{texample}
\begin{lstlisting}
\sqrt[n]{x + y + z}
\end{lstlisting}
    \producing
    $\sqrt[n]{x + y + z}$
  \end{texample}

  \begin{texample}
\begin{lstlisting}
\sqrt{\frac{x + y}{x^2 - y^2}}
\end{lstlisting}
    \producing
    $\displaystyle \sqrt{\frac{x + y}{x^2 - y^2}}$
  \end{texample}
\end{demo}

\subsection{Sommes et intégrales}
\label{sec:math:bases:sommes-et-integrales}

Les sommes et intégrales requièrent un symbole spécial ainsi que des
limites inférieures et supérieures, le cas échéant.
\begin{itemize}
\item Les commandes \cmd{\sum} et \cmd{\int} servent respectivement à tracer les
  symboles de somme $\sum$ et d'intégrale $\int$.
\item Le paquetage \pkg{amsmath} fournit également des commandes comme
  \cmd{\iint} et \cmd{\iiint} pour obtenir des symboles d'intégrales
  multiples finement disposés ($\iint$ et $\iiint$).
\item On entre les éventuelles limites inférieures et supérieures
  comme des indices et des exposants.
  \begin{demo}
    \begin{texample}
\begin{lstlisting}
\sum_{i = 0}^n x_i
\end{lstlisting}
      \producing
      $\displaystyle \sum_{i = 0}^n x_i$
    \end{texample}

    \begin{texample}
\begin{lstlisting}
\int_0^{10} f(x)\, dx
\end{lstlisting}
      \producing
      $\displaystyle \int_0^{10} f(x)\, dx$
    \end{texample}

    \begin{texample}
\begin{lstlisting}
\iint_D f(x, y)\, dx dy
\end{lstlisting}
      \producing
      $\displaystyle \iint_D f(x, y)\, dx dy$
    \end{texample}
  \end{demo}
\item La taille des symboles et la position des limites s'ajustent
  automatiquement selon le contexte. Au fil du texte, la somme et
  l'intégrale simple ci-dessus apparaîtraient comme
  $\sum_{i = 0}^n x_i$ et $\int_0^{10} f(x)\, dx$.
\item Dans une intégrale il est recommandé de séparer l'intégrande de
  l'opérateur de différentiation $dx$ par une espace fine. C'est ce à
  quoi sert la commande \cmd{\,} ci-dessus; voir aussi le tableau de
  la \autopageref{tab:math:espaces}.
\end{itemize}

\subsection{Points de suspension}
\label{sec:math:bases:dots}

Les formules mathématiques comportent fréquemment des points de
suspension dans des suites de variables ou d'opérations. Il faut
éviter de les entrer comme trois points finaux consécutifs, car
l'espacement entre ceux-ci sera trop petit et le résultat,
disgracieux: $...$

\begin{itemize}
\item Le \autoref{tab:math:dots} fournit les commandes {\LaTeX}
  servant à générer divers types de points de suspension.
  \begin{table}
    \caption{Points de suspension}
    \label{tab:math:dots}
    \centering
    \begin{tabular}{lll}
      \toprule
      commande & type de points & exemple \\
      \midrule
      \cmd{\dots} &  sélection automatique \\
      \cmd{\ldots} & points à la ligne de base & $x_1, \ldots, x_n$ \\
      \cmd{\cdots} & points centrés & $x_1 + \cdots + x_n$ \\
      \cmd{\vdots} & points verticaux & $
                                        \begin{matrix}
                                          x_1 \\ \vdots \\ x_n
                                        \end{matrix}$ \\
      \cmd{\ddots} & points diagonaux & $
                                        \begin{matrix}
                                          x_1 &&\\ &\ddots& \\ && x_n
                                        \end{matrix}$ \\
      \bottomrule
    \end{tabular}
  \end{table}
\item Avec \pkg{amsmath}, la commande \cmd{\dots} tâche de
  sélectionner automatiquement entre les points à la ligne de base ou
  les points centrés selon le contexte. Comme le résultat est en
  général le bon, nous recommandons d'utiliser principalement cette
  commande pour insérer des points de suspension en mode mathématique.
  \begin{demo}
    \begin{texample}
\begin{lstlisting}
$x_1, \dots, x_n$
\end{lstlisting}
      \producing
      $x_1, \dots, x_n$
    \end{texample}

    \begin{texample}
\begin{lstlisting}
$x_1, \ldots, x_n$
\end{lstlisting}
      \producing
      $x_1, \ldots, x_n$
    \end{texample}

    \begin{texample}
\begin{lstlisting}
$x_1 + \dots + x_n$
\end{lstlisting}
      \producing
      $x_1 + \dots + x_n$
    \end{texample}

    \begin{texample}
\begin{lstlisting}
$x_1 + \cdots + x_n$
\end{lstlisting}
      \producing
      $x_1 + \cdots + x_n$
    \end{texample}
  \end{demo}
\item Le paquetage \pkg{amsmath} définit également les commandes sémantiques
  \begin{itemize}
  \item \cmd{\dotsc} pour des «points avec des virgules» (\emph{commas});
  \item \cmd{\dotsb} pour des «points avec des opérateurs binaires»;
  \item \cmd{\dotsm} pour des «points de multiplication»;
  \item \cmd{\dotsi} pour des «points avec des intégrales»;
  \item \cmd{\dotso} pour des «autres points» (\emph{other}).
  \end{itemize}
\end{itemize}

\subsection{Texte et espaces}
\label{sec:math:bases:texte}

On l'a vu, en mode mathématique {\LaTeX} traite les lettres comme des
variables et gère automatiquement l'espacement entre les divers
symboles. Or, il n'est pas rare que des formules mathématiques
contiennent du texte (notamment des mots comme «où», «si», «quand»).
De plus, il est parfois souhaitable de pouvoir ajuster les blancs
entre des éléments.
\begin{itemize}
\item La commande de \pkg{amsmath}
\begin{lstlisting}
\text`\marg{texte}'
\end{lstlisting}
  insère \meta{texte} dans une formule mathématique. Le texte est
  inséré tel quel, sans aucune gestion des espaces avant ou après le
  texte. Si des espaces sont nécessaires, ils doivent faire partie de
  \meta{texte}.
  \begin{demo}
    \begin{texample}
\begin{lstlisting}
f(x) = a e^{-ax}
\text{ pour } x > 0
\end{lstlisting}
      \producing
      $f(x) = a e^{-ax} \text{ pour } x > 0$
    \end{texample}
  \end{demo}
\item La commande \cmd{\quad} insère un blanc de $1$~em, soit une
  longueur égale à la taille de la police de caractère courante en
  points (donc $11$~points pour une police de $11$~points). La
  commande \cmd{\qquad} insère le double de cette longueur.\footnote{%
    Bien qu'elles soient surtout utilisées dans le mode mathématique,
    les commandes \cmd{\quad} et \cmd{\qquad} sont également valides
    dans le mode texte.}
  \begin{demo}
    \begin{texample}
\begin{lstlisting}
f(x) = a e^{-ax},
\quad x > 0
\end{lstlisting}
      \producing
      $f(x) = a e^{-ax}, \quad x > 0$
    \end{texample}
  \end{demo}
\item Le \autoref{tab:math:espaces} répertorie et compare les
  différentes commandes qui permettent d'insérer des espaces plus ou
  moins fines entre des éléments dans le mode mathématique:
  \begin{table}
    \caption{Espaces dans le mode mathématique}
    \label{tab:math:espaces}
    \centering
    \begin{tabular}{lll}
      \toprule
      commande & longueur & exemple \\
      \midrule
               & pas d'espace & \spx{} \\
      \cmd{\,} & $3/18$ de \cmdprint{quad} & \spx{\,} \\
      \cmd{\:} & $4/18$ de \cmdprint{quad} & \spx{\:} \\
      \cmd{\;} & $5/18$ de \cmdprint{quad} & \spx{\;} \\
      \cmd{\!} & $-3/18$ de \cmdprint{quad} & \spx{\!} \\
      \cmd{\quad} & $1$~em & \spx{\quad} \\
      \cmd{\qquad} & $2$~em & \spx{\qquad} \\
      \bottomrule
    \end{tabular}
  \end{table}
\end{itemize}

\subsection{Fonctions et opérateurs}
\label{sec:math:bases:fonctions}

Les règles de typographie des équations mathématiques veulent que les
variables apparaissent en \textit{italique}, mais que les noms de
fonctions, eux, apparaissent en \textrm{romain}, comme le texte
standard. Pensons, ici, à des fonctions comme $\sin$ ou $\log$.

\begin{itemize}
\item On sait que l'on ne peut entrer le nom d'une fonction tel quel en mode
  mathématique, car {\LaTeX} interprétera la suite de lettres comme un
  produit de variables:
  \begin{demo}
    \begin{minipage}{0.45\linewidth}
      \begin{texample}
\begin{lstlisting}
$2 sin 30$
\end{lstlisting}
        \producing
        $2 sin 30$
      \end{texample}
    \end{minipage}
    \hfill
    \begin{minipage}{0.45\linewidth}
      \begin{texample}
\begin{lstlisting}
$3 log 2$
\end{lstlisting}
        \producing
        $3 log 2$
      \end{texample}
    \end{minipage}
  \end{demo}
  Utiliser à répétition la commande \cmdprint{text} se révèle peu
  pratique à l'usage.
\item {\LaTeX} définit donc des commandes pour un grand nombre de
  fonctions et d'opérateurs mathématiques standards:
\begin{lstlisting}
\arccos   \cosh    \det    \inf      \limsup   \Pr      \tan
\arcsin   \cot     \dim    \ker      \ln       \sec     \tanh
\arctan   \coth    \exp    \lg       \log      \sin
\arg      \csc     \gcd    \lim      \max      \sinh
\cos      \deg     \hom    \liminf   \min      \sup
\end{lstlisting}
  L'espacement autour des opérateurs mathématiques est également géré
  par {\LaTeX}.
  \begin{demo}
    \begin{minipage}{0.45\linewidth}
      \begin{texample}
\begin{lstlisting}
$2 \sin30$
\end{lstlisting}
        \producing
        $2 \sin30$
      \end{texample}
    \end{minipage}
    \hfill
    \begin{minipage}{0.45\linewidth}
      \begin{texample}
\begin{lstlisting}
$3 \log 2$
\end{lstlisting}
        \producing
        $3 \log 2$
      \end{texample}
    \end{minipage}
  \end{demo}
\item Certaines des fonctions ci-dessus, notamment \cmd{\lim}, acceptent
  des limites comme les symboles de somme et d'intégrale.
  \begin{demo}
    \begin{texample}
\begin{lstlisting}
% au fil du texte
\lim_{x \to 0} x = 0
\end{lstlisting}
      \producing
      $\lim_{x \to 0} x = 0$
    \end{texample}

    \begin{texample}
\begin{lstlisting}
% hors paragraphe
\lim_{x \to 0} x = 0
\end{lstlisting}
      \producing
      $\displaystyle \lim_{x \to 0} x = 0$
    \end{texample}
  \end{demo}
\item La commande \cmd{\DeclareMathOperator} de \pkg{amsmath} permet
  de définir de nouveaux noms de fonctions; consulter la %
  \doc[documentation]{amsldoc}{http://texdoc.net/pkg/amsmath} %
  du paquetage (section 5.1) pour les détails.
\end{itemize}



\section{Symboles mathématiques}
\label{sec:math:symboles}

Outre les chiffres et les lettres de l'alphabet, les claviers
d'ordinateurs comptent somme toute très peu des innombrables symboles
mathématiques utilisés couramment. Pour les représenter dans {\LaTeX},
on aura recours à des commandes qui débutent, comme d'habitude, par le
symbole \verb=\= et dont le nom est habituellement dérivé de la
signification mathématique du symbole.

Si un symbole mathématique a été utilisé quelque part dans une
publication, il y a de fortes chances que sa version existe dans
{\LaTeX}. Il serait donc utopique de tenter de faire ici une recension
des symboles disponibles. Nous nous contenterons d'un avant goût des
principales catégories.

L'ouvrage de référence pour connaître les symboles disponibles dans
{\LaTeX} est la bien nommée %
\doc[Comprehensive {\LaTeX} Symbol List]{compre\-hen\-sive}{http://texdoc.net/pkg/comprehensive}. %
Dans l'édition de 2009, la liste comprenait pas moins de \nombre{5913}
symboles répartis sur plus de 160 pages! On y trouve de tout, des
symboles mathématiques aux pictogrammes de météo ou d'échecs, en
passant par\dots\ des figurines des Simpsons.

\begin{important}
  Les moteurs {\XeTeX} et {\LuaTeX} supportent nativement le code
  source en format Unicode UTF-8 \citep{Unicode:5.0}. Ce standard
  contient des définitions pour plusieurs symboles mathématiques
  \citep{wikipedia:unicode-math}. Cela signifie qu'il est possible
  d'entrer une partie au moins des équations mathématiques avec des
  caractères visibles à l'écran, plutôt qu'avec des commandes {\LaTeX}.
  Nous ne saurions toutefois recommander cette pratique qui rend les
  fichiers source moins compatibles d'un système à un autre.
\end{important}

\subsection{Lettres grecques}
\label{sec:math:symboles:grecques}

On obtient les lettres grecques en {\LaTeX} avec des commandes
correspondant au nom de chaque lettre. Lorsque la commande débute par
une capitale, on obtient une lettre grecque majuscule. Certaines
lettres grecques majuscules n'existent pas car elles sont identiques
aux lettres romaines.

Les tableaux \ref{tab:math:grecques} et \ref{tab:math:Grecques}
présentent l'ensemble des lettres grecques disponibles dans {\LaTeX}.

\begin{table}
  \caption{Lettres grecques minuscules}
  \label{tab:math:grecques}
  \begin{tabularx}{1.0\linewidth}{lXlXlXlX}
    $\alpha$      & \cmd{\alpha}      & $\theta$    & \cmd{\theta} &
    $o$           & o                 & $\tau$      & \cmd{\tau} \\
    $\beta$       & \cmd{\beta}       & $\vartheta$ & \cmd{\vartheta} &
    $\pi$         & \cmd{\pi}         & $\upsilon$  & \cmd{\upsilon} \\
    $\gamma$      & \cmd{\gamma}      & $\iota$     & \cmd{\iota} &
    $\varpi$      & \cmd{\varpi}      & $\phi$      & \cmd{\phi} \\
    $\delta$      & \cmd{\delta}      & $\kappa$    & \cmd{\kappa} &
    $\rho$        & \cmd{\rho}        & $\varphi$   & \cmd{\varphi} \\
    $\epsilon$    & \cmd{\epsilon}    & $\lambda$   & \cmd{\lambda} &
    $\varrho$     & \cmd{\varrho}     & $\chi$      & \cmd{\chi} \\
    $\varepsilon$ & \cmd{\varepsilon} & $\mu$       & \cmd{\mu} &
    $\sigma$      & \cmd{\sigma}      & $\psi$      & \cmd{\psi} \\
    $\zeta$       & \cmd{\zeta}       & $\nu$       & \cmd{\nu} &
    $\varsigma$   & \cmd{\varsigma}   & $\omega$    & \cmd{\omega} \\
    $\eta$        & \cmd{\eta}        & $\xi$       & \cmd{\xi}
  \end{tabularx}
\end{table}

\begin{table}
  \caption{Lettres grecques majuscules}
  \label{tab:math:Grecques}
  \begin{tabularx}{1.0\linewidth}{lXlXlXlX}
    $\Gamma$    & \cmd{\Gamma}   &
    $\Lambda$   & \cmd{\Lambda}  &
    $\Sigma$    & \cmd{\Sigma}   &
    $\Psi$      & \cmd{\Psi}     \\
    $\Delta$    & \cmd{\Delta}   &
    $\Xi$       & \cmd{\Xi}      &
    $\Upsilon$  & \cmd{\Upsilon} &
    $\Omega$    & \cmd{\Omega}   \\
    $\Theta$    & \cmd{\Theta}   &
    $\Pi$       & \cmd{\Pi}      &
    $\Phi$      & \cmd{\Phi}
  \end{tabularx}
\end{table}

\subsection{Lettres modifiées}
\label{tab:math:symboles:mathcal}

Les lettres de l'alphabet, principalement en majuscule, servent
parfois en mathématiques dans des versions modifiées pour représenter
des quantités, notamment les ensembles.

\begin{itemize}
\item La commande \cmd{\mathcal} permet de transfomer un ou plusieurs
  caractères en version dite «calligraphique» dans le mode mathématique.
  \begin{demo}
    \begin{texample}
\begin{lstlisting}
\mathcal{ABC\, xyz}
\end{lstlisting}
      \producing
      %% pris de lucidaot.tex; apparemment lent
      \setmathfont[RawFeature=+ss04]{Lucida Bright Math OT}
      $\mathscr{ABC\, xyz}$
    \end{texample}
  \end{demo}
\item La commande \cmd{\mathbb} fournie, entre autres, par les
  paquetages \pkg{amsfonts} et \pkg{unicode-math} génère des versions
  majuscule ajourée (\emph{blackboard bold}) des lettres de
  l'alphabet. Elles sont principalement utilisée pour représenter les
  ensembles de nombres.
  \begin{demo}
    \begin{texample}
\begin{lstlisting}
\mathbb{NZRC}
\end{lstlisting}
      \producing
      $\mathbb{NZRC}$
    \end{texample}
  \end{demo}
\item Le tableau~213 de la %
  \doc[Comprehensive {\LaTeX} Symbol List]{}{http://texdoc.net/pkg/comprehensive} %
  présente plusieurs autres alphabets spéciaux disponibles en
  mode mathématique.
\item Certaines polices de caractères OpenType contiennent plusieurs
  versions des symboles mathématiques. Par exemple, la police utilisée
  dans le présent document contient deux versions de la police
  calligraphique, celle présentée ci-dessus et celle-ci: %
  \setmathfont[RawFeature=-ss04]{Lucida Bright Math OT}
  $\mathscr{ABC\, xyz}$. Consulter éventuellement la documentation de
  la police pour les détails.
\end{itemize}

\subsection{Opérateurs binaires et relations}
\label{sec:math:symboles:binaires+relations}

Les opérateurs binaires combinent deux quantités pour en former une
troisième; pensons simplement aux opérateurs d'addition $+$ et de
soustraction $-$ que l'on retrouve sur un clavier d'ordinateur normal.
Les relations, quant à elles, servent pour la comparaison entre deux
quantités, comme $<$ et $>$. Le \autoref{tab:math:binaires}
présente une sélection d'opérateurs binaires et
le \autoref{tab:math:relations}  une sélection de relations.

\begin{itemize}
\item La %
  \doc[Comprehensive {\LaTeX} Symbol
  List]{}{http://texdoc.net/pkg/comprehensive} %
  consacre plus d'une dizaine de tableaux aux opérateurs binaires et
  près d'une quarantaine aux relations. C'est dire à quel point les
  tableaux \ref{tab:math:binaires} et \ref{tab:math:relations} ne
  présentent que les principaux éléments à titre indicatif.
\item Certaines relations existent directement en version opposée, ou
  négative (comme $\neq$ ou $\notin$) soit dans {\LaTeX} de base, soit
  avec \pkg{amsmath} ou un autre paquetage. Autrement, il est possible
  de préfixer toute relation de \cmd{\not} pour y supersposer une
  barre oblique $/$.
\end{itemize}

\begin{table}
  \caption{Quelques opérateurs binaires}
  \label{tab:math:binaires}
  \begin{tabularx}{1.0\linewidth}{lXlXlXlX}
    $\times$    & \cmd{\times} &
    $\div$      & \cmd{\div}   &
    $\pm$       & \cmd{\pm}    &
    $\mp$       & \cmd{\mp}    \\
    $\cup$      & \cmd{\cup} &
    $\cap$      & \cmd{\cap} &
    $\setminus$ & \cmd{\setminus} &
    $\cdot$     & \cmd{\cdot}  \\
    $\wedge$    & \cmd{\wedge} &
    $\vee$      & \cmd{\vee} &
    $\oplus$    & \cmd{\oplus} &
    $\otimes$   & \cmd{\otimes}
  \end{tabularx}
\end{table}

\begin{table}
  \caption{Quelques relations et leur négation}
  \label{tab:math:relations}
  \begin{tabularx}{1.0\linewidth}{lXlXlXlX}
    $\leq$      & \cmd{\leq} &
    $\geq$      & \cmd{\geq}   &
    $\neq$      & \cmd{\neq}    &
    $\equiv$    & \cmd{\equiv}    \\
    $\subset$   & \cmd{\subset} &
    $\subseteq$ & \cmd{\subseteq}  &
    $\in$       & \cmd{\in} &
    $\notin$    & \cmd{\notin} \\
    $\nless$    & \cmd{\nless}$^\dagger$ &
    $\ngtr$     & \cmd{\ngtr}$^\dagger$   &
    $\nleq$     & \cmd{\nleq}$^\dagger$    &
    $\ngeq$     & \cmd{\ngeq}$^\dagger$ \\
    \addlinespace
  \end{tabularx}
  \hspace*{1em}{\footnotesize $^\dagger$ requiert \pkg{amsmath}}
\end{table}

\subsection{Flèches}
\label{sec:math:symboles:fleches}

Les flèches de différents types sont souvent utilisées en notation
mathématique, notamment dans les limites ou pour les expressions
logiques. Le \autoref{tab:math:fleches} en présente une sélection.

\begin{itemize}
\item On retrouve les flèches utilisables en notation mathématique
  dans les tableaux 102 à 119 de la %
  \doc[Comprehensive {\LaTeX} Symbol
  List]{}{http://texdoc.net/pkg/comprehensive}. %
  Le document contient divers autres types de flèches, mais celles-ci
  ne sont généralement pas appropriées pour les mathématiques (pensons
  à {\manerrarrow} ou {\faArrowRight}).
\item Le paquetage \pkg{amsmath} fournit plusieurs flèches
  additionnelles ainsi que la négation des plus communes. Ces
  dernières sont d'ailleurs présentées dans le
  \autoref{tab:math:fleches}.
\end{itemize}


\begin{table}
  \caption{Quelques flèches et leur négation}
  \label{tab:math:fleches}
  \begin{tabularx}{1.0\linewidth}{lXlX}
    $\gets$                & \cmd{\leftarrow}\quad \cmd{\gets} &
    $\longleftarrow$       & \cmd{\longleftarrow}              \\
    $\Leftarrow$           & \cmd{\Leftarrow}                  &
    $\Longleftarrow$       & \cmd{\Longleftarrow}              \\
    $\to$                  & \cmd{\rightarrow}\quad \cmd{\to}  &
    $\longrightarrow$      & \cmd{\longrightarrow}             \\
    $\Rightarrow$          & \cmd{\Rightarrow}                 &
    $\Longrightarrow$      & \cmd{\Longrightarrow}             \\
    $\uparrow$             & \cmd{\uparrow}                    &
    $\downarrow$           & \cmd{\downarrow}                  \\
    $\Uparrow$             & \cmd{\Uparrow}                    &
    $\Downarrow$           & \cmd{\Downarrow}                  \\
    $\updownarrow$         & \cmd{\updownarrow}                &
    $\Updownarrow$         & \cmd{\Updownarrow}                \\
    $\leftrightarrow$      & \cmd{\leftrightarrow}             &
    $\longleftrightarrow$  & \cmd{\longleftrightarrow}         \\
    $\Leftrightarrow$      & \cmd{\Leftrightarrow}             &
    $\Longleftrightarrow$  & \cmd{\Longleftrightarrow}         \\
    $\nleftarrow$          & \cmd{\nleftarrow}$^\dagger$         &
    $\nleftrightarrow$     & \cmd{\nleftrightarrow}$^\dagger$    \\
    $\nrightarrow$         & \cmd{\nrightarrow}$^\dagger$        &
    $\nLeftarrow$          & \cmd{\nLeftarrow}$^\dagger$         \\
    $\nLeftrightarrow$     & \cmd{\nLeftrightarrow}$^\dagger$    &
    $\nRightarrow$         & \cmd{\nRightarrow}$^\dagger$        \\
    \addlinespace
  \end{tabularx}
  \hspace*{1em}{\footnotesize $^\dagger$ requiert \pkg{amsmath}}
\end{table}

\subsection{Autres symboles utiles}
\label{sec:math:symboles:autres}

Le \autoref{tab:math:autres} présente quelques autres symboles
mathématiques plus fréquemment utilisés fournis par {\LaTeX}.

\begin{table}[h]
  \caption{Symboles mathématiques divers}
  \label{tab:math:autres}
  \begin{tabularx}{1.0\linewidth}{lXlXlXlX}
    $\infty$ & \cmd{\infty} &
    $\nabla$ & \cmd{\nabla} &
    $\partial$ & \cmd{\partial} &
    $\ell$ & \cmd{\ell} \\
    $\forall$ & \cmd{\forall} &
    $\exists$ & \cmd{\exists} &
    $\emptyset$ & \cmd{\emptyset} &
    $\prime$ & \cmd{\prime} \\
    $\neg$ & \cmd{\neg} &
    $\backslash$ & \cmd{\backslash} &
    $\|$ & \cmd{\|} &
    $\angle$ & \cmd{\angle}
  \end{tabularx}
\end{table}


\begin{exemple}
  Les éléments de cette section et de la précédente permettent déjà de
  créer des équations élaborées. Considérons l'équation suivante:
  \begin{equation*}
    \frac{\Gamma(\alpha)}{\lambda^\alpha} =
    \sum_{j = 0}^\infty \int_j^{j + 1} x^{\alpha - 1} e^{-\lambda x}\,
    dx,
    \quad
    \alpha > 0 \text{ et } \lambda > 0.
  \end{equation*}
  Voici, pièce par pièce, le code {\LaTeX} pour créer cette équation.
  \begin{demo}
    \begin{texample}[0.58\linewidth]
\begin{lstlisting}
\begin{equation*}
\end{lstlisting}
      \producing
      \emph{équation hors paragraphe}
    \end{texample}

    \begin{texample}[0.58\linewidth]
\begin{lstlisting}
\frac{\Gamma(\alpha)}{
  \lambda^\alpha} =
\end{lstlisting}
      \producing
      $\dfrac{\Gamma(\alpha)}{\lambda^\alpha} =$
    \end{texample}

    \begin{texample}[0.58\linewidth]
\begin{lstlisting}
\sum_{j = 0}^\infty \int_j^{j + 1}
\end{lstlisting}
      \producing
      $\displaystyle \sum_{j = 0}^\infty \int_j^{j + 1}$
    \end{texample}

    \begin{texample}[0.58\linewidth]
\begin{lstlisting}
x^{\alpha - 1} e^{-\lambda x}\, dx
\end{lstlisting}
      \producing
      $\displaystyle x^{\alpha - 1} e^{-\lambda x}\, dx$
    \end{texample}

    \begin{texample}[0.58\linewidth]
\begin{lstlisting}
, \quad \alpha > 0 \text{ et }
\lambda > 0.
\end{lstlisting}
      \producing
      $, \quad \alpha > 0 \text{ et } \lambda > 0.$
    \end{texample}

    \begin{texample}[0.58\linewidth]
\begin{lstlisting}
\end{equation*}
\end{lstlisting}
      \producing
      \emph{fin de l'environnement}
    \end{texample}
  \end{demo}
  \qed
\end{exemple}


\section{Équations sur plusieurs lignes et numérotation}
\label{sec:math:align}

Dans ce qui précède, nous n'avons présenté que des équations tenant
sur une seule ligne en mode hors paragraphe. Cette section se penche
sur la manière de représenter des groupes d'équations du type
\begin{align}
  y &= 2x + 4 \\
  y &= 6x - 1
\end{align}
ou des suites d'équations comme
\begin{align*}
  x_{\text{max}}
  &= \sum_{i = 0}^{m - 1} (b - 1) b^i \\
  &= (b - 1) \sum_{i = 0}^{m - 1} b^i \\
  &= b^m - 1.
\end{align*}

Nous recommandons fortement les environnements de \pkg{amsmath} pour
les équations sur plusieurs lignes: ils sont plus polyvalents, plus
simples à utiliser et leur rendu est meilleur. Le
\autoref{tab:math:displays} --- repris presque intégralement de la
documentation de ce paquetage --- compare les différents
environnements pour les équations hors paragraphe.

\begin{table}[p]
  \caption{Comparaison des environnements pour les équations hors
    paragraphe de \pkg{amsmath} (les lignes verticales indiquent les
    marges logiques).}
  \label{tab:math:displays}
  \renewcommand{\theequation}{\arabic{equation}}
  \begin{eqxample}
\begin{lstlisting}
\begin{equation*}
  a = b
\end{equation*}
\end{lstlisting}
    \producing
    \begin{equation*}
      a = b
    \end{equation*}
  \end{eqxample}

  \begin{eqxample}
\begin{lstlisting}
\begin{equation}
  a = b
\end{equation}
\end{lstlisting}
    \producing
    \begin{equation}
      a = b
    \end{equation}
  \end{eqxample}

  \begin{eqxample}
\begin{lstlisting}
\begin{equation}
  \label{xx}
  \begin{split}
    a &= b + c - d \\
    &\phantom{=} + e - f \\
    &= g + h \\
    &= i
  \end{split}
\end{equation}
\end{lstlisting}
    \producing
    \begin{equation}\label{eq:math:xx}
      \begin{split}
        a& =b+c-d\\
        &\phantom{=} +e-f\\
        & =g+h\\
        & =i
      \end{split}
    \end{equation}
  \end{eqxample}

  \begin{eqxample}
\begin{lstlisting}
\begin{multline}
  a + b + c + d + e + f \\
  + i + j + k + l + m + n
\end{multline}
\end{lstlisting}
    \producing
    \begin{multline}
      a+b+c+d+e+f\\
      +i+j+k+l+m+n
    \end{multline}
  \end{eqxample}

  \begin{eqxample}
\begin{lstlisting}
\begin{gather}
  a_1 = b_1 + c_1 \\
  a_2 = b_2 + c_2 - d_2 + e_2
\end{gather}
\end{lstlisting}
    \producing
    \begin{gather}
      a_1=b_1+c_1\\
      a_2=b_2+c_2-d_2+e_2
    \end{gather}
  \end{eqxample}

  \begin{eqxample}
\begin{lstlisting}
\begin{align}
  a_1 &= b_1 + c_1 \\
  a_2 &= b_2 + c_2 - d_2 + e_2
\end{align}
\end{lstlisting}
    \producing
    \begin{align}
      a_1& =b_1+c_1\\
      a_2& =b_2+c_2-d_2+e_2
    \end{align}
  \end{eqxample}

  \begin{eqxample}
\begin{lstlisting}
\begin{align}
  a_{11} &= b_{11} &
  a_{12} &= b_{12} \\
  a_{21} &= b_{21} &
  a_{22} &= b_{22} + c_{22}
\end{align}
\end{lstlisting}
    \producing
    \begin{align}
      a_{11}& =b_{11}&
      a_{12}& =b_{12}\\
      a_{21}& =b_{21}&
      a_{22}& =b_{22}+c_{22}
    \end{align}
  \end{eqxample}
\end{table}

\begin{itemize}
\item L'environnement de base pour les équations alignées sur un
  symbole de relation (en une ou plusieurs colonnes) est \Ie{align}.
  C'est l'environnement le plus utilisé en mode mathématique hormis
  \Ie{equation}.
\item Remarquer, dans le troisième exemple du
  \autoref{tab:math:displays}, comment la commande \cmd{\phantom} sert
  à insérer un blanc exactement de la largeur du symbole $=$ au début
  de la seconde ligne de la suite d'égalités.
\item Pour supprimer la numérotation d'une ligne dans une série
  d'équations numérotées, placer la commande \cmd{\notag} juste avant
  la commande de changement de ligne \verb=\\=.
  \begin{demo}
    \begin{texample}
\begin{lstlisting}
\begin{align}
  a_1 &= b_1 + c_1 \notag \\
  a_2 &= b_2 + c_2 - d_2 + e_2
\end{align}
\end{lstlisting}
      \producing
      \begin{align}
        a_1& =b_1+c_1 \notag \\
        a_2& =b_2+c_2-d_2+e_2
      \end{align}
    \end{texample}
  \end{demo}
\item Les renvois vers des équations numérotées fonctionnent, comme
  partout ailleurs en {\LaTeX}, avec le système d'étiquettes et de
  références \citep{UL:latex:1,sec:tableaux:floats}. Le paquetage
  \pkg{amsmath} fournit également la pratique commande \cmd{\eqref}
  qui place automatiquement le numéro d'équation entre parenthèses.
  \begin{demo}
    \begin{texample}
\begin{lstlisting}
On voit en \eqref{xx} du
tableau `\ref*{tab:math:displays}' que...
\end{lstlisting}
      \producing
      On voit en \eqref{eq:math:xx} du tableau
      \ref*{tab:math:displays} que\dots
    \end{texample}
  \end{demo}
\item L'environnement \Ie{split} sert à apposer un seul numéro à une
  équation affichée sur plusieurs lignes. Il doit être employé à
  l'intérieur d'un autre environnement d'équations hors paragraphe.
\end{itemize}
Consulter le chapitre 3 de la %
\doc[documentation]{amsldoc}{http://texdoc.net/pkg/amsmath} %
du paquetage \pkg{amsmath} pour les détails sur
l'utilisation des environnements du \autoref*{tab:math:displays}.

\begin{conseil}
  Veillez à respecter les règles suivantes pour la mise en forme des
  équations:
  \begin{enumerate}
  \item Qu'elles apparaissent en ligne ou hors paragraphe, les
    équations font partie intégrante de la phrase, aussi les règles de
    ponctuation usuelles s'appliquent-elles aux équations.
  \item Lorsqu'une équation s'étend sur plus d'une ligne, couper
    chaque ligne \emph{avant} un opérateur de sorte que chaque ligne
    constitue une expression mathématique complète (voir les troisième
    et quatrième exemples du \autoref{tab:math:displays}).
  \item Ne numéroter que les équations d'un document auxquelles le
    texte fait référence.
  \end{enumerate}
\end{conseil}



left ... right

vecteurs et matrices

%%%
%%% Exercices
%%%

\section{Exercices}
\label{sec:math:exercices}

\Opensolutionfile{solutions}[solutions-mathematiques]

\begin{Filesave}{solutions}
\section*{Chapitre \ref*{chap:math}}
\addcontentsline{toc}{section}{Chapitre \protect\ref*{chap:math}}

\end{Filesave}



\Closesolutionfile{solutions}

%%% Local Variables:
%%% mode: latex
%%% TeX-engine: xetex
%%% TeX-master: "formation_latex-partie_2"
%%% coding: utf-8
%%% End:


\end{document}

%%% Local Variables:
%%% mode: latex
%%% TeX-master: t
%%% End:
